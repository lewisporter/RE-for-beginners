\begin{center}
\vspace*{\fill}

\Huge \RU{Внимание: это сокращенная LITE-версия}\EN{Warning: this is a shortened LITE-version}\ESph{}\PTBRph{}\PLph{}\ITAph{}!
\normalsize

\bigskip
\bigskip
\bigskip

\Large
\RU{Она примерно в 6 раз короче полной версии (\textasciitilde{}150 страниц) и предназначена для тех,
кто хочет краткого введения в основы reverse engineering.
Здесь нет ничего о MIPS, ARM, OllyDBG, GCC, GDB, IDA, нет задач, примеров, \etc.}
\EN{It is approximately 6 times shorter than full version (\textasciitilde{}150 pages) and intended to those
who wants for very quick introduction to reverse engineering basics.
There are nothing about MIPS, ARM, OllyDBG, GCC, GDB, IDA, there are no exercises, examples, etc.}
\ESph{}\PTBRph{}\PLph{}\ITAph{}
\normalsize

\bigskip
\bigskip
\bigskip

\RU{Если вам всё ещё интересен reverse engineering, полная версия книги всегда доступна на моем сайте}%
\EN{If you still interesting in reverse engineering, full version of the book is always available on my website}\ESph{}\PTBRph{}\PLph{}\ITAph{}: 
\href{http://go.yurichev.com/17009}{beginners.re}.

\vspace*{\fill}
\vfill
\end{center}
