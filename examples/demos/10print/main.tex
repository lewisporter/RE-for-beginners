\section{10 PRINT CHR\$(205.5+RND(1)); : GOTO 10}

\RU{Все примеры здесь для .COM-файлов под MS-DOS}\EN{All examples here are MS-DOS .COM files}.
%FIXME1 -> about .COM files

\index{MS-DOS}
\RU{В}\EN{In} \cite{10PRINT} \RU{можно прочитать об одном из простейших генераторов случайных лабиринтов}
\EN{we can read about one of the most simple possible random maze generators}.
\RU{Он просто бесконечно и случайно печатает символ слэша или обратный слэша, выдавая в итоге что-то вроде}
\EN{It just prints a slash or backslash characters randomly and endlessly, resulting in something like this}:

\begin{figure}[H]
\centering
\includegraphics[scale=\NormalScale]{examples/demos/10print/10print.png}
\end{figure}

\RU{Здесь несколько известных реализаций для 16-битного x86}
\EN{There are a few known implementations for 16-bit x86}.

\subsection{\RU{Версия 42-х байт от Trixter}\EN{Trixter's 42 byte version}}

\newcommand{\FNURLTRIXTER}{\footnote{\url{http://go.yurichev.com/17305}}}

\RU{Листинг взят с его сайта}\EN{The listing was taken from his website}\FNURLTRIXTER, \RU{но комментарии\EMDASH{}мои}
\EN{but the comments are mine}.

\lstinputlisting{examples/demos/10print/10print_42.lst.\LANG}

\index{Intel!8253}
\RU{Псевдослучайное число на самом деле это время, прошедшее со старта системы, получаемое из чипа таймера 8253, 
это значение
увеличивается на единицу 18.2 раза в секунду}\EN{The pseudo-random value here is in fact the time 
that has passed from the system's boot, taken from the 8253 time chip, the value increases by one 18.2 times per second}.

\RU{Записывая ноль в порт}\EN{By writing zero to port} \TT{43h}, 
\RU{мы имеем ввиду что команда это}\EN{we send the command} \q{\RU{выбрать счетчик}\EN{select counter} 0}, 
"counter latch", 
"\RU{двоичный счетчик}\EN{binary counter}" (\RU{а не значение \ac{BCD}}\EN{not a \ac{BCD} value}).

\index{x86!\Instructions!POPF}
\RU{Прерывания снова разрешаются при помощи инструкции}\EN{The interrupts are enabled back with the} \TT{POPF}\RU{, которая
также возвращает флаг \TT{IF}}\EN{ instruction, which restores the \TT{IF} flag as well}.

\index{x86!\Instructions!IN}
\RU{Инструкцию \TT{IN} нельзя использовать с другими регистрами кроме}\EN{It is not possible to 
use the \TT{IN} instruction with registers other than} \TT{AL}, \RU{поэтому здесь перетасовка}
\EN{hence the shuffling}.

\subsection{\RU{Моя попытка укоротить версию Trixter: 27 байт}
\EN{My attempt to reduce Trixter's version: 27 bytes}}

\RU{Мы можем сказать, что мы используем таймер не для того чтобы получить точное время, но псевдослучайное число,
так что мы можем не тратить время (и код) на запрещение прерываний}\EN{We can say that since we use the timer not 
to get a precise time value, but a pseudo-random one, we do not need
to spend time (and code) to disable the interrupts}.
\RU{Еще можно сказать, что так как мы берем бит из младшей 8-битной части, то мы можем считывать только её}
\EN{Another thing we can say is that we need only one bit from the low 8-bit part, so let's read only it}.

\RU{Немного укоротим код и выходит 27 байт}\EN{We can reduced the code slightly and we've got 27 bytes}:

\lstinputlisting{examples/demos/10print/10print_27.lst.\LANG}

\subsection{\RU{Использование случайного мусора в памяти как источника случайных чисел}
\EN{Taking random memory garbage as a source of randomness}}

\RU{Так как это}\EN{Since it is} MS-DOS, \RU{защиты памяти здесь нет вовсе, так что мы можем читать с какого
угодно адреса}\EN{there is no memory protection at all, we can read from whatever address we want}.
\index{x86!\Instructions!LODSB}
\RU{И даже более того: простая инструкция}\EN{Even more than that: a simple} \TT{LODSB} 
\RU{будет читать байт по адресу}\EN{instruction reads a byte from the} \TT{DS:SI}\RU{, но это не проблема
если правильные значения не установлены в регистры, пусть она читает 1) случайные байты; 2) из случайного
места в памяти!}\EN{ address, but it's not a problem
if the registers' values are not set up, let it read 1) random bytes; 2) from a random place in memory!}

\RU{Так что на странице Trixter-а}\EN{It is suggested in Trixter's webpage}\FNURLTRIXTER 
\RU{можно найти предложение использовать}\EN{to use} \TT{LODSB} \RU{без всякой инициализации}\EN{without any setup}.

\index{x86!\Instructions!SCASB}
\RU{Есть также предложение использовать инструкцию}\EN{It is also suggested that the} \TT{SCASB} 
\RU{вместо, потому что она выставляет флаги в соответствии с прочитанным значением}
\EN{instruction can be used instead, because it sets a flag according to the byte it reads}.

\RU{Еще одна идея насчет минимизации кода\EMDASH{}это использовать прерывание DOS}
\EN{Another idea to minimize the code is to use the} \TT{INT 29h} \RU{которое просто печатает символ на экране
из регистра}\EN{DOS syscall, which just prints the character stored in the} \TT{AL}\EN{ register}.

\RU{Это то что сделали}\EN{That is what} Peter Ferrie \AndENRU \HERMIT{}\EN{ did} (11 \AndENRU 
10 \RU{байт}\EN{bytes})
\footnote{\url{http://go.yurichev.com/17087}}:

\lstinputlisting[caption=\HERMIT: 11 \RU{байт}\EN{bytes}]{examples/demos/10print/herm1t_11.lst.\LANG}

\index{x86!\Instructions!SCASB}
\TT{SCASB} \RU{также использует значение в регистре}\EN{also uses the value in the} \TT{AL}\RU{, она вычитает значение
случайного байта в памяти из}
\EN{ register, it subtract a random memory byte's value from the} \TT{5Ch} \RU{в}\EN{value in} \TT{AL}.
\index{x86!\Instructions!JP}
\TT{JP} \RU{это редкая инструкция, здесь она используется для проверки флага четности (PF),
который вычисляется по формуле в листинге}\EN{is a rare instruction, here it used for checking the parity flag (PF), 
which is generated by the formulae in the listing}.
\RU{Как следствие, выводимый символ определяется не каким-то конкретным битом из случайного байта в памяти,
а суммой бит, и это (надеемся) сделает результат более распределенным}\EN{As a consequence, the output character 
is determined not by some bit in a random memory byte, but by a sum of bits, 
this (hopefully) makes the result more distributed}.

\index{x86!\Instructions!SALC}
\index{x86!\Instructions!SETALC}
\index{NEC V20}
\RU{Можно сделать еще короче, если использовать недокументированную x86-инструкцию}\EN{It is possible to 
make this even shorter by using the undocumented x86 instruction} \TT{SALC} (\ac{AKA} \TT{SETALC}) (\q{Set AL CF}).
\RU{Она появилась в}\EN{It was introduced in the NEC V20} \ac{CPU} \RU{и выставляет}\EN{and sets} \TT{AL} \RU{в}\EN{to} 
\TT{0xFF} \RU{если}\EN{if} \TT{CF} \RU{это 1 или 0 если наоборот}\EN{is 1 or to 0 if otherwise}.

\lstinputlisting[caption=Peter Ferrie: 10 \RU{байт}\EN{bytes}]{examples/demos/10print/ferrie_10.lst.\LANG}

\RU{Так что можно избавиться и от условных переходов}\EN{So it is possible to get rid of conditional jumps at all}.
\EN{The }\ac{ASCII}\RU{-код обратного слэща}\EN{ code of backslash} (\q{\textbackslash{}}) 
\RU{это}\EN{is} \TT{0x5C} \AndENRU \TT{0x2F} \RU{для слэша}\EN{for slash} (\q{/}).
\RU{Так что нам нужно конвертировать один (псевдослучайный) бит из флага}\EN{So we need to convert one 
(pseudo-random) bit in the} \TT{CF} \RU{в значение}\EN{flag to a value of} \TT{0x5C} \OrENRU \TT{0x2F}.

\RU{Это делается легко: применяя операцию \q{И} ко всем битам в}\EN{This is done easily: by \TT{AND}-ing all 
bits in} \TT{AL} (\RU{где все 8 бит либо выставлены, либо сброшены}\EN{where all 8 bits are set or cleared}) 
\RU{с}\EN{with} \TT{0x2D} \RU{мы имеем просто}\EN{we have just} 0 \OrENRU \TT{0x2D}.
\RU{Прибавляя значение}\EN{By adding} \TT{0x2F} \RU{к этому значению, мы получаем}\EN{to this value, we get} 
\TT{0x5C} \OrENRU \TT{0x2F}.
\RU{И просто выводим это на экран}\EN{Then we just output it to the screen}.

\subsection{\Conclusion{}}

\index{DOSBox}
\RU{Также стоит отметить, что результат может быть разным в эмуляторе}\EN{It is also worth mentioning that the result may 
be different in} DOSBox, \gls{Windows NT} \RU{и даже}\EN{and even} MS-DOS, 
\RU{из-за разных условий:
чип таймера может эмулироваться по-разному, изначальные значения регистров также могут быть разными}
\EN{due to different
conditions: the timer chip can be emulated differently and the initial register contents may be different as well}.
