\newcommand{\NormalScale}{0.66} % FIXME?
\ifdefined\ebook
\documentclass[a5paper,oneside]{book}
\newcommand{\FigScale}{0.4}
\else
\documentclass[a4paper,oneside]{book}
\newcommand{\FigScale}{\NormalScale} 
\fi

\usepackage{fontspec}
% fonts
\setmonofont{Droid Sans Mono}
\setmainfont[Ligatures=TeX]{PT Sans}
\usepackage{polyglossia}
\defaultfontfeatures{Scale=MatchLowercase} % ensure all fonts have the same 1ex
\usepackage{ucharclasses}
\usepackage{csquotes}

\ifdefined\ENGLISH
\setmainlanguage{english}
\setotherlanguage{russian}
\fi

\ifdefined\RUSSIAN
\setmainlanguage{russian}
\newfontfamily\cyrillicfonttt{Droid Sans Mono}
\setotherlanguage{english}
\fi

\ifdefined\SPANISH
\setmainlanguage{spanish}
\setotherlanguage{english}
\fi

\ifdefined\ITALIAN
\setmainlanguage{italian}
\setotherlanguage{english}
\fi

\ifdefined\BRAZILIAN
\setmainlanguage{portuges}
\setotherlanguage{english}
\fi

\ifdefined\POLISH
\setmainlanguage{polish}
\setotherlanguage{english}
\fi

\usepackage{microtype}
\usepackage{fancyhdr}
\usepackage{listings}
\usepackage{ulem}
\usepackage{url}
\usepackage{graphicx}
\usepackage{makeidx}
\usepackage[backend=biber,style=alphabetic]{biblatex}
%\usepackage{cite}
\usepackage[cm]{fullpage}
\usepackage{color}
\usepackage{fancyvrb}
\usepackage{xspace}
\usepackage{tabularx}
\usepackage{framed}
\usepackage{parskip}
\usepackage{epigraph}
\usepackage{ccicons}
\usepackage[nottoc]{tocbibind}
\usepackage{longtable}
\usepackage[footnote,printonlyused,withpage]{acronym}
\usepackage[table]{xcolor}% http://ctan.org/pkg/xcolor
\usepackage[]{bookmark,hyperref} % must be last
\usepackage[official]{eurosym}

% ************** myref
% http://tex.stackexchange.com/questions/228286/how-to-mix-ref-and-pageref#228292
\ifdefined\RUSSIAN
\newcommand{\myref}[1]{%
  \ref{#1} 
  (стр.~\pageref{#1})%
  }
% FIXME: I wasn't able to force varioref to output russian text...
\else
\usepackage{varioref}
\newcommand{\myref}[1]{\vref{#1}}
\fi
% ************** myref

\usepackage{glossaries}
\usepackage{tikz}
%\usepackage{fixltx2e}
\usepackage{bytefield}

\usepackage{amsmath}
\usepackage{MnSymbol}
\undef\mathdollar 

\usepackage{float}

\usepackage{shorttoc}
\usetikzlibrary{calc,positioning,chains,arrows}
\ifdefined\ebook
\usepackage[margin=0.5in,headheight=11pt]{geometry}
\else
\usepackage[margin=0.5in,headheight=12.5pt]{geometry}
\fi

\ifdefined\RUSSIAN
\renewcommand\lstlistingname{Листинг}
\renewcommand\lstlistlistingname{Листинг}
\fi

%\iffalse
% fancyhdr ********************************************************
\makeatletter
    \let\stdchapter\chapter
    \renewcommand*\chapter{%
    \@ifstar{\starchapter}{\@dblarg\nostarchapter}}
    \newcommand*\starchapter[1]{%
        \stdchapter*{#1}
        \thispagestyle{fancy}
        \markboth{\MakeUppercase{#1}}{}
    }
    \def\nostarchapter[#1]#2{%
        \stdchapter[{#1}]{#2}
        \thispagestyle{fancy}
    }
\makeatother

% taken from http://texblog.org/tag/fancyhdr-font-size/
\newcommand{\changefont}{%
\ifdefined\ebook
    \fontsize{6}{7}\selectfont
\else
    \fontsize{8}{9.5}\selectfont
\fi
}
\fancyhf{}
\fancyhead[L,RO]{\changefont \slshape \rightmark} %section
\fancyhead[R,LO]{\changefont \slshape \leftmark} %chapter
\fancyfoot[C]{\changefont \thepage} %footer
% *****************************************************************
%\fi

\newcommand{\footnoteref}[1]{\textsuperscript{\ref{#1}}}

\definecolor{lstbgcolor}{rgb}{0.94,0.94,0.94}
\makeindex

\newcommand*{\TT}[1]{\texttt{#1}}
\newcommand*{\IT}[1]{\textit{#1}}
\newcommand*{\EN}[1]{\iflanguage{english}{#1}{}}

\ifdefined\RUSSIAN{}
\newcommand*{\RU}[1]{\iflanguage{russian}{#1}{}}
\else
\newcommand*{\RU}[1]{}
\fi

\ifdefined\SPANISH{}
\newcommand*{\ES}[1]{\iflanguage{spanish}{#1}{}}
\else
\newcommand*{\ES}[1]{}
\fi

\ifdefined\ITALIAN{}
\newcommand*{\ITA}[1]{\iflanguage{italian}{#1}{}}
\else
\newcommand*{\ITA}[1]{}
\fi

\ifdefined\BRAZILIAN{}
\newcommand*{\PTBR}[1]{\iflanguage{portuges}{#1}{}}
\else
\newcommand*{\PTBR}[1]{}
\fi

\ifdefined\POLISH{}
\newcommand*{\PL}[1]{\iflanguage{polish}{#1}{}}
\else
\newcommand*{\PL}[1]{}
\fi

\newcommand{\LANG}{\RU{ru}\EN{en}\ES{es}\PTBR{ptbr}\PL{pl}\ITA{it}}

\newcommand{\ESph}{\ES{Spanish text placeholder}}
\newcommand{\PTBRph}{\PTBR{Brazilian Portuguese text placeholder}}
\newcommand{\PLph}{\PL{Polish text placeholder}}
\newcommand{\ITAph}{\ITA{Italian text placeholder}}

\newcommand*{\dittoclosing}{---''---}
\newcommand*{\EMDASH}{\RU{~--- }\EN{---}}
\newcommand*{\AsteriskOne}{${}^{*}$}
\newcommand*{\AsteriskTwo}{${}^{**}$}
\newcommand*{\AsteriskThree}{${}^{***}$}
\newcommand{\q}[1]{\enquote{#1}}
\newcommand{\var}[1]{\textit{#1}}

\newcommand{\ttf}{\TT{f()}\xspace}
\newcommand{\ttfone}{\TT{f1()}\xspace}

% without dot!
\newcommand{\etc}{\RU{и~т.д}\EN{etc}\PTBRph{}\ESph{}\PLph{}\ITAph{}}

% http://tex.stackexchange.com/questions/32160/new-line-after-paragraph
\newcommand{\myparagraph}[1]{\paragraph{#1}\mbox{}\\} 

\newcommand{\figname}{\RU{илл}\EN{fig}\PTBRph{}\ESph{}\PLph{}\ITAph{}.\xspace}
\newcommand{\figref}[1]{\figname{}\ref{#1}\xspace}
\newcommand{\listingname}{\RU{листинг}\EN{listing}\PTBRph{}\ESph{}\PLph{}\ITAph{}.\xspace}
\newcommand{\lstref}[1]{\listingname{}\ref{#1}\xspace}
\newcommand{\bitENRU}{\RU{бит}\EN{bit}\PTBRph{}\ESph{}\PLph{}\ITAph{}\xspace}
\newcommand{\bitsENRU}{\RU{бита}\EN{bits}\PTBRph{}\ESph{}\PLph{}\ITAph{}\xspace}
\newcommand{\Sourcecode}{\RU{Исходный код}\EN{Source code}\PTBRph{}\ESph{}\PLph{}\ITAph{}\xspace}
\newcommand{\Seealso}{\RU{См. также}\EN{See also}\PTBRph{}\ESph{}\PLph{}\ITAph{}\xspace}
\newcommand{\MacOSX}{Mac OS X\xspace}

% FIXME TODO non-overlapping color!
% \newcommand{\headercolor}{\cellcolor{blue!25}}
\newcommand{\headercolor}{}

\newcommand{\tableheader}{\headercolor{} \RU{смещение}\EN{offset}\PTBRph{}\ESph{}\PLph{}\ITAph{} & \headercolor{} \RU{описание}\EN{description}\PTBRph{}\ESph{}}

\newcommand{\IDA}{\ac{IDA}\xspace}

\newcommand{\tracer}{\protect\gls{tracer}\xspace}

\newcommand{\Tchar}{\IT{char}\xspace} 
\newcommand{\Tint}{\IT{int}\xspace}
\newcommand{\Tbool}{\IT{bool}\xspace}
\newcommand{\Tfloat}{\IT{float}\xspace}
\newcommand{\Tdouble}{\IT{double}\xspace}
\newcommand{\Tvoid}{\IT{void}\xspace}
\newcommand{\ITthis}{\IT{this}\xspace}

\newcommand{\Ox}{\TT{/Ox}\xspace}
\newcommand{\Obzero}{\TT{/Ob0}\xspace}
\newcommand{\Othree}{\TT{-O3}\xspace}

\newcommand{\oracle}{Oracle RDBMS\xspace}

\newcommand{\idevices}{iPod/iPhone/iPad\xspace}
\newcommand{\olly}{OllyDbg\xspace}

% common C functions
\newcommand{\printf}{\TT{printf()}\xspace} 
\newcommand{\puts}{\TT{puts()}\xspace} 
\newcommand{\main}{\TT{main()}\xspace} 
\newcommand{\qsort}{\TT{qsort()}\xspace} 
\newcommand{\strlen}{\TT{strlen()}\xspace} 
\newcommand{\scanf}{\TT{scanf()}\xspace} 
\newcommand{\rand}{\TT{rand()}\xspace} 


% for easier fiddling with formatting of all instructions together
\newcommand{\INS}[1]{\TT{#1}\xspace}

% x86 instructions
\newcommand{\ADD}{\INS{ADD}}
\newcommand{\ADRP}{\INS{ADRP}}
\newcommand{\AND}{\INS{AND}}
\newcommand{\CALL}{\INS{CALL}}
\newcommand{\CPUID}{\INS{CPUID}}
\newcommand{\CMP}{\INS{CMP}}
\newcommand{\DEC}{\INS{DEC}}
\newcommand{\FADDP}{\INS{FADDP}}
\newcommand{\FCOM}{\INS{FCOM}}
\newcommand{\FCOMP}{\INS{FCOMP}}
\newcommand{\FCOMI}{\INS{FCOMI}}
\newcommand{\FCOMIP}{\INS{FCOMIP}}
\newcommand{\FUCOM}{\INS{FUCOM}}
\newcommand{\FUCOMI}{\INS{FUCOMI}}
\newcommand{\FUCOMIP}{\INS{FUCOMIP}}
\newcommand{\FUCOMPP}{\INS{FUCOMPP}}
\newcommand{\FDIVR}{\INS{FDIVR}}
\newcommand{\FDIV}{\INS{FDIV}}
\newcommand{\FLD}{\INS{FLD}}
\newcommand{\FMUL}{\INS{FMUL}}
\newcommand{\MUL}{\INS{MUL}}
\newcommand{\FSTP}{\INS{FSTP}}
\newcommand{\FDIVP}{\INS{FDIVP}}
\newcommand{\IDIV}{\INS{IDIV}}
\newcommand{\IMUL}{\INS{IMUL}}
\newcommand{\INC}{\INS{INC}}
\newcommand{\JAE}{\INS{JAE}}
\newcommand{\JA}{\INS{JA}}
\newcommand{\JBE}{\INS{JBE}}
\newcommand{\JB}{\INS{JB}}
\newcommand{\JE}{\INS{JE}}
\newcommand{\JGE}{\INS{JGE}}
\newcommand{\JG}{\INS{JG}}
\newcommand{\JLE}{\INS{JLE}}
\newcommand{\JL}{\INS{JL}}
\newcommand{\JMP}{\INS{JMP}}
\newcommand{\JNE}{\INS{JNE}}
\newcommand{\JNZ}{\INS{JNZ}}
\newcommand{\JNA}{\INS{JNA}}
\newcommand{\JNAE}{\INS{JNAE}}
\newcommand{\JNB}{\INS{JNB}}
\newcommand{\JNBE}{\INS{JNBE}}
\newcommand{\JZ}{\INS{JZ}}
\newcommand{\JP}{\INS{JP}}
\newcommand{\Jcc}{\INS{Jcc}}
\newcommand{\SETcc}{\INS{SETcc}}
\newcommand{\LEA}{\INS{LEA}}
\newcommand{\LOOP}{\INS{LOOP}}
\newcommand{\MOVSX}{\INS{MOVSX}}
\newcommand{\MOVZX}{\INS{MOVZX}}
\newcommand{\MOV}{\INS{MOV}}
\newcommand{\NOP}{\INS{NOP}}
\newcommand{\POP}{\INS{POP}}
\newcommand{\PUSH}{\INS{PUSH}}
\newcommand{\NOT}{\INS{NOT}}
\newcommand{\NOR}{\INS{NOR}}
\newcommand{\RET}{\INS{RET}}
\newcommand{\RETN}{\INS{RETN}}
\newcommand{\SETNZ}{\INS{SETNZ}}
\newcommand{\SETBE}{\INS{SETBE}}
\newcommand{\SETNBE}{\INS{SETNBE}}
\newcommand{\SUB}{\INS{SUB}}
\newcommand{\TEST}{\INS{TEST}}
\newcommand{\TST}{\INS{TST}}
\newcommand{\FNSTSW}{\INS{FNSTSW}}
\newcommand{\SAHF}{\INS{SAHF}}
\newcommand{\XOR}{\INS{XOR}}
\newcommand{\OR}{\INS{OR}}
\newcommand{\SHL}{\INS{SHL}}
\newcommand{\SHR}{\INS{SHR}}
\newcommand{\SAR}{\INS{SAR}}
\newcommand{\LEAVE}{\INS{LEAVE}}
\newcommand{\MOVDQA}{\INS{MOVDQA}}
\newcommand{\MOVDQU}{\INS{MOVDQU}}
\newcommand{\PADDD}{\INS{PADDD}}
\newcommand{\PCMPEQB}{\INS{PCMPEQB}}
\newcommand{\LDR}{\INS{LDR}}
\newcommand{\LSL}{\INS{LSL}}
\newcommand{\LSR}{\INS{LSR}}
\newcommand{\ASR}{\INS{ASR}}
\newcommand{\RSB}{\INS{RSB}}
\newcommand{\BTR}{\INS{BTR}}
\newcommand{\BTS}{\INS{BTS}}
\newcommand{\BTC}{\INS{BTC}}
\newcommand{\LUI}{\INS{LUI}}
\newcommand{\ORI}{\INS{ORI}}
\newcommand{\BIC}{\INS{BIC}}
\newcommand{\EOR}{\INS{EOR}}
\newcommand{\MOVS}{\INS{MOVS}}
\newcommand{\LSLS}{\INS{LSLS}}
\newcommand{\LSRS}{\INS{LSRS}}
\newcommand{\FMRS}{\INS{FMRS}}
\newcommand{\CMOVNE}{\INS{CMOVNE}}
\newcommand{\CMOVNZ}{\INS{CMOVNZ}}
\newcommand{\ROL}{\INS{ROL}}
\newcommand{\CSEL}{\INS{CSEL}}
\newcommand{\SLL}{\INS{SLL}}
\newcommand{\SLLV}{\INS{SLLV}}
\newcommand{\SW}{\INS{SW}}
\newcommand{\LW}{\INS{LW}}

% x86 flags

\newcommand{\ZF}{\TT{ZF}\xspace} 
\newcommand{\CF}{\TT{CF}\xspace} 
\newcommand{\PF}{\TT{PF}\xspace} 

% x86 registers

\newcommand{\AL}{\TT{AL}\xspace} 
\newcommand{\AH}{\TT{AH}\xspace} 
\newcommand{\AX}{\TT{AX}\xspace} 
\newcommand{\EAX}{\TT{EAX}\xspace} 
\newcommand{\EBX}{\TT{EBX}\xspace} 
\newcommand{\ECX}{\TT{ECX}\xspace} 
\newcommand{\EDX}{\TT{EDX}\xspace} 
\newcommand{\DL}{\TT{DL}\xspace} 
\newcommand{\ESI}{\TT{ESI}\xspace} 
\newcommand{\EDI}{\TT{EDI}\xspace} 
\newcommand{\EBP}{\TT{EBP}\xspace} 
\newcommand{\ESP}{\TT{ESP}\xspace} 
\newcommand{\RSP}{\TT{RSP}\xspace} 
\newcommand{\EIP}{\TT{EIP}\xspace} 
\newcommand{\RIP}{\TT{RIP}\xspace} 
\newcommand{\RAX}{\TT{RAX}\xspace} 
\newcommand{\RBX}{\TT{RBX}\xspace} 
\newcommand{\RCX}{\TT{RCX}\xspace} 
\newcommand{\RDX}{\TT{RDX}\xspace} 
\newcommand{\RBP}{\TT{RBP}\xspace} 
\newcommand{\RSI}{\TT{RSI}\xspace} 
\newcommand{\RDI}{\TT{RDI}\xspace} 
\newcommand*{\ST}[1]{\TT{ST(#1)}\xspace}
\newcommand*{\XMM}[1]{\TT{XMM#1}\xspace}

% ARM
\newcommand*{\Reg}[1]{\TT{R#1}\xspace}
\newcommand*{\RegX}[1]{\TT{X#1}\xspace}
\newcommand*{\RegW}[1]{\TT{W#1}\xspace}
\newcommand*{\RegD}[1]{\TT{D#1}\xspace}
\newcommand{\ADREQ}{\TT{ADREQ}\xspace}
\newcommand{\ADRNE}{\TT{ADRNE}\xspace}
\newcommand{\BEQ}{\TT{BEQ}\xspace}

% instructions descriptions
\newcommand{\ASRdesc}{\RU{арифметический сдвиг вправо}\EN{arithmetic shift right}\PTBRph{}\ESph{}\PLph{}\ITAph{}}

% x86 registers tables
% TODO: non-overlapping color!
\newcommand{\RegHeader}{
\RU{
 7 \textsuperscript{(номер байта)} &
 6 &
 5 &
 4 &
 3 &
 2 &
 1 &
 0 }
\EN{
 7th \textsuperscript{(byte number)} &
 6th &
 5th &
 4th &
 3rd &
 2nd &
 1st &
 0th}
\PTBRph{}\ESph{}
\PLph{}
\ITAph{}}

% FIXME навести порядок тут...
\newcommand{\RegTableThree}[5]{
\begin{center}
\begin{tabular}{ | l | l | l | l | l | l | l | l | l |}
\hline
\RegHeader \\
\hline
\multicolumn{8}{ | c | }{#1} \\
\hline
\multicolumn{4}{ | c | }{} & \multicolumn{4}{ c | }{#2} \\
\hline
\multicolumn{6}{ | c | }{} & \multicolumn{2}{ c | }{#3} \\
\hline
\multicolumn{6}{ | c | }{} & #4 & #5 \\
\hline
\end{tabular}
\end{center}
}

\newcommand{\RegTableOne}[5]{\RegTableThree{#1\textsuperscript{x64}}{#2}{#3}{#4}{#5}}

\newcommand{\RegTableTwo}[4]{
\begin{center}
\begin{tabular}{ | l | l | l | l | l | l | l | l | l |}
\hline
\RegHeader \\
\hline
\multicolumn{8}{ | c | }{#1\textsuperscript{x64}} \\
\hline
\multicolumn{4}{ | c | }{} & \multicolumn{4}{ c | }{#2} \\
\hline
\multicolumn{6}{ | c | }{} & \multicolumn{2}{ c | }{#3} \\
\hline
\multicolumn{7}{ | c | }{} & #4\textsuperscript{x64} \\
\hline
\end{tabular}
\end{center}
}

\newcommand{\RegTableFour}[4]{
\begin{center}
\begin{tabular}{ | l | l | l | l | l | l | l | l | l |}
\hline
\RegHeader \\
\hline
\multicolumn{8}{ | c | }{#1} \\
\hline
\multicolumn{4}{ | c | }{} & \multicolumn{4}{ c | }{#2} \\
\hline
\multicolumn{6}{ | c | }{} & \multicolumn{2}{ c | }{#3} \\
\hline
\multicolumn{7}{ | c | }{} & #4 \\
\hline
\end{tabular}
\end{center}
}


\newglossaryentry{tail call}
{
  name=\RU{хвостовая рекурсия}\EN{tail call}\ESph{}\PTBRph{}\PLph{}\ITAph{},
  description={\RU{Это когда компилятор или интерпретатор превращает рекурсию 
  (с которой возможно это проделать, т.е. \IT{хвостовую}) в итерацию для эффективности}
  \EN{It is when the compiler (or interpreter) transforms the recursion (with which it is possible: \IT{tail recursion}) 
  into an iteration for efficiency}\ESph{}\PTBRph{}\PLph{}\ITAph{}: \href{http://go.yurichev.com/17105}{wikipedia}}
}

\newglossaryentry{endianness}
{
  name=endianness,
  description={\RU{Порядок байт}\EN{Byte order}\ESph{}\PTBRph{}\PLph{}\ITAph{}: \myref{sec:endianness}}
}

\newglossaryentry{caller}
{
  name=caller,
  description={\RU{Функция вызывающая другую функцию}\EN{A function calling another}\ESph{}\PTBRph{}\PLph{}\ITAph{}}
}

\newglossaryentry{callee}
{
  name=callee,
  description={\RU{Вызываемая функция}\EN{A function being called by another}\ESph{}\PTBRph{}\PLph{}\ITAph{}}
}

\newglossaryentry{debuggee}
{
  name=debuggee,
  description={\RU{Отлаживаемая программа}\EN{A program being debugged}\ESph{}\PTBRph{}\PLph{}\ITAph{}}
}

\newglossaryentry{leaf function}
{
  name=leaf function,
  description={\RU{Функция не вызывающая больше никаких функций}
  \EN{A function which does not call any other function}\ESph{}\PTBRph{}\PLph{}\ITAph{}}
}

\newglossaryentry{link register}
{
  name=link register,
  description=(RISC) {\RU{Регистр в котором обычно записан адрес возврата.
  Это позволяет вызывать leaf-функции без использования стека, т.е. быстрее}
  \EN{A register where the return address is usually stored.
  This makes it possible to call leaf functions without using the stack, i.e., faster}\ESph{}\PTBRph{}\PLph{}\ITAph{}}
}

\newglossaryentry{anti-pattern}
{
  name=anti-pattern,
  description={\RU{Нечто широко известное как плохое решение}
  \EN{Generally considered as bad practice}\ESph{}\PTBRph{}\PLph{}\ITAph{}}
}

\newglossaryentry{stack pointer}
{
  name=\RU{указатель стека}\EN{stack pointer}\ESph{}\PTBRph{}\PLph{}\ITAph{},
  description={\RU{Регистр указывающий на место в стеке}
  \EN{A register pointing to a place in the stack}\ESph{}\PTBRph{}\PLph{}\ITAph{}}
}

\newglossaryentry{decrement}
{
  name=\RU{декремент}\EN{decrement}\ESph{}\PTBRph{}\PLph{}\ITAph{},
  description={\RU{Уменьшение на 1}\EN{Decrease by 1}\ESph{}\PTBRph{}\PLph{}\ITAph{}}
}

\newglossaryentry{increment}
{
  name=\RU{инкремент}\EN{increment}\ESph{}\PTBRph{}\PLph{}\ITAph{},
  description={\RU{Увеличение на 1}\EN{Increase by 1}\ESph{}\PTBRph{}\PLph{}\ITAph{}}
}

\newglossaryentry{loop unwinding}
{
  name=loop unwinding,
  description={\RU{Это когда вместо организации цикла на $n$ итераций, компилятор генерирует $n$ копий тела
  цикла, для экономии на инструкциях, обеспечивающих сам цикл}
  \EN{It is when a compiler, instead of generating loop code for $n$ iterations, generates just $n$ copies of the
  loop body, in order to get rid of the instructions for loop maintenance}\ESph{}\PTBRph{}\PLph{}\ITAph{}}
}

\newglossaryentry{register allocator}
{
  name=register allocator,
  description={\RU{Функция компилятора распределяющая локальные переменные по регистрам процессора}
  \EN{The part of the compiler that assigns CPU registers to local variables}\ESph{}\PTBRph{}\PLph{}\ITAph{}}
}

\newglossaryentry{quotient}
{
  name=\RU{частное}\EN{quotient}\ESph{}\PTBRph{}\PLph{}\ITAph{},
  description={\RU{Результат деления}\EN{Division result}\ESph{}\PTBRph{}\PLph{}\ITAph{}}
}

\newglossaryentry{product}
{
  name=\RU{произведение}\EN{product}\ESph{}\PTBRph{}\PLph{}\ITAph{},
  description={\RU{Результат умножения}\EN{Multiplication result}\ESph{}\PTBRph{}\PLph{}\ITAph{}}
}

\newglossaryentry{NOP}
{
  name=NOP,
  description={\q{no operation}, \RU{холостая инструкция}\EN{idle instruction}\ESph{}\PTBRph{}\PLph{}\ITAph{}}
}

\newglossaryentry{POKE}
{
  name=POKE,
  description={\RU{Инструкция языка BASIC записывающая байт по определенному адресу}
  	\EN{BASIC language instruction for writing a byte at a specific address}\ESph{}\PTBRph{}\PLph{}\ITAph{}}
}

\newglossaryentry{keygenme}
{
  name=keygenme,
  description={\RU{Программа, имитирующая защиту вымышленной программы, для которой нужно сделать 
  генератор ключей/лицензий}\EN{A program which imitates software protection,
  for which one needs to make a key/license generator}\ESph{}\PTBRph{}\PLph{}\ITAph{}}
}

\newglossaryentry{dongle}
{
  name=dongle,
  description={\RU{Небольшое устройство подключаемое к LPT-порту для принтера (в прошлом) или к USB}
  \EN{Dongle is a small piece of hardware connected to LPT printer port (in past) or to USB}\ESph{}\PTBRph{}\PLph{}\ITAph{}.
  \RU{Исполняло функции security token-а, имела память и, иногда,
  секретную (крипто-)хеширующую функцию}\EN{Its function was similar to a security token, 
  it has some memory and, sometimes, a secret (crypto-)hashing algorithm}\ESph{}\PTBRph{}\PLph{}\ITAph{}}
}

\newglossaryentry{thunk function}
{
  name=thunk function,
  description={\RU{Крохотная функция делающая только одно: вызывающая другую функцию}
  \EN{Tiny function with a single role: call another function}\ESph{}\PTBRph{}\PLph{}\ITAph{}}
}

\newglossaryentry{user mode}
{
  name=user mode,
  description={\RU{Режим CPU с ограниченными возможностями в котором он исполняет прикладное ПО. ср.}
  \EN{A restricted CPU mode in which it all application software code is executed. cf.}\ESph{}\PTBRph{}\PLph{}\ITAph{} \gls{kernel mode}}
}

\newglossaryentry{kernel mode}
{
  name=kernel mode,
  description={\RU{Режим CPU с неограниченными возможностями в котором он исполняет ядро OS и драйвера. ср.}
  \EN{A restrictions-free CPU mode in which the OS kernel and drivers execute. cf.}\ESph{}\PTBRph{}\PLph{}\ITAph{} \gls{user mode}}
}

\newglossaryentry{Windows NT}
{
  name=Windows NT,
  description={Windows NT, 2000, XP, Vista, 7, 8}
}

\newglossaryentry{atomic operation}
{
  name=atomic operation,
  description={
  \q{$\alpha{}\tau{}o\mu{}o\varsigma{}$}
  %\q{atomic}
  \RU{означает \q{неделимый} в греческом языке, так что атомарная операция ---
  это операция которая гарантированно не будет прервана другими тредами}
  \EN{stands for \q{indivisible} in Greek, so an atomic operation is guaranteed not
  to be interrupted by other threads}\ESph{}\PTBRph{}\PLph{}\ITAph{}}
}

% to be proofreaded (begin)
\newglossaryentry{NaN}
{
  name=NaN,
  description={
  	\RU{не число: специальные случаи чисел с плавающей запятой, 
  	обычно сигнализирующие об ошибках}\EN{not a number: 
  	a special cases for floating point numbers, usually signaling about errors}\ESph{}\PTBRph{}\PLph{}\ITAph{}
  }
}

\newglossaryentry{basic block}
{
  name=basic block,
  description={
  	\RU{группа инструкций, не имеющая инструкций переходов,
	а также не имеющая переходов в середину блока извне.
	В \IDA он выглядит как просто список инструкций без строк-разрывов}\EN{a group of 
	instructions that do not have jump/branch instructions, and also don't have
	jumps inside the block from the outside.
	In \IDA it looks just like as a list of instructions without empty lines}\ESph{}\PTBRph{}\PLph{}\ITAph{}
  }
}

\newglossaryentry{NEON}
{
  name=NEON,
  description={\ac{AKA} \q{Advanced SIMD}\EMDASH\ac{SIMD} \RU{от}\EN{from}\ESph{}\PTBRph{}\PLph{}\ITAph{} ARM}
}

\newglossaryentry{reverse engineering}
{
  name=reverse engineering,
  description={\RU{процесс понимания как устроена некая вещь, иногда, с целью клонирования оной}
  \EN{act of understanding how the thing works, sometimes in order to clone it}\ESph{}\PTBRph{}\PLph{}\ITAph{}}
}

\newglossaryentry{compiler intrinsic}
{
  name=compiler intrinsic,
  description={\RU{Специфичная для компилятора функция не являющаяся обычной библиотечной функцией.
	Компилятор вместо её вызова генерирует определенный машинный код.
	Нередко, это псевдофункции для определенной инструкции \ac{CPU}. Читайте больше:}
	\EN{A function specific to a compiler which is not an usual library function.
	The compiler generates a specific machine code instead of a call to it.
	Often, it's a pseudofunction for a specific \ac{CPU} instruction. Read more:}\ESph{}\PTBRph{}\PLph{}\ITAph{} (\myref{sec:compiler_intrinsic})}
}

\newglossaryentry{heap}
{
  name=heap,
  description={\RU{(куча) обычно, большой кусок памяти предоставляемый \ac{OS}, так что прикладное ПО может делить его
  как захочет. malloc()/free() работают с кучей}
  \EN{usually, a big chunk of memory provided by the \ac{OS} so that applications can divide it by themselves as they wish.
  malloc()/free() work with the heap}\ESph{}\PTBRph{}\PLph{}\ITAph{}}
}

\newglossaryentry{name mangling}
{
  name=name mangling,
  description={\RU{применяется как минимум в \Cpp, где компилятору нужно закодировать имя класса,
  метода и типы аргументов в одной
  строке, которая будет внутренним именем функции. читайте также здесь}
  \EN{used at least in \Cpp, where the compiler needs to encode the name of class, method and argument types in one string,
  which will become the internal name of the function. You can read more about it here}\ESph{}\PTBRph{}\PLph{}\ITAph{}: \myref{namemangling}}
}

\newglossaryentry{xoring}
{
  name=xoring,
  description={\RU{нередко применяемое в английском языке, означает применение операции 
  \ac{XOR}}
  \EN{often used in the English language, which implying applying the \ac{XOR} operation}\ESph{}\PTBRph{}\PLph{}\ITAph{}}
}

\newglossaryentry{security cookie}
{
  name=security cookie,
  description={\RU{Случайное значение, разное при каждом исполнении. Читайте больше об этом тут}
  \EN{A random value, different at each execution. You can read more about it here}\ESph{}\PTBRph{}\PLph{}\ITAph{}: \myref{subsec:BO_protection}}
}

\newglossaryentry{tracer}
{
  name=tracer,
  description={\RU{Моя простейшая утилита для отладки. Читайте больше об этом тут}
  \EN{My own simple debugging tool. You can read more about it here}\ESph{}\PTBRph{}\PLph{}\ITAph{}: \myref{tracer}}
}

\newglossaryentry{GiB}
{
  name=GiB,
  description={\RU{Гибибайт: $2^{30}$ или 1024 мебибайт или 1073741824 байт}
  \EN{Gibibyte: $2^{30}$ or 1024 mebibytes or 1073741824 bytes}\ESph{}\PTBRph{}\PLph{}\ITAph{}}
}

\newglossaryentry{CP/M}
{
  name=CP/M,
  description={Control Program for Microcomputers: \RU{очень простая дисковая \ac{OS} использовавшаяся перед}
  \EN{a very basic disk \ac{OS} used before}\ESph{}\PTBRph{}\PLph{}\ITAph{} MS-DOS}
}

\newglossaryentry{stack frame}
{
  name=stack frame,
  description={\RU{Часть стека, в которой хранится информация, связанная с текущей функцией: локальные переменные,
  аргументы функции, \ac{RA}, \etc{}.}\EN{A part of the stack that contains information specific to the current function:
  local variables, function arguments, \ac{RA}, \etc{}}\ESph{}\PTBRph{}\PLph{}\ITAph{}}
}

\newglossaryentry{jump offset}
{
  name=jump offset,
  description={\RU{Часть опкода JMP или Jcc инструкции, просто прибавляется к адресу следующей инструкции,
  и так вычисляется новый \ac{PC}. Может быть отрицательным}\EN{a part of the JMP or Jcc instruction's opcode, 
  to be added to the address
  of the next instruction, and this is how the new \ac{PC} is calculated. May be negative as well}\ESph{}\PTBRph{}\PLph{}\ITAph{}}
}

\newglossaryentry{integral type}
{
  name=\RU{интегральный тип данных}\EN{integral data type}\ESph{}\PTBRph{}\PLph{}\ITAph{},
  description={\RU{обычные числа, но не вещественные. могут использоваться для передачи булевых типов и перечислений (enumerations)}
  \EN{usual numbers, but not a real ones. may be used for passing variables of boolean data type and enumerations}\ESph{}\PTBRph{}\PLph{}\ITAph{}}
}

\newglossaryentry{real number}
{
  name=\RU{вещественное число}\EN{real number}\ESph{}\PTBRph{}\PLph{}\ITAph{},
  description={\RU{числа, которые могут иметь точку. в \CCpp это \Tfloat и \Tdouble}
  \EN{numbers which may contain a dot. this is \Tfloat and \Tdouble in \CCpp}\ESph{}\PTBRph{}\PLph{}\ITAph{}}
}

\newglossaryentry{PDB}
{
  name=PDB,
  description={(Win32) \RU{Файл с отладочной информацией, обычно просто имена функций, 
  но иногда имена аргументов функций и локальных переменных}
  \EN{Debugging information file, usually just function names, but sometimes also function
  arguments and local variables names}\ESph{}\PTBRph{}\PLph{}\ITAph{}}
}

\newglossaryentry{NTAPI}
{
  name=NTAPI,
  description={\RU{\ac{API} доступное только в линии Windows NT. 
  Большей частью не документировано Microsoft-ом}\EN{\ac{API} available only in the Windows NT line. 
  Largely not documented by Microsoft}\ESph{}\PTBRph{}\PLph{}\ITAph{}}
}

\newglossaryentry{stdout}
{
  name=stdout,
  description={standard output}
}

\newglossaryentry{word}
{
  name=word,
  description={\EN{data type fitting in \ac{GPR}}\RU{(слово) тип данных помещающийся в \ac{GPR}\ESph{}\PTBRph{}\PLph{}\ITAph{}}. 
  \RU{В компьютерах старше персональных, память часто измерялась не в байтах, 
  а в словах}\EN{In the computers older than PCs, 
  the memory size was often measured in words rather than bytes}\ESph{}\PTBRph{}\PLph{}\ITAph{}}
}

\newglossaryentry{arithmetic mean}
{
  name=\RU{среднее арифметическое}\EN{arithmetic mean}\ESph{}\PTBRph{}\PLph{}\ITAph{},
  description={\EN{a sum of all values divided by their count}
  \RU{сумма всех значений, разделенная на их количество}\ESph{}\PTBRph{}\PLph{}\ITAph{}}
}

\newcommand{\URLWPDA}
{\RU
 {
  \href{http://go.yurichev.com/17012}{Wikipedia: Выравнивание данных}
 }
 \EN{
  \href{http://go.yurichev.com/17013}{Wikipedia: Data structure alignment}
 }
 \ES{
  \href{URL}{\ESph{}}
 }
 \PL{
  \href{URL}{\PLph{}}
 }
 \PTBR{
  \href{URL}{\PTBRph{}}
 }
 \IT{
  \href{URL}{\ITAph{}}
 }
}

\newcommand{\OracleTablesName}{oracle tables\xspace}
\newcommand{\oracletables}{\OracleTablesName\footnote{\href{http://go.yurichev.com/17014}{yurichev.com}}\xspace}

\newcommand{\WPMAO}
{\RU
{
    \href{http://go.yurichev.com/17015}{Wikipedia: Умножение-сложение}
}
\EN{
    \href{http://go.yurichev.com/17016}{Wikipedia: Multiply–accumulate operation}
}
 \ES{
  \href{URL}{\ESph{}}
 }
 \PL{
  \href{URL}{\PLph{}}
 }
 \PTBR{
  \href{URL}{\PTBRph{}}
 }
 \ITA{
  \href{URL}{\ITAph{}}
 }
}

\newcommand{\BGREPURL}{\href{http://go.yurichev.com/17017}{GitHub}}
\newcommand{\FNMSDNROTxURL}{\footnote{\href{http://go.yurichev.com/17018}{MSDN}}}

\newcommand{\YurichevIDAIDCScripts}{http://go.yurichev.com/17019}

%FIXME: requires PTBR and ES revision (dbmussi)
% for index
\newcommand{\GrepUsage}{\RU{Использование grep}\EN{grep usage}\PTBR{Uso do grep}\ES{Uso de grep}}
\newcommand{\SyntacticSugar}{\RU{Синтаксический сахар}\EN{Syntactic Sugar}\PTBR{Açúcar sintático}\ES{Azúcar sintáctica}}
\newcommand{\CompilerAnomaly}{\RU{Аномалии компиляторов}\EN{Compiler's anomalies}\PTBR{Anomalias do compilador}\ES{Anomalías del compilador}}
\newcommand{\CLanguageElements}{\RU{Элементы языка Си}\EN{C language elements}\PTBR{Elementos da linguagem C}\ES{Elementos del lenguaje C}}
\newcommand{\CStandardLibrary}{\RU{Стандартная библиотека Си}\EN{C standard library}\PTBR{Biblioteca padrão C}\ES{Librería stándar C}}
\newcommand{\Instructions}{\RU{Инструкции}\EN{Instructions}\PTBR{Instruções}\ES{Instrucciones}}
\newcommand{\Pseudoinstructions}{\RU{Псевдоинструкции}\EN{Pseudoinstructions}\PTBR{Pseudo-instruções}\ES{Pseudo-instrucciones}}
\newcommand{\Prefixes}{\RU{Префиксы}\EN{Prefixes}\PTBR{Prefixos}\ES{Prefijos}}

\newcommand{\Flags}{\RU{Флаги}\EN{Flags}\PTBR{Flags}\ES{Flags}}
\newcommand{\Registers}{\RU{Регистры}\EN{Registers}\PTBR{Registradores}\ES{Registros}}
\newcommand{\registers}{\RU{регистры}\EN{registers}\PTBR{registradores}\ES{registros}}
\newcommand{\Stack}{\RU{Стек}\EN{Stack}\PTBR{Pilha}\ES{Pila}}
\newcommand{\Recursion}{\RU{Рекурсия}\EN{Recursion}\PTBR{Recursividade}\ES{Recursión}}
\newcommand{\RAM}{\RU{ОЗУ}\EN{RAM}\PTBR{RAM}\ES{RAM}}
\newcommand{\ROM}{\RU{ПЗУ}\EN{ROM}\PTBR{ROM}\ES{ROM}}
\newcommand{\Pointers}{\RU{Указатели}\EN{Pointers}\PTBR{Ponteiros}\ES{Apuntadores}}
\newcommand{\BufferOverflow}{\RU{Переполнение буфера}\EN{Buffer Overflow}\PTBR{Buffer Overflow}\ES{Buffer Overflow}}
\newcommand{\Conclusion}{\RU{Вывод}\EN{Conclusion}\PTBR{Conclusão}\ES{Conclusión}}

\newcommand{\Exercise}{\RU{Упражнение}\EN{Exercise}\PTBR{Exercício}\ES{Ejercicio}\xspace}
\newcommand{\Exercises}{\RU{Упражнения}\EN{Exercises}\PTBR{Exercícios}\ES{Ejercicios}\xspace}
\newcommand{\Arrays}{\RU{Массивы}\EN{Arrays}\PTBR{Matriz}\ES{Matriz}}
\newcommand{\Cpp}{\RU{Си++}\EN{C++}\PTBR{C++}\ES{C++}\xspace}
\newcommand{\CCpp}{\RU{Си/Си++}\EN{C/C++}\PTBR{C/C++}\ES{C/C++}\xspace}
\newcommand{\NonOptimizing}{\RU{Неоптимизирующий}\EN{Non-optimizing}\PTBR{Sem otimização}\ES{Sin optimización}\xspace}
\newcommand{\Optimizing}{\RU{Оптимизирующий}\EN{Optimizing}\PTBR{Com otimização}\ES{Con optimización}\xspace}
\newcommand{\NonOptimizingKeilVI}{\NonOptimizing Keil 6/2013\xspace}
\newcommand{\OptimizingKeilVI}{\Optimizing Keil 6/2013\xspace}
\newcommand{\NonOptimizingXcodeIV}{\NonOptimizing Xcode 4.6.3 (LLVM)\xspace}
\newcommand{\OptimizingXcodeIV}{\Optimizing Xcode 4.6.3 (LLVM)\xspace}
\newcommand{\ARMMode}{\RU{Режим ARM}\EN{ARM mode}\PTBR{Modo ARM}\ES{Modo ARM}\xspace}
\newcommand{\ThumbMode}{\RU{Режим Thumb}\EN{Thumb mode}\PTBR{Modo Thumb}\ES{Modo Thumb}\xspace}
\newcommand{\ThumbTwoMode}{\RU{Режим Thumb-2}\EN{Thumb-2 mode}\PTBR{Modo Thumb-2}\ES{Modo Thumb-2}\xspace}
\newcommand{\AndENRU}{\RU{и}\EN{and}\PTBR{e}\ES{y}\xspace}
\newcommand{\OrENRU}{\RU{или}\EN{or}\PTBR{ou}\ES{o}\xspace}
\newcommand{\InENRU}{\RU{в}\EN{in}\PTBR{em}\ES{en}\xspace}
\newcommand{\ForENRU}{\RU{для}\EN{for}\PTBR{para}\ES{para}\xspace}
\newcommand{\LineENRU}{\RU{строка}\EN{line}\PTBR{linha}\ES{línea}\xspace}

\newcommand{\DataProcessingInstructionsFootNote}{
	\RU{Эти инструкции также называются}
	\EN{These instructions are also called}
	\PTBR{Estas intruções também são chamadas}
	\ES{Estas instrucciones también se llaman} \q{data processing instructions}
}

% for .bib files
\newcommand{\AlsoAvailableAs}{\RU{Также доступно здесь:}\EN{Also available as}\PTBR{Também disponível como}\ES{También disponible como}\xspace}

% section names
\newcommand{\ShiftsSectionName}{\RU{Сдвиги}\EN{Shifts}\PTBR{Shifts}\ES{Shifts}}
\newcommand{\SignedNumbersSectionName}{\RU{Представление знака в числах}\EN{Signed number representations}}
\newcommand{\HelloWorldSectionName}{Hello, world!}
\newcommand{\SwitchCaseDefaultSectionName}{switch()/case/default}
\newcommand{\PrintfSeveralArgumentsSectionName}{\printf \RU{с несколькими аргументами}\EN{with several arguments}\PTBR{com vários argumentos}\ES{con varios argumentos}}
\newcommand{\BitfieldsChapter}{\RU{Работа с отдельными битами}\EN{Manipulating specific bit(s)}\PTBR{Manipulando bit(s) específicos}\ES{Manipulando bit(s) específicas}}
\newcommand{\ArithOptimizations}{
	\RU{Замена одних арифметических инструкций на другие}
	\EN{Replacing arithmetic instructions to other ones}
	\PTBR{Substituição de instruções aritiméticas por outras}
	\ES{Substituición de instrucciones aritméticas por otros}
	}
\newcommand{\FPUChapterName}{\RU{Работа с FPU}\EN{Floating-point unit}\PTBR{Unidade de Ponto flutuante}\ES{Unidad de Punto flotante}}
\newcommand{\SimpleStringsProcessings}{\RU{Простая работа с Си-строками}\EN{Simple C-strings processing}\PTBR{Processamento de strings C simples}\ES{Procesamiento de strings C simples}}
\newcommand{\DivisionByNineSectionName}{\RU{Деление на 9}\EN{Division by 9}\PTBR{Divisão por 9}\ES{División por 9}}
\newcommand{\Answer}{\RU{Ответ}\EN{Answer}\PTBR{Responda}\ES{Responda}}
\newcommand{\WhatThisCodeDoes}{\RU{Что делает этот код}\EN{What does this code do}\PTBR{O que este código faz}\ES{Lo que hace el código}?}
\newcommand{\WorkingWithFloatAsWithStructSubSubSectionName}{
\RU{Работа с типом float как со структурой}\EN{Working with the float type as with a structure}\PTBR{Trabalhando com o tipo float como uma estrutura}\ES{Trabajando con el tipo float como una estructura}}

\newcommand{\MinesweeperWinXPExampleChapterName}{\RU{Сапёр}\EN{Minesweeper}\PTBR{Campo minado}\ES{Buscaminas} (Windows XP)}

\newcommand{\StructurePackingSectionName}{\RU{Упаковка полей в структуре}\EN{Fields packing in structure}\PTBR{Organização de campos na estrutura}\ES{Organización de campos en la estructura}}
\newcommand{\StructuresChapterName}{\RU{Структуры}\EN{Structures}\PTBR{Estruturas}\ES{Estructuras}}
\newcommand{\PICcode}{\RU{адресно-независимый код}\EN{position-independent code}\PTBR{código independente de posição}\ES{código independiente de la posición}}
\newcommand{\CapitalPICcode}{\RU{Адресно-независимый код}\EN{Position-independent code}\PTBR{Código independente de posição}\ES{Código independiente de lá posición}}
\newcommand{\Loops}{\RU{Циклы}\EN{Loops}\PTBR{Laços}\ES{Lazos}}

% C
\newcommand{\PostIncrement}{\RU{Пост-инкремент}\EN{Post-increment}\PTBR{Pós-incremento}\ES{Post-incremento}}
\newcommand{\PostDecrement}{\RU{Пост-декремент}\EN{Post-decrement}\PTBR{Pós-decremento}\ES{Post-decremento}}
\newcommand{\PreIncrement}{\RU{Пре-инкремент}\EN{Pre-increment}\PTBR{Pré-incremento}\ES{Pre-incremento}}
\newcommand{\PreDecrement}{\RU{Пре-декремент}\EN{Pre-decrement}\PTBR{Pré-decremento}\ES{Pre-decremento}}

% MIPS
\newcommand{\GlobalPointer}{\RU{Глобальный указатель}\EN{Global Pointer}\PTBR{Ponteiro Global}\ES{Puntero Global}}

% other
\newcommand{\garbage}{\RU{мусор}\EN{garbage}\PTBR{Lixo}\ES{Basura}}
\newcommand{\IntelSyntax}{\RU{Синтаксис Intel}\EN{Intel syntax}\PTBR{Sintaxe Intel}\ES{Sintaxis Intel}}
\newcommand{\ATTSyntax}{\RU{Синтаксис AT\&T}\EN{AT\&T syntax}\PTBR{Sintaxe AT\&T}\ES{Sintaxis AT\&T}}
\newcommand{\randomNoise}{\RU{случайный шум}\EN{random noise}\PTBR{Ruído aleatório}\ES{Ruido aleatorio}}
\newcommand{\Example}{\RU{Пример}\EN{Example}\PTBR{Exemplo}\ES{Ejemplo}}
\newcommand{\argument}{\RU{аргумент}\EN{argument}\PTBR{argumento}\ES{argumento}}
\newcommand{\MarkedInIDAAs}{\RU{маркируется в \IDA как}\EN{marked in \IDA as}\PTBR{Marcado no \IDA como}\ES{Marcado en \IDA como}}
\newcommand{\HERMIT}{\RU{Андрей}\EN{Andrey}\PTBR{Andrey}\ES{Andrey} \q{herm1t} \RU{Баранович}\EN{Baranovich}\PTBR{Baranovich}\ES{Baranovich}}
\newcommand{\stepover}{\RU{сделать шаг, не входя в функцию}\EN{step over}\PTBR{passar por cima}\ES{pasar por encima}}
\newcommand{\ShortHotKeyCheatsheet}{\RU{Краткий справочник горячих клавиш}\EN{Hot-keys cheatsheet}\PTBR{Cheatsheet de teclas de atalho}\ES{Cheatsheet de teclas de acceso rápido}}

\newcommand{\assemblyOutput}{\RU{вывод на ассемблере}\EN{assembly output}\PTBR{saída do assembly}\ES{salida de assembly}}



\makeglossaries

\newcommand{\TITLE}{\RU{Reverse Engineering для начинающих}\EN{Reverse Engineering for Beginners}\ESph{}\PTBRph{}\PLph{}\ITAph{}}
\newcommand{\AUTHOR}{\RU{Денис Юричев}\EN{Dennis Yurichev}\ESph{}\PTBRph{}\PLph{}\ITAph{}}
\newcommand{\EMAIL}{dennis(a)yurichev.com}

\hypersetup{
    colorlinks=true,
    allcolors=blue,
    pdfauthor={\AUTHOR},
    pdftitle={\TITLE}
    }

%\ifdefined\RUSSIAN
\newcommand{\LstStyle}{\ttfamily\small}
%\else
%\newcommand{\LstStyle}{\ttfamily}
%\fi 

\lstset{
    backgroundcolor=\color{lstbgcolor},
    basicstyle=\LstStyle,
    breaklines=true,
    %prebreak=\raisebox{0ex}[0ex][0ex]{->},
    %postbreak=\raisebox{0ex}[0ex][0ex]{->},
    prebreak=\raisebox{0ex}[0ex][0ex]{\ensuremath{\rhookswarrow}},
    postbreak=\raisebox{0ex}[0ex][0ex]{\ensuremath{\rcurvearrowse\space}},
    frame=single,
    columns=fullflexible,keepspaces,
    escapeinside=§§,
    inputencoding=utf8
}

\DeclareMathSizes{12}{30}{16}{12}
\EN{
	\bibliography{C_book_en,books,articles,usenet,misc}
}
\PL{
	\bibliography{C_book_pl,books,articles,usenet,misc}
}
\ES{
	\bibliography{C_book_en,books,articles,usenet,misc}
}
\PTBR{
	\bibliography{C_book_en,books,articles,usenet,misc}
}
\RU{
	\bibliography{C_book_ru,books,articles,usenet,misc}
}

\begin{document}

\pagestyle{fancy}

\VerbatimFootnotes

\frontmatter

\include{page_after_cover}
\ifdefined\LITE
\begin{center}
\vspace*{\fill}

\Huge \RU{Внимание: это сокращенная LITE-версия}\EN{Warning: this is a shortened LITE-version}\ESph{}\PTBRph{}\PLph{}\ITAph{}!
\normalsize

\bigskip
\bigskip
\bigskip

\Large
\RU{Она примерно в 6 раз короче полной версии (\textasciitilde{}150 страниц) и предназначена для тех,
кто хочет краткого введения в основы reverse engineering.
Здесь нет ничего о MIPS, ARM, OllyDBG, GCC, GDB, IDA, нет задач, примеров, \etc.}
\EN{It is approximately 6 times shorter than full version (\textasciitilde{}150 pages) and intended to those
who wants for very quick introduction to reverse engineering basics.
There are nothing about MIPS, ARM, OllyDBG, GCC, GDB, IDA, there are no exercises, examples, etc.}
\ESph{}\PTBRph{}\PLph{}\ITAph{}
\normalsize

\bigskip
\bigskip
\bigskip

\RU{Если вам всё ещё интересен reverse engineering, полная версия книги всегда доступна на моем сайте}%
\EN{If you still interesting in reverse engineering, full version of the book is always available on my website}\ESph{}\PTBRph{}\PLph{}\ITAph{}: 
\href{http://go.yurichev.com/17009}{beginners.re}.

\vspace*{\fill}
\vfill
\end{center}

\fi
%\include{survey}

\ifx\LITE\undefined
\shorttoc{\RU{Краткое оглавление}\EN{Abridged contents}\PTBRph{}\ESph{}\PLph{}\ITAph{}}{-1} % Only sections
\fi
\tableofcontents
\cleardoublepage

\cleardoublepage
\section*{\RU{Предисловие}\EN{Preface}\PTBRph{}\ESph{}\PLph{}\ITAph{}}

\iffalse
\RU{Здесь (будет) немного моих заметок о \gls{reverse engineering} на русском языке для начинающих, 
для тех кто хочет научиться понимать создаваемый \CCpp компиляторами код для x86 (коего, 
практически, больше всего остального) и ARM.}
\EN{Here are some of my notes in English for beginners in \gls{reverse engineering}
who would like to learn to understand x86 (which accounts for almost
all executable software in the world) and ARM code created by \CCpp compilers.}
\fi

\RU{У термина \q{\gls{reverse engineering}} несколько популярных значений:
1) исследование скомпилированных
программ; 2) сканирование трехмерной модели для последующего копирования;
3) восстановление структуры СУБД. Настоящий сборник заметок
связан с первым значением.}
\EN{There are several popular meanings of the term \q{\gls{reverse engineering}}:
1) The reverse engineering of software: researching compiled programs;
2) The scanning of 3D structures and the subsequent digital manipulation required order to duplicate them;
3) recreating \ac{DBMS} structure.
This book is about the first meaning.}

\ifx\LITE\undefined
\subsection*{\RU{Рассмотренные темы}\EN{Topics discussed in-depth}\PTBRph{}\ESph{}\PLph{}\ITAph{}}

x86/x64, ARM/ARM64, MIPS, Java/JVM.

\subsection*{\RU{Затронутые темы}\EN{Topics touched upon}\PTBRph{}\ESph{}\PLph{}\ITAph{}}

\oracle (\myref{oracle}),
Itanium (\myref{itanium}),
\RU{донглы для защиты от копирования}\EN{copy-protection dongles} (\myref{dongles}), 
LD\_PRELOAD (\myref{ld_preload}),
\RU{переполнение стека}\EN{stack overflow}, 
\ac{ELF},
\RU{формат файла PE в win32}\EN{win32 PE file format} (\myref{win32_pe}),
x86-64 (\myref{x86-64}),
\RU{критические секции}\EN{critical sections} (\myref{critical_sections}),
\RU{системные вызовы}\EN{syscalls} (\myref{syscalls}), 
\ac{TLS},
\RU{адресно-независимый код}\EN{position-independent code} (\ac{PIC}) (\myref{sec:PIC}), 
profile-guided optimization (\myref{PGO}),
C++ STL (\myref{cpp_STL}),
OpenMP (\myref{openmp}),
SEH (\myref{sec:SEH}).
\fi

\subsection*{\RU{Упражнения и задачи}\EN{Exercises and tasks}\PTBRph{}\ESph{}\PLph{}\ITAph{}}

\dots \RU{все перемещены на отдельный сайт}\EN{are all moved to the separate website}: \url{http://challenges.re}.

\subsection*{\RU{Об авторе}\EN{About the author}\PTBRph{}\ESph{}\PLph{}\ITAph{}}

\begin{tabularx}{\textwidth}{ l X }

\raisebox{-\totalheight}{
\includegraphics[scale=0.60]{Dennis_Yurichev.jpg}
}

&
\RU{Денис Юричев~--- опытный reverse engineer и программист.}%
\EN{Dennis Yurichev is an experienced reverse engineer and programmer.}
\EN{He can be contacted by email:}%
\RU{С ним можно контактировать по емейлу:} \textbf{\EMAIL{}}, 
\EN{or on Skype:}\RU{или по Skype:} \textbf{dennis.yurichev}.

% FIXME: no link. \tablefootnote doesn't work
\end{tabularx}

% subsections:
\input{praise}

\subsection*{\RU{Благодарности}\EN{Thanks}\PTBRph{}\ESph{}\PLph{}\ITAph{}}

\RU{Тем, кто много помогал мне отвечая на массу вопросов}\EN{For patiently answering all my questions}:
\HERMIT, \RU{Слава \q{Avid} Казаков}\EN{Slava \q{Avid} Kazakov}.

\RU{Тем, кто присылал замечания об ошибках и неточностях}\EN{For sending me notes
about mistakes and inaccuracies}:
\RU{Станислав \q{Beaver} Бобрицкий, Александр Лысенко}%
\EN{Stanislav \q{Beaver} Bobrytskyy, Alexander Lysenko}, Shell Rocket, Zhu Ruijin, Changmin Heo.

\RU{Просто помогали разными способами}\EN{For helping me in other ways}:
\RU{Андрей Зубинский}\EN{Andrew Zubinski}, 
Arnaud Patard (rtp \RU{на}\EN{on} \#debian-arm IRC),
\EN{Aliaksandr Autayeu}\RU{Александр Автаев}.

\RU{Переводчикам на китайский язык}\EN{For translating the book into Simplified Chinese}: Antiy Labs (\href{http://antiy.cn}{antiy.cn}) \AndENRU Archer.

\RU{Переводчику на корейский язык}\EN{For translating the book into Korean}: Byungho Min.

\RU{Корректорам}\EN{For proofreading}:
\RU{Александр \q{Lstar} Черненький}\EN{Alexander \q{Lstar} Chernenkiy},
\RU{Владимир Ботов}\EN{Vladimir Botov},
\RU{Андрей Бражук}\EN{Andrei Brazhuk},
\RU{Марк}\EN{Mark} ``Logxen'' \RU{Купер}\EN{Cooper},
Yuan Jochen Kang, Mal Malakov, Lewis Porter, Jarle Thorsen.

\RU{Васил Колев сделал очень много исправлений и указал на многие ошибки.}
\EN{Vasil Kolev did a great amount of work in proofreading and correcting many mistakes.}

\RU{За иллюстрации и обложку: Андрей Нечаевский.}\EN{For illustrations and cover art: Andy Nechaevsky.}

\RU{И ещё всем тем на github.com кто присылал замечания и исправления.}
\EN{Thanks also to all the folks on github.com who have contributed notes and corrections.}

\RU{Было использовано множество пакетов \LaTeX. Их авторов я также хотел бы поблагодарить.}
\EN{Many \LaTeX\ packages were used: I would like to thank the authors as well.}

\subsubsection*{\RU{Жертвователи}\EN{Donors}}

\EN{Those who supported me during the time when I wrote significant part of the book:}%
\RU{Те, кто поддерживал меня во время написании этой книги:}

2 * Oleg Vygovsky (50+100 UAH), 
Daniel Bilar ($\$$50), 
James Truscott ($\$$4.5),
Luis Rocha ($\$$63), 
Joris van de Vis ($\$$127), 
Richard S Shultz ($\$$20), 
Jang Minchang ($\$$20), 
Shade Atlas (5 AUD), 
Yao Xiao ($\$$10),
Pawel Szczur (40 CHF), 
Justin Simms ($\$$20), 
Shawn the R0ck ($\$$27), 
Ki Chan Ahn ($\$$50), 
Triop AB (100 SEK), 
Ange Albertini (\euro{}10+50),
Sergey Lukianov (300 RUR), 
Ludvig Gislason (200 SEK), 
Gérard Labadie (\euro{}40), 
Sergey Volchkov (10 AUD),
Vankayala Vigneswararao ($\$$50),
Philippe Teuwen ($\$$4),
Martin Haeberli ($\$$10),
Victor Cazacov (\euro{}5),
Tobias Sturzenegger (10 CHF),
Sonny Thai ($\$$15),
Bayna AlZaabi ($\$$75),
Redfive B.V. (\euro{}25),
Joona Oskari Heikkilä (\euro{}5),
Marshall Bishop ($\$$50),
Nicolas Werner (\euro{}12),
Jeremy Brown ($\$$100),
Alexandre Borges ($\$$25),
Vladimir Dikovski (\euro{}50),
Jiarui Hong (100.00 SEK),
Jim Di (500 RUR),
Tan Vincent ($\$$30),
Sri Harsha Kandrakota (10 AUD),
Pillay Harish (10 SGD),
Timur Valiev (230 RUR),
Carlos Garcia Prado (\euro{}10),
Salikov Alexander (500 RUR),
Oliver Whitehouse (30 GBP),
Katy Moe ($\$$14),
Maxim Dyakonov ($\$$3),
Sebastian Aguilera (\euro{}20),
Hans-Martin Münch (\euro{}15),
Jarle Thorsen (100 NOK),
Vitaly Osipov ($\$$100),
Yuri Romanov (1000 RUR),
Aliaksandr Autayeu (\euro{}10),
Tudor Azoitei ($\$$40),
Z0vsky (\euro{}10),
Yu Dai ($\$$10). 

\RU{Огромное спасибо каждому!}\EN{Thanks a lot to every donor!}

% subsections
\subsection*{mini-\RU{ЧаВО}\EN{FAQ}}

\newcommand{\HACKINGMdURL}{https://github.com/dennis714/RE-for-beginners/blob/master/HACKING.md}
\newcommand{\FNURLREDDIT}{\footnote{\href{http://go.yurichev.com/17027}{reddit.com/r/ReverseEngineering/}}}

Q: \EN{Why should one learn assembly language these days?}\RU{Зачем в наше время нужно изучать язык ассемблера?}\\
A: \EN{Unless you are an \ac{OS} developer, you probably don't need to code in assembly\EMDASH{}modern compilers 
are much better at performing optimizations than humans}
\RU{Если вы не разработчик \ac{OS}, вам наверное не нужно писать на ассемблере:
современные компиляторы оптимизируют код намного лучше человека}%
\footnote{\RU{Очень хороший текст на эту тему}\EN{A very good text about this topic}: \cite{AgnerFog}}.
\EN{Also, modern \ac{CPU}s are very complex devices and assembly knowledge doesn't really help one to understand their internals.}
\RU{К тому же, современные \ac{CPU} это крайне сложные устройства и знание ассемблера вряд ли
поможет узнать их внутренности.}
\EN{That being said, there are at least two areas where a good understanding of assembly can be helpful: 
First and foremost, security/malware research. It is also a good way to gain a better understanding of your compiled code whilst debugging.}
\RU{Но все-таки остается по крайней мере две области, где знание ассемблера может хорошо
помочь:
1) исследование malware (\IT{зловредов}) с целью анализа; 2) лучшее понимание
вашего скомпилированного кода в процессе отладки.}
\EN{This book is therefore intended for those who want to understand assembly language rather 
than to code in it, which is why there are many examples of compiler output contained within.}
\RU{Таким образом, эта книга предназначена для тех, кто хочет скорее понимать ассемблер,
нежели писать на нем, и вот почему здесь масса примеров, связанных с результатами
работы компиляторов.}\\
\\
Q: \RU{Я кликнул на ссылку внутри PDF-документа, как теперь вернуться назад?}\EN{I clicked on a hyperlink inside a PDF-document, how do I go back?}\\
A: \RU{В Adobe Acrobat Reader нажмите сочетание Alt+LeftArrow.}\EN{In Adobe Acrobat Reader click Alt+LeftArrow.}\\
\\
\ifx\LITE\undefined
Q: \RU{Ваша книга слишком большая! Нет ли чего покороче?}\EN{Your book is huge! Is there anything shorter?}\\
A: \RU{Есть сокращенная lite-версия}\EN{There is shortened, lite version found here}: \url{http://beginners.re/\#lite}.\\
\\
\fi
Q: \RU{Я не могу понять, стоит ли мне заниматься reverse engineering-ом}\EN{I'm not sure if I should try to learn reverse engineering or not}.\\
A: \RU{Наверное, среднее время для освоения сокращенной LITE-версии\EMDASH{}1-2 месяца.}%
\EN{Perhaps, the average time to become familiar with the contents of the shortened LITE-version is 1-2 month(s).}\\
\\
Q: \RU{Могу ли я распечатать эту книгу? Использовать её для обучения?}\EN{May I print this book? Use it for teaching?}\\
A: \RU{Конечно, поэтому книга и лицензирована под лицензией Creative Commons.}\EN{Of course! That's why the book is licensed under the Creative Commons license.}
\EN{One might also want to build one's own version of book\EMDASH{}read \href{\HACKINGMdURL}{here} to find out more.}
\RU{Кто-то может захотеть скомпилировать свою собственную версию книги, читайте \href{\HACKINGMdURL}{здесь} об этом.}\\
\\
Q: \RU{Я хочу перевести вашу книгу на другой язык}\EN{I want to translate your book to some other language}.\\
A: \RU{Прочитайте}\EN{Read} \href{https://github.com/dennis714/RE-for-beginners/blob/master/Translation.md}{\RU{мою заметку для переводчиков}\EN{my note to translators}}.\\
\\
Q: \RU{Как можно найти работу reverse engineer-а}\EN{How does one get a job in reverse engineering}? \\
A: \RU{На reddit, посвященному RE\FNURLREDDIT, время от времени бывают hiring thread}
\EN{There are hiring threads that appear from time to time on reddit, devoted to RE\FNURLREDDIT}
(\href{http://go.yurichev.com/17333}{2013 Q3}, 
\href{http://go.yurichev.com/17334}{2014}).
\RU{Посмотрите там}\EN{Try looking there}.
\EN{A somewhat related hiring thread can be found in the \q{netsec} subreddit}\RU{В смежном субреддите \q{netsec} имеется похожий тред}:
\href{http://go.yurichev.com/17335}{2014 Q2}.\\
\\
\RU{Q: Куда пойти учиться в Украине?\\
A: \href{http://go.yurichev.com/17336}{НТУУ \q{КПИ}: \q{Аналіз програмного коду та бінарних вразливостей}};
\href{http://go.yurichev.com/17337}{факультативы}.\\
\\}
Q: \EN{I have a question}\RU{У меня есть вопрос}...\\
A: \EN{Send it to me by email}\RU{Напишите мне его емейлом} (\EMAIL).


\ifdefined\ebook
\RU{Это версия формата A5 для электронных читалок}\EN{This is the A5-format version for e-book readers}. 
\RU{Хотя, тут всё то же самое, но иллюстрации уменьшены и не очень хорошо читаемы}
\EN{Although the content is mostly the same, the illustrations are resized and probably not readable}. 
\EN{You may try to change scale in your e-book reader.}
\RU{Вы можете попробовать изменить масштаб в вашей читалке.}
\RU{Так или иначе, вы всегда можете посмотреть их в версии формата A4 здесь}
\EN{Otherwise, you can always view them in the A4-format version here}: \href{http://go.yurichev.com/17009}{beginners.re}.
\fi

% {\RU{Целевая аудитория}\EN{Target audience}}

\subsection*{\RU{О переводе на корейский язык}\EN{About the Korean translation}\PTBRph{}\ESph{}\PLph{}\ITAph{}}

\EN{In January 2015, the Acorn publishing company (\href{http://www.acornpub.co.kr}{www.acornpub.co.kr}) in South Korea did a huge amount of work in translating and publishing 
my book (as it was in August 2014) into Korean.}
\RU{В январе 2015, издательство Acorn в Южной Корее сделало много работы в переводе 
и издании моей книги (по состоянию на август 2014) на корейский язык.}

\RU{Она теперь доступна на}\EN{It's now available at} 
\href{http://go.yurichev.com/17343}{\EN{their website}\RU{их сайте}}.

\iffalse
\begin{figure}[H]
\centering
\includegraphics[scale=0.3]{acorn_cover.jpg}
\end{figure}
\fi

\RU{Переводил}\EN{The translator is} Byungho Min (\href{http://go.yurichev.com/17344}{twitter/tais9}).

\EN{The cover art was done by my artistic friend, Andy Nechaevsky}
\RU{Обложку нарисовал мой хороший знакомый художник Андрей Нечаевский}: 
\href{http://go.yurichev.com/17023}{facebook/andydinka}.

\RU{Они также имеют права на издании книги на корейском языке.}
\EN{They also hold the copyright to the Korean translation.}

\RU{Так что если вы хотите иметь \IT{настоящую} книгу на полке на корейском языке и
хотите поддержать мою работу, вы можете купить её.}
\EN{So, if you want to have a \IT{real} book on your shelf in Korean and 
want to support my work, it is now available for purchase.}


\mainmatter

\ifx\LITE\undefined
\include{contents}
\fi

% only parts here!
%FIXME: requires PTBR and ES revision (dbmussi)
\part{\RU{Образцы кода}\EN{Code patterns}\PTBR{Padrões de código}\ES{Patrones de código}}

\RU{\epigraph{Всё познается в сравнении}{Автор неизвестен}}
\EN{\epigraph{Everything is comprehended in comparison}{Author unknown}}
\PTBR{\epigraph{Tudo é relativo}{Autor desconhecido}}
\ES{\epigraph{Todo es relativo}{Autor desconocido}}
% FIXME: english sentence added. (dbmussi) 
% not sure it's correct. (yurichev)
% this is popular Russian proverb and is close to "everything is comprehended in comparison", but the source is lost, however, 
% it's traditionally attributed to all sorts of philosophers..
% I don't know exact analgoue in English language, but OK, let it be so.

\RU{Когда автор этой книги учил Си, а затем \Cpp, он просто писал небольшие фрагменты кода, компилировал и смотрел, что 
получилось на ассемблере. Так было намного проще понять%
\footnote{Честно говоря, он и до сих пор так делаю, когда не понимают, как работает некий код.}.
Он делал это такое количество раз, что связь между кодом на \CCpp и тем, что генерирует компилятор, вбилась в его подсознание достаточно глубоко.
После этого не трудно, глядя на код на ассемблере, сразу в общих чертах понимать, что там было написано на Си. 
Возможно это поможет кому-то ещё.}
\EN{When the author of this book first started learning C and, later, \Cpp, he used to write small pieces of code, compile them, 
and then look at the assembly language output. This made it very easy for him to understand what was going on in the code that he had written.
\footnote{In fact, he still does it when he can't understand what a particular bit of code does.}. 
He did it so many times that the relationship between the \CCpp code and what the compiler produced was imprinted deeply in his mind. 
It's easy to imagine instantly a rough outline of C code's appearance and function. 
Perhaps this technique could be helpful for others.}
\PTBR{Quando o autor deste livro começou a aprender C e, mais tarde, \Cpp, ele costumava escrever pequenos pedaços de código, compilá-los, 
e então olhar a saída em linguagem assembly. Isso tornou muito fácil para ele entender o que estava acontecendo no código que ele tinha escrito.
\footnote{Na verdade, ele ainda faz isso quando não consegue entender o que faz um determinado pedaço de código.}. 
Ele fez isso tantas vezes que o relacionamento entre o código \CCpp code e o que o compilador produzia ficou registrado profundamente em sua mente. 
É fácil imaginar de imediato um esboço da aparência e função do código C. 
Talvez essa técnica poderia ser útil para mais alguém.}
\ES{Cuando el autor de este libro comenzó a aprender C y, más tarde, \Cpp, él solía escribir pequeños trozos de código, compilarlos, 
y luego ver los resultados en lenguaje assembly. Esto lo hizo muy fácil para él entender lo que estaba pasando en el código que había escrito.
\footnote{De hecho, todavia lo hace cuando no puede entender lo que hace una determinada pieza de código.}. 
Él lo hizo tantas veces que la relación entre el código \CCpp y lo que el compilador producido se imprimió profundamente en su mente. 
És fácil imaginar al instante un esbozo de la aparencia y función del código C. 
Quizás esta técnica podría ser útil para otra persona.}

%
%\RU{Здесь много примеров и для x86/x64 и для ARM}\EN{There are a lot of examples for both x86/x64 
%and ARM}.\PTBR{Há uma série de exemplos para ambos x86/x64 e ARM}.\ES{Hay una serie de ejemplos, tanto para x86/x64 y ARM} 
%\RU{Те, кто уже хорошо знаком с одной из архитектур, могут легко пролистывать страницы}
%\EN{Those who already familiar with one of architectures, may freely skim over pages}.
%\PTBR{Aqueles já familiarizados com alguma das arquiteturas, pode ler superficialmente as próximas páginas}.
%\ES{Los que ya están familiarizados con alguna de las arquitecturas, pueden leer superficialmente las páginas siguientes}.

\RU{Иногда здесь используются достаточно древние компиляторы, чтобы получить самый короткий (или простой) фрагмент кода.}%
\EN{Sometimes ancient compilers are used here, in order to get the shortest (or simplest) possible code snippet.}
\PTBR{Em determinadas partes foram usados aqui compiladores muito antigos, para se obter o menor (ou mais simples) snippet possível.}
\ES{En ciertas partes, se han empleado aquí compiladores muy antiguas, con el fin de obtener lo mas corta (o simple) posible snippet.}
\

\ifdefined\IncludeExercises
\section*{\Exercises}

\RU{Когда автор этой книги учил ассемблер, он также часто компилировал короткие функции на Си и затем постепенно 
переписывал их на ассемблер, с целью получить как можно более короткий код.}%
\EN{When the author of this book studied assembly language, he also often compiled small C-functions and then rewrote
them gradually to assembly, trying to make their code as short as possible.}
\PTBR{Quando o autor deste livro estudou a linguagem assembly, ele também frequentemente compilava pequenas funções em C e então as reescrevia gradualmente em assembly, tentando fazer seu código o menor possível.}
\ES{Cuando el autor de este libro estudió la lenguaje assembly, también con frecuencia compilaba pequeñas funciones en C, y reescribia gradualmente en assembly, tratando de hacer el código lo más pequeño posible}
\RU{Наверное, этим не стоит заниматься в наше время на практике (потому что конкурировать с современными
компиляторами в плане эффективности очень трудно), но это очень хороший способ разобраться в ассемблере
лучше.}%
\EN{This probably is not worth doing in real-world scenarios today, 
because it's hard to compete with modern compilers in terms of efficiency. It is, however, a very good way to gain a better understanding of assembly.}
\PTBR{Provavelmente não vale mais à pena fazer isso em cenários reais atualmente, 
porque é difícil competir com os compiladores modernos em termos de eficiência. É, no entanto, uma forma muito boa de obter um melhor entendimento de assembly.}
\ES{Probablemente no vale la pena hacer esto en escenarios reales actualmente, 
porque es dificil competir con los compiladores modernos en términos de eficiencia. Es, sin embargo, una muy buena manera de obtener una mejor compreensión de la assembly}

\RU{Так что вы можете взять любой фрагмент кода на ассемблере в этой книге и постараться сделать его короче.}%
\EN{Feel free, therefore, to take any assembly code from this book and try to make it shorter.}
\PTBR{Sinta-se livre, portanto, para pegar qualquer código assembly deste livro e tentar torná-lo menor.}
\ES{Siéntase libre, por lo tanto, para tomar cualquier código de este libro y tratar de hacerlo más pequeño.}
\RU{Но не забывайте о тестировании своих результатов.}%
\EN{However, don't forget to test what you have written.}
\PTBR{No entanto, não esqueça de testar o que você tiver escrito.}
\ES{Sin embargo, no se olvide de probar lo que has escrito.}
\fi

% rewrote to show that debug\release and optimisations levels are orthogonal concepts.
\section*{\RU{Уровни оптимизации и отладочная информация}\EN{Optimization levels and debug information}\PTBR{Níveis de otimização e informação de depuração}\ES{Níveles de optimización y la información de depuración}}

\RU{Исходный код можно компилировать различными компиляторами с различными уровнями оптимизации.
В типичном компиляторе этих уровней около трёх, где нулевой уровень~--- отключить оптимизацию.
Различают также направления оптимизации кода по размеру и по скорости.}
\EN{Source code can be compiled by different compilers with various optimization levels.
A typical compiler has about three such levels, where level zero means disable optimization.
Optimization can also be targeted towards code size or code speed.}
\PTBR{O código-fonte pode ser compilado por diferentes compiladores com vários níveis de otimização.
Um compilador típico tem cerca de três destes níveis, onde o nível zero significa desativar a otimização.
A otimização também pode ser direcionada para o tamanho do código ou para a velocidade do código.}
\ES{El código fuente puede ser compilado por diferentes compiladores com varios niveles de optimización.
Un compilador típico tiene alredor de tres de esos niveles, donde el nivel cero significa desactivar la optimización.
La optimización también puede dirigirse hacia el tamaño del código o la velocidad de código.}

\RU{Неоптимизирующий компилятор работает быстрее, генерирует более понятный (хотя и более объемный) код.
Оптимизирующий компилятор работает медленнее и старается сгенерировать более быстрый (хотя и не обязательно краткий) код.}
\EN{A non-optimizing compiler is faster and produces more understandable (albeit verbose) code,
whereas an optimizing compiler is slower and tries to produce code that runs faster (but is not necessarily more compact).}
\PTBR{Um compilador sem otimização é mais rápido e produz código mais inteligível (embora maior),
enquanto que um compilador com otimização é mais lento e tenta produzir um código que execute mais rápido (mas não é necessariamente mais compacto).}
\ES{Un compilador sin optimización es más rápido y produce código más inteligible (aunque más grande), 
mientras un compilador con optimización es más lento y trata de producir un código que corre más rápido (pero no necesariamente más compacto).}

\RU{Наряду с уровнями и направлениями оптимизации компилятор может включать в конечный файл отладочную информацию,
производя таким образом код, который легче отлаживать.}
\EN{In addition to optimization levels and direction, a compiler can include in the resulting file some debug information,
thus producing code for easy debugging.}
\PTBR{Além dos níveis e direcionamento da otimização, o compilador pode incluir no arquivo resultante algumas informações de depuração, produzindo assim código para fácil depuração.}
\ES{Además de los niveles y dirección de la otimización, el compilador puede incluir informaciones de depuración en el archivo resultante, produciendo así código para fácil depuración.}

\RU{Одна очень важная черта отладочного кода в том, что он может содержать
связи между каждой строкой в исходном коде и адресом в машинном коде.}
\EN{One of the important features of the ´debug' code is that it might contain links
between each line of the source code and the respective machine code addresses.}
\PTBR{Uma das características importantes do código de ´debug' é que ele pode conter 
ligações entre cada linha do código-fonte e os respectivos endereços de código de máquina.}
\ES{Una de los características importantes del código de ´debug' és que puede contener enlaces entre
cada línea del código fuente y las direcciones de código de máquina respectivos.}
\RU{Оптимизирующие компиляторы обычно генерируют код, где целые строки из исходного кода
могут быть оптимизированы и не присутствовать в итоговом машинном коде.}
\EN{Optimizing compilers, on the other hand, tend to produce output where entire lines of source code
can be optimized away and thus not even be present in the resulting machine code.}
\PTBR{Compiladores com otimização, por outro lado, tendem a produzir uma saída onde linhas inteiras de código-fonte podem ser otimizadas a ponto de serem removidas e portanto não estarem presentes no código de máquina resultante.}
\ES{Compiladores con optimización, por otro lado, tienden a producir una salida donde líneas enteras de código fuente pueden ser optimizados al punto de ser eliminados y por consiguiente no estar presentes en el código de máquina resultante.}

\RU{Практикующий reverse engineer обычно сталкивается с обоими версиями, потому что некоторые разработчики
включают оптимизацию, некоторые другие\EMDASH{}нет. Вот почему мы постараемся поработать с примерами для обоих версий.}
\EN{Reverse engineers can encounter either version, simply because some developers turn on the compiler's optimization flags and others do not. 
Because of this, we'll try to work on examples of both debug and release versions of the code featured in this book, where possible.}
\PTBR{Engenheiros Reversos podem encontrar ambas as versões, simplesmente porque alguns desenvolvedores ativam as flags de otimização do compilador e outros não ativam. 
Por causa disso, nós tentaremos trabalhar em exemplos de ambas as versões de debug e release do código destacado neste livro, onde possível.}
\ES{Ingenieros Inversos pueden encontrar ambas versiones, simplesmente porque alguns desarrolladores activan los flags de optimización del compilador, y otros no activan. 
Debido a esto, vamos a tratar de trabajar con ejemplos de ambas versiones de debug y release del código resaltado en este libro, cuando sea posible.}

\chapter{\RU{Краткое введение в CPU}\EN{A short introduction to the CPU}\PTBR{Uma breve introdução à CPU}\ES{Una breve introducción a la CPU}}

\EN{The}\PTBR{A}\ES{La} \ac{CPU} \RU{это устройство исполняющее все программы}\EN{is the device that executes the machine code a program consists of}\PTBR{é o dispositivo que executa o código de máquina que consiste num programa}\ES{es el dispositivo que ejecuta el código de máquina que constituye un programa}.

\textbf{\RU{Немного терминологии}\EN{A short glossary}\PTBR{Um pequeno glossário}\ES{Un breve glosario}:}

\begin{description}
\item[\RU{Инструкция}\EN{Instruction}\PTBR{Instrução}\ES{Instrucción}]: \RU{примитивная команда}\EN{A primitive}\PTBR{Um primitivo}\ES{Una primitiva}
	\ac{CPU}\RU{.} \EN{command.}\PTBR{comando.}\ES{comando.}
\RU{Простейшие примеры: перемещение между регистрами, работа с памятью, примитивные арифметические операции}%
\EN{The simplest examples include: moving data between registers, working with memory, primitive arithmetic operations}
\PTBR{Os exemplos mais simples incluem: mover dados entre registradores, trabalhar com a memória, operações aritiméticas primitivas}
\ES{Los ejemplos más simples incluyen: mover datos entre registros, trabajar con la memoria, operaciones aritméticas primitivas}.
\RU{Как правило, каждый}\EN{As a rule, each}\PTBR{Como regra geral, cada}\ES{Como regla general, cada} \ac{CPU} \RU{имеет свой набор инструкций}\EN{has its own instruction set architecture}\PTBR{tem seu próprio conjunto de instruções}\ES{tiene su proprio conjunto de instrucciones} 
(\ac{ISA}).

\item[\RU{Машинный код}\EN{Machine code}]: \RU{код понимаемый}\EN{Code that the}\PTBR{Código que a}\ES{Código que la} \ac{CPU}\EN{ directly processes}\PTBR{ processa diretamente}\ES{ procesa directamente}. 
\RU{Каждая инструкция обычно кодируется несколькими байтами}\EN{Each instruction is usually encoded by several bytes}\PTBR{Cada instrução é normalmente codificada em vários bytes}\ES{Cada instrucción generalmente se codifica por vários bytes}.

\item[\RU{Язык ассемблера}\EN{Assembly language}\PTBR{Linguagem assembly}\ES{Lenguaje assembly}]: 
\RU{машинный код плюс некоторые расширения, призванные облегчить труд программиста: макросы, имена, \etc.}
\EN{Mnemonic code and some extensions like macros that are intended to make a programmer's life easier.}
\PTBR{Código mnemônico e algumas extensões como macros que têm a finalidade de facilitar a vida do programamdor.}
\ES{Código mnemónico y algunas extensiones como macros que destinados a hacer la vida del programador más fácil.}

\item[\RU{Регистр CPU}\EN{CPU register}\PTBR{Registradores da CPU}\ES{Registros de la CPU}]: 
\RU{Каждый}\EN{Each}\PTBR{Cada}\ES{Cada} \ac{CPU} \RU{имеет некоторый фиксированный набор регистров общего назначения}\EN{has a fixed set of general purpose registers}\PTBR{tem um conjunto fixo de registradores de propósito geral}\ES{tiene un conjunto fijo de registros de propósito general} (\ac{GPR}).
$\approx 8$ \InENRU x86, $\approx 16$ \InENRU x86-64, $\approx 16$ \InENRU ARM.
\RU{Проще всего понимать регистр как временную переменную без типа}%
\EN{The easiest way to understand a register is to think of it as an untyped temporary variable}
\PTBR{A forma mais fácil de entender um registrador é pensar nele como uma variável temporária não tipada}
\ES{La forma más fácil de entender un registro es pensar en ello como una variable temporal sin tipo}.
\RU{Можно представить, что вы пишете на \ac{PL} высокого уровня и у вас только 8 переменных шириной 32 (или 64) бита}%
\EN{Imagine if you were working with a high-level \ac{PL} and could only use eight 32-bit (or 64-bit) variables}
\PTBR{Imagine que você estivesse trabalhando com uma \ac{PL} de alto nível e pudesse usar apenas oito variáveis de 32-bit (ou de 64-bit)}
\ES{Imagine si estuviera trabajando con una \ac{PL} de alto nivel y sólo podría utilizar ocho variables de 32-bit (o de 64-bit)}.
\RU{Можно сделать очень много используя только их}\EN{Yet a lot can be done using just these}\PTBR{No entanto, muito ainda pode ser feito usando apenas eles}\ES{Sin embargo mucho se puede hacer usando sólo estos}!
\end{description}

\RU{Откуда взялась разница между машинным кодом и \ac{PL} высокого уровня?
Ответ в том, что люди и \ac{CPU}-ы отличаются друг от друга\EMDASH{}}
\EN{One might wonder why there needs to be a difference between machine code and a \ac{PL}.
The answer lies in the fact that humans and \ac{CPU}s are not alike\EMDASH{}}
\PTBR{Alguém poderia perguntar por que é preciso haver diferença entre código de máquina e uma \ac{PL} de alto nível.
A resposta reside no fato de que humanos e \ac{CPU}s não são iguais\EMDASH{}}
\ES{Uno podría perguntarse por qué es necessário que haya diferencia entre el código de la máquina y una lenguaje de programación de alto nivel.
La respuesta está en el hecho de que los seres humanos y CPUs no son iguales\EMDASH{}}
\RU{человеку проще писать на \ac{PL} высокого уровня вроде \CCpp, Java, Python, 
а \ac{CPU} проще работать с абстракциями куда более низкого уровня}%
\EN{it is much easier for humans to use a high-level \ac{PL} like \CCpp, Java, Python, etc., 
but it is easier for a \ac{CPU} to use a much lower level of abstraction.}
\PTBR{É muito mais fácil para os humanos usar uma \ac{PL} de alto nível como \CCpp, Java, Python, etc.,
mas é muito mais fácil para a \ac{CPU} usar um nível de abstração muito menor.}
\ES{És mucho más fácil para los humanos utilizar un \ac{PL} de alto nivel como \CCpp, Java, Python, etc., 
pero és más fácil para una \ac{CPU} utilizar un nivel mucho más bajo de abstración.}
\RU{Возможно, можно было бы придумать \ac{CPU} исполняющий код \ac{PL} высокого уровня, но он был бы значительно сложнее, чем те, что мы имеем сегодня.}
\EN{Perhaps it would be possible to invent a \ac{CPU} that can execute high-level \ac{PL} code, but it would be many times more complex than the \ac{CPU}s we know of today.}
\PTBR{talvez fosse possível inventar uma \ac{CPU} que pudesse executar código feito em \ac{PL} alto nível, mas seria inúmeras vezes mais complexa do que as \ac{CPU}s que conhecemos hoje.}
\ES{Tal vez sería posible inventar una \ac{CPU} que podría ejecutar código de \ac{PL} de alto nivel, pero sería muchas veces más compleja que las \ac{CPU}s que conocemos hoy.}
% A note on the experiments in this area (like the LISP machines http://en.wikipedia.org/wiki/Lisp_machine
% might be useful
\RU{И наоборот, человеку очень неудобно писать на ассемблере из-за его низкоуровневости,
к тому же, крайне трудно обойтись без мелких ошибок.}
\EN{In a similar fashion, it is very inconvenient for humans to write in assembly language,
due to it being so low-level and difficult to write in without making a huge number of annoying mistakes.}
\PTBR{De forma semelhante, é muito inconveninente para os seres humanos escrever em linguagem assembly, 
devido ao fato dela ser tão baixo nível e difícil de escrever sem comenter uma enorme quantidade de erros irritantes.}
\ES{En uma manera similar, es muy incómodo para los seres humanos escribir en lenguaje assembly, 
debido a que es tan bajo nivel y difícil escribir sin hacer una gran cantidade de errores molestos.}
\RU{Программа, переводящая код из \ac{PL} высокого уровня в ассемблер называется \IT{компилятором}%
\footnote{
	\RU{В более старой русскоязычной литературе также часто встречается термин \q{транслятор}.}
	\EN{Old-school Russian literature also use term \q{translator}.}
	\ESph{}
	\PTBRph{}\PLph{}\ITAph{}
}.}
\EN{The program that converts the high-level \ac{PL} code into assembly is called a \IT{compiler}.}
\PTBR{O programa que converte o código de \ac{PL} de alto nível em assembly é chamado \IT{compiler}.}
\ES{El programa que convierte el código de \ac{PL} de alto nivel en assembly se llama \IT{compiler}.}
% TODO1 add about linker: "компоновщик" и "редактор связей" в русскоязычной лит-ре

\ifx\LITE\undefined
\section{\RU{Несколько слов о разнице между \ac{ISA}}\EN{A couple of words about different \ac{ISA}s}\PTBR{Algumas palavras a respeito de diferentes \ac{ISA}s}\ES{Algunas palabras sobre diferentes \ac{ISA}s}}

\RU{x86 всегда был архитектурой с опкодами переменной длины, так что когда пришла 64-битная эра,
расширения x64 не очень сильно повлияли на \ac{ISA}.}
\RU{В x86 до сих пор есть масса инструкций, появившихся в 16-битном 8086 и присутствующих в самых последних
процессорах.}
\EN{The x86 \ac{ISA} has always been one with variable-length opcodes, so when the 64-bit era came, 
the x64 extensions did not impact the \ac{ISA} very significantly. In fact, the x86 \ac{ISA} still contains a lot of instructions that first appeared in 16-bit 8086 CPU, yet are still found in the CPUs of today.}
\PTBR{O \ac{ISA} x86 sempre possuiu opcodes de tamanho variável, então com a chegada da era do 64-bit, 
as extensões x64 não impactaram a \ac{ISA} de forma muito significante. De fato, o \ac{ISA} x86 ainda contém uma série de instrucões que surgiram inicialmente na CPU 8086 16-bit, mas ainda são encontradas nas CPUs de hoje em dia.}
\ES{El \ac{ISA} x86 siempre ha tenido opcodes de tamaño variable, de modo que cuanco llegó la era de 64-bit, 
las extensiones x64 no impactan el \ac{ISA} de manera muy significativa. De hecho, el \ac{ISA} x86 aún contiene una gran cantidade de instrucciones que primero aparecieron en CPU 8086 16-bit, pero aún se encuentran en las CPUs de hoy.}
\PLph{}\ITAph{}\\
\\
\index{ARM!\ARMMode}%
\index{ARM!\ThumbMode}%
\index{ARM!\ThumbTwoMode}%

\RU{ARM это \ac{RISC}-процессор разработанный с учетом опкодов одинаковой длины, что было некоторым преимуществом в прошлом.}
\EN{ARM is a \ac{RISC} \ac{CPU} designed with constant-length opcode in mind, which had some advantages in the past.}
\PTBR{ARM é uma \ac{CPU} \ac{RISC} desenvolvido com a idéia de opcodes com tamanho constante, o que trouxe algumas vantagens no passado.}
\ES{ARM és una \ac{CPU} \ac{RISC} diseñado con la idea de opcodes con tamaño constante, que tenía algunas ventajas en el pasado.}
\RU{Так что в самом начале все инструкции ARM кодировались 4-мя байтами}%
\EN{In the very beginning, all ARM instructions were encoded in 4 bytes}%
\PTBR{Bem no início, todas as instruções ARM foram codificadas em 4 bytes}%
\ES{En el principio, todas las instrucciones ARM fueron codificados en 4 bytes}%
\ifx\LITE\undefined
\footnote{\RU{Кстати,
инструкции фиксированного размера удобны тем, что всегда можно легко узнать адрес 
следующей (или предыдущей) инструкции. Эта особенность будет рассмотрена в секции об операторе 
switch()~(\myref{sec:SwitchARMLot}).}
\EN{By the way, fixed-length instructions are handy because one can calculate the next (or previous) 
instruction address without effort. This feature will be discussed in the switch() operator~(\myref{sec:SwitchARMLot}) section.}
\PTBR{A propósito, instruções de tamanho fixo são úteis porque se pode calcular o endereço da próxima instrução (ou da anterior) sem esforço. Esta característica será discutida na seção do operador switch() ~(\myref{sec:SwitchARMLot}).}
\ES{Dicho sea de paso, las instrucciones de longitud fija son muy útiles porque se puede calcular la dirección de instrucción siguiente (o anterior) sin esfuerzo. Esta característica se discutirá en la sección de el operador switch() ~(\myref{sec:SwitchARMLot}).}
}%
\fi
.
\RU{Это то, что сейчас называется \q{режим ARM}}\EN{This is now referred to as \q{ARM mode}}\PTBR{Este é atualmente referenciado como \q{ARM mode}}\ES{Esto actualmente se conoce como \q{ARM mode}}.

\RU{Потом они подумали, что это не очень экономично}\EN{Then they thought it wasn't as frugal as they first imagined}\PTBR{Então concluiu-se que não era tão econômico quanto se imaginou a princípio.}\ES{Entonces se llegó a la conclusión que no era tan económico como se imaginó al princípio}.
\RU{На самом деле, самые используемые инструкции\footnote{А это MOV/PUSH/CALL/Jcc} процессора на практике могут быть закодированы
c использованием меньшего количества информации.}
\EN{In fact, most used \ac{CPU} instructions\footnote{These are MOV/PUSH/CALL/Jcc} in real world applications can be encoded using less information.}
\PTBR{Na verdade, as instruções de \ac{CPU} mais utilizadas \footnote{São estas MOV/PUSH/CALL/Jcc} em aplicações do mundo real podem ser codificadas usando menos informação.}
\ES{En realidad, la mayoría de las instrucciones de \ac{CPU} utilizados \footnote{Son estos MOV/PUSH/CALL/Jcc} en aplicaciones del mundo real pueden ser codificados utilizando menos información.}
\RU{Так что они добавили другую \ac{ISA} с названием Thumb, где каждая инструкция кодируется всего лишь
2-мя байтами.}
\EN{They therefore added another \ac{ISA}, called Thumb, where each instruction was encoded in just 2 bytes.}
\PTBR{Foi adicionado então outro \ac{ISA}, chamado Thumb, onde cada instrução era codificada em apenas 2 bytes.}
\ES{Por lo tanto añadieron otra \ac{ISA}, llamado Thumb, donde cada instrucción fue codificada en sólo 2 bytes.}
\RU{Теперь это называется \q{режим Thumb}}\EN{This is now referred as \q{Thumb mode}}\PTBR{Este é conhecido como \q{Thumb mode}}\ES{Esto se conoce como \q{Thumb mode}}.
\RU{Но не все инструкции ARM могут быть закодированы в двух байтах, так что набор инструкций Thumb ограниченный.}
\EN{However, not \IT{all} ARM instructions can be encoded in just 2 bytes, so the Thumb instruction set is somewhat limited.}
\PTBR{No entanto, nem \IT{all} instruções ARM podem ser codificadas em apenas 2 bytes, então o conjunto de instruções Thumb é de certa forma limitado.}
\ES{No obstante, no todas las instrucciones ARM pueden ser codificadas en apenas 2 bytes, entonces el conjunto de instrucciones Thumb es algo limitada.}
\RU{Код, скомпилированный для режима ARM и Thumb может сосуществовать в одной программе.}
\EN{It is worth noting that code compiled for ARM mode and Thumb mode may of course coexist within one single program.}
\PTBR{É interessante notar que códigos compilados para os modos ARM e Thumb podem, conforme esperado, coexistir num mesmo programa.}
\ES{Es importante destacar que el código compilado para el modo ARM y para el modo Thumb pueden, por supuesto, coexistir dentro de un solo programa.}

\RU{Затем создатели ARM решили, что Thumb можно расширить: так появился Thumb-2 (в ARMv7).}
\EN{The ARM creators thought Thumb could be extended, giving rise to Thumb-2, which appeared in ARMv7.}
\PTBR{Os criadores do ARM concluíram que o Thumb poderia ser extendido, dando origem ao Thumb-2, que apareceu no ARMv7.}
\ES{Los creadores de ARM concluyeron que se podría extender el Thumb, dando origem al Thumb-2, que apareció en el ARMv7.}
\RU{Thumb-2 это всё ещё двухбайтные инструкции, но некоторые новые инструкции имеют длину 4 байта.}
\EN{Thumb-2 still uses 2-byte instructions, but has some new instructions which have the size of 4 bytes.}
\PTBR{Thumb-2 ainda usa instruções de 2 bytes, mas possui algumas novas instruções com 4 bytes de tamanho.}
\ES{Thumb-2 sigue utilizando instrucciones de 2 bytes, pero tiene algunas nuevas instrucciones que tienen el tamaño de 4 bytes.}
\RU{Распространено заблуждение, что Thumb-2\EMDASH{}это смесь ARM и Thumb. Это не верно. Режим Thumb-2 был дополнен до
более полной поддержки возможностей процессора и теперь может легко конкурировать с режимом ARM.
Основное количество приложений для \idevices скомпилировано для набора инструкций Thumb-2, потому что Xcode
делает так по умолчанию.}
\EN{There is a common misconception that Thumb-2 is a mix of ARM and Thumb. This is incorrect. 
Rather, Thumb-2 was extended to fully support all processor features so it could
compete with ARM mode\EMDASH{}a goal that was clearly achieved, as the majority of applications for \idevices are compiled for the Thumb-2 instruction set (admittedly, largely due to the fact that Xcode does this by default).}
\PTBR{Há um equívoco comum que Thumb-2 é uma mistura de ARM e Thumb. Isso é incorreto.
Em vez disso, Thumb-2 foi extendido para suportar completamente todos os recursos de processador de forma que ele pudesse competir com o modo ARM\EMDASH{}um objetivo que foi claramente alcançado, uma vez que a maioria das aplicações para \idevices são compiladas para o conjunto de instruções do Thumb-2 (admitidamente, principalmente devido ao fato que o Xcode faz isso por padrão).}
\ES{Hay una idea errónea de que Thumb-2 es una mezcla de ARM y Thumb. Esto es incorrecto. 
Más bien, se extendió Thumb-2 para apoyar plenamente todas las características de processador por lo que podría 
competir con el modo ARM\EMDASH{}un objetivo que se logró con claridad, ya que la mayoria de aplicacciones para \idevices son compmilados para el conjunto de instrucciones del Thumb-2 (la verdade es, en gran parte debido al hecho de que Xcode hace esto por defecto).}
\RU{Потом появился 64-битный ARM. Это \ac{ISA} снова с 4-байтными опкодами, без дополнительного режима Thumb.}
\EN{Later the 64-bit ARM came out. This \ac{ISA} has 4-byte opcodes, and lacked the need of any additional Thumb mode.}
\PTBR{Posteriormente o ARM 64-bit foi lançado. Este \ac{ISA} tem opcodes de 4 bytes, e descarta a necessidade de qualquer modo Thumb adicional.}
\ES{Más tarde, el ARM 64-bit salió. Este \ac{ISA} tiene opcodes de 4 bytes, y descarta la necesidade de cualquier modo Thumb adicional.}
\RU{Но 64-битные требования повлияли на \ac{ISA}, так что теперь у нас 3 набора инструкций ARM:
режим ARM, режим Thumb (включая Thumb-2) и ARM64.}
\EN{However, the 64-bit requirements affected the \ac{ISA}, resulting in us now having three ARM instruction sets: ARM mode, Thumb mode (including Thumb-2) and ARM64.}
\PTBR{No entanto, os requisitos de 64-bit afetaram o \ac{ISA}, resultando em termos atualmente três conjuntos de instruções ARM: ARM mode, Thumb mode (incluindo Thumb-2) e ARM64.}
\ES{Pero, los requisitos de 64-bit afectaron la \ac{ISA}, resultando en ahora tenermos tres conjuntos de instrucciones ARM: ARM mode, Thumb mode (incluyendo Thumb-2) y ARM64.}
\RU{Эти наборы инструкций частично пересекаются, но можно сказать, это скорее разные наборы, нежели вариации одного.}%
\EN{These \ac{ISA}s intersect partially, but it can be said that they are different \ac{ISA}s, rather than variations of the same one.}
\PTBR{Estes \ac{ISA}s se intersecionam parcialmente, porém podemos dizer que são \ac{ISA}s diferentes, ao invés de variações do mesmo.}
\ES{Estos \ac{ISA}s se intersectan parcialmente, pero puede ser más bien decir que son \ac{ISA}s diferentes, en lugar de variaciones de lo mismo.}
\RU{Следовательно, в этой книге постараемся добавлять фрагменты кода на всех трех ARM \ac{ISA}.}
\EN{Therefore, we would try to add fragments of code in all three ARM \ac{ISA}s in this book.}
\PTBR{Portanto, gostaríamos de tentar adicionar pedaços de código dos três \ac{ISA}s do ARM neste livro.}
\ES{Por lo tanto, nos gustaría intentar añadir fragmentos de código de los tres \ac{ISA}s del ARM en este libro.}

\index{PowerPC}%
\index{MIPS}%
\index{Alpha AXP}%

\RU{Существует ещё много \ac{RISC} \ac{ISA} с опкодами фиксированной 32-битной длины~--- это как минимум}
\EN{There are, by the way, many other \ac{RISC} \ac{ISA}s with fixed length 32-bit opcodes, such as}
\PTBR{Existem, a propósito, muitos outros \ac{RISC} \ac{ISA}s com opcodes de tamanho fixo de 32-bit, como}
\ES{Hay, por cierto, muchos otros \ac{RISC} \ac{ISA}s con opcodes de tamaño fijo de 32-bit, tales como}
MIPS, PowerPC \AndENRU Alpha AXP.
\fi

% chapters
\chapter{%
\RU{Простейшая функция}%
\EN{The simplest Function}%
\ES{Spanish text here}%
\PTBR{Brazilian portuguese text here}%
}

\RU{Наверное, простейшая из возможных функций это та что возвращает некоторую константу:}%
\EN{The simplest possible function is arguably one that simply returns a constant value:}

\RU{Вот, например}\EN{Here it is}:

\lstinputlisting[caption=\EN{\CCpp Code}\RU{Код на \CCpp}]{patterns/00_ret/1.c}

\RU{Скомпилируем её!}
\EN{Lets compile it!}

\section{x86}

\RU{И вот что делает оптимизирующий GCC}\EN{Here's what both the optimizing GCC and MSVC compilers produce on the x86 platform}:

\lstinputlisting[caption=\Optimizing GCC/MSVC (\assemblyOutput)]{patterns/00_ret/1.s}

\index{x86!\Instructions!RET}
\RU{Здесь только две инструкции. Первая помещает значение 123 в регистр \EAX, который используется
для передачи возвращаемых значений. Вторая это \RET, которая возвращает управление в вызывающую функцию.}
\EN{There are just two instructions: the first places the value 123 into the \EAX register, which is used by convention for storing the return
value and the second one is \RET, which returns execution to the \gls{caller}.}
\RU{Вызывающая функция возьмет результат из регистра \EAX.}
\EN{The caller will take the result from the \EAX register.}

\ifdefined\IncludeARM
\section{ARM}

\RU{А что насчет ARM?}\EN{There are a few differences on the ARM platform:}

\lstinputlisting[caption=\OptimizingKeilVI (\ARMMode) ASM Output]{patterns/00_ret/1_Keil_ARM_O3.s}

\RU{ARM использует регистр \Reg{0} для возврата значений, так что здесь 123 помещается в \Reg{0}.}
\EN{ARM uses the register \Reg{0} for returning the results of functions, so 123 is copied into \Reg{0}.}

\RU{Адрес возврата (\ac{RA}) в ARM не сохраняется в локальном стеке, а в регистре \ac{LR}.
Так что инструкция \TT{BX LR} делает переход по этому адресу, и это то же самое что и вернуть управление
в вызывающую ф-цию.}
%Maybe explain what a link register is, or if it is just a normal register, say so?
\EN{The return address is not saved on the local stack in the ARM \ac{ISA}, but rather in the link register, 
so the \TT{BX LR} instruction causes execution to jump to that address\EMDASH{}effectively returning execution to the \gls{caller}.}
\fi

\index{ARM!\Instructions!MOV}
\index{x86!\Instructions!MOV}
\RU{Нужно отметить, что название инструкции \MOV в x86 и ARM сбивает с толку.}
\EN{It is worth noting that \MOV is a misleading name for the instruction in both x86 and ARM \ac{ISA}s. }
\RU{На самом деле, данные не \IT{перемещаются}, а скорее \IT{копируются}.}
\EN{The data is not in fact \IT{moved}, but \IT{copied}.}

\ifdefined\IncludeMIPS
\section{MIPS}

\label{MIPS_leaf_function_ex1}
\RU{Есть два способа называть регистры в мире MIPS.}
\EN{There are two naming conventions used in the world of MIPS when naming registers:}
\RU{По номеру (от \$0 до \$31) или по псевдоимени (\$V0, \$A0, \etc{}.).}
\EN{by number (from \$0 to \$31) or by pseudoname (\$V0, \$A0, \etc{}).}
\RU{Вывод на ассемблере в GCC показывает регистры по номерам:}
\EN{The GCC assembly output below lists registers by number:}

\lstinputlisting[caption=\Optimizing GCC 4.4.5 (\assemblyOutput)]{patterns/00_ret/MIPS.s}

\dots \RU{а \IDA\EMDASH{}по псевдоименам}\EN{while \IDA does it\EMDASH{}by their pseudonames}:

\lstinputlisting[caption=\Optimizing GCC 4.4.5 (IDA)]{patterns/00_ret/MIPS_IDA.lst}

\RU{Так что регистр \$2 (или \$V0) используется для возврата значений.}
\EN{The \$2 (or \$V0) register is used to store the function's return value.}
\index{MIPS!\Pseudoinstructions!LI}
LI \RU{это}\EN{stands for} ``Load Immediate'' \EN{and is the MIPS equivalent to MOV}.

\index{MIPS!\Instructions!J}
\RU{Другая инструкция это инструкция перехода (J или JR), которая возвращает управление в 
\glslink{caller}{вызывающую ф-цию}, переходя по адресу в регистре \$31 (или \$RA).}
\EN{The other instruction is the jump instruction (J or JR) which returns the execution flow to the \gls{caller},
jumping to the address in the \$31 (or \$RA) register.}
\RU{Это аналог регистра \ac{LR} в ARM.}
\EN{This is the register analogous to \ac{LR} in ARM.}

\RU{Но почему инструкция загрузки (LI) и инструкция перехода (J или JR) поменены местами?}
\index{MIPS!Branch delay slot}
\RU{Это артефакт \ac{RISC} и называется он}
\EN{You might be wondering why positions of the the load instruction (LI) and the jump instruction (J or JR) are swapped. This is due to a \ac{RISC} feature called} ``branch delay slot''.
\RU{На самом деле, нам не нужно вникать в эти детали.}
\RU{Нужно просто запомнить: в MIPS инструкция после инструкции перехода исполняется \IT{перед} 
инструкцией перехода.}
\EN{The reason this happens is a quirk in the architecture of some RISC \ac{ISA}s and isn't important for our purposes - we just need to remember that in MIPS, the instruction following a jump or branch instruction
is executed \IT{before} the jump/brunch instruction itself.}
\RU{Таким образом, инструкция перехода всегда поменена местами с той, которая должна быть исполнена перед ней.}
\EN{As a consequence, branch instructions always swap places with the instruction which must be executed beforehand.}
% A footnote/link to http://en.wikipedia.org/wiki/Delay_slot#Branch_delay_slots or
% something similar might be useful for the people more interested in it.

\subsection{\RU{Еще кое-что об именах инструкций и регистров в MIPS}\EN{A note about MIPS instruction/register names}}

\RU{Имена регистров и инструкций в мире MIPS традиционно пишутся в нижнем регистре.}
\EN{Register and instruction names in the world of MIPS are traditionally written in lowercase.}
\RU{Но мы будем использовать верхний регистр, потому что имена инструкций и регистров других \ac{ISA} в этой книге так же в верхнем регистре.}
\EN{However, for the sake of consistency, we'll stick to using uppercase letters, as it is the convention followed by all other \ac{ISA}s featured this book.}

\fi

\chapter{\HelloWorldSectionName}
\label{sec:helloworld}

\RU{Продолжим, используя знаменитый пример из книги}
\EN{Let's use the famous example from the book}
``The C programming Language''\cite{Kernighan:1988:CPL:576122}:

\lstinputlisting{patterns/01_helloworld/hw.c}

\section{x86}

\input{patterns/01_helloworld/MSVC_x86}
\ifdefined\IncludeGCC
\input{patterns/01_helloworld/GCC_x86}
\fi

\section{x86-64}
\input{patterns/01_helloworld/MSVC_x64}
\ifdefined\IncludeGCC
\input{patterns/01_helloworld/GCC_x64}
\fi

\ifdefined\IncludeGCC
\input{patterns/01_helloworld/GCC_one_more}
\fi
\ifdefined\IncludeARM
\input{patterns/01_helloworld/ARM/main}
\fi
\ifdefined\IncludeMIPS
\input{patterns/01_helloworld/MIPS/main}
\fi

\section{\Conclusion{}}

\RU{Основная разница между кодом x86/ARM и x64/ARM64 в том, что указатель на строку теперь 64-битный.}
\EN{The main difference between x86/ARM and x64/ARM64 code is that the pointer to the string is now 64-bits in length.}
\RU{Действительно, ведь для того современные \ac{CPU} и стали 64-битными, потому что подешевела память,
её теперь можно поставить в компьютер намного больше, и чтобы её адресовать, 32-х бит уже
недостаточно.}
\EN{Indeed, modern \ac{CPU}s are now 64-bit due to both the reduced cost of memory and the greater demand for it by modern applications. 
We can add much more memory to our computers than 32-bit pointers are able to address.}
\RU{Поэтому все указатели теперь 64-битные.}\EN{As such, all pointers are now 64-bit.}

% sections
\ifdefined\IncludeExercises
\input{patterns/01_helloworld/exercises}
\fi

\chapter{\RU{Пролог и эпилог функций}\EN{Function prologue and epilogue}}
\label{sec:prologepilog}
\index{Function epilogue}
\index{Function prologue}

\RU{Пролог функции это инструкции в самом начале функции. Как правило это что-то вроде такого
фрагмента кода:}
\EN{A function prologue is a sequence of instructions at the start of a function. It often looks something like the following
code fragment:}

\begin{lstlisting}
    push    ebp
    mov     ebp, esp
    sub     esp, X
\end{lstlisting}

\RU{Эти инструкции делают следующее: сохраняют значение регистра \EBP на будущее, выставляют \EBP равным \ESP, 
затем подготавливают место в стеке для хранения локальных переменных.}
\EN{What these instruction do: save the value in the \EBP register,
set the value of the \EBP register to the value of the \ESP and then allocate space on the stack 
for local variables.}

\RU{\EBP сохраняет свое значение на протяжении всей функции, он будет использоваться здесь для доступа 
к локальным переменным и аргументам. Можно было бы использовать и \ESP, но он постоянно меняется и 
это не очень удобно.}
\EN{The value in the \EBP stays the same over the period of the function execution and is to be used for local variables and 
arguments access. 
For the same purpose one can use \ESP, but since it changes over time this approach is not too convenient.}

\RU{Эпилог функции аннулирует выделенное место в стеке, восстанавливает значение \EBP на старое и возвращает 
управление в вызывающую функцию:}
\EN{The function epilogue frees the allocated space in the stack, returns the value in the \EBP register back to its initial state 
and returns the control flow to the \gls{callee}:}

\begin{lstlisting}
    mov    esp, ebp
    pop    ebp
    ret    0
\end{lstlisting}

% what about calling convention?
\RU{Пролог и эпилог функции обычно находятся в дизассемблерах для отделения функций друг от друга.}
\EN{Function prologues and epilogues are usually detected in disassemblers for function delimitation.}

\section{\Recursion}

\index{\Recursion}
\RU{Наличие эпилога и пролога может несколько ухудшить эффективность рекурсии.}
\EN{Epilogues and prologues can negatively affect the recursion performance.}

\EN{More about recursion in this book}\RU{Больше о рекурсии в этой книге}: 
\myref{Recursion_and_tail_call}.

\chapter{\Stack}
\label{sec:stack}
\index{\Stack}

\RU{Стек в информатике~--- это одна из наиболее фундаментальных структур данных}%
\EN{The stack is one of the most fundamental data structures in computer science}%
\footnote{\href{http://go.yurichev.com/17119}{wikipedia.org/wiki/Call\_stack}}.

\RU{Технически это просто блок памяти в памяти процесса + регистр \ESP в x86 или \RSP в x64, либо \ac{SP} в ARM, который указывает где-то в пределах этого блока.}
\EN{Technically, it is just a block of memory in process memory along with the \ESP or \RSP register in x86 or x64, or the \ac{SP} register in ARM, as a pointer within that block.}

\index{ARM!\Instructions!PUSH}
\index{ARM!\Instructions!POP}
\index{x86!\Instructions!PUSH}
\index{x86!\Instructions!POP}
\RU{Часто используемые инструкции для работы со стеком~--- это \PUSH и \POP (в x86 и Thumb-режиме ARM). 
\PUSH уменьшает \ESP/\RSP/\ac{SP} на 4 в 32-битном режиме (или на 8 в 64-битном),
затем записывает по адресу, на который указывает \ESP/\RSP/\ac{SP}, содержимое своего единственного операнда.}
\EN{The most frequently used stack access instructions are \PUSH and \POP (in both x86 and ARM Thumb-mode). 
\PUSH subtracts from \ESP/\RSP/\ac{SP} 4 in 32-bit mode (or 8 in 64-bit mode) and then writes the contents of its sole operand to the memory address pointed by \ESP/\RSP/\ac{SP}.} 

\RU{\POP это обратная операция~--- сначала достает из \glslink{stack pointer}{указателя стека} значение и помещает его в операнд 
(который очень часто является регистром) и затем увеличивает указатель стека на 4 (или 8).}
\EN{\POP is the reverse operation: retrieve the data from the memory location that \ac{SP} points to, 
load it into the instruction operand (often a register) and then add 4 (or 8) to the \gls{stack pointer}.}

\RU{В самом начале \glslink{stack pointer}{регистр-указатель} указывает на конец стека.}
\EN{After stack allocation, the \gls{stack pointer} points at the bottom of the stack.}
\RU{\PUSH уменьшает \glslink{stack pointer}{регистр-указатель}, а \POP~--- увеличивает.}
\EN{\PUSH decreases the \gls{stack pointer} and \POP increases it.}
\RU{Конец стека находится в начале блока памяти, выделенного под стек. Это странно, но это так.}
\EN{The bottom of the stack is actually at the beginning of the memory allocated for the stack block. 
It seems strange, but that's the way it is.}

\ifdefined\IncludeARM
\RU{В процессоре ARM, тем не менее, есть поддержка стеков, растущих как в сторону уменьшения, так и в
сторону увеличения.}
\EN{ARM supports both descending and ascending stacks.}
\index{ARM!\Instructions!STMFD}
\index{ARM!\Instructions!LDMFD}
\index{ARM!\Instructions!STMED}
\index{ARM!\Instructions!LDMED}
\index{ARM!\Instructions!STMFA}
\index{ARM!\Instructions!LDMFA}
\index{ARM!\Instructions!STMEA}
\index{ARM!\Instructions!LDMEA}

\RU{Например, инструкции}\EN{For example the} 
\ac{STMFD}/\ac{LDMFD}, \ac{STMED}/\ac{LDMED} 
\RU{предназначены для descending-стека 
(растет назад, начиная с высоких адресов в сторону низких).}
\EN{instructions are intended to deal with a descending stack 
(grows downwards, starting with a high address and progressing to a lower one).}
\RU{Инструкции}\EN{The}
\ac{STMFA}/\ac{LDMFA}, \ac{STMEA}/\ac{LDMEA} 
\RU{предназначены для ascending-стека 
(растет вперед, начиная с низких адресов в сторону высоких).}
\EN{instructions are intended to deal with an ascending stack 
(grows upwards, starting from a low address and progressing to a higher one).}
\fi

% It might be worth mentioning that STMED and STMEA write first,
% and then move the pointer,
% and that LDMED and LDMEA move the pointer first, and then read.
% In other words, ARM not only lets the stack grow in a non-standard direction,
% but also in a non-standard order.
% Maybe this can be in the glossary, which would explain why E stands for "empty".

\section{\RU{Почему стек растет в обратную сторону?}\EN{Why does the stack grow backwards?}}

\RU{Интуитивно мы можем подумать, что, как и любая другая структура данных, стек мог бы расти вперед, 
т.е. в сторону увеличения адресов}\EN{Intuitively, we might think that the stack grows upwards, i.e. towards
higher addresses, like any other data structure}.

\RU{Причина, почему стек растет назад, вероятно, историческая}%
\EN{The reason that the stack grows backward is probably historical}.
\RU{Когда компьютеры были большие и занимали целую комнату, было очень легко разделить сегмент на две части:
для \glslink{heap}{кучи} и для стека}\EN{When the computers were big and occupied a whole room, 
it was easy to divide memory into two parts, one for the \gls{heap} and one for the stack}.
\RU{Заранее было неизвестно, насколько большой может быть \glslink{heap}{куча} или стек, 
так что это решение было самым простым}\EN{Of course, 
it was unknown how big the \gls{heap} and the stack would be during program execution, 
so this solution was the simplest possible}.

\begin{center}
	\begin{tikzpicture}
	\tikzstyle{every path}=[thick]

	\node [rectangle,draw,minimum width=6cm, minimum height=2cm] (memory) {};
	\node [] [right=0.2cm of memory.west] (heap) {Heap};
	\node [] [left=0.2cm of memory.east] (stack) {Stack};

	\node [] (center1) [right=2cm of memory.west] {};
	\node [] (center2) [left=2cm of memory.east] {};

	\draw [->] (heap) -- (center1);
	\draw [->] (stack) -- (center2);

	\node [] [above left=1.1cm and 0.2cm of heap] (t1) {\RU{Начало кучи}\EN{Start of heap}};
	\node [] [above right=1.1cm and 0.2cm of stack] (t2) {\RU{Вершина стека}\EN{Start of stack}};

	\draw [->] (t1) -- (memory.west);
	\draw [->] (t2) -- (memory.east);

	\end{tikzpicture}
\end{center}

\RU{В}\EN{In} \cite{Ritchie74} \RU{можно прочитать}\EN{we can read}:

\begin{framed}
\begin{quotation}
The user-core part of an image is divided into three logical segments. The program text segment begins at location 0 in the virtual address space. During execution, this segment is write-protected and a single copy of it is shared among all processes executing the same program. At the first 8K byte boundary above the program text segment in the virtual address space begins a nonshared, writable data segment, the size of which may be extended by a system call. Starting at the highest address in the virtual address space is a stack segment, which automatically grows downward as the hardware's stack pointer fluctuates.
\end{quotation}
\end{framed}

\RU{Это немного напоминает как некоторые студенты
пишут два конспекта в одной тетрадке:
первый конспект начинается обычным образом, второй пишется с конца, перевернув тетрадку.
Конспекты могут встретиться где-то посредине, в случае недостатка свободного места.}
\EN{This reminds us how some students write two lecture notes using only one notebook:
notes for the first lecture are written as usual, 
and notes for the second one are written from the end of notebook, by flipping it.
Notes may meet each other somewhere in between, in case of lack of free space.}
% I think if we want to expand on this analogy,
% one might remember that the line number increases as as you go down a page.
% So when you decrease the address when pushing to the stack, visually,
% the stack does grow upwards.
% Of course, the problem is that in most human languages,
% just as with computers,
% we write downwards, so this direction is what makes buffer overflows so messy.

\section{\RU{Для чего используется стек?}\EN{What is the stack used for?}}

% subsections
\input{patterns/02_stack/01_saving_ret_addr}
\input{patterns/02_stack/02_args_passing}
\input{patterns/02_stack/03_local_vars}
\input{patterns/02_stack/04_alloca/main}
\input{patterns/02_stack/05_SEH}
\input{patterns/02_stack/06_BO_protection}

\subsection{\EN{Automatic deallocation of data in stack}\RU{Автоматическое освобождение данных в стеке}}

\RU{Возможно, причина хранения локальных переменных и SEH-записей в стеке в том, что после выхода из функции, всё эти данные освобождаются автоматически,
используя только одну инструкцию корректирования указателя стека (часто это ADD).}
\EN{Perhaps, the reason for storing local variables and SEH records in the stack is that they are freed automatically upon function exit,
using just one instruction to correct the stack pointer (it is often ADD).}
\RU{Аргументы функций, можно сказать, тоже освобождаются автоматически в конце функции.}
\EN{Function arguments, as we could say, are also deallocated automatically at the end of function.}
\RU{А всё что хранится в куче (\IT{heap}) нужно освобождать явно.}
\EN{In contrast, everything stored in the \IT{heap} must be deallocated explicitly.}

% sections
\input{patterns/02_stack/07_layout}
\ifx\LITE\undefined
\input{patterns/02_stack/08_noise/main}
\fi
\ifdefined\IncludeExercises
\input{patterns/02_stack/exercises}
\fi

\clearpage
\section{\RU{Простейшее четырехбайтное XOR-шифрование}\EN{Simplest possible 4-byte XOR encryption}}

\RU{Если при XOR-шифровании применялся шаблон длинее байта, например, 4-байтный, то его также легко
увидеть.}
\EN{If longer pattern was used while XOR-encryption, for example, 4 byte pattern, it's easy
to spot it as well.}
\RU{Например, вот начало файла kernel32.dll (32-битная версия из Windows Server 2008):}
\EN{As example, here is beginning of kernel32.dll file (32-bit version from Windows Server 2008):}

\begin{figure}[H]
\centering
\includegraphics[scale=\FigScale]{ff/XOR/4byte/original1.png}
\caption{\EN{Original file}\RU{Оригинальный файл}}
\end{figure}

\clearpage
\RU{Вот он же, но \q{зашифрованный} 4-байтным ключем:}
\EN{Here is it \q{encrypted} by 4-byte key:}

\begin{figure}[H]
\centering
\includegraphics[scale=\FigScale]{ff/XOR/4byte/encrypted1.png}
\caption{\EN{\q{Encrypted} file}\RU{\q{Зашифрованный} файл}}
\end{figure}

\RU{Очень легко увидеть повторяющиеся 4 символа.}
\EN{It's very easy to spot recurring 4 symbols.}
\RU{Ведь в заголовке PE-файла много длинных нулевых областей, из-за которых ключ становится видным.}
\EN{Indeed, PE-file header has a lot of long zero lacunes, which is the reason why key became visible.}

\clearpage
\RU{Вот начало PE-заголовка в 16-ричном виде:}
\EN{Here is beginning of PE-header in hexadecimal form:}

\begin{figure}[H]
\centering
\includegraphics[scale=\FigScale]{ff/XOR/4byte/original2.png}
\caption{PE-\EN{header}\RU{заголовок}}
\end{figure}

\clearpage
\RU{И вот он же, \q{зашифрованный}:}
\EN{Here is it \q{encrypted}:}

\begin{figure}[H]
\centering
\includegraphics[scale=\FigScale]{ff/XOR/4byte/encrypted2.png}
\caption{\EN{\q{Encrypted} PE-header}\RU{\q{Зашифрованный} PE-заголовок}}
\end{figure}

\RU{Легко увидеть визуально, что ключ это следующие 4 байта}
\EN{It's easy to spot that key is the following 4 bytes}: \TT{8C 61 D2 63}.
\RU{Используя эту информацию, довольно легко расшифровать весь файл.}
\EN{It's easy to decrypt the whole file using this information.}

\RU{Таким образом, важно помнить эти свойства PE-файлов:
1) в PE-заголовке много нулевых областей;
2) все PE-секции дополняются нулями до границы страницы (4096 байт), 
так что после всех секций обычно имеются длинные нулевые области.}
\EN{So this is important to remember these property of PE-files:
1) PE-header has many zero lacunas;
2) all PE-sections padded with zeroes by page border (4096 bytes),
so long zero lacunas usually present after all sections.}

\RU{Некоторые другие форматы файлов могут также иметь длинные нулевые области.}
\EN{Some other file formats may contain long zero lacunas.}
\RU{Это очень типично для файлов, используемых научным и инженерным ПО.}
\EN{It's very typical for files used by scientific and engineering software.}

\RU{Для тех, кто самостоятельно хочет изучить эти файлы, то их можно скачать здесь:}
\EN{For those who wants to inspect these files on one's own, they are downloadable there:}
\url{http://go.yurichev.com/17352}.

\subsection{\Exercise}

\begin{itemize}
	\item \url{http://challenges.re/50}
\end{itemize}


\chapter{scanf()}
\index{\CStandardLibrary!scanf()}
\label{label_scanf}

\RU{Теперь попробуем использовать scanf().}\EN{Now let's use scanf().}

% sections
\input{patterns/04_scanf/1_simple/main}
\input{patterns/04_scanf/2_global/main}
\input{patterns/04_scanf/3_checking_retval/main}

\section{\Exercise}

\begin{itemize}
	\item \url{http://challenges.re/53}
\end{itemize}


\clearpage
\section{\RU{Простейшее четырехбайтное XOR-шифрование}\EN{Simplest possible 4-byte XOR encryption}}

\RU{Если при XOR-шифровании применялся шаблон длинее байта, например, 4-байтный, то его также легко
увидеть.}
\EN{If longer pattern was used while XOR-encryption, for example, 4 byte pattern, it's easy
to spot it as well.}
\RU{Например, вот начало файла kernel32.dll (32-битная версия из Windows Server 2008):}
\EN{As example, here is beginning of kernel32.dll file (32-bit version from Windows Server 2008):}

\begin{figure}[H]
\centering
\includegraphics[scale=\FigScale]{ff/XOR/4byte/original1.png}
\caption{\EN{Original file}\RU{Оригинальный файл}}
\end{figure}

\clearpage
\RU{Вот он же, но \q{зашифрованный} 4-байтным ключем:}
\EN{Here is it \q{encrypted} by 4-byte key:}

\begin{figure}[H]
\centering
\includegraphics[scale=\FigScale]{ff/XOR/4byte/encrypted1.png}
\caption{\EN{\q{Encrypted} file}\RU{\q{Зашифрованный} файл}}
\end{figure}

\RU{Очень легко увидеть повторяющиеся 4 символа.}
\EN{It's very easy to spot recurring 4 symbols.}
\RU{Ведь в заголовке PE-файла много длинных нулевых областей, из-за которых ключ становится видным.}
\EN{Indeed, PE-file header has a lot of long zero lacunes, which is the reason why key became visible.}

\clearpage
\RU{Вот начало PE-заголовка в 16-ричном виде:}
\EN{Here is beginning of PE-header in hexadecimal form:}

\begin{figure}[H]
\centering
\includegraphics[scale=\FigScale]{ff/XOR/4byte/original2.png}
\caption{PE-\EN{header}\RU{заголовок}}
\end{figure}

\clearpage
\RU{И вот он же, \q{зашифрованный}:}
\EN{Here is it \q{encrypted}:}

\begin{figure}[H]
\centering
\includegraphics[scale=\FigScale]{ff/XOR/4byte/encrypted2.png}
\caption{\EN{\q{Encrypted} PE-header}\RU{\q{Зашифрованный} PE-заголовок}}
\end{figure}

\RU{Легко увидеть визуально, что ключ это следующие 4 байта}
\EN{It's easy to spot that key is the following 4 bytes}: \TT{8C 61 D2 63}.
\RU{Используя эту информацию, довольно легко расшифровать весь файл.}
\EN{It's easy to decrypt the whole file using this information.}

\RU{Таким образом, важно помнить эти свойства PE-файлов:
1) в PE-заголовке много нулевых областей;
2) все PE-секции дополняются нулями до границы страницы (4096 байт), 
так что после всех секций обычно имеются длинные нулевые области.}
\EN{So this is important to remember these property of PE-files:
1) PE-header has many zero lacunas;
2) all PE-sections padded with zeroes by page border (4096 bytes),
so long zero lacunas usually present after all sections.}

\RU{Некоторые другие форматы файлов могут также иметь длинные нулевые области.}
\EN{Some other file formats may contain long zero lacunas.}
\RU{Это очень типично для файлов, используемых научным и инженерным ПО.}
\EN{It's very typical for files used by scientific and engineering software.}

\RU{Для тех, кто самостоятельно хочет изучить эти файлы, то их можно скачать здесь:}
\EN{For those who wants to inspect these files on one's own, they are downloadable there:}
\url{http://go.yurichev.com/17352}.

\subsection{\Exercise}

\begin{itemize}
	\item \url{http://challenges.re/50}
\end{itemize}


\chapter{\RU{Ещё о возвращаемых результатах}\EN{More about results returning}}

\index{x86!\Registers!EAX}
\RU{Результат выполнения функции в x86 обычно возвращается}%
\EN{In x86, the result of function execution is usually returned}%
\footnote{\Seealso: 
MSDN: Return Values (C++): \href{http://go.yurichev.com/17258}{MSDN}}
\RU{через регистр \EAX, 
а если результат имеет тип байт или символ (\Tchar), 
то в самой младшей части \EAX~--- \AL. Если функция возвращает число с плавающей запятой, 
то будет использован регистр FPU \ST{0}.
\ifdefined\IncludeARM
\index{ARM!\Registers!R0}
В ARM обычно результат возвращается в регистре \Reg{0}.
\fi
}
\EN{in the \EAX register. 
If it is byte type or a character (\Tchar), then the lowest part of register \EAX (\AL) is used. 
If a function returns a \Tfloat number, the FPU register \ST{0} is used instead.
\ifdefined\IncludeARM
\index{ARM!\Registers!R0}
In ARM, the result is usually returned in the \Reg{0} register.
\fi
}

\section{\RU{Попытка использовать результат функции возвращающей \Tvoid}
\EN{Attempt to use the result of a function returning \Tvoid}}

\RU{Кстати, что будет, если возвращаемое значение в функции \main объявлять не как \Tint, а как \Tvoid?}
\EN{So, what if the \main function return value was declared of type \Tvoid and not \Tint?}

\RU{Т.н. startup-код вызывает \main примерно так:}
\EN{The so-called startup-code is calling \main roughly as follows:}

\begin{lstlisting}
push envp
push argv
push argc
call main
push eax
call exit
\end{lstlisting}

\RU{Иными словами:}\EN{In other words:}

\begin{lstlisting}
exit(main(argc,argv,envp));
\end{lstlisting}

\RU{Если вы объявите \main как \Tvoid, и ничего не будете возвращать явно (при помощи выражения \IT{return}), 
то в единственный аргумент exit() попадет
то, что лежало в регистре \EAX на момент выхода из \main.}
\EN{If you declare \main as \Tvoid, nothing is to be returned explicitly 
(using the \IT{return} statement),
then something random, that was stored in the \EAX register at the end of \main becomes 
the sole argument of the exit() function.}
\RU{Там, скорее всего, будет какие-то случайное число, оставшееся от работы вашей функции.
Так что код завершения программы будет псевдослучайным.}
\EN{Most likely, there will be a random value, left from your function execution,
so the exit code of program is pseudorandom.}\PTBRph{}\ESph{}\PLph{}\ITAph{} \\

\RU{Мы можем это проиллюстрировать}\EN{We can illustrate this fact}. 
\RU{Заметьте, что у функции}\EN{Please note that here the} \main
\RU{тип возвращаемого значения именно}\EN{function has a} \Tvoid\EN{ return type}:

\begin{lstlisting}
#include <stdio.h>

void main()
{
	printf ("Hello, world!\n");
};
\end{lstlisting}

\RU{Скомпилируем в}\EN{Let's compile it in} Linux.

\index{puts() \RU{вместо}\EN{instead of} printf()}
GCC 4.8.1 \RU{заменила}\EN{replaced} \printf \RU{на}\EN{with} \puts 
\ifx\LITE\undefined
(\RU{мы видели это прежде}\EN{we have seen this before}: \myref{puts})
\fi
, \RU{но это нормально, потому что}\EN{but that's OK, since} \puts \RU{возвращает количество
выведенных символов, так же как и}\EN{returns the number of characters printed out, just like} \printf.
\RU{Обратите внимание на то, что}\EN{Please notice that} \EAX \RU{не обнуляется перед выходом их}\EN{is not 
zeroed before} \main\EN{'s end}.
\RU{Это значит что \EAX перед выходом из \main содержит то, что \puts оставляет там.}
\EN{This implies that the value of \EAX at the end of \main contains what \puts has left there.}

\begin{lstlisting}[caption=GCC 4.8.1]
.LC0:
	.string	"Hello, world!"
main:
	push	ebp
	mov	ebp, esp
	and	esp, -16
	sub	esp, 16
	mov	DWORD PTR [esp], OFFSET FLAT:.LC0
	call	puts
	leave
	ret
\end{lstlisting}

\index{bash}
\RU{Напишем небольшой скрипт на bash, показывающий статус возврата (\q{exit status} или \q{exit code})}
\EN{Let' s write a bash script that shows the exit status}:

\begin{lstlisting}[caption=tst.sh]
#!/bin/sh
./hello_world
echo $?
\end{lstlisting}

\RU{И запустим}\EN{And run it}:

\begin{lstlisting}
$ tst.sh 
Hello, world!
14
\end{lstlisting}

14 \RU{это как раз количество выведенных символов}\EN{is the number of characters printed}.

\section{\RU{Что если не использовать результат функции?}\EN{What if we do not use the function result?}}

\RU{\printf возвращает количество успешно выведенных символов, но результат работы этой функции 
редко используется на практике.}
\EN{\printf returns the count of characters successfully output, but the result of this function 
is rarely used in practice.}
\RU{Можно даже явно вызывать функции, чей смысл именно в возвращаемых значениях, но явно не использовать их:}
\EN{It is also possible to call a function whose essence is in returning a value, and not use it:}

\begin{lstlisting}
int f()
{
    // skip first 3 random values
    rand();
    rand();
    rand();
    // and use 4th
    return rand();
};
\end{lstlisting}

\EN{The result of the rand() function is left in \EAX, in all four cases.}
\RU{Результат работы rand() остается в \EAX во всех четырех случаях.}
\EN{But in the first 3 cases, the value in \EAX is just thrown away.}
\RU{Но в первых трех случаях значение, лежащее в \EAX, просто выбрасывается.}

\ifx\LITE\undefined
\section{\RU{Возврат структуры}\EN{Returning a structure}}

\index{\CLanguageElements!return}
\RU{Вернемся к тому факту, что возвращаемое значение остается в регистре \EAX.}
\EN{Let's go back to the fact that the return value is left in the \EAX register.}
\RU{Вот почему старые компиляторы Си не способны создавать функции, возвращающие нечто большее, нежели 
помещается 
в один регистр (обычно тип \Tint), а когда нужно, приходится возвращать через указатели, указываемые 
в аргументах.}
\EN{That is why old C compilers cannot create functions capable of returning something that does not fit in one 
register (usually \Tint), but if one needs it, one have to return information via pointers passed 
as function's arguments.}
\RU{Так что как правило, если функция должна вернуть несколько значений, она возвращает только одно, 
а остальные~--- через указатели.}
\EN{So, usually, if a function needs to return several values, it returns only one, and 
all the rest---via pointers.}
\RU{Хотя позже и стало возможным, вернуть, скажем, целую структуру, но этот метод до сих пор не 
очень популярен. 
Если функция должна вернуть структуру, вызывающая функция должна сама, скрыто и прозрачно для программиста, 
выделить место и передать указатель на него в качестве первого аргумента. Это почти то же самое 
что и сделать это вручную, но компилятор прячет это.}
\EN{Now it has become possible to return, let's say, an entire structure, but that is still not very popular. 
If a function has to return a large structure, the \gls{caller} must allocate it and pass a pointer to it via the first argument, transparently for the programmer. 
That is almost the same as to pass a pointer in the first argument manually, but the compiler hides it.}

\RU{Небольшой пример:}\EN{Small example:}

\lstinputlisting{patterns/06_return_results/6_1.c}

\dots \RU{получим}\EN{what we got} (MSVC 2010 \Ox):

\lstinputlisting{patterns/06_return_results/6_1.asm}

\RU{\TT{\$T3853} это имя внутреннего макроса для передачи указателя на структуру.}
\EN{The macro name for internal passing of pointer to a structure here is \TT{\$T3853}.}

\index{\CLanguageElements!C99}
\RU{Этот пример можно даже переписать, используя расширения C99}\EN{This example can be rewritten using
the C99 language extensions}:

\lstinputlisting{patterns/06_return_results/6_1_C99.c}

\lstinputlisting[caption=GCC 4.8.1]{patterns/06_return_results/6_1_C99.asm}

\RU{Как видно, функция просто заполняет поля в структуре, выделенной вызывающей функцией. 
Как если бы передавался просто указатель на структуру.
Так что никаких проблем с эффективностью нет.}
\EN{As we see, the function is just filling the structure's fields allocated by
the caller function,
as if a pointer to the structure was passed.
So there are no performance drawbacks.}
\fi

\ifx\LITE\undefined
\clearpage
\section{\RU{Простейшее четырехбайтное XOR-шифрование}\EN{Simplest possible 4-byte XOR encryption}}

\RU{Если при XOR-шифровании применялся шаблон длинее байта, например, 4-байтный, то его также легко
увидеть.}
\EN{If longer pattern was used while XOR-encryption, for example, 4 byte pattern, it's easy
to spot it as well.}
\RU{Например, вот начало файла kernel32.dll (32-битная версия из Windows Server 2008):}
\EN{As example, here is beginning of kernel32.dll file (32-bit version from Windows Server 2008):}

\begin{figure}[H]
\centering
\includegraphics[scale=\FigScale]{ff/XOR/4byte/original1.png}
\caption{\EN{Original file}\RU{Оригинальный файл}}
\end{figure}

\clearpage
\RU{Вот он же, но \q{зашифрованный} 4-байтным ключем:}
\EN{Here is it \q{encrypted} by 4-byte key:}

\begin{figure}[H]
\centering
\includegraphics[scale=\FigScale]{ff/XOR/4byte/encrypted1.png}
\caption{\EN{\q{Encrypted} file}\RU{\q{Зашифрованный} файл}}
\end{figure}

\RU{Очень легко увидеть повторяющиеся 4 символа.}
\EN{It's very easy to spot recurring 4 symbols.}
\RU{Ведь в заголовке PE-файла много длинных нулевых областей, из-за которых ключ становится видным.}
\EN{Indeed, PE-file header has a lot of long zero lacunes, which is the reason why key became visible.}

\clearpage
\RU{Вот начало PE-заголовка в 16-ричном виде:}
\EN{Here is beginning of PE-header in hexadecimal form:}

\begin{figure}[H]
\centering
\includegraphics[scale=\FigScale]{ff/XOR/4byte/original2.png}
\caption{PE-\EN{header}\RU{заголовок}}
\end{figure}

\clearpage
\RU{И вот он же, \q{зашифрованный}:}
\EN{Here is it \q{encrypted}:}

\begin{figure}[H]
\centering
\includegraphics[scale=\FigScale]{ff/XOR/4byte/encrypted2.png}
\caption{\EN{\q{Encrypted} PE-header}\RU{\q{Зашифрованный} PE-заголовок}}
\end{figure}

\RU{Легко увидеть визуально, что ключ это следующие 4 байта}
\EN{It's easy to spot that key is the following 4 bytes}: \TT{8C 61 D2 63}.
\RU{Используя эту информацию, довольно легко расшифровать весь файл.}
\EN{It's easy to decrypt the whole file using this information.}

\RU{Таким образом, важно помнить эти свойства PE-файлов:
1) в PE-заголовке много нулевых областей;
2) все PE-секции дополняются нулями до границы страницы (4096 байт), 
так что после всех секций обычно имеются длинные нулевые области.}
\EN{So this is important to remember these property of PE-files:
1) PE-header has many zero lacunas;
2) all PE-sections padded with zeroes by page border (4096 bytes),
so long zero lacunas usually present after all sections.}

\RU{Некоторые другие форматы файлов могут также иметь длинные нулевые области.}
\EN{Some other file formats may contain long zero lacunas.}
\RU{Это очень типично для файлов, используемых научным и инженерным ПО.}
\EN{It's very typical for files used by scientific and engineering software.}

\RU{Для тех, кто самостоятельно хочет изучить эти файлы, то их можно скачать здесь:}
\EN{For those who wants to inspect these files on one's own, they are downloadable there:}
\url{http://go.yurichev.com/17352}.

\subsection{\Exercise}

\begin{itemize}
	\item \url{http://challenges.re/50}
\end{itemize}


\fi
\chapter{\RU{Оператор GOTO}\EN{GOTO operator}}

\RU{Оператор GOTO считается анти-паттерном}\EN{The GOTO operator is generally considered as anti-pattern.} 
\cite{Dijkstra:1968:LEG:362929.362947}, 
\RU{но тем не менее, его можно использовать в разумных пределах}
\EN{Nevertheless, it can be used reasonably} \cite{Knuth:1974:SPG:356635.356640}, \cite[1.3.2]{CBook}.

\RU{Вот простейший пример}\EN{Here is a very simple example}:

\lstinputlisting{patterns/065_GOTO/goto.c}

\RU{Вот что мы получаем в}\EN{Here is what we have got in} MSVC 2012:

\lstinputlisting[caption=MSVC 2012]{patterns/065_GOTO/MSVC_goto.asm}

\RU{Выражение \IT{goto} заменяется инструкцией \JMP, которая работает точно также:
безусловный переход в другое место.}
\EN{The \IT{goto} statement has been simply replaced by a \JMP instruction, which has the same
effect: unconditional jump to another place.}

\RU{Вызов второго \printf может исполнится только при помощи человеческого вмешательства,
используя отладчик или модифицирование кода.}
\EN{The second \printf could be executed only with human intervention, 
by using a debugger or by patching the code.}
\PTBRph{}\ESph{}\PLph{}\ITAph{}\\
\\
\ifdefined\IncludeHiew
\clearpage
\RU{Это также может быть простым упражнением на модификацию кода.}
\EN{This could also be useful as a simple patching exercise.}
\RU{Откроем исполняемый файл в}\EN{Let's open the resulting executable in} Hiew:

\begin{figure}[H]
\centering
\includegraphics[scale=\FigScale]{patterns/065_GOTO/hiew1.png}
\caption{Hiew}
\label{fig:goto_hiew1}
\end{figure}

\clearpage
\RU{Поместите курсор по адресу}\EN{Place the cursor to address} \JMP (\TT{0x410}), 
\RU{нажмите}\EN{press} F3 (\RU{редактирование}\EN{edit}), \RU{нажмите два нуля, так что
опкод становится}\EN{press zero twice, so the opcode becomes} \TT{EB 00}:

\begin{figure}[H]
\centering
\includegraphics[scale=\FigScale]{patterns/065_GOTO/hiew2.png}
\caption{Hiew}
\label{fig:goto_hiew2}
\end{figure}

\RU{Второй байт опкода \JMP это относительное смещение от перехода. 0 означает место
прямо после текущей инструкции.}
\EN{The second byte of the \JMP opcode denotes the relative offset for the jump, 0 means the point
right after the current instruction.}
\RU{Теперь \JMP не будет пропускать следующий вызов \printf.}
\EN{So now \JMP not skipping the second \printf call.}

\RU{Нажмите F9 (запись) и выйдите.}
\EN{Press F9 (save) and exit.}
\RU{Теперь мы запускаем исполняемый файл и видим это}\EN{Now if we run the executable we should see 
this}:

\begin{figure}[H]
\centering
\includegraphics[scale=\NormalScale]{patterns/065_GOTO/result.png}
\caption{\RU{Результат}\EN{Patched executable output}}
\label{fig:goto_result}
\end{figure}

\RU{Подобного же эффекта можно достичь, если заменить инструкцию \JMP на две инструкции \NOP.}
\EN{The same result could be achieved by replacing the \JMP instruction with 2 \NOP instructions.}
\RU{\NOP имеет опкод \TT{0x90} и длину в 1 байт, так что нужно 2 инструкции для замены.}
\EN{\NOP has an opcode of \TT{0x90} and length of 1 byte, so we need 2 instructions as \JMP replacement (which is 2 bytes in size).}
\fi

\section{\RU{Мертвый код}\EN{Dead code}}

\RU{Вызов второго \printf также называется \q{мертвым кодом} (\q{dead code}) 
в терминах компиляторов.}
\EN{The second \printf call is also called \q{dead code} in compiler terms.}
\RU{Это значит, что он никогда не будет исполнен.}
\EN{This means that the code will never be executed.}
\EN{So when you compile this example with optimizations, the compiler removes \q{dead code}, leaving
no trace of it:}
\RU{Так что если вы компилируете этот пример с оптимизацией, компилятор удаляет \q{мертвый
код} не оставляя следа:}

\lstinputlisting[caption=\Optimizing MSVC 2012]{patterns/065_GOTO/MSVC_goto_Ox.asm}

\RU{Впрочем, строку}\EN{However, the compiler forgot to remove the} \q{skip me!} \RU{компилятор 
убрать забыл}\EN{string}.

%Note: cl "/Ox" option for maximum optimisation does get rid of "skip me" string as well

\ifdefined\IncludeExercises
\section{\Exercise}

% TODO debugger example can fit here
\RU{Попробуйте добиться того же самого в вашем любимом компиляторе и отладчике.}
\EN{Try to achieve the same result using your favorite compiler and debugger.}
\fi

\clearpage
\section{\RU{Простейшее четырехбайтное XOR-шифрование}\EN{Simplest possible 4-byte XOR encryption}}

\RU{Если при XOR-шифровании применялся шаблон длинее байта, например, 4-байтный, то его также легко
увидеть.}
\EN{If longer pattern was used while XOR-encryption, for example, 4 byte pattern, it's easy
to spot it as well.}
\RU{Например, вот начало файла kernel32.dll (32-битная версия из Windows Server 2008):}
\EN{As example, here is beginning of kernel32.dll file (32-bit version from Windows Server 2008):}

\begin{figure}[H]
\centering
\includegraphics[scale=\FigScale]{ff/XOR/4byte/original1.png}
\caption{\EN{Original file}\RU{Оригинальный файл}}
\end{figure}

\clearpage
\RU{Вот он же, но \q{зашифрованный} 4-байтным ключем:}
\EN{Here is it \q{encrypted} by 4-byte key:}

\begin{figure}[H]
\centering
\includegraphics[scale=\FigScale]{ff/XOR/4byte/encrypted1.png}
\caption{\EN{\q{Encrypted} file}\RU{\q{Зашифрованный} файл}}
\end{figure}

\RU{Очень легко увидеть повторяющиеся 4 символа.}
\EN{It's very easy to spot recurring 4 symbols.}
\RU{Ведь в заголовке PE-файла много длинных нулевых областей, из-за которых ключ становится видным.}
\EN{Indeed, PE-file header has a lot of long zero lacunes, which is the reason why key became visible.}

\clearpage
\RU{Вот начало PE-заголовка в 16-ричном виде:}
\EN{Here is beginning of PE-header in hexadecimal form:}

\begin{figure}[H]
\centering
\includegraphics[scale=\FigScale]{ff/XOR/4byte/original2.png}
\caption{PE-\EN{header}\RU{заголовок}}
\end{figure}

\clearpage
\RU{И вот он же, \q{зашифрованный}:}
\EN{Here is it \q{encrypted}:}

\begin{figure}[H]
\centering
\includegraphics[scale=\FigScale]{ff/XOR/4byte/encrypted2.png}
\caption{\EN{\q{Encrypted} PE-header}\RU{\q{Зашифрованный} PE-заголовок}}
\end{figure}

\RU{Легко увидеть визуально, что ключ это следующие 4 байта}
\EN{It's easy to spot that key is the following 4 bytes}: \TT{8C 61 D2 63}.
\RU{Используя эту информацию, довольно легко расшифровать весь файл.}
\EN{It's easy to decrypt the whole file using this information.}

\RU{Таким образом, важно помнить эти свойства PE-файлов:
1) в PE-заголовке много нулевых областей;
2) все PE-секции дополняются нулями до границы страницы (4096 байт), 
так что после всех секций обычно имеются длинные нулевые области.}
\EN{So this is important to remember these property of PE-files:
1) PE-header has many zero lacunas;
2) all PE-sections padded with zeroes by page border (4096 bytes),
so long zero lacunas usually present after all sections.}

\RU{Некоторые другие форматы файлов могут также иметь длинные нулевые области.}
\EN{Some other file formats may contain long zero lacunas.}
\RU{Это очень типично для файлов, используемых научным и инженерным ПО.}
\EN{It's very typical for files used by scientific and engineering software.}

\RU{Для тех, кто самостоятельно хочет изучить эти файлы, то их можно скачать здесь:}
\EN{For those who wants to inspect these files on one's own, they are downloadable there:}
\url{http://go.yurichev.com/17352}.

\subsection{\Exercise}

\begin{itemize}
	\item \url{http://challenges.re/50}
\end{itemize}


\chapter{\SwitchCaseDefaultSectionName}
\index{\CLanguageElements!switch}

% sections
\input{patterns/08_switch/1_few/main}
\input{patterns/08_switch/2_lot/main}
% TODO What's the difference between 3 and 4? Seems to be the same...
% it is fallthrough from 3 to 4 :) --DY
\input{patterns/08_switch/3_several_cases/main}
\input{patterns/08_switch/4_fallthrough/main}

\ifdefined\IncludeExercises
\section{\Exercises}

\subsection{\Exercise \#1}
\label{exercise_switch_1}

\RU{Вполне возможно переделать пример на Си в листинге \myref{switch_lot_c} так, чтобы при компиляции
получалось даже ещё меньше кода, но работать всё будет точно так же.}
\EN{It's possible to rework the C example in \myref{switch_lot_c} in such way that the compiler
can produce even smaller code, but will work just the same.}
\RU{Попробуйте этого добиться}\EN{Try to achieve it}.

% \RU{Подсказка}\EN{Hint}: \printf \EN{may be called only from a single place}\RU{вполне может 
% вызываться только из одного места}.
\fi

\clearpage
\section{\RU{Простейшее четырехбайтное XOR-шифрование}\EN{Simplest possible 4-byte XOR encryption}}

\RU{Если при XOR-шифровании применялся шаблон длинее байта, например, 4-байтный, то его также легко
увидеть.}
\EN{If longer pattern was used while XOR-encryption, for example, 4 byte pattern, it's easy
to spot it as well.}
\RU{Например, вот начало файла kernel32.dll (32-битная версия из Windows Server 2008):}
\EN{As example, here is beginning of kernel32.dll file (32-bit version from Windows Server 2008):}

\begin{figure}[H]
\centering
\includegraphics[scale=\FigScale]{ff/XOR/4byte/original1.png}
\caption{\EN{Original file}\RU{Оригинальный файл}}
\end{figure}

\clearpage
\RU{Вот он же, но \q{зашифрованный} 4-байтным ключем:}
\EN{Here is it \q{encrypted} by 4-byte key:}

\begin{figure}[H]
\centering
\includegraphics[scale=\FigScale]{ff/XOR/4byte/encrypted1.png}
\caption{\EN{\q{Encrypted} file}\RU{\q{Зашифрованный} файл}}
\end{figure}

\RU{Очень легко увидеть повторяющиеся 4 символа.}
\EN{It's very easy to spot recurring 4 symbols.}
\RU{Ведь в заголовке PE-файла много длинных нулевых областей, из-за которых ключ становится видным.}
\EN{Indeed, PE-file header has a lot of long zero lacunes, which is the reason why key became visible.}

\clearpage
\RU{Вот начало PE-заголовка в 16-ричном виде:}
\EN{Here is beginning of PE-header in hexadecimal form:}

\begin{figure}[H]
\centering
\includegraphics[scale=\FigScale]{ff/XOR/4byte/original2.png}
\caption{PE-\EN{header}\RU{заголовок}}
\end{figure}

\clearpage
\RU{И вот он же, \q{зашифрованный}:}
\EN{Here is it \q{encrypted}:}

\begin{figure}[H]
\centering
\includegraphics[scale=\FigScale]{ff/XOR/4byte/encrypted2.png}
\caption{\EN{\q{Encrypted} PE-header}\RU{\q{Зашифрованный} PE-заголовок}}
\end{figure}

\RU{Легко увидеть визуально, что ключ это следующие 4 байта}
\EN{It's easy to spot that key is the following 4 bytes}: \TT{8C 61 D2 63}.
\RU{Используя эту информацию, довольно легко расшифровать весь файл.}
\EN{It's easy to decrypt the whole file using this information.}

\RU{Таким образом, важно помнить эти свойства PE-файлов:
1) в PE-заголовке много нулевых областей;
2) все PE-секции дополняются нулями до границы страницы (4096 байт), 
так что после всех секций обычно имеются длинные нулевые области.}
\EN{So this is important to remember these property of PE-files:
1) PE-header has many zero lacunas;
2) all PE-sections padded with zeroes by page border (4096 bytes),
so long zero lacunas usually present after all sections.}

\RU{Некоторые другие форматы файлов могут также иметь длинные нулевые области.}
\EN{Some other file formats may contain long zero lacunas.}
\RU{Это очень типично для файлов, используемых научным и инженерным ПО.}
\EN{It's very typical for files used by scientific and engineering software.}

\RU{Для тех, кто самостоятельно хочет изучить эти файлы, то их можно скачать здесь:}
\EN{For those who wants to inspect these files on one's own, they are downloadable there:}
\url{http://go.yurichev.com/17352}.

\subsection{\Exercise}

\begin{itemize}
	\item \url{http://challenges.re/50}
\end{itemize}


\chapter{\SimpleStringsProcessings}
\index{\CStandardLibrary!strlen()}
\index{\CLanguageElements!while}

% sections
\input{patterns/10_strings/1_strlen/main}

\clearpage
\section{\RU{Простейшее четырехбайтное XOR-шифрование}\EN{Simplest possible 4-byte XOR encryption}}

\RU{Если при XOR-шифровании применялся шаблон длинее байта, например, 4-байтный, то его также легко
увидеть.}
\EN{If longer pattern was used while XOR-encryption, for example, 4 byte pattern, it's easy
to spot it as well.}
\RU{Например, вот начало файла kernel32.dll (32-битная версия из Windows Server 2008):}
\EN{As example, here is beginning of kernel32.dll file (32-bit version from Windows Server 2008):}

\begin{figure}[H]
\centering
\includegraphics[scale=\FigScale]{ff/XOR/4byte/original1.png}
\caption{\EN{Original file}\RU{Оригинальный файл}}
\end{figure}

\clearpage
\RU{Вот он же, но \q{зашифрованный} 4-байтным ключем:}
\EN{Here is it \q{encrypted} by 4-byte key:}

\begin{figure}[H]
\centering
\includegraphics[scale=\FigScale]{ff/XOR/4byte/encrypted1.png}
\caption{\EN{\q{Encrypted} file}\RU{\q{Зашифрованный} файл}}
\end{figure}

\RU{Очень легко увидеть повторяющиеся 4 символа.}
\EN{It's very easy to spot recurring 4 symbols.}
\RU{Ведь в заголовке PE-файла много длинных нулевых областей, из-за которых ключ становится видным.}
\EN{Indeed, PE-file header has a lot of long zero lacunes, which is the reason why key became visible.}

\clearpage
\RU{Вот начало PE-заголовка в 16-ричном виде:}
\EN{Here is beginning of PE-header in hexadecimal form:}

\begin{figure}[H]
\centering
\includegraphics[scale=\FigScale]{ff/XOR/4byte/original2.png}
\caption{PE-\EN{header}\RU{заголовок}}
\end{figure}

\clearpage
\RU{И вот он же, \q{зашифрованный}:}
\EN{Here is it \q{encrypted}:}

\begin{figure}[H]
\centering
\includegraphics[scale=\FigScale]{ff/XOR/4byte/encrypted2.png}
\caption{\EN{\q{Encrypted} PE-header}\RU{\q{Зашифрованный} PE-заголовок}}
\end{figure}

\RU{Легко увидеть визуально, что ключ это следующие 4 байта}
\EN{It's easy to spot that key is the following 4 bytes}: \TT{8C 61 D2 63}.
\RU{Используя эту информацию, довольно легко расшифровать весь файл.}
\EN{It's easy to decrypt the whole file using this information.}

\RU{Таким образом, важно помнить эти свойства PE-файлов:
1) в PE-заголовке много нулевых областей;
2) все PE-секции дополняются нулями до границы страницы (4096 байт), 
так что после всех секций обычно имеются длинные нулевые области.}
\EN{So this is important to remember these property of PE-files:
1) PE-header has many zero lacunas;
2) all PE-sections padded with zeroes by page border (4096 bytes),
so long zero lacunas usually present after all sections.}

\RU{Некоторые другие форматы файлов могут также иметь длинные нулевые области.}
\EN{Some other file formats may contain long zero lacunas.}
\RU{Это очень типично для файлов, используемых научным и инженерным ПО.}
\EN{It's very typical for files used by scientific and engineering software.}

\RU{Для тех, кто самостоятельно хочет изучить эти файлы, то их можно скачать здесь:}
\EN{For those who wants to inspect these files on one's own, they are downloadable there:}
\url{http://go.yurichev.com/17352}.

\subsection{\Exercise}

\begin{itemize}
	\item \url{http://challenges.re/50}
\end{itemize}


\ifx\LITE\undefined
\chapter{\FPUChapterName}
\label{sec:FPU}

\newcommand{\FNURLSTACK}{\footnote{\href{http://go.yurichev.com/17123}{wikipedia.org/wiki/Stack\_machine}}}
\newcommand{\FNURLFORTH}{\footnote{\href{http://go.yurichev.com/17124}{wikipedia.org/wiki/Forth\_(programming\_language)}}}
\newcommand{\FNURLIEEE}{\footnote{\href{http://go.yurichev.com/17125}{wikipedia.org/wiki/IEEE\_floating\_point}}}
\newcommand{\FNURLSP}{\footnote{\href{http://go.yurichev.com/17126}{wikipedia.org/wiki/Single-precision\_floating-point\_format}}}
\newcommand{\FNURLDP}{\footnote{\href{http://go.yurichev.com/17127}{wikipedia.org/wiki/Double-precision\_floating-point\_format}}}
\newcommand{\FNURLEP}{\footnote{\href{http://go.yurichev.com/17128}{wikipedia.org/wiki/Extended\_precision}}}

\RU{\ac{FPU}\EMDASH блок в процессоре работающий с числами с плавающей запятой.}
\EN{The \ac{FPU} is a device within the main \ac{CPU}, specially designed to deal with floating point numbers.}
\RU{Раньше он назывался \q{сопроцессором} и он стоит немного в стороне от \ac{CPU}.}
\EN{It was called \q{coprocessor} in the past and it stays somewhat aside of the main \ac{CPU}.}

\section{IEEE 754}

\RU{Число с плавающей точкой в формате IEEE 754 состоит из \IT{знака}, \IT{мантиссы}\footnote{\IT{significand} или \IT{fraction} 
в англоязычной литературе} и \IT{экспоненты}.}
\EN{A number in the IEEE 754 format consists of a \IT{sign}, a \IT{significand} (also called \IT{fraction}) and an \IT{exponent}.}

\section{x86}

\RU{Перед изучением \ac{FPU} в x86 полезно ознакомиться с тем как работают стековые машины\FNURLSTACK 
или ознакомиться с основами языка Forth\FNURLFORTH.}
\EN{It is worth looking into stack machines\FNURLSTACK or learning the basics of the Forth language\FNURLFORTH,
before studying the \ac{FPU} in x86.}

\index{Intel!80486}
\index{Intel!FPU}
\RU{Интересен факт, что в свое время (до 80486) сопроцессор был отдельным чипом на материнской плате, 
и вследствие его высокой цены, он не всегда присутствовал. Его можно было докупить и установить отдельно}%
\EN{It is interesting to know that in the past (before the 80486 CPU) the coprocessor was a separate chip 
and it was not always pre-installed on the motherboard. It was possible to buy it separately and install it}%
\footnote{\RU{Например, Джон Кармак использовал в своей игре Doom числа с фиксированной запятой 
(\href{http://go.yurichev.com/17357}{ru.wikipedia.org/wiki/Число\_с\_фиксированной\_запятой}), хранящиеся
в обычных 32-битных \ac{GPR} (16 бит на целую часть и 16 на дробную),
чтобы Doom работал на 32-битных компьютерах без FPU, т.е. 80386 и 80486 SX.}
\EN{For example, John Carmack used fixed-point arithmetic 
(\href{http://go.yurichev.com/17356}{wikipedia.org/wiki/Fixed-point\_arithmetic}) values in his Doom video game, stored in 
32-bit \ac{GPR} registers (16 bit for integral part and another 16 bit for fractional part), so Doom
could work on 32-bit computers without FPU, i.e., 80386 and 80486 SX.}}.
\RU{Начиная с 80486 DX в состав процессора всегда входит FPU.}
\EN{Starting with the 80486 DX CPU, the \ac{FPU} is integrated in the \ac{CPU}.}

\index{x86!\Instructions!FWAIT}
\RU{Этот факт может напоминать такой рудимент как наличие инструкции \TT{FWAIT}, 
которая заставляет
\ac{CPU} ожидать, пока \ac{FPU} закончит работу}\EN{The \TT{FWAIT} instruction reminds us of that fact---it
switches the \ac{CPU} to a waiting state, so it can wait until the \ac{FPU} is done with its work}.
\RU{Другой рудимент это тот факт, что опкоды \ac{FPU}-инструкций начинаются с т.н. \q{escape}-опкодов 
(\TT{D8..DF}) как опкоды, передающиеся в отдельный сопроцессор.}
\EN{Another rudiment is the fact that the \ac{FPU} instruction 
opcodes start with the so called \q{escape}-opcodes (\TT{D8..DF}), i.e., 
opcodes passed to a separate coprocessor.}

\index{IEEE 754}
\label{FPU_is_stack}
\RU{FPU имеет стек из восьми 80-битных регистров:}
\EN{The FPU has a stack capable to holding 8 80-bit registers, and each register can hold a number 
in the IEEE 754\FNURLIEEE format.}
\RU{\ST{0}..\ST{7}. Для краткости, IDA и \olly отображают \ST{0} как \TT{ST},
что в некоторых учебниках и документациях означает \q{Stack Top} (\q{вершина стека}).}
\RU{Каждый регистр может содержать число в формате IEEE 754\FNURLIEEE.}
\EN{They are \ST{0}..\ST{7}. For brevity, IDA and \olly show \ST{0} as \TT{ST}, 
which is represented in some textbooks and manuals as \q{Stack Top}.}

\section{ARM, MIPS, x86/x64 SIMD}

\RU{В ARM и MIPS FPU это не стек, а просто набор регистров.}
\EN{In ARM and MIPS the FPU is not a stack, but a set of registers.}
\RU{Такая же идеология применяется в расширениях SIMD в процессорах x86/x64.}
\EN{The same ideology is used in the SIMD extensions of x86/x64 CPUs.}

\section{\CCpp}

\index{float}
\index{double}
\RU{В стандартных \CCpp имеются два типа для работы с числами с плавающей запятой: 
\Tfloat (\IT{число одинарной точности}\FNURLSP, 32 бита)
\footnote{Формат представления чисел с плавающей точкой одинарной точности затрагивается в разделе 
\IT{\WorkingWithFloatAsWithStructSubSubSectionName}~(\myref{sec:floatasstruct}).}
и \Tdouble (\IT{число двойной точности}\FNURLDP, 64 бита).}
\EN{The standard \CCpp languages offer at least two floating number types, \Tfloat (\IT{single-precision}\FNURLSP, 32 bits)
\footnote{the single precision floating point number format is also addressed in 
the \IT{\WorkingWithFloatAsWithStructSubSubSectionName}~(\myref{sec:floatasstruct}) section}
and \Tdouble (\IT{double-precision}\FNURLDP, 64 bits).}

\index{long double}
\RU{GCC также поддерживает тип \IT{long double} (\IT{extended precision}\FNURLEP, 80 бит), но MSVC~--- нет.}
\EN{GCC also supports the \IT{long double} type (\IT{extended precision}\FNURLEP, 80 bit), which MSVC doesn't.}

\RU{Несмотря на то, что \Tfloat занимает столько же места, сколько и \Tint на 32-битной архитектуре, 
представление чисел, разумеется, совершенно другое.}
\EN{The \Tfloat type requires the same number of bits as the \Tint type in 32-bit environments, 
but the number representation is completely different.}

\input{patterns/12_FPU/1_simple/main}
\input{patterns/12_FPU/2_passing_floats/main}
\input{patterns/12_FPU/3_comparison/main}

\section{\RU{Стек, калькуляторы и обратная польская запись}\EN{Stack, calculators and reverse Polish notation}}

\index{\RU{Обратная польская запись}\EN{Reverse Polish notation}}
\RU{Теперь понятно, почему некоторые старые калькуляторы использовали обратную польскую запись%
\footnote{\href{http://go.yurichev.com/17355}{ru.wikipedia.org/wiki/Обратная\_польская\_запись}}.}
\EN{Now we undestand why some old calculators used reverse Polish notation
\footnote{\href{http://go.yurichev.com/17354}{wikipedia.org/wiki/Reverse\_Polish\_notation}}.}
\RU{Например для сложения 12 и 34 нужно было набрать 12, потом 34, потом нажать знак \q{плюс}.}
\EN{For example, for addition of 12 and 34 one has to enter 12, then 34, then press \q{plus} sign.}
\RU{Это потому что старые калькуляторы просто реализовали стековую машину и это было куда проще, 
чем обрабатывать сложные выражения со скобками.}
\EN{It's because old calculators were just stack machine implementations, and this was much simpler
than to handle complex parenthesized expressions.}
\section{x64}

\RU{О том, как происходит работа с числами с плавающей запятой в x86-64, читайте здесь: \myref{floating_SIMD}.}
\EN{On how floating point numbers are processed in x86-64, read more here: \myref{floating_SIMD}.}

% sections
\ifdefined\IncludeExercises
\input{patterns/12_FPU/exercises}
\fi

\fi
\clearpage
\section{\RU{Простейшее четырехбайтное XOR-шифрование}\EN{Simplest possible 4-byte XOR encryption}}

\RU{Если при XOR-шифровании применялся шаблон длинее байта, например, 4-байтный, то его также легко
увидеть.}
\EN{If longer pattern was used while XOR-encryption, for example, 4 byte pattern, it's easy
to spot it as well.}
\RU{Например, вот начало файла kernel32.dll (32-битная версия из Windows Server 2008):}
\EN{As example, here is beginning of kernel32.dll file (32-bit version from Windows Server 2008):}

\begin{figure}[H]
\centering
\includegraphics[scale=\FigScale]{ff/XOR/4byte/original1.png}
\caption{\EN{Original file}\RU{Оригинальный файл}}
\end{figure}

\clearpage
\RU{Вот он же, но \q{зашифрованный} 4-байтным ключем:}
\EN{Here is it \q{encrypted} by 4-byte key:}

\begin{figure}[H]
\centering
\includegraphics[scale=\FigScale]{ff/XOR/4byte/encrypted1.png}
\caption{\EN{\q{Encrypted} file}\RU{\q{Зашифрованный} файл}}
\end{figure}

\RU{Очень легко увидеть повторяющиеся 4 символа.}
\EN{It's very easy to spot recurring 4 symbols.}
\RU{Ведь в заголовке PE-файла много длинных нулевых областей, из-за которых ключ становится видным.}
\EN{Indeed, PE-file header has a lot of long zero lacunes, which is the reason why key became visible.}

\clearpage
\RU{Вот начало PE-заголовка в 16-ричном виде:}
\EN{Here is beginning of PE-header in hexadecimal form:}

\begin{figure}[H]
\centering
\includegraphics[scale=\FigScale]{ff/XOR/4byte/original2.png}
\caption{PE-\EN{header}\RU{заголовок}}
\end{figure}

\clearpage
\RU{И вот он же, \q{зашифрованный}:}
\EN{Here is it \q{encrypted}:}

\begin{figure}[H]
\centering
\includegraphics[scale=\FigScale]{ff/XOR/4byte/encrypted2.png}
\caption{\EN{\q{Encrypted} PE-header}\RU{\q{Зашифрованный} PE-заголовок}}
\end{figure}

\RU{Легко увидеть визуально, что ключ это следующие 4 байта}
\EN{It's easy to spot that key is the following 4 bytes}: \TT{8C 61 D2 63}.
\RU{Используя эту информацию, довольно легко расшифровать весь файл.}
\EN{It's easy to decrypt the whole file using this information.}

\RU{Таким образом, важно помнить эти свойства PE-файлов:
1) в PE-заголовке много нулевых областей;
2) все PE-секции дополняются нулями до границы страницы (4096 байт), 
так что после всех секций обычно имеются длинные нулевые области.}
\EN{So this is important to remember these property of PE-files:
1) PE-header has many zero lacunas;
2) all PE-sections padded with zeroes by page border (4096 bytes),
so long zero lacunas usually present after all sections.}

\RU{Некоторые другие форматы файлов могут также иметь длинные нулевые области.}
\EN{Some other file formats may contain long zero lacunas.}
\RU{Это очень типично для файлов, используемых научным и инженерным ПО.}
\EN{It's very typical for files used by scientific and engineering software.}

\RU{Для тех, кто самостоятельно хочет изучить эти файлы, то их можно скачать здесь:}
\EN{For those who wants to inspect these files on one's own, they are downloadable there:}
\url{http://go.yurichev.com/17352}.

\subsection{\Exercise}

\begin{itemize}
	\item \url{http://challenges.re/50}
\end{itemize}


\clearpage
\section{\RU{Простейшее четырехбайтное XOR-шифрование}\EN{Simplest possible 4-byte XOR encryption}}

\RU{Если при XOR-шифровании применялся шаблон длинее байта, например, 4-байтный, то его также легко
увидеть.}
\EN{If longer pattern was used while XOR-encryption, for example, 4 byte pattern, it's easy
to spot it as well.}
\RU{Например, вот начало файла kernel32.dll (32-битная версия из Windows Server 2008):}
\EN{As example, here is beginning of kernel32.dll file (32-bit version from Windows Server 2008):}

\begin{figure}[H]
\centering
\includegraphics[scale=\FigScale]{ff/XOR/4byte/original1.png}
\caption{\EN{Original file}\RU{Оригинальный файл}}
\end{figure}

\clearpage
\RU{Вот он же, но \q{зашифрованный} 4-байтным ключем:}
\EN{Here is it \q{encrypted} by 4-byte key:}

\begin{figure}[H]
\centering
\includegraphics[scale=\FigScale]{ff/XOR/4byte/encrypted1.png}
\caption{\EN{\q{Encrypted} file}\RU{\q{Зашифрованный} файл}}
\end{figure}

\RU{Очень легко увидеть повторяющиеся 4 символа.}
\EN{It's very easy to spot recurring 4 symbols.}
\RU{Ведь в заголовке PE-файла много длинных нулевых областей, из-за которых ключ становится видным.}
\EN{Indeed, PE-file header has a lot of long zero lacunes, which is the reason why key became visible.}

\clearpage
\RU{Вот начало PE-заголовка в 16-ричном виде:}
\EN{Here is beginning of PE-header in hexadecimal form:}

\begin{figure}[H]
\centering
\includegraphics[scale=\FigScale]{ff/XOR/4byte/original2.png}
\caption{PE-\EN{header}\RU{заголовок}}
\end{figure}

\clearpage
\RU{И вот он же, \q{зашифрованный}:}
\EN{Here is it \q{encrypted}:}

\begin{figure}[H]
\centering
\includegraphics[scale=\FigScale]{ff/XOR/4byte/encrypted2.png}
\caption{\EN{\q{Encrypted} PE-header}\RU{\q{Зашифрованный} PE-заголовок}}
\end{figure}

\RU{Легко увидеть визуально, что ключ это следующие 4 байта}
\EN{It's easy to spot that key is the following 4 bytes}: \TT{8C 61 D2 63}.
\RU{Используя эту информацию, довольно легко расшифровать весь файл.}
\EN{It's easy to decrypt the whole file using this information.}

\RU{Таким образом, важно помнить эти свойства PE-файлов:
1) в PE-заголовке много нулевых областей;
2) все PE-секции дополняются нулями до границы страницы (4096 байт), 
так что после всех секций обычно имеются длинные нулевые области.}
\EN{So this is important to remember these property of PE-files:
1) PE-header has many zero lacunas;
2) all PE-sections padded with zeroes by page border (4096 bytes),
so long zero lacunas usually present after all sections.}

\RU{Некоторые другие форматы файлов могут также иметь длинные нулевые области.}
\EN{Some other file formats may contain long zero lacunas.}
\RU{Это очень типично для файлов, используемых научным и инженерным ПО.}
\EN{It's very typical for files used by scientific and engineering software.}

\RU{Для тех, кто самостоятельно хочет изучить эти файлы, то их можно скачать здесь:}
\EN{For those who wants to inspect these files on one's own, they are downloadable there:}
\url{http://go.yurichev.com/17352}.

\subsection{\Exercise}

\begin{itemize}
	\item \url{http://challenges.re/50}
\end{itemize}


\chapter[\RU{Линейный конгруэнтный генератор}\EN{Linear congruential generator}]
{\RU{Линейный конгруэнтный генератор как генератор псевдослучайных чисел}\EN{Linear congruential generator as pseudorandom number generator}}
\index{\CStandardLibrary!rand()}
\label{LCG_simple}

\RU{Линейный конгруэнтный генератор, пожалуй, самый простой способ генерировать псевдослучайные числа.}
\EN{The linear congruential generator is probably the simplest possible way to generate random numbers.}
\RU{Он не в почете в наше время\footnote{Вихрь Мерсенна куда лучше}, но он настолько прост
(только одно умножение, одно сложение и одна операция \q{И}),
что мы можем использовать его в качестве примера.}
\EN{It's not in favour in modern times\footnote{Mersenne twister is better}, but it's so simple 
(just one multiplication, one addition and one AND operation), 
we can use it as an example.}

\lstinputlisting{patterns/145_LCG/rand.c.\LANG}

\RU{Здесь две функции: одна используется для инициализации внутреннего состояния, а вторая
вызывается собственно для генерации псевдослучайных чисел.}
\EN{There are two functions: the first one is used to initialize the internal state, and the second one is called
to generate pseudorandom numbers.}

\RU{Мы видим что в алгоритме применяются две константы}\EN{We see that two constants are used in the algorithm}.
\RU{Они взяты из}\EN{They are taken from} \cite{Numerical}.
\RU{Определим их используя выражение \CCpp \TT{\#define}. Это макрос.}
\EN{Let's define them using a \TT{\#define} \CCpp statement. It's a macro.}
\RU{Разница между макросом в \CCpp и константой в том, что все макросы заменяются на значения препроцессором
\CCpp и они не занимают места в памяти как переменные.}
\EN{The difference between a \CCpp macro and a constant is that all macros are replaced 
with their value by \CCpp preprocessor,
and they don't take any memory, unlike variables.}
\RU{А константы, напротив, это переменные только для чтения.}
\EN{In contrast, a constant is a read-only variable.}
\RU{Можно взять указатель (или адрес) переменной-константы, но это невозможно сделать с макросом.}
\EN{It's possible to take a pointer (or address) of a constant variable, but impossible to do so with a macro.}

\RU{Последняя операция \q{И} нужна, потому что согласно стандарту Си \TT{my\_rand()} должна возвращать значение в пределах
0..32767.}
\EN{The last AND operation is needed because by C-standard \TT{my\_rand()} has to return a value in 
the 0..32767 range.}
\RU{Если вы хотите получать 32-битные псевдослучайные значения, просто уберите последнюю операцию \q{И}.}
\EN{If you want to get 32-bit pseudorandom values, just omit the last AND operation.}

\section{x86}

\lstinputlisting[caption=\Optimizing MSVC 2013]{patterns/145_LCG/rand_MSVC_2013_x86_Ox.asm}

\RU{Вот мы это и видим: обе константы встроены в код.}
\EN{Here we see it: both constants are embedded into the code.}
\RU{Память для них не выделяется.}\EN{There is no memory allocated for them.}
\RU{Функция \TT{my\_srand()} просто копирует входное значение во внутреннюю переменную \TT{rand\_state}.}
\EN{The \TT{my\_srand()} function just copies its input value into the internal \TT{rand\_state} variable.}

\RU{\TT{my\_rand()} берет её, вычисляет следующее состояние \TT{rand\_state}, 
обрезает его и оставляет в регистре EAX.}
\EN{\TT{my\_rand()} takes it, calculates the next \TT{rand\_state}, cuts it and leaves it in the EAX register.}

\RU{Неоптимизированная версия побольше}\EN{The non-optimized version is more verbose}:

\lstinputlisting[caption=\NonOptimizing MSVC 2013]{patterns/145_LCG/rand_MSVC_2013_x86.asm}

\section{x64}

\RU{Версия для x64 почти такая же, и использует 32-битные регистры вместо 64-битных
(потому что мы работаем здесь с переменными типа \Tint).}
\EN{The x64 version is mostly the same and uses 32-bit registers instead of 64-bit ones 
(because we are working with \Tint values here).}
\RU{Но функция \TT{my\_srand()} берет входной аргумент из регистра \ECX, а не из стека:}
\EN{But \TT{my\_srand()} takes its input argument from the \ECX register rather than from stack:}

\lstinputlisting[caption=\Optimizing MSVC 2013 x64]{patterns/145_LCG/rand_MSVC_2013_x64_Ox.asm.\LANG}

\ifdefined\IncludeGCC
\RU{GCC делает почти такой же код}\EN{GCC compiler generates mostly the same code}.
\fi

\ifdefined\IncludeARM
\section{32-bit ARM}

\lstinputlisting[caption=\OptimizingKeilVI (\ARMMode)]{patterns/145_LCG/rand.s_Keil_ARM_O3.s.\LANG}

\RU{В ARM инструкцию невозможно встроить 32-битную константу, так что Keil-у приходится размещать
их отдельно и дополнительно загружать.}
\EN{It's not possible to embed 32-bit constants into ARM instructions, so Keil has to place them externally
and load them additionally.}

\RU{Вот еще что интересно: константу 0x7FFF также нельзя встроить.}
\EN{One interesting thing is that it's not possible to embed the 0x7FFF constant as well.}
\RU{Поэтому Keil сдвигает \TT{rand\_state} влево на 17 бит и затем сдвигает вправо на 17 бит.}
\EN{So what Keil does is shifting \TT{rand\_state} left by 17 bits and then shifting it right by 17 bits.}
\RU{Это аналогично \CCpp{}-выражению $(rand\_state \ll 17) \gg 17$.}
\EN{This is analogous to the $(rand\_state \ll 17) \gg 17$ statement in \CCpp.}
\RU{Выглядит как бессмысленная операция, но тем не менее, что она делает это очищает старшие 17 бит, оставляя
младшие 15 бит нетронутыми, и это наша цель, в конце концов.}
\EN{It seems to be useless operation, but
what it does is clearing the high 17 bits, leaving the low 15 bits intact, and that's our goal after all.}
\ESph{}\PTBRph{}\PLph{}\ITAph{}\\
\\
\Optimizing Keil \RU{для режима Thumb делает почти такой же код}\EN{for Thumb mode generates mostly the same code}.
\fi

\ifdefined\IncludeMIPS
\input{patterns/145_LCG/MIPS}
\fi

\ifx\LITE\undefined
\section{\RU{Версия этого примера для многопоточной среды}\EN{Thread-safe version of the example}}

\RU{Версия примера для многопоточной среды будет рассмотрена позже}%
\EN{The thread-safe version of the example is to be demonstrated later}: \myref{LCG_TLS}.
\fi

\clearpage
\section{\RU{Простейшее четырехбайтное XOR-шифрование}\EN{Simplest possible 4-byte XOR encryption}}

\RU{Если при XOR-шифровании применялся шаблон длинее байта, например, 4-байтный, то его также легко
увидеть.}
\EN{If longer pattern was used while XOR-encryption, for example, 4 byte pattern, it's easy
to spot it as well.}
\RU{Например, вот начало файла kernel32.dll (32-битная версия из Windows Server 2008):}
\EN{As example, here is beginning of kernel32.dll file (32-bit version from Windows Server 2008):}

\begin{figure}[H]
\centering
\includegraphics[scale=\FigScale]{ff/XOR/4byte/original1.png}
\caption{\EN{Original file}\RU{Оригинальный файл}}
\end{figure}

\clearpage
\RU{Вот он же, но \q{зашифрованный} 4-байтным ключем:}
\EN{Here is it \q{encrypted} by 4-byte key:}

\begin{figure}[H]
\centering
\includegraphics[scale=\FigScale]{ff/XOR/4byte/encrypted1.png}
\caption{\EN{\q{Encrypted} file}\RU{\q{Зашифрованный} файл}}
\end{figure}

\RU{Очень легко увидеть повторяющиеся 4 символа.}
\EN{It's very easy to spot recurring 4 symbols.}
\RU{Ведь в заголовке PE-файла много длинных нулевых областей, из-за которых ключ становится видным.}
\EN{Indeed, PE-file header has a lot of long zero lacunes, which is the reason why key became visible.}

\clearpage
\RU{Вот начало PE-заголовка в 16-ричном виде:}
\EN{Here is beginning of PE-header in hexadecimal form:}

\begin{figure}[H]
\centering
\includegraphics[scale=\FigScale]{ff/XOR/4byte/original2.png}
\caption{PE-\EN{header}\RU{заголовок}}
\end{figure}

\clearpage
\RU{И вот он же, \q{зашифрованный}:}
\EN{Here is it \q{encrypted}:}

\begin{figure}[H]
\centering
\includegraphics[scale=\FigScale]{ff/XOR/4byte/encrypted2.png}
\caption{\EN{\q{Encrypted} PE-header}\RU{\q{Зашифрованный} PE-заголовок}}
\end{figure}

\RU{Легко увидеть визуально, что ключ это следующие 4 байта}
\EN{It's easy to spot that key is the following 4 bytes}: \TT{8C 61 D2 63}.
\RU{Используя эту информацию, довольно легко расшифровать весь файл.}
\EN{It's easy to decrypt the whole file using this information.}

\RU{Таким образом, важно помнить эти свойства PE-файлов:
1) в PE-заголовке много нулевых областей;
2) все PE-секции дополняются нулями до границы страницы (4096 байт), 
так что после всех секций обычно имеются длинные нулевые области.}
\EN{So this is important to remember these property of PE-files:
1) PE-header has many zero lacunas;
2) all PE-sections padded with zeroes by page border (4096 bytes),
so long zero lacunas usually present after all sections.}

\RU{Некоторые другие форматы файлов могут также иметь длинные нулевые области.}
\EN{Some other file formats may contain long zero lacunas.}
\RU{Это очень типично для файлов, используемых научным и инженерным ПО.}
\EN{It's very typical for files used by scientific and engineering software.}

\RU{Для тех, кто самостоятельно хочет изучить эти файлы, то их можно скачать здесь:}
\EN{For those who wants to inspect these files on one's own, they are downloadable there:}
\url{http://go.yurichev.com/17352}.

\subsection{\Exercise}

\begin{itemize}
	\item \url{http://challenges.re/50}
\end{itemize}


\ifx\LITE\undefined
\chapter{\RU{Объединения (union)}\EN{Unions}}

\EN{\CCpp \IT{union} is mostly used for interpreting a variable (or memory block) of one data type as a variable of another data type.}
\RU{\IT{union} в \CCpp используется в основном для интерпертации переменной (или блока памяти) одного типа как переменной другого типа.}

% sections
\input{patterns/17_unions/FPU_PRNG/main}
\input{patterns/17_unions/epsilon/main}

\section{\RU{Быстрое вычисление квадратного корня}\EN{Fast square root calculation}}

\RU{Вот где еще можно на практике применить трактовку типа \Tfloat как целочисленного, это быстрое вычисление квадратного корня.}%
\EN{Another well-known algorithm where \Tfloat is interpreted as integer is fast calculation of square root.}

\begin{lstlisting}[caption=\EN{The source code is taken from Wikipedia}\RU{Исходный код взят из Wikipedia}: \url{http://go.yurichev.com/17364}]
/* Assumes that float is in the IEEE 754 single precision floating point format
 * and that int is 32 bits. */
float sqrt_approx(float z)
{
    int val_int = *(int*)&z; /* Same bits, but as an int */
    /*
     * To justify the following code, prove that
     *
     * ((((val_int / 2^m) - b) / 2) + b) * 2^m = ((val_int - 2^m) / 2) + ((b + 1) / 2) * 2^m)
     *
     * where
     *
     * b = exponent bias
     * m = number of mantissa bits
     *
     * .
     */
 
    val_int -= 1 << 23; /* Subtract 2^m. */
    val_int >>= 1; /* Divide by 2. */
    val_int += 1 << 29; /* Add ((b + 1) / 2) * 2^m. */
 
    return *(float*)&val_int; /* Interpret again as float */
}
\end{lstlisting}

\RU{В качестве упражнения, вы можете попробовать скомпилировать эту функцию и разобраться, как она работает.}
\EN{As an exercise, you can try to compile this function and to understand, how it works.}\ESph{}\PTBRph{}\PLph{}\ITAph{}\\
\\
\RU{Имеется также известный алгоритм быстрого вычисления}\EN{There is also well-known algorithm of fast calculation of} $\frac{1}{\sqrt{x}}$.
\index{Quake III Arena}
\RU{Алгоритм стал известным, вероятно потому, что был применен в Quake III Arena.}%
\EN{Algorithm became popular, supposedly, because it was used in Quake III Arena.}

\RU{Описание алгоритма есть в}\EN{Algorithm description is present in} Wikipedia:
\EN{\url{http://go.yurichev.com/17360}}\RU{\url{http://go.yurichev.com/17361}}.


\clearpage
\section{\RU{Простейшее четырехбайтное XOR-шифрование}\EN{Simplest possible 4-byte XOR encryption}}

\RU{Если при XOR-шифровании применялся шаблон длинее байта, например, 4-байтный, то его также легко
увидеть.}
\EN{If longer pattern was used while XOR-encryption, for example, 4 byte pattern, it's easy
to spot it as well.}
\RU{Например, вот начало файла kernel32.dll (32-битная версия из Windows Server 2008):}
\EN{As example, here is beginning of kernel32.dll file (32-bit version from Windows Server 2008):}

\begin{figure}[H]
\centering
\includegraphics[scale=\FigScale]{ff/XOR/4byte/original1.png}
\caption{\EN{Original file}\RU{Оригинальный файл}}
\end{figure}

\clearpage
\RU{Вот он же, но \q{зашифрованный} 4-байтным ключем:}
\EN{Here is it \q{encrypted} by 4-byte key:}

\begin{figure}[H]
\centering
\includegraphics[scale=\FigScale]{ff/XOR/4byte/encrypted1.png}
\caption{\EN{\q{Encrypted} file}\RU{\q{Зашифрованный} файл}}
\end{figure}

\RU{Очень легко увидеть повторяющиеся 4 символа.}
\EN{It's very easy to spot recurring 4 symbols.}
\RU{Ведь в заголовке PE-файла много длинных нулевых областей, из-за которых ключ становится видным.}
\EN{Indeed, PE-file header has a lot of long zero lacunes, which is the reason why key became visible.}

\clearpage
\RU{Вот начало PE-заголовка в 16-ричном виде:}
\EN{Here is beginning of PE-header in hexadecimal form:}

\begin{figure}[H]
\centering
\includegraphics[scale=\FigScale]{ff/XOR/4byte/original2.png}
\caption{PE-\EN{header}\RU{заголовок}}
\end{figure}

\clearpage
\RU{И вот он же, \q{зашифрованный}:}
\EN{Here is it \q{encrypted}:}

\begin{figure}[H]
\centering
\includegraphics[scale=\FigScale]{ff/XOR/4byte/encrypted2.png}
\caption{\EN{\q{Encrypted} PE-header}\RU{\q{Зашифрованный} PE-заголовок}}
\end{figure}

\RU{Легко увидеть визуально, что ключ это следующие 4 байта}
\EN{It's easy to spot that key is the following 4 bytes}: \TT{8C 61 D2 63}.
\RU{Используя эту информацию, довольно легко расшифровать весь файл.}
\EN{It's easy to decrypt the whole file using this information.}

\RU{Таким образом, важно помнить эти свойства PE-файлов:
1) в PE-заголовке много нулевых областей;
2) все PE-секции дополняются нулями до границы страницы (4096 байт), 
так что после всех секций обычно имеются длинные нулевые области.}
\EN{So this is important to remember these property of PE-files:
1) PE-header has many zero lacunas;
2) all PE-sections padded with zeroes by page border (4096 bytes),
so long zero lacunas usually present after all sections.}

\RU{Некоторые другие форматы файлов могут также иметь длинные нулевые области.}
\EN{Some other file formats may contain long zero lacunas.}
\RU{Это очень типично для файлов, используемых научным и инженерным ПО.}
\EN{It's very typical for files used by scientific and engineering software.}

\RU{Для тех, кто самостоятельно хочет изучить эти файлы, то их можно скачать здесь:}
\EN{For those who wants to inspect these files on one's own, they are downloadable there:}
\url{http://go.yurichev.com/17352}.

\subsection{\Exercise}

\begin{itemize}
	\item \url{http://challenges.re/50}
\end{itemize}


\fi
\clearpage
\section{\RU{Простейшее четырехбайтное XOR-шифрование}\EN{Simplest possible 4-byte XOR encryption}}

\RU{Если при XOR-шифровании применялся шаблон длинее байта, например, 4-байтный, то его также легко
увидеть.}
\EN{If longer pattern was used while XOR-encryption, for example, 4 byte pattern, it's easy
to spot it as well.}
\RU{Например, вот начало файла kernel32.dll (32-битная версия из Windows Server 2008):}
\EN{As example, here is beginning of kernel32.dll file (32-bit version from Windows Server 2008):}

\begin{figure}[H]
\centering
\includegraphics[scale=\FigScale]{ff/XOR/4byte/original1.png}
\caption{\EN{Original file}\RU{Оригинальный файл}}
\end{figure}

\clearpage
\RU{Вот он же, но \q{зашифрованный} 4-байтным ключем:}
\EN{Here is it \q{encrypted} by 4-byte key:}

\begin{figure}[H]
\centering
\includegraphics[scale=\FigScale]{ff/XOR/4byte/encrypted1.png}
\caption{\EN{\q{Encrypted} file}\RU{\q{Зашифрованный} файл}}
\end{figure}

\RU{Очень легко увидеть повторяющиеся 4 символа.}
\EN{It's very easy to spot recurring 4 symbols.}
\RU{Ведь в заголовке PE-файла много длинных нулевых областей, из-за которых ключ становится видным.}
\EN{Indeed, PE-file header has a lot of long zero lacunes, which is the reason why key became visible.}

\clearpage
\RU{Вот начало PE-заголовка в 16-ричном виде:}
\EN{Here is beginning of PE-header in hexadecimal form:}

\begin{figure}[H]
\centering
\includegraphics[scale=\FigScale]{ff/XOR/4byte/original2.png}
\caption{PE-\EN{header}\RU{заголовок}}
\end{figure}

\clearpage
\RU{И вот он же, \q{зашифрованный}:}
\EN{Here is it \q{encrypted}:}

\begin{figure}[H]
\centering
\includegraphics[scale=\FigScale]{ff/XOR/4byte/encrypted2.png}
\caption{\EN{\q{Encrypted} PE-header}\RU{\q{Зашифрованный} PE-заголовок}}
\end{figure}

\RU{Легко увидеть визуально, что ключ это следующие 4 байта}
\EN{It's easy to spot that key is the following 4 bytes}: \TT{8C 61 D2 63}.
\RU{Используя эту информацию, довольно легко расшифровать весь файл.}
\EN{It's easy to decrypt the whole file using this information.}

\RU{Таким образом, важно помнить эти свойства PE-файлов:
1) в PE-заголовке много нулевых областей;
2) все PE-секции дополняются нулями до границы страницы (4096 байт), 
так что после всех секций обычно имеются длинные нулевые области.}
\EN{So this is important to remember these property of PE-files:
1) PE-header has many zero lacunas;
2) all PE-sections padded with zeroes by page border (4096 bytes),
so long zero lacunas usually present after all sections.}

\RU{Некоторые другие форматы файлов могут также иметь длинные нулевые области.}
\EN{Some other file formats may contain long zero lacunas.}
\RU{Это очень типично для файлов, используемых научным и инженерным ПО.}
\EN{It's very typical for files used by scientific and engineering software.}

\RU{Для тех, кто самостоятельно хочет изучить эти файлы, то их можно скачать здесь:}
\EN{For those who wants to inspect these files on one's own, they are downloadable there:}
\url{http://go.yurichev.com/17352}.

\subsection{\Exercise}

\begin{itemize}
	\item \url{http://challenges.re/50}
\end{itemize}


\ifx\LITE\undefined
\clearpage
\section{\RU{Простейшее четырехбайтное XOR-шифрование}\EN{Simplest possible 4-byte XOR encryption}}

\RU{Если при XOR-шифровании применялся шаблон длинее байта, например, 4-байтный, то его также легко
увидеть.}
\EN{If longer pattern was used while XOR-encryption, for example, 4 byte pattern, it's easy
to spot it as well.}
\RU{Например, вот начало файла kernel32.dll (32-битная версия из Windows Server 2008):}
\EN{As example, here is beginning of kernel32.dll file (32-bit version from Windows Server 2008):}

\begin{figure}[H]
\centering
\includegraphics[scale=\FigScale]{ff/XOR/4byte/original1.png}
\caption{\EN{Original file}\RU{Оригинальный файл}}
\end{figure}

\clearpage
\RU{Вот он же, но \q{зашифрованный} 4-байтным ключем:}
\EN{Here is it \q{encrypted} by 4-byte key:}

\begin{figure}[H]
\centering
\includegraphics[scale=\FigScale]{ff/XOR/4byte/encrypted1.png}
\caption{\EN{\q{Encrypted} file}\RU{\q{Зашифрованный} файл}}
\end{figure}

\RU{Очень легко увидеть повторяющиеся 4 символа.}
\EN{It's very easy to spot recurring 4 symbols.}
\RU{Ведь в заголовке PE-файла много длинных нулевых областей, из-за которых ключ становится видным.}
\EN{Indeed, PE-file header has a lot of long zero lacunes, which is the reason why key became visible.}

\clearpage
\RU{Вот начало PE-заголовка в 16-ричном виде:}
\EN{Here is beginning of PE-header in hexadecimal form:}

\begin{figure}[H]
\centering
\includegraphics[scale=\FigScale]{ff/XOR/4byte/original2.png}
\caption{PE-\EN{header}\RU{заголовок}}
\end{figure}

\clearpage
\RU{И вот он же, \q{зашифрованный}:}
\EN{Here is it \q{encrypted}:}

\begin{figure}[H]
\centering
\includegraphics[scale=\FigScale]{ff/XOR/4byte/encrypted2.png}
\caption{\EN{\q{Encrypted} PE-header}\RU{\q{Зашифрованный} PE-заголовок}}
\end{figure}

\RU{Легко увидеть визуально, что ключ это следующие 4 байта}
\EN{It's easy to spot that key is the following 4 bytes}: \TT{8C 61 D2 63}.
\RU{Используя эту информацию, довольно легко расшифровать весь файл.}
\EN{It's easy to decrypt the whole file using this information.}

\RU{Таким образом, важно помнить эти свойства PE-файлов:
1) в PE-заголовке много нулевых областей;
2) все PE-секции дополняются нулями до границы страницы (4096 байт), 
так что после всех секций обычно имеются длинные нулевые области.}
\EN{So this is important to remember these property of PE-files:
1) PE-header has many zero lacunas;
2) all PE-sections padded with zeroes by page border (4096 bytes),
so long zero lacunas usually present after all sections.}

\RU{Некоторые другие форматы файлов могут также иметь длинные нулевые области.}
\EN{Some other file formats may contain long zero lacunas.}
\RU{Это очень типично для файлов, используемых научным и инженерным ПО.}
\EN{It's very typical for files used by scientific and engineering software.}

\RU{Для тех, кто самостоятельно хочет изучить эти файлы, то их можно скачать здесь:}
\EN{For those who wants to inspect these files on one's own, they are downloadable there:}
\url{http://go.yurichev.com/17352}.

\subsection{\Exercise}

\begin{itemize}
	\item \url{http://challenges.re/50}
\end{itemize}


\fi
\chapter{\RU{64 бита}\EN{64 bits}}

\section{x86-64}
\index{x86-64}
\label{x86-64}

\RU{Это расширение x86-архитуктуры до 64 бит.}\EN{It is a 64-bit extension to the x86 architecture.}

\RU{С точки зрения начинающего reverse engineer-а, наиболее важные отличия от 32-битного x86 это:}
\EN{From the reverse engineer's perspective, the most important changes are:}

\index{\CLanguageElements!\Pointers}
\begin{itemize}

\item
\RU{Почти все регистры (кроме FPU и SIMD) расширены до 64-бит и получили префикс R-. 
И еще 8 регистров добавлено. 
В итоге имеются эти \ac{GPR}-ы:}
\EN{Almost all registers (except FPU and SIMD) were extended to 64 bits and got a R- prefix.
8 additional registers wer added.
Now \ac{GPR}'s are:} \RAX, \RBX, \RCX, \RDX, 
\RBP, \RSP, \RSI, \RDI, \Reg{8}, \Reg{9}, \Reg{10}, 
\Reg{11}, \Reg{12}, \Reg{13}, \Reg{14}, \Reg{15}. 

\RU{К ним также можно обращаться так же, как и прежде. Например, для доступа к младшим 32 битам \TT{RAX} 
можно использовать \EAX:}
\EN{It is still possible to access the \IT{older} register parts as usual. 
For example, it is possible to access the lower 32-bit part of the \TT{RAX} register using \EAX:}

\RegTableOne{RAX}{EAX}{AX}{AH}{AL}

\RU{У новых регистров \TT{R8-R15} также имеются их \IT{младшие части}: \TT{R8D-R15D} 
(младшие 32-битные части), 
\TT{R8W-R15W} (младшие 16-битные части), \TT{R8L-R15L} (младшие 8-битные части).}
\EN{The new \TT{R8-R15} registers also have their \IT{lower parts}: \TT{R8D-R15D} (lower 32-bit parts),
\TT{R8W-R15W} (lower 16-bit parts), \TT{R8L-R15L} (lower 8-bit parts).}

\RegTableFour{R8}{R8D}{R8W}{R8L}

\RU{Удвоено количество SIMD-регистров: с 8 до 16:}
\EN{The number of SIMD registers was doubled from 8 to 16:} \XMM{0}-\XMM{15}.

\item
\RU{В win64 передача всех параметров немного иная, это немного похоже на fastcall 
\ifx\LITE\undefined
(\myref{fastcall})
\fi
.
Первые 4 аргумента записываются в регистры \RCX, \RDX, \Reg{8}, \Reg{9}, а остальные ~--- в стек. 
Вызывающая функция также должна подготовить место из 32 байт чтобы вызываемая функция могла сохранить 
там первые 4 аргумента и использовать эти регистры по своему усмотрению. 
Короткие функции могут использовать аргументы прямо из регистров, но б\'{о}льшие функции могут сохранять 
их значения на будущее.}
\EN{In Win64, the function calling convention is slightly different, somewhat resembling fastcall
\ifx\LITE\undefined
(\myref{fastcall})
\fi
.
The first 4 arguments are stored in the \RCX, \RDX, \Reg{8}, \Reg{9} registers, the rest~---in the stack.
The \gls{caller} function must also allocate 32 bytes so the \gls{callee} may save there 4 first arguments and use these 
registers for its own needs.
Short functions may use arguments just from registers, but larger ones may save their values on the stack.}

\RU{Соглашение }System V AMD64 ABI (Linux, *BSD, \MacOSX)\cite{SysVABI} \RU{также напоминает}\EN{also somewhat resembles}
fastcall, \RU{использует 6 регистров}\EN{it uses 6 registers} 
\RDI, \RSI, \RDX, \RCX, \Reg{8}, \Reg{9} \RU{для первых шести аргументов}\EN{for the first 6 arguments}.
\RU{Остальные передаются через стек}\EN{All the rest are passed via the stack}.

\ifx\LITE\undefined
\RU{См. также в соответствующем разделе о способах передачи аргументов через стек}
\EN{See also the section on calling conventions}~(\myref{sec:callingconventions}).
\fi

\item
\RU{\Tint в \CCpp остается 32-битным для совместимости.}
\EN{The \CCpp \Tint type is still 32-bit for compatibility.}

\item
\RU{Все указатели теперь 64-битные}\EN{All pointers are 64-bit now}.

\RU{На это иногда сетуют: ведь теперь для хранения всех указателей нужно в 2 раза больше места 
в памяти, в т.ч. и в кэш-памяти, не смотря на то что x64-процессоры могут адресовать только 48 бит
внешней \ac{RAM}.}
\EN{This provokes irritation sometimes: now one needs twice as much memory for storing pointers,
including cache memory, despite the fact that x64 \ac{CPU}s can address only 48 bits of external 
\ac{RAM}.}

\end{itemize}

\index{Register allocation}
\RU{Из-за того, что регистров общего пользования теперь вдвое больше, у компиляторов теперь больше 
свободного места для маневра, называемого \glslink{register allocator}{register allocation}.
Для нас это означает, что в итоговом коде будет меньше локальных переменных.}
\EN{Since now the number of registers is doubled, the compilers have more space for maneuvering called 
\glslink{register allocator}{register allocation}.
For us this implies that the emitted code containing less number of local variables.}

\ifx\LITE\undefined
\index{DES}
\RU{Для примера, функция вычисляющая первый S-бокс алгоритма шифрования DES, 
она обрабатывает сразу 32/64/128/256 значений, в зависимости от типа \TT{DES\_type} (uint32, uint64, SSE2 или AVX), 
методом bitslice DES (больше об этом методе читайте здесь~(\myref{bitslicedes})):}
\EN{For example, the function that calculates the first S-box of the DES encryption algorithm processes
32/64/128/256 values at once (depending on \TT{DES\_type} type (uint32, uint64, SSE2 or AVX)) 
using the bitslice DES method
(read more about this technique here ~(\myref{bitslicedes})):}

\lstinputlisting{patterns/20_x64/19_1.c}

\RU{Здесь много локальных переменных. Конечно, далеко не все они будут в локальном стеке. 
Компилируем обычным MSVC 2008 с опцией \Ox:}
\EN{There are a lot of local variables. 
Of course, not all those going into the local stack.
Let's compile it with MSVC 2008 with \Ox option:}

\lstinputlisting[caption=\Optimizing MSVC 2008]{patterns/20_x64/19_2_msvc_Ox.asm}

\RU{5 переменных компилятору пришлось разместить в локальном стеке.}
\EN{5 variables were allocated in the local stack by the compiler.}

\RU{Теперь попробуем то же самое только в 64-битной версии MSVC 2008:}
\EN{Now let's try the same thing in the 64-bit version of MSVC 2008:}

\lstinputlisting[caption=\Optimizing MSVC 2008]{patterns/20_x64/19_3_msvc_x64.asm}

\RU{Компилятор ничего не выделил в локальном стеке, а \TT{x36} это синоним для \TT{a5}.}
\EN{Nothing was allocated in the local stack by the compiler, \TT{x36} is synonym for \TT{a5}.}
\fi

\iffalse
% FIXME1 невнятно
\RU{Кстати, видно, что функция сохраняет регистры \RCX, \RDX в отведенных для 
этого вызываемой функцией местах, 
а \Reg{8} и \Reg{9} не сохраняет, а начинает использовать их сразу.}
\EN{By the way, we can see here that the function saved the \RCX and \RDX registers in space allocated by the \gls{caller},
but \Reg{8} and \Reg{9} were not saved but used from the beginning.}
\fi

\RU{Кстати, существуют процессоры с еще большим количеством \ac{GPR}, например, 
Itanium ~--- 128 регистров.}
\EN{By the way, there are CPUs with much more \ac{GPR}'s, e.g. Itanium (128 registers).}

\ifdefined\IncludeARM
\section{ARM}

\RU{64-битные инструкции появились в}\EN{64-bit instructions appeared in} ARMv8.
\fi

\ifx\LITE\undefined
\section{\RU{Числа с плавающей запятой}\EN{Float point numbers}}

\RU{О том как происходит работа с числами с плавающей запятой в x86-64, читайте здесь: \myref{floating_SIMD}.}
\EN{How floating point numbers are processed in x86-64 is explained here: \myref{floating_SIMD}.}
\fi

\ifx\LITE\undefined
\clearpage
\section{\RU{Простейшее четырехбайтное XOR-шифрование}\EN{Simplest possible 4-byte XOR encryption}}

\RU{Если при XOR-шифровании применялся шаблон длинее байта, например, 4-байтный, то его также легко
увидеть.}
\EN{If longer pattern was used while XOR-encryption, for example, 4 byte pattern, it's easy
to spot it as well.}
\RU{Например, вот начало файла kernel32.dll (32-битная версия из Windows Server 2008):}
\EN{As example, here is beginning of kernel32.dll file (32-bit version from Windows Server 2008):}

\begin{figure}[H]
\centering
\includegraphics[scale=\FigScale]{ff/XOR/4byte/original1.png}
\caption{\EN{Original file}\RU{Оригинальный файл}}
\end{figure}

\clearpage
\RU{Вот он же, но \q{зашифрованный} 4-байтным ключем:}
\EN{Here is it \q{encrypted} by 4-byte key:}

\begin{figure}[H]
\centering
\includegraphics[scale=\FigScale]{ff/XOR/4byte/encrypted1.png}
\caption{\EN{\q{Encrypted} file}\RU{\q{Зашифрованный} файл}}
\end{figure}

\RU{Очень легко увидеть повторяющиеся 4 символа.}
\EN{It's very easy to spot recurring 4 symbols.}
\RU{Ведь в заголовке PE-файла много длинных нулевых областей, из-за которых ключ становится видным.}
\EN{Indeed, PE-file header has a lot of long zero lacunes, which is the reason why key became visible.}

\clearpage
\RU{Вот начало PE-заголовка в 16-ричном виде:}
\EN{Here is beginning of PE-header in hexadecimal form:}

\begin{figure}[H]
\centering
\includegraphics[scale=\FigScale]{ff/XOR/4byte/original2.png}
\caption{PE-\EN{header}\RU{заголовок}}
\end{figure}

\clearpage
\RU{И вот он же, \q{зашифрованный}:}
\EN{Here is it \q{encrypted}:}

\begin{figure}[H]
\centering
\includegraphics[scale=\FigScale]{ff/XOR/4byte/encrypted2.png}
\caption{\EN{\q{Encrypted} PE-header}\RU{\q{Зашифрованный} PE-заголовок}}
\end{figure}

\RU{Легко увидеть визуально, что ключ это следующие 4 байта}
\EN{It's easy to spot that key is the following 4 bytes}: \TT{8C 61 D2 63}.
\RU{Используя эту информацию, довольно легко расшифровать весь файл.}
\EN{It's easy to decrypt the whole file using this information.}

\RU{Таким образом, важно помнить эти свойства PE-файлов:
1) в PE-заголовке много нулевых областей;
2) все PE-секции дополняются нулями до границы страницы (4096 байт), 
так что после всех секций обычно имеются длинные нулевые области.}
\EN{So this is important to remember these property of PE-files:
1) PE-header has many zero lacunas;
2) all PE-sections padded with zeroes by page border (4096 bytes),
so long zero lacunas usually present after all sections.}

\RU{Некоторые другие форматы файлов могут также иметь длинные нулевые области.}
\EN{Some other file formats may contain long zero lacunas.}
\RU{Это очень типично для файлов, используемых научным и инженерным ПО.}
\EN{It's very typical for files used by scientific and engineering software.}

\RU{Для тех, кто самостоятельно хочет изучить эти файлы, то их можно скачать здесь:}
\EN{For those who wants to inspect these files on one's own, they are downloadable there:}
\url{http://go.yurichev.com/17352}.

\subsection{\Exercise}

\begin{itemize}
	\item \url{http://challenges.re/50}
\end{itemize}


\fi
\ifdefined\IncludeARM
\clearpage
\section{\RU{Простейшее четырехбайтное XOR-шифрование}\EN{Simplest possible 4-byte XOR encryption}}

\RU{Если при XOR-шифровании применялся шаблон длинее байта, например, 4-байтный, то его также легко
увидеть.}
\EN{If longer pattern was used while XOR-encryption, for example, 4 byte pattern, it's easy
to spot it as well.}
\RU{Например, вот начало файла kernel32.dll (32-битная версия из Windows Server 2008):}
\EN{As example, here is beginning of kernel32.dll file (32-bit version from Windows Server 2008):}

\begin{figure}[H]
\centering
\includegraphics[scale=\FigScale]{ff/XOR/4byte/original1.png}
\caption{\EN{Original file}\RU{Оригинальный файл}}
\end{figure}

\clearpage
\RU{Вот он же, но \q{зашифрованный} 4-байтным ключем:}
\EN{Here is it \q{encrypted} by 4-byte key:}

\begin{figure}[H]
\centering
\includegraphics[scale=\FigScale]{ff/XOR/4byte/encrypted1.png}
\caption{\EN{\q{Encrypted} file}\RU{\q{Зашифрованный} файл}}
\end{figure}

\RU{Очень легко увидеть повторяющиеся 4 символа.}
\EN{It's very easy to spot recurring 4 symbols.}
\RU{Ведь в заголовке PE-файла много длинных нулевых областей, из-за которых ключ становится видным.}
\EN{Indeed, PE-file header has a lot of long zero lacunes, which is the reason why key became visible.}

\clearpage
\RU{Вот начало PE-заголовка в 16-ричном виде:}
\EN{Here is beginning of PE-header in hexadecimal form:}

\begin{figure}[H]
\centering
\includegraphics[scale=\FigScale]{ff/XOR/4byte/original2.png}
\caption{PE-\EN{header}\RU{заголовок}}
\end{figure}

\clearpage
\RU{И вот он же, \q{зашифрованный}:}
\EN{Here is it \q{encrypted}:}

\begin{figure}[H]
\centering
\includegraphics[scale=\FigScale]{ff/XOR/4byte/encrypted2.png}
\caption{\EN{\q{Encrypted} PE-header}\RU{\q{Зашифрованный} PE-заголовок}}
\end{figure}

\RU{Легко увидеть визуально, что ключ это следующие 4 байта}
\EN{It's easy to spot that key is the following 4 bytes}: \TT{8C 61 D2 63}.
\RU{Используя эту информацию, довольно легко расшифровать весь файл.}
\EN{It's easy to decrypt the whole file using this information.}

\RU{Таким образом, важно помнить эти свойства PE-файлов:
1) в PE-заголовке много нулевых областей;
2) все PE-секции дополняются нулями до границы страницы (4096 байт), 
так что после всех секций обычно имеются длинные нулевые области.}
\EN{So this is important to remember these property of PE-files:
1) PE-header has many zero lacunas;
2) all PE-sections padded with zeroes by page border (4096 bytes),
so long zero lacunas usually present after all sections.}

\RU{Некоторые другие форматы файлов могут также иметь длинные нулевые области.}
\EN{Some other file formats may contain long zero lacunas.}
\RU{Это очень типично для файлов, используемых научным и инженерным ПО.}
\EN{It's very typical for files used by scientific and engineering software.}

\RU{Для тех, кто самостоятельно хочет изучить эти файлы, то их можно скачать здесь:}
\EN{For those who wants to inspect these files on one's own, they are downloadable there:}
\url{http://go.yurichev.com/17352}.

\subsection{\Exercise}

\begin{itemize}
	\item \url{http://challenges.re/50}
\end{itemize}


\fi
\ifdefined\IncludeMIPS
\clearpage
\section{\RU{Простейшее четырехбайтное XOR-шифрование}\EN{Simplest possible 4-byte XOR encryption}}

\RU{Если при XOR-шифровании применялся шаблон длинее байта, например, 4-байтный, то его также легко
увидеть.}
\EN{If longer pattern was used while XOR-encryption, for example, 4 byte pattern, it's easy
to spot it as well.}
\RU{Например, вот начало файла kernel32.dll (32-битная версия из Windows Server 2008):}
\EN{As example, here is beginning of kernel32.dll file (32-bit version from Windows Server 2008):}

\begin{figure}[H]
\centering
\includegraphics[scale=\FigScale]{ff/XOR/4byte/original1.png}
\caption{\EN{Original file}\RU{Оригинальный файл}}
\end{figure}

\clearpage
\RU{Вот он же, но \q{зашифрованный} 4-байтным ключем:}
\EN{Here is it \q{encrypted} by 4-byte key:}

\begin{figure}[H]
\centering
\includegraphics[scale=\FigScale]{ff/XOR/4byte/encrypted1.png}
\caption{\EN{\q{Encrypted} file}\RU{\q{Зашифрованный} файл}}
\end{figure}

\RU{Очень легко увидеть повторяющиеся 4 символа.}
\EN{It's very easy to spot recurring 4 symbols.}
\RU{Ведь в заголовке PE-файла много длинных нулевых областей, из-за которых ключ становится видным.}
\EN{Indeed, PE-file header has a lot of long zero lacunes, which is the reason why key became visible.}

\clearpage
\RU{Вот начало PE-заголовка в 16-ричном виде:}
\EN{Here is beginning of PE-header in hexadecimal form:}

\begin{figure}[H]
\centering
\includegraphics[scale=\FigScale]{ff/XOR/4byte/original2.png}
\caption{PE-\EN{header}\RU{заголовок}}
\end{figure}

\clearpage
\RU{И вот он же, \q{зашифрованный}:}
\EN{Here is it \q{encrypted}:}

\begin{figure}[H]
\centering
\includegraphics[scale=\FigScale]{ff/XOR/4byte/encrypted2.png}
\caption{\EN{\q{Encrypted} PE-header}\RU{\q{Зашифрованный} PE-заголовок}}
\end{figure}

\RU{Легко увидеть визуально, что ключ это следующие 4 байта}
\EN{It's easy to spot that key is the following 4 bytes}: \TT{8C 61 D2 63}.
\RU{Используя эту информацию, довольно легко расшифровать весь файл.}
\EN{It's easy to decrypt the whole file using this information.}

\RU{Таким образом, важно помнить эти свойства PE-файлов:
1) в PE-заголовке много нулевых областей;
2) все PE-секции дополняются нулями до границы страницы (4096 байт), 
так что после всех секций обычно имеются длинные нулевые области.}
\EN{So this is important to remember these property of PE-files:
1) PE-header has many zero lacunas;
2) all PE-sections padded with zeroes by page border (4096 bytes),
so long zero lacunas usually present after all sections.}

\RU{Некоторые другие форматы файлов могут также иметь длинные нулевые области.}
\EN{Some other file formats may contain long zero lacunas.}
\RU{Это очень типично для файлов, используемых научным и инженерным ПО.}
\EN{It's very typical for files used by scientific and engineering software.}

\RU{Для тех, кто самостоятельно хочет изучить эти файлы, то их можно скачать здесь:}
\EN{For those who wants to inspect these files on one's own, they are downloadable there:}
\url{http://go.yurichev.com/17352}.

\subsection{\Exercise}

\begin{itemize}
	\item \url{http://challenges.re/50}
\end{itemize}


\fi

\clearpage
\section{\RU{Простейшее четырехбайтное XOR-шифрование}\EN{Simplest possible 4-byte XOR encryption}}

\RU{Если при XOR-шифровании применялся шаблон длинее байта, например, 4-байтный, то его также легко
увидеть.}
\EN{If longer pattern was used while XOR-encryption, for example, 4 byte pattern, it's easy
to spot it as well.}
\RU{Например, вот начало файла kernel32.dll (32-битная версия из Windows Server 2008):}
\EN{As example, here is beginning of kernel32.dll file (32-bit version from Windows Server 2008):}

\begin{figure}[H]
\centering
\includegraphics[scale=\FigScale]{ff/XOR/4byte/original1.png}
\caption{\EN{Original file}\RU{Оригинальный файл}}
\end{figure}

\clearpage
\RU{Вот он же, но \q{зашифрованный} 4-байтным ключем:}
\EN{Here is it \q{encrypted} by 4-byte key:}

\begin{figure}[H]
\centering
\includegraphics[scale=\FigScale]{ff/XOR/4byte/encrypted1.png}
\caption{\EN{\q{Encrypted} file}\RU{\q{Зашифрованный} файл}}
\end{figure}

\RU{Очень легко увидеть повторяющиеся 4 символа.}
\EN{It's very easy to spot recurring 4 symbols.}
\RU{Ведь в заголовке PE-файла много длинных нулевых областей, из-за которых ключ становится видным.}
\EN{Indeed, PE-file header has a lot of long zero lacunes, which is the reason why key became visible.}

\clearpage
\RU{Вот начало PE-заголовка в 16-ричном виде:}
\EN{Here is beginning of PE-header in hexadecimal form:}

\begin{figure}[H]
\centering
\includegraphics[scale=\FigScale]{ff/XOR/4byte/original2.png}
\caption{PE-\EN{header}\RU{заголовок}}
\end{figure}

\clearpage
\RU{И вот он же, \q{зашифрованный}:}
\EN{Here is it \q{encrypted}:}

\begin{figure}[H]
\centering
\includegraphics[scale=\FigScale]{ff/XOR/4byte/encrypted2.png}
\caption{\EN{\q{Encrypted} PE-header}\RU{\q{Зашифрованный} PE-заголовок}}
\end{figure}

\RU{Легко увидеть визуально, что ключ это следующие 4 байта}
\EN{It's easy to spot that key is the following 4 bytes}: \TT{8C 61 D2 63}.
\RU{Используя эту информацию, довольно легко расшифровать весь файл.}
\EN{It's easy to decrypt the whole file using this information.}

\RU{Таким образом, важно помнить эти свойства PE-файлов:
1) в PE-заголовке много нулевых областей;
2) все PE-секции дополняются нулями до границы страницы (4096 байт), 
так что после всех секций обычно имеются длинные нулевые области.}
\EN{So this is important to remember these property of PE-files:
1) PE-header has many zero lacunas;
2) all PE-sections padded with zeroes by page border (4096 bytes),
so long zero lacunas usually present after all sections.}

\RU{Некоторые другие форматы файлов могут также иметь длинные нулевые области.}
\EN{Some other file formats may contain long zero lacunas.}
\RU{Это очень типично для файлов, используемых научным и инженерным ПО.}
\EN{It's very typical for files used by scientific and engineering software.}

\RU{Для тех, кто самостоятельно хочет изучить эти файлы, то их можно скачать здесь:}
\EN{For those who wants to inspect these files on one's own, they are downloadable there:}
\url{http://go.yurichev.com/17352}.

\subsection{\Exercise}

\begin{itemize}
	\item \url{http://challenges.re/50}
\end{itemize}


\ifx\LITE\undefined
\clearpage
\section{\RU{Простейшее четырехбайтное XOR-шифрование}\EN{Simplest possible 4-byte XOR encryption}}

\RU{Если при XOR-шифровании применялся шаблон длинее байта, например, 4-байтный, то его также легко
увидеть.}
\EN{If longer pattern was used while XOR-encryption, for example, 4 byte pattern, it's easy
to spot it as well.}
\RU{Например, вот начало файла kernel32.dll (32-битная версия из Windows Server 2008):}
\EN{As example, here is beginning of kernel32.dll file (32-bit version from Windows Server 2008):}

\begin{figure}[H]
\centering
\includegraphics[scale=\FigScale]{ff/XOR/4byte/original1.png}
\caption{\EN{Original file}\RU{Оригинальный файл}}
\end{figure}

\clearpage
\RU{Вот он же, но \q{зашифрованный} 4-байтным ключем:}
\EN{Here is it \q{encrypted} by 4-byte key:}

\begin{figure}[H]
\centering
\includegraphics[scale=\FigScale]{ff/XOR/4byte/encrypted1.png}
\caption{\EN{\q{Encrypted} file}\RU{\q{Зашифрованный} файл}}
\end{figure}

\RU{Очень легко увидеть повторяющиеся 4 символа.}
\EN{It's very easy to spot recurring 4 symbols.}
\RU{Ведь в заголовке PE-файла много длинных нулевых областей, из-за которых ключ становится видным.}
\EN{Indeed, PE-file header has a lot of long zero lacunes, which is the reason why key became visible.}

\clearpage
\RU{Вот начало PE-заголовка в 16-ричном виде:}
\EN{Here is beginning of PE-header in hexadecimal form:}

\begin{figure}[H]
\centering
\includegraphics[scale=\FigScale]{ff/XOR/4byte/original2.png}
\caption{PE-\EN{header}\RU{заголовок}}
\end{figure}

\clearpage
\RU{И вот он же, \q{зашифрованный}:}
\EN{Here is it \q{encrypted}:}

\begin{figure}[H]
\centering
\includegraphics[scale=\FigScale]{ff/XOR/4byte/encrypted2.png}
\caption{\EN{\q{Encrypted} PE-header}\RU{\q{Зашифрованный} PE-заголовок}}
\end{figure}

\RU{Легко увидеть визуально, что ключ это следующие 4 байта}
\EN{It's easy to spot that key is the following 4 bytes}: \TT{8C 61 D2 63}.
\RU{Используя эту информацию, довольно легко расшифровать весь файл.}
\EN{It's easy to decrypt the whole file using this information.}

\RU{Таким образом, важно помнить эти свойства PE-файлов:
1) в PE-заголовке много нулевых областей;
2) все PE-секции дополняются нулями до границы страницы (4096 байт), 
так что после всех секций обычно имеются длинные нулевые области.}
\EN{So this is important to remember these property of PE-files:
1) PE-header has many zero lacunas;
2) all PE-sections padded with zeroes by page border (4096 bytes),
so long zero lacunas usually present after all sections.}

\RU{Некоторые другие форматы файлов могут также иметь длинные нулевые области.}
\EN{Some other file formats may contain long zero lacunas.}
\RU{Это очень типично для файлов, используемых научным и инженерным ПО.}
\EN{It's very typical for files used by scientific and engineering software.}

\RU{Для тех, кто самостоятельно хочет изучить эти файлы, то их можно скачать здесь:}
\EN{For those who wants to inspect these files on one's own, they are downloadable there:}
\url{http://go.yurichev.com/17352}.

\subsection{\Exercise}

\begin{itemize}
	\item \url{http://challenges.re/50}
\end{itemize}


\clearpage
\section{\RU{Простейшее четырехбайтное XOR-шифрование}\EN{Simplest possible 4-byte XOR encryption}}

\RU{Если при XOR-шифровании применялся шаблон длинее байта, например, 4-байтный, то его также легко
увидеть.}
\EN{If longer pattern was used while XOR-encryption, for example, 4 byte pattern, it's easy
to spot it as well.}
\RU{Например, вот начало файла kernel32.dll (32-битная версия из Windows Server 2008):}
\EN{As example, here is beginning of kernel32.dll file (32-bit version from Windows Server 2008):}

\begin{figure}[H]
\centering
\includegraphics[scale=\FigScale]{ff/XOR/4byte/original1.png}
\caption{\EN{Original file}\RU{Оригинальный файл}}
\end{figure}

\clearpage
\RU{Вот он же, но \q{зашифрованный} 4-байтным ключем:}
\EN{Here is it \q{encrypted} by 4-byte key:}

\begin{figure}[H]
\centering
\includegraphics[scale=\FigScale]{ff/XOR/4byte/encrypted1.png}
\caption{\EN{\q{Encrypted} file}\RU{\q{Зашифрованный} файл}}
\end{figure}

\RU{Очень легко увидеть повторяющиеся 4 символа.}
\EN{It's very easy to spot recurring 4 symbols.}
\RU{Ведь в заголовке PE-файла много длинных нулевых областей, из-за которых ключ становится видным.}
\EN{Indeed, PE-file header has a lot of long zero lacunes, which is the reason why key became visible.}

\clearpage
\RU{Вот начало PE-заголовка в 16-ричном виде:}
\EN{Here is beginning of PE-header in hexadecimal form:}

\begin{figure}[H]
\centering
\includegraphics[scale=\FigScale]{ff/XOR/4byte/original2.png}
\caption{PE-\EN{header}\RU{заголовок}}
\end{figure}

\clearpage
\RU{И вот он же, \q{зашифрованный}:}
\EN{Here is it \q{encrypted}:}

\begin{figure}[H]
\centering
\includegraphics[scale=\FigScale]{ff/XOR/4byte/encrypted2.png}
\caption{\EN{\q{Encrypted} PE-header}\RU{\q{Зашифрованный} PE-заголовок}}
\end{figure}

\RU{Легко увидеть визуально, что ключ это следующие 4 байта}
\EN{It's easy to spot that key is the following 4 bytes}: \TT{8C 61 D2 63}.
\RU{Используя эту информацию, довольно легко расшифровать весь файл.}
\EN{It's easy to decrypt the whole file using this information.}

\RU{Таким образом, важно помнить эти свойства PE-файлов:
1) в PE-заголовке много нулевых областей;
2) все PE-секции дополняются нулями до границы страницы (4096 байт), 
так что после всех секций обычно имеются длинные нулевые области.}
\EN{So this is important to remember these property of PE-files:
1) PE-header has many zero lacunas;
2) all PE-sections padded with zeroes by page border (4096 bytes),
so long zero lacunas usually present after all sections.}

\RU{Некоторые другие форматы файлов могут также иметь длинные нулевые области.}
\EN{Some other file formats may contain long zero lacunas.}
\RU{Это очень типично для файлов, используемых научным и инженерным ПО.}
\EN{It's very typical for files used by scientific and engineering software.}

\RU{Для тех, кто самостоятельно хочет изучить эти файлы, то их можно скачать здесь:}
\EN{For those who wants to inspect these files on one's own, they are downloadable there:}
\url{http://go.yurichev.com/17352}.

\subsection{\Exercise}

\begin{itemize}
	\item \url{http://challenges.re/50}
\end{itemize}


\fi
\clearpage
\section{\RU{Простейшее четырехбайтное XOR-шифрование}\EN{Simplest possible 4-byte XOR encryption}}

\RU{Если при XOR-шифровании применялся шаблон длинее байта, например, 4-байтный, то его также легко
увидеть.}
\EN{If longer pattern was used while XOR-encryption, for example, 4 byte pattern, it's easy
to spot it as well.}
\RU{Например, вот начало файла kernel32.dll (32-битная версия из Windows Server 2008):}
\EN{As example, here is beginning of kernel32.dll file (32-bit version from Windows Server 2008):}

\begin{figure}[H]
\centering
\includegraphics[scale=\FigScale]{ff/XOR/4byte/original1.png}
\caption{\EN{Original file}\RU{Оригинальный файл}}
\end{figure}

\clearpage
\RU{Вот он же, но \q{зашифрованный} 4-байтным ключем:}
\EN{Here is it \q{encrypted} by 4-byte key:}

\begin{figure}[H]
\centering
\includegraphics[scale=\FigScale]{ff/XOR/4byte/encrypted1.png}
\caption{\EN{\q{Encrypted} file}\RU{\q{Зашифрованный} файл}}
\end{figure}

\RU{Очень легко увидеть повторяющиеся 4 символа.}
\EN{It's very easy to spot recurring 4 symbols.}
\RU{Ведь в заголовке PE-файла много длинных нулевых областей, из-за которых ключ становится видным.}
\EN{Indeed, PE-file header has a lot of long zero lacunes, which is the reason why key became visible.}

\clearpage
\RU{Вот начало PE-заголовка в 16-ричном виде:}
\EN{Here is beginning of PE-header in hexadecimal form:}

\begin{figure}[H]
\centering
\includegraphics[scale=\FigScale]{ff/XOR/4byte/original2.png}
\caption{PE-\EN{header}\RU{заголовок}}
\end{figure}

\clearpage
\RU{И вот он же, \q{зашифрованный}:}
\EN{Here is it \q{encrypted}:}

\begin{figure}[H]
\centering
\includegraphics[scale=\FigScale]{ff/XOR/4byte/encrypted2.png}
\caption{\EN{\q{Encrypted} PE-header}\RU{\q{Зашифрованный} PE-заголовок}}
\end{figure}

\RU{Легко увидеть визуально, что ключ это следующие 4 байта}
\EN{It's easy to spot that key is the following 4 bytes}: \TT{8C 61 D2 63}.
\RU{Используя эту информацию, довольно легко расшифровать весь файл.}
\EN{It's easy to decrypt the whole file using this information.}

\RU{Таким образом, важно помнить эти свойства PE-файлов:
1) в PE-заголовке много нулевых областей;
2) все PE-секции дополняются нулями до границы страницы (4096 байт), 
так что после всех секций обычно имеются длинные нулевые области.}
\EN{So this is important to remember these property of PE-files:
1) PE-header has many zero lacunas;
2) all PE-sections padded with zeroes by page border (4096 bytes),
so long zero lacunas usually present after all sections.}

\RU{Некоторые другие форматы файлов могут также иметь длинные нулевые области.}
\EN{Some other file formats may contain long zero lacunas.}
\RU{Это очень типично для файлов, используемых научным и инженерным ПО.}
\EN{It's very typical for files used by scientific and engineering software.}

\RU{Для тех, кто самостоятельно хочет изучить эти файлы, то их можно скачать здесь:}
\EN{For those who wants to inspect these files on one's own, they are downloadable there:}
\url{http://go.yurichev.com/17352}.

\subsection{\Exercise}

\begin{itemize}
	\item \url{http://challenges.re/50}
\end{itemize}


\ifx\LITE\undefined
\clearpage
\section{\RU{Простейшее четырехбайтное XOR-шифрование}\EN{Simplest possible 4-byte XOR encryption}}

\RU{Если при XOR-шифровании применялся шаблон длинее байта, например, 4-байтный, то его также легко
увидеть.}
\EN{If longer pattern was used while XOR-encryption, for example, 4 byte pattern, it's easy
to spot it as well.}
\RU{Например, вот начало файла kernel32.dll (32-битная версия из Windows Server 2008):}
\EN{As example, here is beginning of kernel32.dll file (32-bit version from Windows Server 2008):}

\begin{figure}[H]
\centering
\includegraphics[scale=\FigScale]{ff/XOR/4byte/original1.png}
\caption{\EN{Original file}\RU{Оригинальный файл}}
\end{figure}

\clearpage
\RU{Вот он же, но \q{зашифрованный} 4-байтным ключем:}
\EN{Here is it \q{encrypted} by 4-byte key:}

\begin{figure}[H]
\centering
\includegraphics[scale=\FigScale]{ff/XOR/4byte/encrypted1.png}
\caption{\EN{\q{Encrypted} file}\RU{\q{Зашифрованный} файл}}
\end{figure}

\RU{Очень легко увидеть повторяющиеся 4 символа.}
\EN{It's very easy to spot recurring 4 symbols.}
\RU{Ведь в заголовке PE-файла много длинных нулевых областей, из-за которых ключ становится видным.}
\EN{Indeed, PE-file header has a lot of long zero lacunes, which is the reason why key became visible.}

\clearpage
\RU{Вот начало PE-заголовка в 16-ричном виде:}
\EN{Here is beginning of PE-header in hexadecimal form:}

\begin{figure}[H]
\centering
\includegraphics[scale=\FigScale]{ff/XOR/4byte/original2.png}
\caption{PE-\EN{header}\RU{заголовок}}
\end{figure}

\clearpage
\RU{И вот он же, \q{зашифрованный}:}
\EN{Here is it \q{encrypted}:}

\begin{figure}[H]
\centering
\includegraphics[scale=\FigScale]{ff/XOR/4byte/encrypted2.png}
\caption{\EN{\q{Encrypted} PE-header}\RU{\q{Зашифрованный} PE-заголовок}}
\end{figure}

\RU{Легко увидеть визуально, что ключ это следующие 4 байта}
\EN{It's easy to spot that key is the following 4 bytes}: \TT{8C 61 D2 63}.
\RU{Используя эту информацию, довольно легко расшифровать весь файл.}
\EN{It's easy to decrypt the whole file using this information.}

\RU{Таким образом, важно помнить эти свойства PE-файлов:
1) в PE-заголовке много нулевых областей;
2) все PE-секции дополняются нулями до границы страницы (4096 байт), 
так что после всех секций обычно имеются длинные нулевые области.}
\EN{So this is important to remember these property of PE-files:
1) PE-header has many zero lacunas;
2) all PE-sections padded with zeroes by page border (4096 bytes),
so long zero lacunas usually present after all sections.}

\RU{Некоторые другие форматы файлов могут также иметь длинные нулевые области.}
\EN{Some other file formats may contain long zero lacunas.}
\RU{Это очень типично для файлов, используемых научным и инженерным ПО.}
\EN{It's very typical for files used by scientific and engineering software.}

\RU{Для тех, кто самостоятельно хочет изучить эти файлы, то их можно скачать здесь:}
\EN{For those who wants to inspect these files on one's own, they are downloadable there:}
\url{http://go.yurichev.com/17352}.

\subsection{\Exercise}

\begin{itemize}
	\item \url{http://challenges.re/50}
\end{itemize}


\fi
\input{tools}
\ifx\LITE\undefined
\input{examples/examples}
\part{\RU{Примеры разбора закрытых (proprietary) форматов файлов}\EN{Examples of reversing proprietary file formats}\ESph{}\PTBRph{}\PLph{}\ITAph{}}

% chapters
\clearpage
\section{\RU{Простейшее четырехбайтное XOR-шифрование}\EN{Simplest possible 4-byte XOR encryption}}

\RU{Если при XOR-шифровании применялся шаблон длинее байта, например, 4-байтный, то его также легко
увидеть.}
\EN{If longer pattern was used while XOR-encryption, for example, 4 byte pattern, it's easy
to spot it as well.}
\RU{Например, вот начало файла kernel32.dll (32-битная версия из Windows Server 2008):}
\EN{As example, here is beginning of kernel32.dll file (32-bit version from Windows Server 2008):}

\begin{figure}[H]
\centering
\includegraphics[scale=\FigScale]{ff/XOR/4byte/original1.png}
\caption{\EN{Original file}\RU{Оригинальный файл}}
\end{figure}

\clearpage
\RU{Вот он же, но \q{зашифрованный} 4-байтным ключем:}
\EN{Here is it \q{encrypted} by 4-byte key:}

\begin{figure}[H]
\centering
\includegraphics[scale=\FigScale]{ff/XOR/4byte/encrypted1.png}
\caption{\EN{\q{Encrypted} file}\RU{\q{Зашифрованный} файл}}
\end{figure}

\RU{Очень легко увидеть повторяющиеся 4 символа.}
\EN{It's very easy to spot recurring 4 symbols.}
\RU{Ведь в заголовке PE-файла много длинных нулевых областей, из-за которых ключ становится видным.}
\EN{Indeed, PE-file header has a lot of long zero lacunes, which is the reason why key became visible.}

\clearpage
\RU{Вот начало PE-заголовка в 16-ричном виде:}
\EN{Here is beginning of PE-header in hexadecimal form:}

\begin{figure}[H]
\centering
\includegraphics[scale=\FigScale]{ff/XOR/4byte/original2.png}
\caption{PE-\EN{header}\RU{заголовок}}
\end{figure}

\clearpage
\RU{И вот он же, \q{зашифрованный}:}
\EN{Here is it \q{encrypted}:}

\begin{figure}[H]
\centering
\includegraphics[scale=\FigScale]{ff/XOR/4byte/encrypted2.png}
\caption{\EN{\q{Encrypted} PE-header}\RU{\q{Зашифрованный} PE-заголовок}}
\end{figure}

\RU{Легко увидеть визуально, что ключ это следующие 4 байта}
\EN{It's easy to spot that key is the following 4 bytes}: \TT{8C 61 D2 63}.
\RU{Используя эту информацию, довольно легко расшифровать весь файл.}
\EN{It's easy to decrypt the whole file using this information.}

\RU{Таким образом, важно помнить эти свойства PE-файлов:
1) в PE-заголовке много нулевых областей;
2) все PE-секции дополняются нулями до границы страницы (4096 байт), 
так что после всех секций обычно имеются длинные нулевые области.}
\EN{So this is important to remember these property of PE-files:
1) PE-header has many zero lacunas;
2) all PE-sections padded with zeroes by page border (4096 bytes),
so long zero lacunas usually present after all sections.}

\RU{Некоторые другие форматы файлов могут также иметь длинные нулевые области.}
\EN{Some other file formats may contain long zero lacunas.}
\RU{Это очень типично для файлов, используемых научным и инженерным ПО.}
\EN{It's very typical for files used by scientific and engineering software.}

\RU{Для тех, кто самостоятельно хочет изучить эти файлы, то их можно скачать здесь:}
\EN{For those who wants to inspect these files on one's own, they are downloadable there:}
\url{http://go.yurichev.com/17352}.

\subsection{\Exercise}

\begin{itemize}
	\item \url{http://challenges.re/50}
\end{itemize}


\ifdefined\IncludeMSDOS
\chapter{\RU{Файл сохранения состояния в игре Millenium}\EN{Millenium game save file}}
\label{Millenium_DOS_game}
\index{MS-DOS}

\RU{Игра}\EN{The} \q{Millenium Return to Earth} \RU{под DOS довольно древняя (1991), позволяющая
добывать ресурсы, строить корабли, снаряжать их на другие планеты,\etc{}.}
\EN{is an ancient DOS game (1991), that allows you to mine resources, build ships,
equip them on other planets, and so on}\footnote{\RU{Её можно скачать бесплатно}\EN{It can be downloaded for free}
\href{http://go.yurichev.com/17316}{\RU{здесь}\EN{here}}}.

\RU{Как и многие другие игры, она позволяет сохранять состояние игры в файл.}
\EN{Like many other games, it allows you to save all game state into a file.}

\RU{Посмотрим, сможем ли мы найти что-нибудь в нем}\EN{Let's see if we can find something in it}.

\clearpage
\RU{В игре есть шахта}\EN{So there is a mine in the game}.
\RU{Шахты на некоторых планетах работают быстрее, на некоторых других --- медленнее}\EN{Mines at some planets 
work faster, or slower on others}. 
\RU{Набор ресурсов также разный}\EN{The set of resources is also different}.

\RU{Здесь видно, какие ресурсы добыты в этот момент}\EN{Here we can see what resources are mined at the time}: 

\begin{figure}[H]
\centering
\includegraphics[scale=\FigScale]{ff/millenium/1.png}
\caption{\RU{Шахта: первое состояние}\EN{Mine: state 1}}
\label{fig:mill_1}
\end{figure}

\RU{Сохраним состояние игры}\EN{Let's save a game state}.
\RU{Это файл размером}\EN{This is a file of size} 9538 \RU{байт}\EN{bytes}.

\RU{Подождем несколько \q{дней} здесь в игре и теперь в шахте добыто больше ресурсов}%
\EN{Let's wait some \q{days} here in the game, and now we've got more resources from the mine}:

\begin{figure}[H]
\centering
\includegraphics[scale=\FigScale]{ff/millenium/2.png}
\caption{\RU{Шахта: второе состояние}\EN{Mine: state 2}}
\label{fig:mill_2}
\end{figure}

\RU{Снова сохраним состояние игры}\EN{Let's sav game state again}.

\RU{Теперь просто попробуем сравнить оба файла побайтово используя простую утилиту FC под DOS/Windows:}
\EN{Now let's try to just do binary comparison of the save files using the simple DOS/Windows FC utility:}

\lstinputlisting{ff/millenium/fc_result.txt}

\RU{Вывод здесь неполный, там было больше отличий, но мы обрежем результат до самого интересного.}%
\EN{The output is incomplete here, there are more differences, but we will cut result to show the most interesting.}

\RU{В первой версии у нас было 14 единиц водорода (hydrogen) и 102 --- кислорода (oxygen).}
\EN{In the first state, we have 14 \q{units} of hydrogen and 102 \q{units} of oxygen.}
\RU{Во второй версии у нас 22 и 155 единиц соответственно.}
\EN{We have 22 and 155 \q{units} respectively in the second state.}
\RU{Если эти значения сохраняются в файл, мы должны увидеть разницу}\EN{If these values are saved into 
the save file, we would see this in the difference}.
\RU{И она действительно есть}\EN{And indeed we do}. 
\RU{Там}\EN{There is} 0x0E (14) \RU{на позиции}\EN{at position} 0xBDA \RU{и это значение}\EN{and this value is} 
0x16 (22) \RU{в новой версии файла}\EN{in the new version of the file}.
\RU{Это, наверное, водород}\EN{This is probably hydrogen}.
\RU{Там также}\EN{There is} 0x66 (102) \RU{на позиции}\EN{at position} 0xBDC \RU{в старой версии и}\EN{in the old 
version and} 0x9B (155) \RU{в новой версии файла}\EN{in the new version of the file}. 
\RU{Это, наверное, кислород}\EN{This seems to be the oxygen}.

\RU{Обе версии файла доступны на сайте, для тех кто хочет их изучить (или поэкспериментировать)}%
\EN{Both files are available on the website for those who wants to inspect them (or experiment) more}: 
\href{http://go.yurichev.com/17212}{beginners.re}.

\clearpage
\RU{Новую версию файла откроем в Hiew и отметим значения, связанные с ресурсами, добытыми на шахте в игре}%
\EN{Here is the new version of file opened in Hiew, we marked the values related to the resources mined in the game}: 

\begin{figure}[H]
\centering
\includegraphics[scale=\FigScale]{ff/millenium/hiew3.png}
\caption{Hiew: \RU{первое состояние}\EN{state 1}}
\label{fig:mill_hiew3}
\end{figure}

\RU{Проверим каждое, и это они}\EN{Let's check each, and these are}.
\RU{Это явно 16-битные значения: не удивительно для 16-битной программы под DOS, где \Tint имел длину в 16 бит.}
\EN{These are clearly 16-bit values: not a strange thing for 16-bit DOS software where the \Tint type has 16-bit width.}

\clearpage
\RU{Проверим наши предположения}\EN{Let's check our assumptions}.
\RU{Запишем 1234 (0x4D2) на первой позиции (это должен быть водород)}%
\EN{We will write the 1234 (0x4D2) value at the first position (this must be hydrogen)}:

\begin{figure}[H]
\centering
\includegraphics[scale=\FigScale]{ff/millenium/hiew4.png}
\caption{Hiew: \RU{запишем там}\EN{let's write 1234} (0x4D2)\EN{ there}}
\label{fig:mill_hiew4}
\end{figure}

\RU{Затем загрузим измененный файл в игру и посмотрим на статистику в шахте}%
\EN{Then we will load the changed file in the game and took a look at mine statistics}:

\begin{figure}[H]
\centering
\includegraphics[scale=\FigScale]{ff/millenium/5.png}
\caption{\RU{Проверим значение водорода}\EN{Let's check for hydrogen value}}
\label{fig:mill_5}
\end{figure}

\RU{Так что да, это оно}\EN{So yes, this is it}.

\clearpage
\RU{Попробуем пройти игру как можно быстрее, установим максимальные значения везде}\EN{Now let's try to 
finish the game as soon as possible, set the maximal values everywhere}:

\begin{figure}[H]
\centering
\includegraphics[scale=\FigScale]{ff/millenium/hiew7.png}
\caption{Hiew: \RU{установим максимальные значения}\EN{let's set maximal values}}
\label{fig:mill_hiew7}
\end{figure}

0xFFFF \RU{это}\EN{is} 65535, \RU{так что да, у нас много ресурсов теперь}\EN{so yes, we now have a 
lot of resources}:

\begin{figure}[H]
\centering
\includegraphics[scale=\FigScale]{ff/millenium/6.png}
\caption{\RU{Все ресурсы теперь действительно}\EN{All resources are} 65535 (0xFFFF)\EN{ indeed}}
\label{fig:mill_6}
\end{figure}

\clearpage
\RU{Пропустим еще несколько \q{дней} в игре и видим что-то неладное}\EN{Let's skip some \q{days} in the game and oops}! 
\RU{Некоторых ресурсов стало меньше}\EN{We have a lower amount of some resources}:

\begin{figure}[H]
\centering
\includegraphics[scale=\FigScale]{ff/millenium/8.png}
\caption{\RU{Переполнение переменных ресурсов}\EN{Resource variables overflow}}
\label{fig:mill_8}
\end{figure}

\RU{Это просто переполнение}\EN{That's just overflow}. 
\RU{Разработчик игры вероятно никогда не думал, что значения ресурсов будут такими большими,
так что, здесь, наверное, нет проверок на переполнение, но шахта в игре \q{работает}, ресурсы добавляются,
отсюда и переполнение.}
\EN{The game's developer probably didn't think about such high amounts of resources,
so there are probably no overflow checks, but the mine is \q{working} in the game, resources are added,
hence the overflows.}
\RU{Вероятно, не нужно было жадничать}\EN{Apparently, it was a bad idea to be that greedy}.

\RU{Здесь наверняка еще какие-то значения в этом файле}\EN{There are probably a lot of more values 
saved in this file}.

\RU{Так что это очень простой способ читинга в играх}\EN{So this is very simple method of cheating in games}.
\RU{Файл с таблицей очков также можно легко модифицировать}\EN{High score files often can be easily 
patched like that}.

\EN{More about files and memory snapshots comparing}\RU{Еще насчет сравнения файлов и снимков памяти}: 
\myref{snapshots_comparing}.

\fi
\ifdefined\IncludeOracle
\clearpage
\section{\RU{Простейшее четырехбайтное XOR-шифрование}\EN{Simplest possible 4-byte XOR encryption}}

\RU{Если при XOR-шифровании применялся шаблон длинее байта, например, 4-байтный, то его также легко
увидеть.}
\EN{If longer pattern was used while XOR-encryption, for example, 4 byte pattern, it's easy
to spot it as well.}
\RU{Например, вот начало файла kernel32.dll (32-битная версия из Windows Server 2008):}
\EN{As example, here is beginning of kernel32.dll file (32-bit version from Windows Server 2008):}

\begin{figure}[H]
\centering
\includegraphics[scale=\FigScale]{ff/XOR/4byte/original1.png}
\caption{\EN{Original file}\RU{Оригинальный файл}}
\end{figure}

\clearpage
\RU{Вот он же, но \q{зашифрованный} 4-байтным ключем:}
\EN{Here is it \q{encrypted} by 4-byte key:}

\begin{figure}[H]
\centering
\includegraphics[scale=\FigScale]{ff/XOR/4byte/encrypted1.png}
\caption{\EN{\q{Encrypted} file}\RU{\q{Зашифрованный} файл}}
\end{figure}

\RU{Очень легко увидеть повторяющиеся 4 символа.}
\EN{It's very easy to spot recurring 4 symbols.}
\RU{Ведь в заголовке PE-файла много длинных нулевых областей, из-за которых ключ становится видным.}
\EN{Indeed, PE-file header has a lot of long zero lacunes, which is the reason why key became visible.}

\clearpage
\RU{Вот начало PE-заголовка в 16-ричном виде:}
\EN{Here is beginning of PE-header in hexadecimal form:}

\begin{figure}[H]
\centering
\includegraphics[scale=\FigScale]{ff/XOR/4byte/original2.png}
\caption{PE-\EN{header}\RU{заголовок}}
\end{figure}

\clearpage
\RU{И вот он же, \q{зашифрованный}:}
\EN{Here is it \q{encrypted}:}

\begin{figure}[H]
\centering
\includegraphics[scale=\FigScale]{ff/XOR/4byte/encrypted2.png}
\caption{\EN{\q{Encrypted} PE-header}\RU{\q{Зашифрованный} PE-заголовок}}
\end{figure}

\RU{Легко увидеть визуально, что ключ это следующие 4 байта}
\EN{It's easy to spot that key is the following 4 bytes}: \TT{8C 61 D2 63}.
\RU{Используя эту информацию, довольно легко расшифровать весь файл.}
\EN{It's easy to decrypt the whole file using this information.}

\RU{Таким образом, важно помнить эти свойства PE-файлов:
1) в PE-заголовке много нулевых областей;
2) все PE-секции дополняются нулями до границы страницы (4096 байт), 
так что после всех секций обычно имеются длинные нулевые области.}
\EN{So this is important to remember these property of PE-files:
1) PE-header has many zero lacunas;
2) all PE-sections padded with zeroes by page border (4096 bytes),
so long zero lacunas usually present after all sections.}

\RU{Некоторые другие форматы файлов могут также иметь длинные нулевые области.}
\EN{Some other file formats may contain long zero lacunas.}
\RU{Это очень типично для файлов, используемых научным и инженерным ПО.}
\EN{It's very typical for files used by scientific and engineering software.}

\RU{Для тех, кто самостоятельно хочет изучить эти файлы, то их можно скачать здесь:}
\EN{For those who wants to inspect these files on one's own, they are downloadable there:}
\url{http://go.yurichev.com/17352}.

\subsection{\Exercise}

\begin{itemize}
	\item \url{http://challenges.re/50}
\end{itemize}


\chapter{\oracle: \EN{.MSB-files}\RU{.MSB-файлы}\ESph{}\PTBRph{}\PLph{}\ITAph{}}
\index{\oracle}
\EN{\epigraph{When working toward the solution of a problem, it always helps if you know the answer.}{Murphy's Laws, Rule of Accuracy}}
\RU{\epigraph{Работая над решением задачи, всегда полезно знать ответ.}{Законы Мерфи, правило точности}}

\RU{Это бинарный файл, содержащий сообщения об ошибках вместе с их номерами.}
\EN{This is a binary file that contains error messages with their corresponding numbers.}
\RU{Давайте попробуем понять его формат и найти способ распаковать его}\EN{Let's try to understand 
its format and find a way to unpack it}.

\RU{В \oracle имеются файлы с сообщениями об ошибках в текстовом виде, так что мы можем сравнивать файлы:
текстовый и запакованный бинарный}\EN{There are \oracle error message files in text form, 
so we can compare the text and packed binary files}
\footnote{\EN{Open-source text files don't exist in \oracle for every .MSB file, so that's why we will work on their file format}
\RU{Текстовые файлы с открытым кодом в \oracle имеются не для каждого .MSB-файла, вот почему мы будем работать над его форматом}}.

\RU{Это начало файла}\EN{This is the beginning of the} ORAUS.MSG \RU{без ненужных комментариев}\EN{text file 
with some irrelevant comments stripped}:

\begin{lstlisting}[caption=\RU{Начало файла}\EN{Beginning of} ORAUS.MSG \RU{без комментариев}\EN{file without comments}]
00000, 00000, "normal, successful completion"
00001, 00000, "unique constraint (%s.%s) violated"
00017, 00000, "session requested to set trace event"
00018, 00000, "maximum number of sessions exceeded"
00019, 00000, "maximum number of session licenses exceeded"
00020, 00000, "maximum number of processes (%s) exceeded"
00021, 00000, "session attached to some other process; cannot switch session"
00022, 00000, "invalid session ID; access denied"
00023, 00000, "session references process private memory; cannot detach session"
00024, 00000, "logins from more than one process not allowed in single-process mode"
00025, 00000, "failed to allocate %s"
00026, 00000, "missing or invalid session ID"
00027, 00000, "cannot kill current session"
00028, 00000, "your session has been killed"
00029, 00000, "session is not a user session"
00030, 00000, "User session ID does not exist."
00031, 00000, "session marked for kill"
...
\end{lstlisting}

\RU{Первое число\EMDASH{}это код ошибки}\EN{The first number is the error code}.
\RU{Второе это, вероятно, могут быть дополнительные флаги}\EN{The second is perhaps maybe some additional flags}.

\clearpage
\RU{Давайте откроем бинарный файл}\EN{Now let's open the} ORAUS.MSB 
\RU{и найдем эти текстовые строки}\EN{binary file and find these text strings}. 
\RU{И вот они}\EN{And there are}:

\begin{figure}[H]
\centering
\includegraphics[scale=\FigScale]{ff/Oracle_MSB/1.png}
\caption{Hiew: \RU{первый блок}\EN{first block}}
\label{fig:oracle_MSB_1}
\end{figure}

\RU{Мы видим текстовые строки (включая те, с которых начинается файл ORAUS.MSG) перемежаемые с какими-то
бинарными значениями}\EN{We see the text strings (including those from the beginning of the ORAUS.MSG file) 
interleaved with some binary values}.
\RU{Мы можем довольно быстро обнаружить что главная часть бинарного файла поделена на блоки размером 0x200 (512)
байт}\EN{By quick investigation, we can see that main part of the binary file is divided by blocks of 
size 0x200 (512) bytes}.

\clearpage
\RU{Посмотрим содержимое первого блока}\EN{Let's see the contents of the first block}:

\begin{figure}[H]
\centering
\includegraphics[scale=\FigScale]{ff/Oracle_MSB/2.png}
\caption{Hiew: \RU{первый блок}\EN{first block}}
\label{fig:oracle_MSB_2}
\end{figure}

\RU{Мы видим тексты первых сообщений об ошибках}\EN{Here we see the texts of the first messages errors}.
\RU{Что мы видим еще, так это то, что здесь нет нулевых байтов между сообщениями}\EN{What we also see is 
that there are no zero bytes between the error messages}.
\RU{Это значит, что это не оканчивающиеся нулем Си-строки}\EN{This implies that these are not null-terminated C strings}.
\RU{Как следствие, длина каждого сообщения об ошибке должна быть как-то закодирована}\EN{As a consequence, 
the length of each error message must be encoded somehow}.
\RU{Попробуем также найти номера ошибок}\EN{Let's also try to find the error numbers}.
\RU{Файл}\EN{The} ORAUS.MSG \RU{начинается с таких}\EN{files starts with these}: 
0, 1, 17 (0x11), 18 (0x12), 19 (0x13), 20 (0x14), 21 (0x15), 22 (0x16), 23 (0x17), 24 (0x18)...
\RU{Найдем эти числа в начале блока и отметим их красными линиями}\EN{We will find these numbers in the beginning 
of the block and mark them with red lines}.
\RU{Период между кодами ошибок 6 байт}\EN{The period between error codes is 6 bytes}.
\RU{Это значит, здесь, наверное, 6 байт информации выделено для каждого сообщения об ошибке.}
\EN{This implies that there are probably 6 bytes of information allocated for each error message.}

\RU{Первое 16-битное значение (здесь 0xA или 10) означает количество сообщений в блоке: это можно проверить глядя на другие блоки.}%
\EN{The first 16-bit value (0xA here or 10) mean the number of messages in each block: this can be checked by investigating other blocks.}
\RU{Действительно: сообщения об ошибках имеют произвольный размер}\EN{Indeed: the error messages have arbitrary size}. 
\RU{Некоторые длиннее, некоторые короче}\EN{Some are longer, some are shorter}. 
\RU{Но длина блока всегда фиксирована, следовательно, никогда не знаешь, сколько сообщений можно запаковать
в каждый блок}\EN{But block size is always fixed, hence,
you never know how many text messages can be packed in each block}.

\RU{Как мы уже отметили, так как это не оканчивающиеся нулем Си-строки, длина строки должна быть закодирована где-то.}%
\EN{As we already noted, since these are not null-terminating C strings, their size must be encoded somewhere.}
\RU{Длина первой строки}\EN{The size of the first string} \q{normal, successful completion} \RU{это}\EN{is} 
29 (0x1D) \RU{байт}\EN{bytes}.
\RU{Длина второй строки}\EN{The size of the second string} \q{unique constraint (\%s.\%s) violated} 
\RU{это}\EN{is} 34 (0x22) \RU{байт}\EN{bytes}.
\RU{Мы не можем отыскать этих значений}\EN{We can't find these values} (0x1D \OrENRU/\AndENRU 0x22) 
\RU{в блоке}\EN{in the block}.

\RU{А вот еще кое-что}\EN{There is also another thing}.
\oracle \RU{должен как-то определять позицию строки, которую он должен загрузить, верно}
\EN{has to determine the position of the string it needs to load in the block, right}?
\RU{Первая строка}\EN{The first string} \q{normal, successful completion} \RU{начинается с позиции}\EN{starts 
at  position} 0x1444 (\RU{если считать с начала бинарного файла}\EN{if we count starting at the beginning of the file}) \RU{или с}\EN{or at} 0x44 (\RU{от начала блока}\EN{from the block's start}).
\RU{Вторая строка}\EN{The second string} \q{unique constraint (\%s.\%s) violated} 
\RU{начинается с позиции}\EN{starts at position} 0x1461 (\RU{от начала файла}\EN{from the
file's start}) \RU{или с}\EN{or at} 0x61 (\RU{считая с начала блока}\EN{from the at the block's start}).
\RU{Эти числа}\EN{These numbers} (0x44 \AndENRU 0x61) \RU{нам знакомы}\EN{are familiar somehow}! 
\RU{Мы их можем легко отыскать в начале блока}\EN{We can clearly see them at the start of the block}.

\RU{Так что, каждый 6-байтный блок это}\EN{So, each 6-byte block is}:

\begin{itemize}
\item 16-\RU{битный номер ошибки}\EN{bit error number}; 
\item 16-\RU{битный ноль (может быть, дополнительные флаги}\EN{bit zero (maybe additional flags)}; 
\item 16-\RU{битная начальная позиция текстовой строки внутри текущего блока}\EN{bit starting position of 
the text string within the current block}.
\end{itemize}

\RU{Мы можем быстро проверить остальные значения чтобы удостовериться в своей правоте}%
\EN{We can quickly check the other values and be sure our guess is correct}.
\RU{И здесь еще последний \q{пустой} 6-байтный блок с нулевым номером ошибки и начальной позицией за последним
символом последнего сообщения об ошибке.}\EN{And there is also the last \q{dummy} 6-byte block 
with an error number of zero and starting position beyond the last error message's last character.}
\RU{Может быть именно так и определяется длина сообщения}\EN{Probably that's how text message length is 
determined}?
\RU{Мы просто перебираем 6-байтные блоки в поисках нужного номера ошибки, затем
мы узнаем позицию текстовой строки, затем мы узнаем позицию следующей текстовой строки глядя на
следующий 6-байтный блок!}\EN{We just enumerate 6-byte blocks to find the error number
we need, then we get the text string's position, then we get the position of the text string by looking at the next
6-byte block!}
\RU{Так мы определяем границы строки}\EN{This way we determine the string's boundaries}!
\RU{Этот метод позволяет сэкономить место в файле не записывая длину строки}\EN{This method allows to 
save some space by not saving the text string's size in the file}!
\RU{Нельзя сказать, что экономия памяти большая, но это интересный трюк.}%
\EN{It's not possible to say it saves a lot of space, but it's a clever trick}.

\clearpage
\RU{Вернемся к заголовку .MSB-файла}\EN{Let's back to the header of .MSB-file}:

\begin{figure}[H]
\centering
\includegraphics[scale=\FigScale]{ff/Oracle_MSB/3.png}
\caption{Hiew: \RU{заголовок файла}\EN{file header}}
\label{fig:oracle_MSB_3}
\end{figure}

\RU{Теперь мы можем быстро найти количество блоков (отмечено красным)}\EN{Now we can quickly find the number of blocks in the file 
(marked by red)}.
\RU{Проверяем другие .MSB-файлы и оказывается что это справедливо для всех}\EN{We can checked other .MSB-files and we see that it's true 
for all of them}.
\RU{Здесь есть много других значений, но мы не будем разбираться с ними, так как наша задача (утилита для распаковки) уже решена.}
\EN{There are a lot of other values, but we will not investigate them, since our job (an unpacking utility) was done.}
\RU{А если бы мы писали запаковщик .MSB-файлов, тогда нам наверное пришлось бы понять, зачем нужны остальные.}
\EN{If we have to write a .MSB file packer, we would probably need to understand the meaning of the other values.}

\clearpage
\RU{Тут еще есть таблица после заголовка, вероятно, содержащая 16-битные значения}\EN{There is also a 
table that came after the header which probably contains 16-bit values}:

\begin{figure}[H]
\centering
\includegraphics[scale=\FigScale]{ff/Oracle_MSB/4.png}
\caption{Hiew: \RU{таблица }last\_errnos\EN{ table}}
\label{fig:oracle_MSB_4}
\end{figure}

\RU{Их длина может быть определена визуально (здесь нарисованы красные линии).}%
\EN{Their size can be determined visually (red lines are drawn here).}
\RU{Когда мы сдампили эти значения, мы обнаружили, что каждое 16-битное число\EMDASH{}это последний код ошибки для каждого блока.}%
\EN{While dumping these values, we have found that each 16-bit number is the last error code for each block.}

\RU{Так вот как \oracle быстро находит сообщение об ошибке}\EN{So that's how \oracle quickly finds the error message}:

\begin{itemize}
\item \RU{загружает таблицу, которую мы назовем}\EN{load a table we will call} last\_errnos 
(\RU{содержащую последний номер ошибки для каждого блока}\EN{that contains the last error number for each block});
\item \RU{находит блок содержащий код ошибки, полагая что все коды ошибок увеличиваются и внутри каждого блока
и также в файле}\EN{find a block that contains the error code we need, assuming all error codes 
increase across each block and across the file as well};
\item \RU{загружает соответствующий блок}\EN{load the specific block};
\item \RU{перебирает 6-байтные структуры, пока не найдется соответствующий номер ошибки}\EN{enumerate the 6-byte 
structures until the specific error number is found};
\item \RU{находит позицию первого символа из текущего 6-байтного блока}\EN{get the position of the first 
character from the current 6-byte block};
\item \RU{находит позицию последнего символа из следующего 6-байтного блока}\EN{get the position of the last 
character from the next 6-byte block};
\item \RU{загружает все символы сообщения в этих пределах}\EN{load all characters of the message in this range}.
\end{itemize}

\RU{Это программа на Си которую мы написали для распаковки .MSB-файлов}
\EN{This is C program that we wrote which unpacks .MSB-files}:
\href{http://go.yurichev.com/17213}{beginners.re}.

\RU{И еще два файла которые были использованы в этом примере}
\EN{There are also the two files which were used in the example} 
(\oracle 11.1.0.6):
\href{http://go.yurichev.com/17214}{beginners.re},
\href{http://go.yurichev.com/17215}{beginners.re}.

\section{\RU{Вывод}\EN{Summary}}

\RU{Этот метод, наверное, слишком олд-скульный для современных компьютеров}\EN{The method is probably too 
old-school for modern computers}.
\RU{Возможно, формат этого файла был разработан в середине 1980-х кем-то, кто программировал для мейнфреймов,
учитывая экономию памяти и места на дисках}\EN{Supposedly, this file format was developed in the mid-80's by 
someone who also coded for \IT{big iron} with
memory/disk space economy in mind}.
\RU{Тем не менее, это интересная (хотя и простая) задача на разбор проприетарного формата файла без
заглядывания в код \oracle}\EN{Nevertheless, it was an interesting and yet easy task 
to understand a proprietary file format without looking into \oracle's code}.

\fi

\clearpage
\section{\RU{Простейшее четырехбайтное XOR-шифрование}\EN{Simplest possible 4-byte XOR encryption}}

\RU{Если при XOR-шифровании применялся шаблон длинее байта, например, 4-байтный, то его также легко
увидеть.}
\EN{If longer pattern was used while XOR-encryption, for example, 4 byte pattern, it's easy
to spot it as well.}
\RU{Например, вот начало файла kernel32.dll (32-битная версия из Windows Server 2008):}
\EN{As example, here is beginning of kernel32.dll file (32-bit version from Windows Server 2008):}

\begin{figure}[H]
\centering
\includegraphics[scale=\FigScale]{ff/XOR/4byte/original1.png}
\caption{\EN{Original file}\RU{Оригинальный файл}}
\end{figure}

\clearpage
\RU{Вот он же, но \q{зашифрованный} 4-байтным ключем:}
\EN{Here is it \q{encrypted} by 4-byte key:}

\begin{figure}[H]
\centering
\includegraphics[scale=\FigScale]{ff/XOR/4byte/encrypted1.png}
\caption{\EN{\q{Encrypted} file}\RU{\q{Зашифрованный} файл}}
\end{figure}

\RU{Очень легко увидеть повторяющиеся 4 символа.}
\EN{It's very easy to spot recurring 4 symbols.}
\RU{Ведь в заголовке PE-файла много длинных нулевых областей, из-за которых ключ становится видным.}
\EN{Indeed, PE-file header has a lot of long zero lacunes, which is the reason why key became visible.}

\clearpage
\RU{Вот начало PE-заголовка в 16-ричном виде:}
\EN{Here is beginning of PE-header in hexadecimal form:}

\begin{figure}[H]
\centering
\includegraphics[scale=\FigScale]{ff/XOR/4byte/original2.png}
\caption{PE-\EN{header}\RU{заголовок}}
\end{figure}

\clearpage
\RU{И вот он же, \q{зашифрованный}:}
\EN{Here is it \q{encrypted}:}

\begin{figure}[H]
\centering
\includegraphics[scale=\FigScale]{ff/XOR/4byte/encrypted2.png}
\caption{\EN{\q{Encrypted} PE-header}\RU{\q{Зашифрованный} PE-заголовок}}
\end{figure}

\RU{Легко увидеть визуально, что ключ это следующие 4 байта}
\EN{It's easy to spot that key is the following 4 bytes}: \TT{8C 61 D2 63}.
\RU{Используя эту информацию, довольно легко расшифровать весь файл.}
\EN{It's easy to decrypt the whole file using this information.}

\RU{Таким образом, важно помнить эти свойства PE-файлов:
1) в PE-заголовке много нулевых областей;
2) все PE-секции дополняются нулями до границы страницы (4096 байт), 
так что после всех секций обычно имеются длинные нулевые области.}
\EN{So this is important to remember these property of PE-files:
1) PE-header has many zero lacunas;
2) all PE-sections padded with zeroes by page border (4096 bytes),
so long zero lacunas usually present after all sections.}

\RU{Некоторые другие форматы файлов могут также иметь длинные нулевые области.}
\EN{Some other file formats may contain long zero lacunas.}
\RU{Это очень типично для файлов, используемых научным и инженерным ПО.}
\EN{It's very typical for files used by scientific and engineering software.}

\RU{Для тех, кто самостоятельно хочет изучить эти файлы, то их можно скачать здесь:}
\EN{For those who wants to inspect these files on one's own, they are downloadable there:}
\url{http://go.yurichev.com/17352}.

\subsection{\Exercise}

\begin{itemize}
	\item \url{http://challenges.re/50}
\end{itemize}


\fi
\input{reading}
\part*{\RU{Послесловие}\EN{Afterword}}
\addcontentsline{toc}{part}{\RU{Послесловие}\EN{Afterword}}

\chapter{\RU{Вопросы?}\EN{Questions?}}

\RU{Совершенно по любым вопросам вы можете не раздумывая писать автору}%
\EN{Do not hesitate to mail any questions to the author}: \TT{<\EMAIL>}

\EN{Any suggestions what also should be added to my book?}%
\RU{Есть идеи о том, что ещё можно добавить в эту книгу?}
 
\RU{Пожалуйста, присылайте мне информацию о замеченных ошибках (включая грамматические),}
\EN{Please, do not hesitate to send me any corrections (including grammar (you see how horrible my English is?)),}\etc.\\
\\
\RU{Автор много работает над книгой, поэтому номера страниц, листингов, \etc. очень часто меняются.}%
\EN{The author is working on the book a lot, so the page and listing numbers, \etc. are changing very rapidly.}
\RU{Пожалуйста, в своих письмах мне не ссылайтесь на номера страниц и листингов.}%
\EN{Please, do not refer to page and listing numbers in your emails to me.}
\RU{Есть метод проще: сделайте скриншот страницы, затем в графическом редакторе подчеркните место, где вы видите
ошибку, и отправьте автору. Так он может исправить её намного быстрее.}%
\EN{There is a much simpler method: make a screenshot of the page, in a graphics editor underline the place where you see the error,
and send it to me. He'll fix it much faster.}
\RU{Ну а если вы знакомы с git и \LaTeX, вы можете исправить ошибку прямо в исходных текстах:}\EN{And if you familiar with git and \LaTeX\, you can fix the error right in the source code:}\\
\href{http://go.yurichev.com/17089}{GitHub}.\\
\\
\EN{Do not worry to bother me while writing me about any petty mistakes you found, even if you are not very confident.
I'm writing for beginners, after all, so beginners' opinions and comments are crucial for my job.}
\RU{Не бойтесь побеспокоить меня написав мне о какой-то мелкой ошибке, даже если вы не очень уверены.
Я всё-таки пишу для начинающих, поэтому мнение и коментарии именно начинающих очень важны для моей работы.}




\ifdefined\LITE
\begin{center}
\vspace*{\fill}

\Huge \RU{Внимание: это сокращенная LITE-версия}\EN{Warning: this is a shortened LITE-version}\ESph{}\PTBRph{}\PLph{}\ITAph{}!
\normalsize

\bigskip
\bigskip
\bigskip

\Large
\RU{Она примерно в 6 раз короче полной версии (\textasciitilde{}150 страниц) и предназначена для тех,
кто хочет краткого введения в основы reverse engineering.
Здесь нет ничего о MIPS, ARM, OllyDBG, GCC, GDB, IDA, нет задач, примеров, \etc.}
\EN{It is approximately 6 times shorter than full version (\textasciitilde{}150 pages) and intended to those
who wants for very quick introduction to reverse engineering basics.
There are nothing about MIPS, ARM, OllyDBG, GCC, GDB, IDA, there are no exercises, examples, etc.}
\ESph{}\PTBRph{}\PLph{}\ITAph{}
\normalsize

\bigskip
\bigskip
\bigskip

\RU{Если вам всё ещё интересен reverse engineering, полная версия книги всегда доступна на моем сайте}%
\EN{If you still interesting in reverse engineering, full version of the book is always available on my website}\ESph{}\PTBRph{}\PLph{}\ITAph{}: 
\href{http://go.yurichev.com/17009}{beginners.re}.

\vspace*{\fill}
\vfill
\end{center}

\fi

\ifx\LITE\undefined
\part*{\RU{Приложение}\EN{Appendix}}
\appendix
\addcontentsline{toc}{part}{\RU{Приложение}\EN{Appendix}}

% chapters
\clearpage
\section{\RU{Простейшее четырехбайтное XOR-шифрование}\EN{Simplest possible 4-byte XOR encryption}}

\RU{Если при XOR-шифровании применялся шаблон длинее байта, например, 4-байтный, то его также легко
увидеть.}
\EN{If longer pattern was used while XOR-encryption, for example, 4 byte pattern, it's easy
to spot it as well.}
\RU{Например, вот начало файла kernel32.dll (32-битная версия из Windows Server 2008):}
\EN{As example, here is beginning of kernel32.dll file (32-bit version from Windows Server 2008):}

\begin{figure}[H]
\centering
\includegraphics[scale=\FigScale]{ff/XOR/4byte/original1.png}
\caption{\EN{Original file}\RU{Оригинальный файл}}
\end{figure}

\clearpage
\RU{Вот он же, но \q{зашифрованный} 4-байтным ключем:}
\EN{Here is it \q{encrypted} by 4-byte key:}

\begin{figure}[H]
\centering
\includegraphics[scale=\FigScale]{ff/XOR/4byte/encrypted1.png}
\caption{\EN{\q{Encrypted} file}\RU{\q{Зашифрованный} файл}}
\end{figure}

\RU{Очень легко увидеть повторяющиеся 4 символа.}
\EN{It's very easy to spot recurring 4 symbols.}
\RU{Ведь в заголовке PE-файла много длинных нулевых областей, из-за которых ключ становится видным.}
\EN{Indeed, PE-file header has a lot of long zero lacunes, which is the reason why key became visible.}

\clearpage
\RU{Вот начало PE-заголовка в 16-ричном виде:}
\EN{Here is beginning of PE-header in hexadecimal form:}

\begin{figure}[H]
\centering
\includegraphics[scale=\FigScale]{ff/XOR/4byte/original2.png}
\caption{PE-\EN{header}\RU{заголовок}}
\end{figure}

\clearpage
\RU{И вот он же, \q{зашифрованный}:}
\EN{Here is it \q{encrypted}:}

\begin{figure}[H]
\centering
\includegraphics[scale=\FigScale]{ff/XOR/4byte/encrypted2.png}
\caption{\EN{\q{Encrypted} PE-header}\RU{\q{Зашифрованный} PE-заголовок}}
\end{figure}

\RU{Легко увидеть визуально, что ключ это следующие 4 байта}
\EN{It's easy to spot that key is the following 4 bytes}: \TT{8C 61 D2 63}.
\RU{Используя эту информацию, довольно легко расшифровать весь файл.}
\EN{It's easy to decrypt the whole file using this information.}

\RU{Таким образом, важно помнить эти свойства PE-файлов:
1) в PE-заголовке много нулевых областей;
2) все PE-секции дополняются нулями до границы страницы (4096 байт), 
так что после всех секций обычно имеются длинные нулевые области.}
\EN{So this is important to remember these property of PE-files:
1) PE-header has many zero lacunas;
2) all PE-sections padded with zeroes by page border (4096 bytes),
so long zero lacunas usually present after all sections.}

\RU{Некоторые другие форматы файлов могут также иметь длинные нулевые области.}
\EN{Some other file formats may contain long zero lacunas.}
\RU{Это очень типично для файлов, используемых научным и инженерным ПО.}
\EN{It's very typical for files used by scientific and engineering software.}

\RU{Для тех, кто самостоятельно хочет изучить эти файлы, то их можно скачать здесь:}
\EN{For those who wants to inspect these files on one's own, they are downloadable there:}
\url{http://go.yurichev.com/17352}.

\subsection{\Exercise}

\begin{itemize}
	\item \url{http://challenges.re/50}
\end{itemize}


\ifdefined\IncludeARM
\clearpage
\section{\RU{Простейшее четырехбайтное XOR-шифрование}\EN{Simplest possible 4-byte XOR encryption}}

\RU{Если при XOR-шифровании применялся шаблон длинее байта, например, 4-байтный, то его также легко
увидеть.}
\EN{If longer pattern was used while XOR-encryption, for example, 4 byte pattern, it's easy
to spot it as well.}
\RU{Например, вот начало файла kernel32.dll (32-битная версия из Windows Server 2008):}
\EN{As example, here is beginning of kernel32.dll file (32-bit version from Windows Server 2008):}

\begin{figure}[H]
\centering
\includegraphics[scale=\FigScale]{ff/XOR/4byte/original1.png}
\caption{\EN{Original file}\RU{Оригинальный файл}}
\end{figure}

\clearpage
\RU{Вот он же, но \q{зашифрованный} 4-байтным ключем:}
\EN{Here is it \q{encrypted} by 4-byte key:}

\begin{figure}[H]
\centering
\includegraphics[scale=\FigScale]{ff/XOR/4byte/encrypted1.png}
\caption{\EN{\q{Encrypted} file}\RU{\q{Зашифрованный} файл}}
\end{figure}

\RU{Очень легко увидеть повторяющиеся 4 символа.}
\EN{It's very easy to spot recurring 4 symbols.}
\RU{Ведь в заголовке PE-файла много длинных нулевых областей, из-за которых ключ становится видным.}
\EN{Indeed, PE-file header has a lot of long zero lacunes, which is the reason why key became visible.}

\clearpage
\RU{Вот начало PE-заголовка в 16-ричном виде:}
\EN{Here is beginning of PE-header in hexadecimal form:}

\begin{figure}[H]
\centering
\includegraphics[scale=\FigScale]{ff/XOR/4byte/original2.png}
\caption{PE-\EN{header}\RU{заголовок}}
\end{figure}

\clearpage
\RU{И вот он же, \q{зашифрованный}:}
\EN{Here is it \q{encrypted}:}

\begin{figure}[H]
\centering
\includegraphics[scale=\FigScale]{ff/XOR/4byte/encrypted2.png}
\caption{\EN{\q{Encrypted} PE-header}\RU{\q{Зашифрованный} PE-заголовок}}
\end{figure}

\RU{Легко увидеть визуально, что ключ это следующие 4 байта}
\EN{It's easy to spot that key is the following 4 bytes}: \TT{8C 61 D2 63}.
\RU{Используя эту информацию, довольно легко расшифровать весь файл.}
\EN{It's easy to decrypt the whole file using this information.}

\RU{Таким образом, важно помнить эти свойства PE-файлов:
1) в PE-заголовке много нулевых областей;
2) все PE-секции дополняются нулями до границы страницы (4096 байт), 
так что после всех секций обычно имеются длинные нулевые области.}
\EN{So this is important to remember these property of PE-files:
1) PE-header has many zero lacunas;
2) all PE-sections padded with zeroes by page border (4096 bytes),
so long zero lacunas usually present after all sections.}

\RU{Некоторые другие форматы файлов могут также иметь длинные нулевые области.}
\EN{Some other file formats may contain long zero lacunas.}
\RU{Это очень типично для файлов, используемых научным и инженерным ПО.}
\EN{It's very typical for files used by scientific and engineering software.}

\RU{Для тех, кто самостоятельно хочет изучить эти файлы, то их можно скачать здесь:}
\EN{For those who wants to inspect these files on one's own, they are downloadable there:}
\url{http://go.yurichev.com/17352}.

\subsection{\Exercise}

\begin{itemize}
	\item \url{http://challenges.re/50}
\end{itemize}


\fi
\ifdefined\IncludeMIPS
\chapter{MIPS}

\section{\Registers}
\label{MIPS_registers_ref}

\index{MIPS!O32}
( \RU{Соглашение о вызовах O32}\EN{O32 calling convention} )

\subsection{\RU{Регистры общего пользования}\EN{General purpose registers} \ac{GPR}}

\begin{center}
\begin{tabular}{ | l | l | l | }
\hline
\cellcolor{blue!25} \RU{Номер}\EN{Number} & 
\cellcolor{blue!25} \RU{Псевдоимя}\EN{Pseudoname} & 
\cellcolor{blue!25} \RU{Описание}\EN{Description} \\
\hline
\$0             & \$ZERO          & \RU{Всегда ноль. Запись в этот регистр работает как холостая инструкция}\EN{Always zero. Writing to this register is effectively an idle instruction} (\ac{NOP}). \\
\hline
\$1             & \$AT            & \RU{Используется как временный регистр для ассемблерных макросов и псевдоинструкций}\EN{Used as a temporary register for assembly macros and pseudoinstructions}. \\
\hline
\$2 \dots \$3   & \$V0 \dots \$V1 & \RU{Здесь возвращается результат функции}\EN{Function result is returned here}. \\
\hline
\$4 \dots \$7   & \$A0 \dots \$A3 & \RU{Аргументы функции}\EN{Function arguments}. \\
\hline
\$8 \dots \$15  & \$T0 \dots \$T7 & \RU{Используется для временных данных}\EN{Used for temporary data}. \\
\hline
\$16 \dots \$23 & \$S0 \dots \$S7 & \RU{Используется для временных данных}\EN{Used for temporary data}\AsteriskOne{}. \\
\hline
\$24 \dots \$25 & \$T8 \dots \$T9 & \RU{Используется для временных данных}\EN{Used for temporary data}. \\
\hline
\$26 \dots \$27 & \$K0 \dots \$K1 & \RU{Зарезервировано для ядра \ac{OS}}\EN{Reserved for \ac{OS} kernel}. \\
\hline
\$28            & \$GP            & \RU{Глобальный указатель}\EN{Global Pointer}\AsteriskTwo{}. \\
\hline
\$29            & \$SP            & \ac{SP}\AsteriskOne{}. \\
\hline
\$30            & \$FP            & \ac{FP}\AsteriskOne{}. \\
\hline
\$31            & \$RA            & \ac{RA}. \\
\hline
n/a             & PC              & \ac{PC}. \\
\hline
n/a             & HI              & \RU{старшие 32 бита результата умножения или остаток от деления}\EN{high 32 bit of multiplication or division remainder}\AsteriskThree{}. \\
\hline
n/a             & LO              & \RU{младшие 32 бита результата умножения или результат деления}\EN{low 32 bit of multiplication and division remainder}\AsteriskThree{}. \\
\hline
\end{tabular}
\end{center}

\subsection{\RU{Регистры для работы с числами с плавающей точкой}\EN{Floating-point registers}}
\label{MIPS_FPU_registers}

\begin{center}
\begin{tabular}{ | l | l | l | }
\hline
\cellcolor{blue!25} \RU{Название}\EN{Name} & \cellcolor{blue!25} \RU{Описание}\EN{Description} \\
\hline
\$F0..\$F1   & \RU{Здесь возвращается результат функции}\EN{Function result returned here}. \\
\hline
\$F2..\$F3   & \RU{Не используется}\EN{Not used}. \\
\hline
\$F4..\$F11  & \RU{Используется для временных данных}\EN{Used for temporary data}. \\
\hline
\$F12..\$F15 & \RU{Первые два аргумента функции}\EN{First two function arguments}. \\
\hline
\$F16..\$F19 & \RU{Используется для временных данных}\EN{Used for temporary data}. \\
\hline
\$F20..\$F31 & \RU{Используется для временных данных}\EN{Used for temporary data}\AsteriskOne{}. \\
\hline
%fcr31 & Control/status register. \\
%\hline
\end{tabular}
\end{center}

\AsteriskOne{}\EMDASH{}\Gls{callee} \RU{должен сохранять}\EN{must preserve the value}.\\
\AsteriskTwo{}\EMDASH{}\Gls{callee} \RU{должен сохранять}\EN{must preserve the value} (\RU{кроме \ac{PIC}-кода}
\EN{except in \ac{PIC} code}).\\
\index{MIPS!\Instructions!MFLO}
\index{MIPS!\Instructions!MFHI}
\AsteriskThree{}\EMDASH{}\RU{доступны используя инструкции}\EN{accessible using the} \TT{MFHI} \AndENRU \TT{MFLO}\EN{ instructions}.\\

\section{\Instructions}

\RU{Есть три типа инструкций}\EN{There are 3 kinds of instructions}:

\begin{itemize}

\item \RU{Тип R: имеющие 3 регистра. R-инструкции обычно имеют такой вид:}
\EN{R-type: those which have 3 registers. R-instruction usually have the following form:}

\begin{lstlisting}
instruction destination, source1, source2
\end{lstlisting}

\RU{Важно помнить что если первый и второй регистр один и тот же, IDA может показать инструкцию в сокращенной
форме:}
\EN{One important thing to remember is that when the first and second register are the same, 
IDA may show the instruction in its shorter form:}

\begin{lstlisting}
instruction destination/source1, source2
\end{lstlisting}

\RU{Это немного напоминает Интеловский синтаксис ассемблера x86.}
\EN{That somewhat reminds us of the Intel syntax for x86 assembly language.}

\item \RU{Тип I: имеющие 2 регистра и 16-битное \q{immediate}-значение.}
\EN{I-type: those which have 2 registers and a 16-bit immediate value.}

\item \RU{Тип J: инструкции перехода, имеют 26 бит для кодирования смещения.}
\EN{J-type: jump/branch instructions, have 26 bits for encoding the offset.}

\end{itemize}

\subsection{\RU{Инструкции перехода}\EN{Jump instructions}}

\RU{Какая разница между инструкциями начинающихся с B- (BEQ, B, \etc{}.) и с J- (JAL, JALR, \etc{}.)?}
\EN{What is the difference between B- instructions (BEQ, B, \etc{}) and J- ones (JAL, JALR, \etc{})?}

\RU{B-инструкции имеют тип I, так что, смещение в этих инструкциях кодируется как 16-битное значение.}
\EN{The B-instructions have an I-type, hence, the B-instructions' offset is encoded as a 16-bit immediate.}
\RU{Инструкции JR и JALR имеют тип R, и они делают переход по абсолютному адресу указанному в регистре.}
\EN{JR and JALR are R-type and jump to an absolute address specified in a register.}
\RU{J и JAL имеют тип J, так что смещение кодируется как 26-битное значение.}
\EN{J and JAL are J-type, hence the offset is encoded as a 26-bit immediate.}

\RU{Коротко говоря, в B-инструкциях можно кодировать условие 
(B на самом деле это псевдоинструкция для \TT{BEQ \$ZERO, \$ZERO, LABEL}),
а в J-инструкциях нельзя.}
\EN{In short, B-instructions can encode a condition 
(B is in fact pseudoinstruction for \TT{BEQ \$ZERO, \$ZERO, LABEL}), 
while J-instructions can't.}

\fi
\ifx\LITE\undefined
\input{appendix/GCC_library}
\input{appendix/MSVC_library}
\input{appendix/cheatsheets}
\fi

\fi
\part*{\RU{Список принятых сокращений}\EN{Acronyms used}\ESph{}\PTBRph{}\PLph{}\ITAph{}}
\addcontentsline{toc}{part}{\RU{Список принятых сокращений}\EN{Acronyms used}\ESph{}\PTBRph{}\PLph{}\ITAph{}}
\begin{acronym}
\RU{
	\acro{OS}[ОС]{Операционная Система}
	\acro{FAQ}[ЧаВО]{Часто задаваемые вопросы}
	\acro{OOP}[ООП]{Объектно-Ориентированное Программирование}
	\acro{PL}[ЯП]{Язык Программирования}
	\acro{PRNG}[ГПСЧ]{Генератор псевдослучайных чисел}
	\acro{ROM}[ПЗУ]{Постоянное запоминающее устройство}
	\acro{ALU}[АЛУ]{Арифметико-логическое устройство}
}
\EN{
	\acro{OS}{Operating System}
	\acro{FAQ}{Frequently Asked Questions}
	\acro{OOP}{Object-Oriented Programming}
	\acro{PL}{Programming language}
	\acro{PRNG}{Pseudorandom number generator}
	\acro{ROM}{Read-only memory}
	\acro{ALU}{Arithmetic logic unit}
}
\ES{
	\acro{OS}{\ESph{}}
	\acro{FAQ}{\ESph{}}
	\acro{OOP}{\ESph{}}
	\acro{PL}{\ESph{}}
	\acro{PRNG}{\ESph{}}
	\acro{ROM}{\ESph{}}
	\acro{ALU}{\ESph{}}
}
\PTBR{
	\acro{OS}{\PTBRph{}}
	\acro{FAQ}{\PTBRph{}}
	\acro{OOP}{\PTBRph{}}
	\acro{PL}{\PTBRph{}}
	\acro{PRNG}{\PTBRph{}}
	\acro{ROM}{\PTBRph{}}
	\acro{ALU}{\PTBRph{}}
}
\PL{
	\acro{OS}{\PLph{}}
	\acro{FAQ}{\PLph{}}
	\acro{OOP}{\PLph{}}
	\acro{PL}{\PLph{}}
	\acro{PRNG}{\PLph{}}
	\acro{ROM}{\PLph{}}
	\acro{ALU}{\PLph{}}
}
\ITA{
	\acro{OS}{\ITAph{}}
	\acro{FAQ}{\ITAph{}}
	\acro{OOP}{\ITAph{}}
	\acro{PL}{\ITAph{}}
	\acro{PRNG}{\ITAph{}}
	\acro{ROM}{\ITAph{}}
	\acro{ALU}{\ITAph{}}
}
\acro{RA}{\RU{Адрес возврата}\EN{Return Address}\PTBRph{}\ESph{}\PLph{}\ITAph{}}
\acro{PE}{Portable Executable: \myref{win32_pe}}
\acro{SP}{\gls{stack pointer}. SP/ESP/RSP \InENRU x86/x64. SP \InENRU ARM.}
\acro{DLL}{Dynamic-link library}
\acro{PC}{Program Counter. IP/EIP/RIP \InENRU x86/64. PC \InENRU ARM.}
\acro{LR}{Link Register}
\acro{IDA}{
	\RU{Интерактивный дизассемблер и отладчик, разработан \href{https://hex-rays.com/}{Hex-Rays}}
	\EN{Interactive Disassembler and debugger developed by \href{https://hex-rays.com/}{Hex-Rays}}
	\ESph{}
	\PTBRph{}
	\PLph{}
	\ITAph{}
}
\acro{IAT}{Import Address Table}
\acro{INT}{Import Name Table}
\acro{RVA}{Relative Virtual Address}
\acro{VA}{Virtual Address}
\acro{OEP}{Original Entry Point}
\acro{MSVC}{Microsoft Visual C++}
\acro{MSVS}{Microsoft Visual Studio}
\acro{ASLR}{Address Space Layout Randomization}
\acro{MFC}{Microsoft Foundation Classes}
\acro{TLS}{Thread Local Storage}
\acro{AKA}{Also Known As\RU{ (Также известный как)}\ESph{}\PTBRph{}\PLph{}\ITAph{}}
\acro{CRT}{C runtime library
\ifx\LITE\undefined
: \myref{sec:CRT}
\fi
}
\acro{CPU}{Central processing unit}
\acro{FPU}{Floating-point unit}
\acro{CISC}{Complex instruction set computing}
\acro{RISC}{Reduced instruction set computing}
\acro{GUI}{Graphical user interface}
\acro{RTTI}{Run-time type information}
\acro{BSS}{Block Started by Symbol}
\acro{SIMD}{Single instruction, multiple data}
\acro{BSOD}{Black Screen of Death}
\acro{DBMS}{Database management systems}
\acro{ISA}{Instruction Set Architecture\RU{ (Архитектура набора команд)}}
\acro{CGI}{Common Gateway Interface}
\acro{HPC}{High-Performance Computing}
\acro{SOC}{System on Chip}
\acro{SEH}{Structured Exception Handling
\ifx\LITE\undefined
: \myref{sec:SEH}
\fi
}
\acro{ELF}{\RU{Формат исполняемых файлов, использующийся в Linux и некоторых других *NIX}
	\EN{Executable file format widely used in *NIX systems including Linux}\ESph{}\PTBRph{}\PLph{}\ITAph{}}
\acro{TIB}{Thread Information Block}
\acro{TEA}{Tiny Encryption Algorithm}
\acro{PIC}{Position Independent Code: \myref{sec:PIC}}
\acro{NAN}{Not a Number}
\acro{NOP}{No OPeration}
\acro{BEQ}{(PowerPC, ARM) Branch if Equal}
\acro{BNE}{(PowerPC, ARM) Branch if Not Equal}
\acro{BLR}{(PowerPC) Branch to Link Register}
\acro{XOR}{eXclusive OR\RU{ (исключающее \q{ИЛИ})}}
\acro{MCU}{Microcontroller unit}
\acro{RAM}{Random-access memory}
\acro{EGA}{Enhanced Graphics Adapter}
\acro{VGA}{Video Graphics Array}
\acro{API}{Application programming interface}
\acro{ASCII}{American Standard Code for Information Interchange}
\acro{ASCIIZ}{ASCII Zero (\RU{ASCII-строка заканчивающаяся нулем}\EN{null-terminated ASCII string})}
\acro{IA64}{Intel Architecture 64 (Itanium): \myref{itanium}}
\acro{EPIC}{Explicitly parallel instruction computing}
\acro{OOE}{Out-of-order execution}
\acro{MSDN}{Microsoft Developer Network}
\acro{MSB}{Most significant bit/byte\RU{ (самый старший бит/байт)}}
\acro{LSB}{Least significant bit/byte\RU{ (самый младший бит/байт)}}
\acro{STL}{(\Cpp) Standard Template Library: \myref{sec:STL}}
\acro{PODT}{(\Cpp) Plain Old Data Type}
\acro{HDD}{Hard disk drive}
\acro{VM}{Virtual Memory\RU{ (виртуальная память)}}
\acro{WRK}{Windows Research Kernel}
\acro{GPR}{General Purpose Registers\RU{ (регистры общего пользования)}}
\acro{SSDT}{System Service Dispatch Table}
\acro{RE}{Reverse Engineering}
\acro{SSE}{Streaming SIMD Extensions}
\acro{BCD}{Binary-coded decimal}
\acro{BOM}{Byte order mark}
\acro{GDB}{GNU debugger}
\acro{FP}{Frame Pointer}
\acro{MBR}{Master Boot Record}
\acro{JPE}{Jump Parity Even (\RU{инструкция x86}\EN{x86 instruction})}
\acro{CIDR}{Classless Inter-Domain Routing}
\acro{STMFD}{Store Multiple Full Descending (\RU{инструкция ARM}\EN{ARM instruction})}
\acro{LDMFD}{Load Multiple Full Descending (\RU{инструкция ARM}\EN{ARM instruction})}
\acro{STMED}{Store Multiple Empty Descending (\RU{инструкция ARM}\EN{ARM instruction})}
\acro{LDMED}{Load Multiple Empty Descending (\RU{инструкция ARM}\EN{ARM instruction})}
\acro{STMFA}{Store Multiple Full Ascending (\RU{инструкция ARM}\EN{ARM instruction})}
\acro{LDMFA}{Load Multiple Full Ascending (\RU{инструкция ARM}\EN{ARM instruction})}
\acro{STMEA}{Store Multiple Empty Ascending (\RU{инструкция ARM}\EN{ARM instruction})}
\acro{LDMEA}{Load Multiple Empty Ascending (\RU{инструкция ARM}\EN{ARM instruction})}
\acro{APSR}{(ARM) Application Program Status Register}
\acro{FPSCR}{(ARM) Floating-Point Status and Control Register}
\acro{PID}{\RU{ID программы/процесса}\EN{Program/process ID}\ESph{}\PTBRph{}\PLph{}\ITAph{}}
\acro{LF}{Line feed\RU{ (подача строки)} (10 \OrENRU '\textbackslash{}n' \InENRU \CCpp)}
\acro{CR}{Carriage return\RU{ (возврат каретки)} (13 \OrENRU '\textbackslash{}r' \InENRU \CCpp)}
\acro{RFC}{Request for Comments}
\acro{TOS}{Top Of Stack\RU{ (вершина стека)}}
\acro{LVA}{(Java) Local Variable Array\RU{ (массив локальных переменных)}}
\acro{JVM}{Java virtual machine}
\acro{JIT}{Just-in-time compilation}
\acro{EOF}{End of file\RU{ (конец файла)}}
\end{acronym}


\bookmarksetup{startatroot}

\clearpage
\phantomsection
\addcontentsline{toc}{chapter}{\RU{Глоссарий}\EN{Glossary}\PTBRph{}\ESph{}\PLph{}\ITAph{}}
\printglossaries

\clearpage
\phantomsection
\printindex

\clearpage
\phantomsection
\addcontentsline{toc}{chapter}{\RU{Библиография}\EN{Bibliography}\PTBRph{}\ESph{}\PLph{}\ITAph{}}
\printbibliography

\end{document}
