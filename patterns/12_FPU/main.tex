\chapter{\FPUChapterName}
\label{sec:FPU}

\newcommand{\FNURLSTACK}{\footnote{\href{http://go.yurichev.com/17123}{wikipedia.org/wiki/Stack\_machine}}}
\newcommand{\FNURLFORTH}{\footnote{\href{http://go.yurichev.com/17124}{wikipedia.org/wiki/Forth\_(programming\_language)}}}
\newcommand{\FNURLIEEE}{\footnote{\href{http://go.yurichev.com/17125}{wikipedia.org/wiki/IEEE\_floating\_point}}}
\newcommand{\FNURLSP}{\footnote{\href{http://go.yurichev.com/17126}{wikipedia.org/wiki/Single-precision\_floating-point\_format}}}
\newcommand{\FNURLDP}{\footnote{\href{http://go.yurichev.com/17127}{wikipedia.org/wiki/Double-precision\_floating-point\_format}}}
\newcommand{\FNURLEP}{\footnote{\href{http://go.yurichev.com/17128}{wikipedia.org/wiki/Extended\_precision}}}

\RU{\ac{FPU}\EMDASH блок в процессоре работающий с числами с плавающей запятой.}
\EN{The \ac{FPU} is a device within the main \ac{CPU}, specially designed to deal with floating point numbers.}
\RU{Раньше он назывался \q{сопроцессором} и он стоит немного в стороне от \ac{CPU}.}
\EN{It was called \q{coprocessor} in the past and it stays somewhat aside of the main \ac{CPU}.}

\section{IEEE 754}

\RU{Число с плавающей точкой в формате IEEE 754 состоит из \IT{знака}, \IT{мантиссы}\footnote{\IT{significand} или \IT{fraction} 
в англоязычной литературе} и \IT{экспоненты}.}
\EN{A number in the IEEE 754 format consists of a \IT{sign}, a \IT{significand} (also called \IT{fraction}) and an \IT{exponent}.}

\section{x86}

\RU{Перед изучением \ac{FPU} в x86 полезно ознакомиться с тем как работают стековые машины\FNURLSTACK 
или ознакомиться с основами языка Forth\FNURLFORTH.}
\EN{It is worth looking into stack machines\FNURLSTACK or learning the basics of the Forth language\FNURLFORTH,
before studying the \ac{FPU} in x86.}

\index{Intel!80486}
\index{Intel!FPU}
\RU{Интересен факт, что в свое время (до 80486) сопроцессор был отдельным чипом на материнской плате, 
и вследствие его высокой цены, он не всегда присутствовал. Его можно было докупить и установить отдельно}%
\EN{It is interesting to know that in the past (before the 80486 CPU) the coprocessor was a separate chip 
and it was not always pre-installed on the motherboard. It was possible to buy it separately and install it}%
\footnote{\RU{Например, Джон Кармак использовал в своей игре Doom числа с фиксированной запятой 
(\href{http://go.yurichev.com/17357}{ru.wikipedia.org/wiki/Число\_с\_фиксированной\_запятой}), хранящиеся
в обычных 32-битных \ac{GPR} (16 бит на целую часть и 16 на дробную),
чтобы Doom работал на 32-битных компьютерах без FPU, т.е. 80386 и 80486 SX.}
\EN{For example, John Carmack used fixed-point arithmetic 
(\href{http://go.yurichev.com/17356}{wikipedia.org/wiki/Fixed-point\_arithmetic}) values in his Doom video game, stored in 
32-bit \ac{GPR} registers (16 bit for integral part and another 16 bit for fractional part), so Doom
could work on 32-bit computers without FPU, i.e., 80386 and 80486 SX.}}.
\RU{Начиная с 80486 DX в состав процессора всегда входит FPU.}
\EN{Starting with the 80486 DX CPU, the \ac{FPU} is integrated in the \ac{CPU}.}

\index{x86!\Instructions!FWAIT}
\RU{Этот факт может напоминать такой рудимент как наличие инструкции \TT{FWAIT}, 
которая заставляет
\ac{CPU} ожидать, пока \ac{FPU} закончит работу}\EN{The \TT{FWAIT} instruction reminds us of that fact---it
switches the \ac{CPU} to a waiting state, so it can wait until the \ac{FPU} is done with its work}.
\RU{Другой рудимент это тот факт, что опкоды \ac{FPU}-инструкций начинаются с т.н. \q{escape}-опкодов 
(\TT{D8..DF}) как опкоды, передающиеся в отдельный сопроцессор.}
\EN{Another rudiment is the fact that the \ac{FPU} instruction 
opcodes start with the so called \q{escape}-opcodes (\TT{D8..DF}), i.e., 
opcodes passed to a separate coprocessor.}

\index{IEEE 754}
\label{FPU_is_stack}
\RU{FPU имеет стек из восьми 80-битных регистров:}
\EN{The FPU has a stack capable to holding 8 80-bit registers, and each register can hold a number 
in the IEEE 754\FNURLIEEE format.}
\RU{\ST{0}..\ST{7}. Для краткости, IDA и \olly отображают \ST{0} как \TT{ST},
что в некоторых учебниках и документациях означает \q{Stack Top} (\q{вершина стека}).}
\RU{Каждый регистр может содержать число в формате IEEE 754\FNURLIEEE.}
\EN{They are \ST{0}..\ST{7}. For brevity, IDA and \olly show \ST{0} as \TT{ST}, 
which is represented in some textbooks and manuals as \q{Stack Top}.}

\section{ARM, MIPS, x86/x64 SIMD}

\RU{В ARM и MIPS FPU это не стек, а просто набор регистров.}
\EN{In ARM and MIPS the FPU is not a stack, but a set of registers.}
\RU{Такая же идеология применяется в расширениях SIMD в процессорах x86/x64.}
\EN{The same ideology is used in the SIMD extensions of x86/x64 CPUs.}

\section{\CCpp}

\index{float}
\index{double}
\RU{В стандартных \CCpp имеются два типа для работы с числами с плавающей запятой: 
\Tfloat (\IT{число одинарной точности}\FNURLSP, 32 бита)
\footnote{Формат представления чисел с плавающей точкой одинарной точности затрагивается в разделе 
\IT{\WorkingWithFloatAsWithStructSubSubSectionName}~(\myref{sec:floatasstruct}).}
и \Tdouble (\IT{число двойной точности}\FNURLDP, 64 бита).}
\EN{The standard \CCpp languages offer at least two floating number types, \Tfloat (\IT{single-precision}\FNURLSP, 32 bits)
\footnote{the single precision floating point number format is also addressed in 
the \IT{\WorkingWithFloatAsWithStructSubSubSectionName}~(\myref{sec:floatasstruct}) section}
and \Tdouble (\IT{double-precision}\FNURLDP, 64 bits).}

\index{long double}
\RU{GCC также поддерживает тип \IT{long double} (\IT{extended precision}\FNURLEP, 80 бит), но MSVC~--- нет.}
\EN{GCC also supports the \IT{long double} type (\IT{extended precision}\FNURLEP, 80 bit), which MSVC doesn't.}

\RU{Несмотря на то, что \Tfloat занимает столько же места, сколько и \Tint на 32-битной архитектуре, 
представление чисел, разумеется, совершенно другое.}
\EN{The \Tfloat type requires the same number of bits as the \Tint type in 32-bit environments, 
but the number representation is completely different.}

\clearpage
\section{\RU{Простейшее четырехбайтное XOR-шифрование}\EN{Simplest possible 4-byte XOR encryption}}

\RU{Если при XOR-шифровании применялся шаблон длинее байта, например, 4-байтный, то его также легко
увидеть.}
\EN{If longer pattern was used while XOR-encryption, for example, 4 byte pattern, it's easy
to spot it as well.}
\RU{Например, вот начало файла kernel32.dll (32-битная версия из Windows Server 2008):}
\EN{As example, here is beginning of kernel32.dll file (32-bit version from Windows Server 2008):}

\begin{figure}[H]
\centering
\includegraphics[scale=\FigScale]{ff/XOR/4byte/original1.png}
\caption{\EN{Original file}\RU{Оригинальный файл}}
\end{figure}

\clearpage
\RU{Вот он же, но \q{зашифрованный} 4-байтным ключем:}
\EN{Here is it \q{encrypted} by 4-byte key:}

\begin{figure}[H]
\centering
\includegraphics[scale=\FigScale]{ff/XOR/4byte/encrypted1.png}
\caption{\EN{\q{Encrypted} file}\RU{\q{Зашифрованный} файл}}
\end{figure}

\RU{Очень легко увидеть повторяющиеся 4 символа.}
\EN{It's very easy to spot recurring 4 symbols.}
\RU{Ведь в заголовке PE-файла много длинных нулевых областей, из-за которых ключ становится видным.}
\EN{Indeed, PE-file header has a lot of long zero lacunes, which is the reason why key became visible.}

\clearpage
\RU{Вот начало PE-заголовка в 16-ричном виде:}
\EN{Here is beginning of PE-header in hexadecimal form:}

\begin{figure}[H]
\centering
\includegraphics[scale=\FigScale]{ff/XOR/4byte/original2.png}
\caption{PE-\EN{header}\RU{заголовок}}
\end{figure}

\clearpage
\RU{И вот он же, \q{зашифрованный}:}
\EN{Here is it \q{encrypted}:}

\begin{figure}[H]
\centering
\includegraphics[scale=\FigScale]{ff/XOR/4byte/encrypted2.png}
\caption{\EN{\q{Encrypted} PE-header}\RU{\q{Зашифрованный} PE-заголовок}}
\end{figure}

\RU{Легко увидеть визуально, что ключ это следующие 4 байта}
\EN{It's easy to spot that key is the following 4 bytes}: \TT{8C 61 D2 63}.
\RU{Используя эту информацию, довольно легко расшифровать весь файл.}
\EN{It's easy to decrypt the whole file using this information.}

\RU{Таким образом, важно помнить эти свойства PE-файлов:
1) в PE-заголовке много нулевых областей;
2) все PE-секции дополняются нулями до границы страницы (4096 байт), 
так что после всех секций обычно имеются длинные нулевые области.}
\EN{So this is important to remember these property of PE-files:
1) PE-header has many zero lacunas;
2) all PE-sections padded with zeroes by page border (4096 bytes),
so long zero lacunas usually present after all sections.}

\RU{Некоторые другие форматы файлов могут также иметь длинные нулевые области.}
\EN{Some other file formats may contain long zero lacunas.}
\RU{Это очень типично для файлов, используемых научным и инженерным ПО.}
\EN{It's very typical for files used by scientific and engineering software.}

\RU{Для тех, кто самостоятельно хочет изучить эти файлы, то их можно скачать здесь:}
\EN{For those who wants to inspect these files on one's own, they are downloadable there:}
\url{http://go.yurichev.com/17352}.

\subsection{\Exercise}

\begin{itemize}
	\item \url{http://challenges.re/50}
\end{itemize}


\clearpage
\section{\RU{Простейшее четырехбайтное XOR-шифрование}\EN{Simplest possible 4-byte XOR encryption}}

\RU{Если при XOR-шифровании применялся шаблон длинее байта, например, 4-байтный, то его также легко
увидеть.}
\EN{If longer pattern was used while XOR-encryption, for example, 4 byte pattern, it's easy
to spot it as well.}
\RU{Например, вот начало файла kernel32.dll (32-битная версия из Windows Server 2008):}
\EN{As example, here is beginning of kernel32.dll file (32-bit version from Windows Server 2008):}

\begin{figure}[H]
\centering
\includegraphics[scale=\FigScale]{ff/XOR/4byte/original1.png}
\caption{\EN{Original file}\RU{Оригинальный файл}}
\end{figure}

\clearpage
\RU{Вот он же, но \q{зашифрованный} 4-байтным ключем:}
\EN{Here is it \q{encrypted} by 4-byte key:}

\begin{figure}[H]
\centering
\includegraphics[scale=\FigScale]{ff/XOR/4byte/encrypted1.png}
\caption{\EN{\q{Encrypted} file}\RU{\q{Зашифрованный} файл}}
\end{figure}

\RU{Очень легко увидеть повторяющиеся 4 символа.}
\EN{It's very easy to spot recurring 4 symbols.}
\RU{Ведь в заголовке PE-файла много длинных нулевых областей, из-за которых ключ становится видным.}
\EN{Indeed, PE-file header has a lot of long zero lacunes, which is the reason why key became visible.}

\clearpage
\RU{Вот начало PE-заголовка в 16-ричном виде:}
\EN{Here is beginning of PE-header in hexadecimal form:}

\begin{figure}[H]
\centering
\includegraphics[scale=\FigScale]{ff/XOR/4byte/original2.png}
\caption{PE-\EN{header}\RU{заголовок}}
\end{figure}

\clearpage
\RU{И вот он же, \q{зашифрованный}:}
\EN{Here is it \q{encrypted}:}

\begin{figure}[H]
\centering
\includegraphics[scale=\FigScale]{ff/XOR/4byte/encrypted2.png}
\caption{\EN{\q{Encrypted} PE-header}\RU{\q{Зашифрованный} PE-заголовок}}
\end{figure}

\RU{Легко увидеть визуально, что ключ это следующие 4 байта}
\EN{It's easy to spot that key is the following 4 bytes}: \TT{8C 61 D2 63}.
\RU{Используя эту информацию, довольно легко расшифровать весь файл.}
\EN{It's easy to decrypt the whole file using this information.}

\RU{Таким образом, важно помнить эти свойства PE-файлов:
1) в PE-заголовке много нулевых областей;
2) все PE-секции дополняются нулями до границы страницы (4096 байт), 
так что после всех секций обычно имеются длинные нулевые области.}
\EN{So this is important to remember these property of PE-files:
1) PE-header has many zero lacunas;
2) all PE-sections padded with zeroes by page border (4096 bytes),
so long zero lacunas usually present after all sections.}

\RU{Некоторые другие форматы файлов могут также иметь длинные нулевые области.}
\EN{Some other file formats may contain long zero lacunas.}
\RU{Это очень типично для файлов, используемых научным и инженерным ПО.}
\EN{It's very typical for files used by scientific and engineering software.}

\RU{Для тех, кто самостоятельно хочет изучить эти файлы, то их можно скачать здесь:}
\EN{For those who wants to inspect these files on one's own, they are downloadable there:}
\url{http://go.yurichev.com/17352}.

\subsection{\Exercise}

\begin{itemize}
	\item \url{http://challenges.re/50}
\end{itemize}


\clearpage
\section{\RU{Простейшее четырехбайтное XOR-шифрование}\EN{Simplest possible 4-byte XOR encryption}}

\RU{Если при XOR-шифровании применялся шаблон длинее байта, например, 4-байтный, то его также легко
увидеть.}
\EN{If longer pattern was used while XOR-encryption, for example, 4 byte pattern, it's easy
to spot it as well.}
\RU{Например, вот начало файла kernel32.dll (32-битная версия из Windows Server 2008):}
\EN{As example, here is beginning of kernel32.dll file (32-bit version from Windows Server 2008):}

\begin{figure}[H]
\centering
\includegraphics[scale=\FigScale]{ff/XOR/4byte/original1.png}
\caption{\EN{Original file}\RU{Оригинальный файл}}
\end{figure}

\clearpage
\RU{Вот он же, но \q{зашифрованный} 4-байтным ключем:}
\EN{Here is it \q{encrypted} by 4-byte key:}

\begin{figure}[H]
\centering
\includegraphics[scale=\FigScale]{ff/XOR/4byte/encrypted1.png}
\caption{\EN{\q{Encrypted} file}\RU{\q{Зашифрованный} файл}}
\end{figure}

\RU{Очень легко увидеть повторяющиеся 4 символа.}
\EN{It's very easy to spot recurring 4 symbols.}
\RU{Ведь в заголовке PE-файла много длинных нулевых областей, из-за которых ключ становится видным.}
\EN{Indeed, PE-file header has a lot of long zero lacunes, which is the reason why key became visible.}

\clearpage
\RU{Вот начало PE-заголовка в 16-ричном виде:}
\EN{Here is beginning of PE-header in hexadecimal form:}

\begin{figure}[H]
\centering
\includegraphics[scale=\FigScale]{ff/XOR/4byte/original2.png}
\caption{PE-\EN{header}\RU{заголовок}}
\end{figure}

\clearpage
\RU{И вот он же, \q{зашифрованный}:}
\EN{Here is it \q{encrypted}:}

\begin{figure}[H]
\centering
\includegraphics[scale=\FigScale]{ff/XOR/4byte/encrypted2.png}
\caption{\EN{\q{Encrypted} PE-header}\RU{\q{Зашифрованный} PE-заголовок}}
\end{figure}

\RU{Легко увидеть визуально, что ключ это следующие 4 байта}
\EN{It's easy to spot that key is the following 4 bytes}: \TT{8C 61 D2 63}.
\RU{Используя эту информацию, довольно легко расшифровать весь файл.}
\EN{It's easy to decrypt the whole file using this information.}

\RU{Таким образом, важно помнить эти свойства PE-файлов:
1) в PE-заголовке много нулевых областей;
2) все PE-секции дополняются нулями до границы страницы (4096 байт), 
так что после всех секций обычно имеются длинные нулевые области.}
\EN{So this is important to remember these property of PE-files:
1) PE-header has many zero lacunas;
2) all PE-sections padded with zeroes by page border (4096 bytes),
so long zero lacunas usually present after all sections.}

\RU{Некоторые другие форматы файлов могут также иметь длинные нулевые области.}
\EN{Some other file formats may contain long zero lacunas.}
\RU{Это очень типично для файлов, используемых научным и инженерным ПО.}
\EN{It's very typical for files used by scientific and engineering software.}

\RU{Для тех, кто самостоятельно хочет изучить эти файлы, то их можно скачать здесь:}
\EN{For those who wants to inspect these files on one's own, they are downloadable there:}
\url{http://go.yurichev.com/17352}.

\subsection{\Exercise}

\begin{itemize}
	\item \url{http://challenges.re/50}
\end{itemize}



\section{\RU{Стек, калькуляторы и обратная польская запись}\EN{Stack, calculators and reverse Polish notation}}

\index{\RU{Обратная польская запись}\EN{Reverse Polish notation}}
\RU{Теперь понятно, почему некоторые старые калькуляторы использовали обратную польскую запись%
\footnote{\href{http://go.yurichev.com/17355}{ru.wikipedia.org/wiki/Обратная\_польская\_запись}}.}
\EN{Now we undestand why some old calculators used reverse Polish notation
\footnote{\href{http://go.yurichev.com/17354}{wikipedia.org/wiki/Reverse\_Polish\_notation}}.}
\RU{Например для сложения 12 и 34 нужно было набрать 12, потом 34, потом нажать знак \q{плюс}.}
\EN{For example, for addition of 12 and 34 one has to enter 12, then 34, then press \q{plus} sign.}
\RU{Это потому что старые калькуляторы просто реализовали стековую машину и это было куда проще, 
чем обрабатывать сложные выражения со скобками.}
\EN{It's because old calculators were just stack machine implementations, and this was much simpler
than to handle complex parenthesized expressions.}
\section{x64}

\RU{О том, как происходит работа с числами с плавающей запятой в x86-64, читайте здесь: \myref{floating_SIMD}.}
\EN{On how floating point numbers are processed in x86-64, read more here: \myref{floating_SIMD}.}

% sections
\ifdefined\IncludeExercises
\section{\Exercises}

\begin{itemize}
	\item \url{http://challenges.re/60}
	\item \url{http://challenges.re/61}
\end{itemize}


\fi
