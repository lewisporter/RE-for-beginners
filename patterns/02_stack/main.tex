\chapter{\Stack}
\label{sec:stack}
\index{\Stack}

\RU{Стек в информатике~--- это одна из наиболее фундаментальных структур данных}%
\EN{The stack is one of the most fundamental data structures in computer science}%
\footnote{\href{http://go.yurichev.com/17119}{wikipedia.org/wiki/Call\_stack}}.

\RU{Технически это просто блок памяти в памяти процесса + регистр \ESP в x86 или \RSP в x64, либо \ac{SP} в ARM, который указывает где-то в пределах этого блока.}
\EN{Technically, it is just a block of memory in process memory along with the \ESP or \RSP register in x86 or x64, or the \ac{SP} register in ARM, as a pointer within that block.}

\index{ARM!\Instructions!PUSH}
\index{ARM!\Instructions!POP}
\index{x86!\Instructions!PUSH}
\index{x86!\Instructions!POP}
\RU{Часто используемые инструкции для работы со стеком~--- это \PUSH и \POP (в x86 и Thumb-режиме ARM). 
\PUSH уменьшает \ESP/\RSP/\ac{SP} на 4 в 32-битном режиме (или на 8 в 64-битном),
затем записывает по адресу, на который указывает \ESP/\RSP/\ac{SP}, содержимое своего единственного операнда.}
\EN{The most frequently used stack access instructions are \PUSH and \POP (in both x86 and ARM Thumb-mode). 
\PUSH subtracts from \ESP/\RSP/\ac{SP} 4 in 32-bit mode (or 8 in 64-bit mode) and then writes the contents of its sole operand to the memory address pointed by \ESP/\RSP/\ac{SP}.} 

\RU{\POP это обратная операция~--- сначала достает из \glslink{stack pointer}{указателя стека} значение и помещает его в операнд 
(который очень часто является регистром) и затем увеличивает указатель стека на 4 (или 8).}
\EN{\POP is the reverse operation: retrieve the data from the memory location that \ac{SP} points to, 
load it into the instruction operand (often a register) and then add 4 (or 8) to the \gls{stack pointer}.}

\RU{В самом начале \glslink{stack pointer}{регистр-указатель} указывает на конец стека.}
\EN{After stack allocation, the \gls{stack pointer} points at the bottom of the stack.}
\RU{\PUSH уменьшает \glslink{stack pointer}{регистр-указатель}, а \POP~--- увеличивает.}
\EN{\PUSH decreases the \gls{stack pointer} and \POP increases it.}
\RU{Конец стека находится в начале блока памяти, выделенного под стек. Это странно, но это так.}
\EN{The bottom of the stack is actually at the beginning of the memory allocated for the stack block. 
It seems strange, but that's the way it is.}

\ifdefined\IncludeARM
\RU{В процессоре ARM, тем не менее, есть поддержка стеков, растущих как в сторону уменьшения, так и в
сторону увеличения.}
\EN{ARM supports both descending and ascending stacks.}
\index{ARM!\Instructions!STMFD}
\index{ARM!\Instructions!LDMFD}
\index{ARM!\Instructions!STMED}
\index{ARM!\Instructions!LDMED}
\index{ARM!\Instructions!STMFA}
\index{ARM!\Instructions!LDMFA}
\index{ARM!\Instructions!STMEA}
\index{ARM!\Instructions!LDMEA}

\RU{Например, инструкции}\EN{For example the} 
\ac{STMFD}/\ac{LDMFD}, \ac{STMED}/\ac{LDMED} 
\RU{предназначены для descending-стека 
(растет назад, начиная с высоких адресов в сторону низких).}
\EN{instructions are intended to deal with a descending stack 
(grows downwards, starting with a high address and progressing to a lower one).}
\RU{Инструкции}\EN{The}
\ac{STMFA}/\ac{LDMFA}, \ac{STMEA}/\ac{LDMEA} 
\RU{предназначены для ascending-стека 
(растет вперед, начиная с низких адресов в сторону высоких).}
\EN{instructions are intended to deal with an ascending stack 
(grows upwards, starting from a low address and progressing to a higher one).}
\fi

% It might be worth mentioning that STMED and STMEA write first,
% and then move the pointer,
% and that LDMED and LDMEA move the pointer first, and then read.
% In other words, ARM not only lets the stack grow in a non-standard direction,
% but also in a non-standard order.
% Maybe this can be in the glossary, which would explain why E stands for "empty".

\section{\RU{Почему стек растет в обратную сторону?}\EN{Why does the stack grow backwards?}}

\RU{Интуитивно мы можем подумать, что, как и любая другая структура данных, стек мог бы расти вперед, 
т.е. в сторону увеличения адресов}\EN{Intuitively, we might think that the stack grows upwards, i.e. towards
higher addresses, like any other data structure}.

\RU{Причина, почему стек растет назад, вероятно, историческая}%
\EN{The reason that the stack grows backward is probably historical}.
\RU{Когда компьютеры были большие и занимали целую комнату, было очень легко разделить сегмент на две части:
для \glslink{heap}{кучи} и для стека}\EN{When the computers were big and occupied a whole room, 
it was easy to divide memory into two parts, one for the \gls{heap} and one for the stack}.
\RU{Заранее было неизвестно, насколько большой может быть \glslink{heap}{куча} или стек, 
так что это решение было самым простым}\EN{Of course, 
it was unknown how big the \gls{heap} and the stack would be during program execution, 
so this solution was the simplest possible}.

\begin{center}
	\begin{tikzpicture}
	\tikzstyle{every path}=[thick]

	\node [rectangle,draw,minimum width=6cm, minimum height=2cm] (memory) {};
	\node [] [right=0.2cm of memory.west] (heap) {Heap};
	\node [] [left=0.2cm of memory.east] (stack) {Stack};

	\node [] (center1) [right=2cm of memory.west] {};
	\node [] (center2) [left=2cm of memory.east] {};

	\draw [->] (heap) -- (center1);
	\draw [->] (stack) -- (center2);

	\node [] [above left=1.1cm and 0.2cm of heap] (t1) {\RU{Начало кучи}\EN{Start of heap}};
	\node [] [above right=1.1cm and 0.2cm of stack] (t2) {\RU{Вершина стека}\EN{Start of stack}};

	\draw [->] (t1) -- (memory.west);
	\draw [->] (t2) -- (memory.east);

	\end{tikzpicture}
\end{center}

\RU{В}\EN{In} \cite{Ritchie74} \RU{можно прочитать}\EN{we can read}:

\begin{framed}
\begin{quotation}
The user-core part of an image is divided into three logical segments. The program text segment begins at location 0 in the virtual address space. During execution, this segment is write-protected and a single copy of it is shared among all processes executing the same program. At the first 8K byte boundary above the program text segment in the virtual address space begins a nonshared, writable data segment, the size of which may be extended by a system call. Starting at the highest address in the virtual address space is a stack segment, which automatically grows downward as the hardware's stack pointer fluctuates.
\end{quotation}
\end{framed}

\RU{Это немного напоминает как некоторые студенты
пишут два конспекта в одной тетрадке:
первый конспект начинается обычным образом, второй пишется с конца, перевернув тетрадку.
Конспекты могут встретиться где-то посредине, в случае недостатка свободного места.}
\EN{This reminds us how some students write two lecture notes using only one notebook:
notes for the first lecture are written as usual, 
and notes for the second one are written from the end of notebook, by flipping it.
Notes may meet each other somewhere in between, in case of lack of free space.}
% I think if we want to expand on this analogy,
% one might remember that the line number increases as as you go down a page.
% So when you decrease the address when pushing to the stack, visually,
% the stack does grow upwards.
% Of course, the problem is that in most human languages,
% just as with computers,
% we write downwards, so this direction is what makes buffer overflows so messy.

\section{\RU{Для чего используется стек?}\EN{What is the stack used for?}}

% subsections
\subsection{\RU{Сохранение адреса возврата управления}
\EN{Save the function's return address}}

\subsubsection{x86}

\index{x86!\Instructions!CALL}
\RU{При вызове другой функции через \CALL сначала в стек записывается адрес, указывающий на место после 
инструкции \CALL, затем делается безусловный переход (почти как \TT{JMP}) на адрес, указанный в операнде.} 
\EN{When calling another function with a \CALL instruction, the address of the point exactly after the \CALL instruction is saved 
to the stack and then an unconditional jump to the address in the CALL operand is executed.} 

\index{x86!\Instructions!PUSH}
\index{x86!\Instructions!JMP}
\RU{\CALL~--- это аналог пары инструкций \TT{PUSH address\_after\_call / JMP}.}
\EN{The \CALL instruction is equivalent to a \TT{PUSH address\_after\_call / JMP operand} instruction pair.}

\index{x86!\Instructions!RET}
\index{x86!\Instructions!POP}
\RU{\RET вытаскивает из стека значение и передает управление по этому адресу~--- 
это аналог пары инструкций \TT{POP tmp / JMP tmp}.}
\EN{\RET fetches a value from the stack and jumps to it~---that is equivalent to a \TT{POP tmp / JMP tmp} instruction pair.}

\index{\Stack!\RU{Переполнение стека}\EN{Stack overflow}}
\index{\Recursion}
\RU{Крайне легко устроить переполнение стека, запустив бесконечную рекурсию:}
\EN{Overflowing the stack is straightforward. Just run eternal recursion:}

\begin{lstlisting}
void f()
{
	f();
};
\end{lstlisting}

\RU{MSVC 2008 предупреждает о проблеме:}\EN{MSVC 2008 reports the problem:}

\begin{lstlisting}
c:\tmp6>cl ss.cpp /Fass.asm
Microsoft (R) 32-bit C/C++ Optimizing Compiler Version 15.00.21022.08 for 80x86
Copyright (C) Microsoft Corporation.  All rights reserved.

ss.cpp
c:\tmp6\ss.cpp(4) : warning C4717: 'f' : recursive on all control paths, function will cause runtime stack overflow
\end{lstlisting}

\dots \RU{но, тем не менее, создает нужный код}\EN{but generates the right code anyway}:

\begin{lstlisting}
?f@@YAXXZ PROC						; f
; File c:\tmp6\ss.cpp
; Line 2
	push	ebp
	mov	ebp, esp
; Line 3
	call	?f@@YAXXZ				; f
; Line 4
	pop	ebp
	ret	0
?f@@YAXXZ ENDP						; f
\end{lstlisting}

\dots \RU{причем, если включить оптимизацию (\Ox), то будет даже интереснее, без переполнения стека, 
но работать будет \IT{корректно}\footnote{здесь ирония}:}
\EN{Also if we turn on the compiler optimization (\Ox option) the optimized code will not overflow the stack 
and will work \IT{correctly}\footnote{irony here} instead:}

\begin{lstlisting}
?f@@YAXXZ PROC						; f
; File c:\tmp6\ss.cpp
; Line 2
$LL3@f:
; Line 3
	jmp	SHORT $LL3@f
?f@@YAXXZ ENDP						; f
\end{lstlisting}

\ifdefined\IncludeGCC
\RU{GCC 4.4.1 генерирует точно такой же код в обоих случаях, хотя и не предупреждает о проблеме.}
\EN{GCC 4.4.1 generates similar code in both cases without, however,  issuing any warning about the problem.}
\fi

\ifdefined\IncludeARM
\subsubsection{ARM}

\index{ARM!\Registers!Link Register}
\RU{Программы для ARM также используют стек для сохранения \ac{RA}, куда нужно вернуться, но несколько иначе}\EN{ARM
programs also use the stack for saving return addresses, but differently}.
\RU{Как уже упоминалось в секции}\EN{As mentioned in} \q{\HelloWorldSectionName}~(\myref{sec:hw_ARM}),
\RU{\ac{RA} записывается в регистр}\EN{the \ac{RA} is saved to the} \ac{LR} (\gls{link register}).
\RU{Но если есть необходимость вызывать какую-то другую функцию и использовать регистр \ac{LR} ещё
раз, его значение желательно сохранить}%
\EN{If one needs, however, to call another function and use the \ac{LR} register
one more time, its value has to be saved}.
\index{Function prologue}
\RU{Обычно это происходит в прологе функции, часто мы видим там инструкцию вроде}
\EN{Usually it is saved in the function prologue. Often, we see instructions like}
\index{ARM!\Instructions!PUSH}
\index{ARM!\Instructions!POP}
\TT{PUSH {R4-R7,LR}} \RU{, а в эпилоге}\EN{along with this instruction in epilogue}
\TT{POP {R4-R7,PC}}\RU{~--- так сохраняются регистры, которые будут использоваться в текущей функции, в том числе}
\EN{---thus register values
to be used in the function are saved in the stack, including} \ac{LR}.

\index{ARM!Leaf function}
\RU{Тем не менее, если некая функция не вызывает никаких более функций, в терминологии \ac{RISC} она называется}
\EN{Nevertheless, if a function never calls any other function, in \ac{RISC} terminology it is called a}
\IT{\gls{leaf function}}\footnote{\href{http://go.yurichev.com/17064}{infocenter.arm.com/help/index.jsp?topic=/com.arm.doc.faqs/ka13785.html}}. 
\RU{Как следствие, \q{leaf}-функция не сохраняет регистр \ac{LR} (потому что не изменяет его).}
\EN{As a consequence, leaf functions do not save the \ac{LR} register (because they don't modify it).}
\RU{А если эта функция небольшая, использует мало регистров, она может не использовать стек вообще}%
\EN{If such function is small and uses a small number of registers, it may not use the stack at all}.
\RU{Таким образом, в ARM возможен вызов небольших leaf-функций не используя стек.}
\EN{Thus, it is possible to call leaf functions without using the stack,}
\RU{Это может быть быстрее чем в старых x86, ведь внешняя память для стека не используется}%
\EN{which can be faster than on older x86 machines because external RAM is not used for the stack}%
\footnote{\RU{Когда-то, очень давно, на PDP-11 и VAX на инструкцию CALL (вызов других функций) могло тратиться
вплоть до 50\% времени (возможно из-за работы с памятью),
поэтому считалось, что много небольших функций это \glslink{anti-pattern}{анти-паттерн}}%
\EN{Some time ago, on PDP-11 and VAX, the CALL instruction (calling other functions) was expensive; up to 50\%
of execution time might be spent on it, so it was considered that having a big number of small functions is an \gls{anti-pattern}} \cite[Chapter 4, Part II]{Raymond:2003:AUP:829549}.}.
\RU{Либо это может быть полезным для тех ситуаций, когда память для стека ещё не выделена, либо недоступна}%
\EN{This can be also useful for situations when memory for the stack is not yet allocated or not available}.

\EN{Some examples of leaf functions:}\RU{Некоторые примеры таких функций:}
\myref{ARM_leaf_example1}, \myref{ARM_leaf_example2}, 
\myref{ARM_leaf_example3}, \myref{ARM_leaf_example4}, \myref{ARM_leaf_example5},
\myref{ARM_leaf_example6}, \myref{ARM_leaf_example7}, \myref{ARM_leaf_example10}.
\fi

\subsection{\RU{Передача параметров функции}\EN{Passing function arguments}}

\RU{Самый распространенный способ передачи параметров в x86 называется}
\EN{The most popular way to pass parameters in x86 is called} \q{cdecl}:

\begin{lstlisting}
push arg3
push arg2
push arg1
call f
add esp, 12 ; 4*3=12
\end{lstlisting}

\RU{Вызываемая функция получает свои параметры также через указатель стека.}
\EN{\Gls{callee} functions get their arguments via the stack pointer.}

\RU{Следовательно, так расположены значения в стеке перед исполнением самой первой инструкции
функции \ttf{}:}
\EN{Therefore, this is how the argument values are located in the stack before the execution
of the \ttf{} function's very first instruction:}

\begin{center}
\begin{tabular}{ | l | l | }
\hline
ESP & \RU{адрес возврата}\EN{return address} \\
\hline
ESP+4 & \argument \#1, \MarkedInIDAAs{} \TT{arg\_0} \\
\hline
ESP+8 & \argument \#2, \MarkedInIDAAs{} \TT{arg\_4} \\
\hline
ESP+0xC & \argument \#3, \MarkedInIDAAs{} \TT{arg\_8} \\
\hline
\dots & \dots \\
\hline
\end{tabular}
\end{center}

\ifx\LITE\undefined
\RU{См. также в соответствующем разделе о других способах передачи аргументов через стек}
\EN{For more information on other calling conventions see also section}~(\myref{sec:callingconventions}).
\fi
\RU{Важно отметить, что, в общем, никто не заставляет программистов передавать параметры именно через стек,
это не является требованием к исполняемому коду.}
\EN{It is worth noting that nothing obliges programmers to pass arguments through the stack. It is not a requirement.}
\RU{Вы можете делать это совершенно иначе, не используя стек вообще.}
\EN{One could implement any other method without using the stack at all.}

\RU{К примеру, можно выделять в \glslink{heap}{куче} место для аргументов, 
заполнять их и передавать в функцию указатель на это место через \EAX. И это вполне будет работать}%
\EN{For example, it is possible to allocate a space for arguments in the \gls{heap}, fill it and pass it to a function 
via a pointer to this block in the \EAX register. This will work}%
\footnote{\RU{Например, в книге Дональда Кнута \q{Искусство программирования}, в разделе 1.4.1 
посвященном подпрограммам \cite[раздел 1.4.1]{Knuth:1998:ACP:521463}, 
мы можем прочитать о возможности располагать параметры для вызываемой подпрограммы после инструкции \JMP,
передающей управление подпрограмме. Кнут описывает, что это было особенно удобно для компьютеров IBM System/360.}%
\EN{For example, in the \q{The Art of Computer Programming} book by Donald Knuth, 
in section 1.4.1 dedicated to subroutines \cite[section 1.4.1]{Knuth:1998:ACP:521463},
we could read that one way to supply arguments to a subroutine is simply to list them after the \JMP instruction
passing control to subroutine. Knuth explains that this method was particularly convenient on IBM System/360.}}.
\RU{Однако традиционно сложилось, что в x86 и ARM передача аргументов происходит именно через стек.}
% I am unsure about what this comment means.
% My guess is that the arguments are put in the memory position after
% the jump instruction, so you could say:
% "one way to supply arguments to a subroutine is simply to list them in memory
% after the \JMP instruction that passes control to the subroutine."
% Right now, "after" also sounds like it refers to the time after
% the jump happens, which I think is too late.
\EN{However, it is a convenient custom in x86 and ARM to use the stack for this purpose.} \\
\\
\RU{Кстати, вызываемая функция не имеет информации о количестве переданных ей аргументов.}
\EN{By the way, the \gls{callee} function does not have any information about how many arguments were passed.}
\RU{Функции Си с переменным количеством аргументов (как \printf) определяют их количество по 
спецификаторам строки формата (начинающиеся со знака \%).}
\EN{C functions with a variable number of arguments (like \printf) determine their number using format string  specifiers (which begin with the \% symbol).}
\RU{Если написать что-то вроде}\EN{If we write something like} 

\begin{lstlisting}
printf("%d %d %d", 1234);
\end{lstlisting}

\printf \RU{выведет 1234, затем ещё два случайных числа, которые волею случая оказались в стеке рядом.}
\EN{will print 1234, and then two random numbers, which were lying next to it in the stack.}\\
\\
\RU{Вот почему не так уж и важно, как объявлять функцию \main}
\EN{That's why it is not very important how we declare the \main function}: \RU{как}\EN{as} \main, 
\TT{main(int argc, char *argv[])} 
\RU{либо}\EN{or} \TT{main(int argc, char *argv[], char *envp[])}.

\RU{В реальности, \ac{CRT}-код вызывает \main примерно так:}
\EN{In fact, the \ac{CRT}-code is calling \main roughly as:}

\begin{lstlisting}
push envp
push argv
push argc
call main
...
\end{lstlisting}

\RU{Если вы объявляете \main без аргументов, они, тем не менее, присутствуют в стеке, но не используются.}
\EN{If you declare \main as \main without arguments, they are, nevertheless, still present in the stack, but
are not used.}
\RU{Если вы объявите \main как}\EN{If you declare \main as} \TT{main(int argc, char *argv[])}, 
\RU{вы можете использовать два первых аргумента, а третий останется для вашей функции \q{невидимым}.}
\EN{you will be able to use first two arguments, and the third will remain \q{invisible} for your function.}
\RU{Более того, можно даже объявить}\EN{Even more, it is possible to declare} \TT{main(int argc)}, 
\RU{и это будет работать}\EN{and it will work}.


\subsection{\RU{Хранение локальных переменных}\EN{Local variable storage}}

\RU{Функция может выделить для себя некоторое место в стеке для локальных переменных, просто отодвинув 
\glslink{stack pointer}{указатель стека} глубже к концу стека.}
\EN{A function could allocate space in the stack for its local variables just by decreasing 
the \gls{stack pointer} towards the stack bottom.}
% I think here, "stack bottom" means the lowest address in the stack space,
% but the reader might also think it means towards the top of the stack space,
% like in a pop, so you might change "towards the stack bottom" to
% "towards the lowest address of the stack", or just take it out,
% since "decreasing" also suggests that.
\RU{Это очень быстро вне зависимости от количества локальных переменных.}
\EN{Hence, it's very fast, no matter how many local variables are defined.}

\RU{Хранить локальные переменные в стеке не является необходимым требованием. 
Вы можете хранить локальные переменные где угодно. 
Но по традиции всё сложилось так.}
\EN{It is also not a requirement to store local variables in the stack.
You could store local variables wherever you like, 
but traditionally this is how it's done.}

\subsection{x86: \RU{Функция alloca()}\EN{alloca() function}}
\label{alloca}
\index{\CStandardLibrary!alloca()}
\RU{Интересен случай с функцией \TT{alloca()}}%
\EN{It is worth noting the \TT{alloca()} function}\footnote{
\RU{В MSVC, реализацию функции можно посмотреть в файлах}%
\EN{In MSVC, the function implementation can be found in} 
  \TT{alloca16.asm} 
  \AndENRU 
  \TT{chkstk.asm} 
  \InENRU 
  \TT{C:\textbackslash{}Program Files (x86)\textbackslash{}Microsoft Visual Studio 10.0\textbackslash{}VC\textbackslash{}crt\textbackslash{}src\textbackslash{}intel}}. 

\RU{Эта функция работает как \TT{malloc()}, но выделяет память прямо в стеке.} 
\EN{This function works like \TT{malloc()}, but allocates memory directly on the stack.}

\RU{Память освобождать через \TT{free()} не нужно, так как эпилог функции~(\myref{sec:prologepilog})
вернет \ESP в изначальное состояние и выделенная память просто \IT{выкидывается}.}
\EN{The allocated memory chunk does not need to be freed via a \TT{free()} function call, since the 
function epilogue~(\myref{sec:prologepilog}) returns \ESP back to its initial state and 
the allocated memory is just \IT{dropped}.}

\RU{Интересна реализация функции \TT{alloca()}.}
\EN{It is worth noting how \TT{alloca()} is implemented.}

\RU{Эта функция, если упрощенно, просто сдвигает \ESP вглубь стека 
на столько байт, сколько вам нужно и возвращает \ESP в качестве указателя на выделенный блок.}
\EN{In simple terms, this function just shifts \ESP downwards toward the stack bottom by the number of bytes you need and sets \ESP as a pointer to the \IT{allocated} block.}
\RU{Попробуем:}\EN{Let's try:}

\lstinputlisting{patterns/02_stack/04_alloca/2_1.c}

\RU{Функция \TT{\_snprintf()} работает так же, как и \printf, только вместо выдачи результата в \gls{stdout} (т.е. на терминал или в консоль),
записывает его в буфер \TT{buf}. Функция \puts выдает содержимое буфера \TT{buf} в \gls{stdout}. Конечно, можно было бы
заменить оба этих вызова на один \printf, но здесь нужно проиллюстрировать использование небольшого буфера.}%
\EN{\TT{\_snprintf()} function works just like \printf, but instead of dumping the result into \gls{stdout} (e.g., to terminal or 
console), it writes it to the \TT{buf} buffer. Function \puts copies the contents of \TT{buf} to \gls{stdout}. Of course, these two
function calls might be replaced by one \printf call, but we have to illustrate small buffer usage.}

\subsubsection{MSVC}

\RU{Компилируем}\EN{Let's compile} (MSVC 2010):

\lstinputlisting[caption=MSVC 2010]{patterns/02_stack/04_alloca/2_2_msvc.asm}

\index{Compiler intrinsic}
\RU {Единственный параметр в \TT{alloca()} передается через \EAX, а не как обычно через стек}%
\EN{The sole \TT{alloca()} argument is passed via \EAX (instead of pushing it into the stack)}%
\footnote{
\RU{Это потому, что alloca()~--- это не сколько функция, сколько т.н. \IT{compiler intrinsic}%
\ifx\LITE\undefined
(\myref{sec:compiler_intrinsic})
\fi
}%
\EN{It is because alloca() is rather a compiler intrinsic 
\ifx\LITE\undefined
(\myref{sec:compiler_intrinsic}) 
\fi
than a normal function}.

\RU{Одна из причин, почему здесь нужна именно функция, а не несколько инструкций прямо в коде в том, что в реализации 
функции alloca() от \ac{MSVC}
есть также код, читающий из только что выделенной памяти, чтобы \ac{OS} подключила физическую память к этому региону \ac{VM}.}
\EN{One of the reasons we need a separate function instead of just a couple of instructions in the code,
is because the \ac{MSVC} alloca() implementation also has code which reads from the memory just allocated, in order to let the \ac{OS} map
physical memory to this \ac{VM} region.}
}.
\RU{После вызова \TT{alloca()} \ESP указывает на блок в 600 байт, который 
мы можем использовать под \TT{buf}.}
\EN{After the \TT{alloca()} call, \ESP points to the block of 600 bytes and we can 
use it as memory for the \TT{buf} array.}

\ifdefined\IncludeGCC
\subsubsection{GCC + \IntelSyntax}

\RU{А GCC 4.4.1 обходится без вызова других функций:}
\EN{GCC 4.4.1 does the same without calling external functions:}

\lstinputlisting[caption=GCC 4.7.3]{patterns/02_stack/04_alloca/2_1_gcc_intel_O3.asm.\LANG}

\subsubsection{GCC + \ATTSyntax}

\RU{Посмотрим на тот же код, только в синтаксисе AT\&T}\EN{Let's see the same code, but in AT\&T syntax}:

\lstinputlisting[caption=GCC 4.7.3]{patterns/02_stack/04_alloca/2_1_gcc_ATT_O3.s}

\index{\ATTSyntax}
\RU{Всё то же самое, что и в прошлом листинге.}\EN{The code is the same as in the previous listing.}

\RU{Кстати}\EN{By the way}, \TT{movl \$3, 20(\%esp)}%
\RU{~--- это аналог}\EN{corresponds to} \TT{mov DWORD PTR [esp+20], 3}
\RU{в синтаксисе Intel.}\EN{ in Intel-syntax.}
\RU{Адресация памяти в виде \IT{регистр+смещение} записывается в синтаксисе AT\&T как \TT{смещение(\%{регистр})}.}
\EN{In the AT\&T syntax, the register+offset format of addressing memory looks like \TT{offset(\%{register})}.}
\fi

\subsection{(Windows) SEH}
\index{Windows!Structured Exception Handling}

\RU{В стеке хранятся записи \ac{SEH} для функции (если они присутствуют)}%
\EN{\ac{SEH} records are also stored on the stack (if they are present).}.

\ifx\LITE\undefined
\RU{Читайте больше о нем здесь}\EN{Read more about it}: (\myref{sec:SEH}).
\fi

\input{patterns/02_stack/06_BO_protection}

\subsection{\EN{Automatic deallocation of data in stack}\RU{Автоматическое освобождение данных в стеке}}

\RU{Возможно, причина хранения локальных переменных и SEH-записей в стеке в том, что после выхода из функции, всё эти данные освобождаются автоматически,
используя только одну инструкцию корректирования указателя стека (часто это ADD).}
\EN{Perhaps, the reason for storing local variables and SEH records in the stack is that they are freed automatically upon function exit,
using just one instruction to correct the stack pointer (it is often ADD).}
\RU{Аргументы функций, можно сказать, тоже освобождаются автоматически в конце функции.}
\EN{Function arguments, as we could say, are also deallocated automatically at the end of function.}
\RU{А всё что хранится в куче (\IT{heap}) нужно освобождать явно.}
\EN{In contrast, everything stored in the \IT{heap} must be deallocated explicitly.}

% sections
\section{\RU{Разметка типичного стека}\EN{A typical stack layout}}

\RU{Разметка типичного стека в 32-битной среде
перед исполнением самой первой инструкции функции выглядит так:}
\EN{A typical stack layout in a 32-bit environment at the start of a function, 
before the first instruction execution looks like this:}

\begin{center}
\begin{tabular}{ | l | l | }
\hline
\dots & \dots \\
\hline
ESP-0xC & \RU{локальная переменная}\EN{local variable} \#2, \MarkedInIDAAs{} \TT{var\_8} \\
\hline
ESP-8 & \RU{локальная переменная}\EN{local variable} \#1, \MarkedInIDAAs{} \TT{var\_4} \\
\hline
ESP-4 & \RU{сохраненное значение}\EN{saved value of} \EBP \\
\hline
ESP & \RU{адрес возврата}\EN{return address} \\
\hline
ESP+4 & \argument \#1, \MarkedInIDAAs{} \TT{arg\_0} \\
\hline
ESP+8 & \argument \#2, \MarkedInIDAAs{} \TT{arg\_4} \\
\hline
ESP+0xC & \argument \#3, \MarkedInIDAAs{} \TT{arg\_8} \\
\hline
\dots & \dots \\
\hline
\end{tabular}
\end{center}
% I think this only applies to RISC architectures
% that don't have a POP instruction that only lets you read one value
% (ie. ARM and MIPS).
% In x86, the return address is saved before entering the function,
% and the function does not have the chance to save the frame pointer.
% Also, you should mention that this is how the stack looks like
% right after the function prologue,
% which some readers might think is the first instruction,
% but is needed to save the frame pointer.

\ifx\LITE\undefined
\section{\RU{Мусор в стеке}\EN{Noise in stack}}

\RU{Часто в этой книге говорится о \q{шуме} или \q{мусоре} в стеке или памяти.}
\EN{Often in this book \q{noise} or \q{garbage} values in the stack or memory are mentioned.}
\RU{Откуда он берется}\EN{Where do they come from}?
\RU{Это то, что осталось там после исполнения предыдущих функций.}
\EN{These are what was left in there after other functions' executions.}
\RU{Короткий пример}\EN{Short example}:

\lstinputlisting{patterns/02_stack/08_noise/st.c}

\RU{Компилируем}\EN{Compiling}\dots

\lstinputlisting[caption=\NonOptimizing MSVC 2010]{patterns/02_stack/08_noise/st.asm}

\RU{Компилятор поворчит немного}\EN{The compiler will grumble a little bit}\dots

\begin{lstlisting}
c:\Polygon\c>cl st.c /Fast.asm /MD
Microsoft (R) 32-bit C/C++ Optimizing Compiler Version 16.00.40219.01 for 80x86
Copyright (C) Microsoft Corporation.  All rights reserved.

st.c
c:\polygon\c\st.c(11) : warning C4700: uninitialized local variable 'c' used
c:\polygon\c\st.c(11) : warning C4700: uninitialized local variable 'b' used
c:\polygon\c\st.c(11) : warning C4700: uninitialized local variable 'a' used
Microsoft (R) Incremental Linker Version 10.00.40219.01
Copyright (C) Microsoft Corporation.  All rights reserved.

/out:st.exe
st.obj
\end{lstlisting}

\RU{Но когда мы запускаем}\EN{But when we run the compiled program}\dots

\begin{lstlisting}
c:\Polygon\c>st
1, 2, 3
\end{lstlisting}

\RU{Ох. Вот это странно. Мы ведь не устанавливали значения никаких переменных в}\EN{Oh, 
what a weird thing! We did not set any variables in} \TT{f2()}. 
\RU{Эти значения --- это \q{привидения}, которые всё ещё в стеке.}
\EN{These are \q{ghosts} values, which are still in the stack.}

\clearpage
\RU{Загрузим пример в}\EN{Let's load the example into} \olly:

\begin{figure}[H]
\centering
\includegraphics[scale=\FigScale]{patterns/02_stack/08_noise/olly1.png}
\caption{\olly: \TT{f1()}}
\label{fig:stack_noise_olly1}
\end{figure}

\RU{Когда}\EN{When} \TT{f1()} \RU{заполняет переменные}\EN{assigns the variables} $a$, $b$ \AndENRU $c$ 
\RU{они сохраняются по адресу}\EN{, their values are stored at the address} \TT{0x1FF860} 
\RU{\etc{}.}\EN{and so on.}

\clearpage
\RU{А когда исполняется}\EN{And when} \TT{f2()}\EN{ executes}:

\begin{figure}[H]
\centering
\includegraphics[scale=\FigScale]{patterns/02_stack/08_noise/olly2.png}
\caption{\olly: \TT{f2()}}
\label{fig:stack_noise_olly2}
\end{figure}

... $a$, $b$ \AndENRU $c$ \RU{в функции}\EN{of} \TT{f2()} \RU{находятся по тем же адресам!}
\EN{are located at the same addresses!}
\RU{Пока никто не перезаписал их, так что они здесь в нетронутом виде.}
\EN{No one has overwritten the values yet, so at that point they are still untouched.}

\RU{Для создания такой странной ситуации несколько функций должны исполняться друг за другом
и \ac{SP} должен быть одинаковым при входе в функции, т.е. у функций должно быть равное количество
аргументов). Тогда локальные переменные будут расположены в том же месте стека.}
\EN{So, for this weird situation to occur, several functions have to be called one after another and
\ac{SP} has to be the same at each function entry (i.e., they have the same number
of arguments). Then the local variables will be located at the same positions in the stack.}

\RU{Подводя итоги, все значения в стеке (да и памяти вообще) это значения оставшиеся от 
исполнения предыдущих функций.}
\EN{Summarizing, all values in the stack (and memory cells in general) 
have values left there from previous function executions.}
\RU{Строго говоря, они не случайны, они скорее непредсказуемы.}
\EN{They are not random in the strict sense, but rather have unpredictable values.}

\RU{А как иначе}\EN{Is there another option}?
\RU{Можно было бы очищать части стека перед исполнением каждой функции,
но это слишком много лишней (и ненужной) работы.}
\EN{It probably would be possible to clear portions of the stack before each function execution,
but that's too much extra (and unnecessary) work.}

\subsection{MSVC 2013}

\EN{The example was compiled by}\RU{Этот пример был скомпилирован в} MSVC 2010.
\EN{But the reader of this book made attempt to compile this example in MSVC 2013, ran it, and got all 3 numbers reversed:}%
\RU{Но один читатель этой книги сделал попытку скомпилировать пример в MSVC 2013, запустил и увидел 3 числа в обратном порядке:}

\begin{lstlisting}
c:\Polygon\c>st
3, 2, 1
\end{lstlisting}

\EN{Why?}\RU{Почему?}

\EN{I also compiled this example in MSVC 2013 and saw this:}%
\RU{Я также попробовал скомпилировать этот пример в MSVC 2013 и увидел это:}

\begin{lstlisting}[caption=MSVC 2013]
_a$ = -12						; size = 4
_b$ = -8						; size = 4
_c$ = -4						; size = 4
_f2	PROC

...

_f2	ENDP

_c$ = -12						; size = 4
_b$ = -8						; size = 4
_a$ = -4						; size = 4
_f1	PROC

...

_f1	ENDP
\end{lstlisting}

\EN{Unlike MSVC 2010, MSVC 2013 allocated a/b/c variables in function \TT{f2()} in reverse order.}%
\RU{В отличии от MSVC 2010, MSVC 2013 разместил переменные a/b/c в функции \TT{f2()} в обратном порядке.}
\EN{And this is completely correct, because \CCpp standards has no rule, in which order local variables must be allocated in the local stack, if at all.}%
\RU{И это полностью корректно, потому что в стандартах \CCpp нет правила, в каком порядке локальные переменные должны быть размещены в локальном стеке,
если вообще.}
\EN{The reason of difference is because MSVC 2010 has one way to do it, and MSVC 2013 has probably something changed inside of compiler guts, so it behaves
slightly different.}%
\RU{Разница есть из-за того что MSVC 2010 делает это одним способом, а в MSVC 2013, вероятно, что-то немного изменили во внутренностях компилятора,
так что он ведет себя слегка иначе.}


\fi
\ifdefined\IncludeExercises
\section{\Exercises}

\begin{itemize}
	\item \url{http://challenges.re/51}
	\item \url{http://challenges.re/52}
\end{itemize}


\fi
