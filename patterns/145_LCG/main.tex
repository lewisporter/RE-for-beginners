\chapter[\RU{Линейный конгруэнтный генератор}\EN{Linear congruential generator}]
{\RU{Линейный конгруэнтный генератор как генератор псевдослучайных чисел}\EN{Linear congruential generator as pseudorandom number generator}}
\index{\CStandardLibrary!rand()}
\label{LCG_simple}

\RU{Линейный конгруэнтный генератор, пожалуй, самый простой способ генерировать псевдослучайные числа.}
\EN{The linear congruential generator is probably the simplest possible way to generate random numbers.}
\RU{Он не в почете в наше время\footnote{Вихрь Мерсенна куда лучше}, но он настолько прост
(только одно умножение, одно сложение и одна операция \q{И}),
что мы можем использовать его в качестве примера.}
\EN{It's not in favour in modern times\footnote{Mersenne twister is better}, but it's so simple 
(just one multiplication, one addition and one AND operation), 
we can use it as an example.}

\lstinputlisting{patterns/145_LCG/rand.c.\LANG}

\RU{Здесь две функции: одна используется для инициализации внутреннего состояния, а вторая
вызывается собственно для генерации псевдослучайных чисел.}
\EN{There are two functions: the first one is used to initialize the internal state, and the second one is called
to generate pseudorandom numbers.}

\RU{Мы видим что в алгоритме применяются две константы}\EN{We see that two constants are used in the algorithm}.
\RU{Они взяты из}\EN{They are taken from} \cite{Numerical}.
\RU{Определим их используя выражение \CCpp \TT{\#define}. Это макрос.}
\EN{Let's define them using a \TT{\#define} \CCpp statement. It's a macro.}
\RU{Разница между макросом в \CCpp и константой в том, что все макросы заменяются на значения препроцессором
\CCpp и они не занимают места в памяти как переменные.}
\EN{The difference between a \CCpp macro and a constant is that all macros are replaced 
with their value by \CCpp preprocessor,
and they don't take any memory, unlike variables.}
\RU{А константы, напротив, это переменные только для чтения.}
\EN{In contrast, a constant is a read-only variable.}
\RU{Можно взять указатель (или адрес) переменной-константы, но это невозможно сделать с макросом.}
\EN{It's possible to take a pointer (or address) of a constant variable, but impossible to do so with a macro.}

\RU{Последняя операция \q{И} нужна, потому что согласно стандарту Си \TT{my\_rand()} должна возвращать значение в пределах
0..32767.}
\EN{The last AND operation is needed because by C-standard \TT{my\_rand()} has to return a value in 
the 0..32767 range.}
\RU{Если вы хотите получать 32-битные псевдослучайные значения, просто уберите последнюю операцию \q{И}.}
\EN{If you want to get 32-bit pseudorandom values, just omit the last AND operation.}

\section{x86}

\lstinputlisting[caption=\Optimizing MSVC 2013]{patterns/145_LCG/rand_MSVC_2013_x86_Ox.asm}

\RU{Вот мы это и видим: обе константы встроены в код.}
\EN{Here we see it: both constants are embedded into the code.}
\RU{Память для них не выделяется.}\EN{There is no memory allocated for them.}
\RU{Функция \TT{my\_srand()} просто копирует входное значение во внутреннюю переменную \TT{rand\_state}.}
\EN{The \TT{my\_srand()} function just copies its input value into the internal \TT{rand\_state} variable.}

\RU{\TT{my\_rand()} берет её, вычисляет следующее состояние \TT{rand\_state}, 
обрезает его и оставляет в регистре EAX.}
\EN{\TT{my\_rand()} takes it, calculates the next \TT{rand\_state}, cuts it and leaves it in the EAX register.}

\RU{Неоптимизированная версия побольше}\EN{The non-optimized version is more verbose}:

\lstinputlisting[caption=\NonOptimizing MSVC 2013]{patterns/145_LCG/rand_MSVC_2013_x86.asm}

\section{x64}

\RU{Версия для x64 почти такая же, и использует 32-битные регистры вместо 64-битных
(потому что мы работаем здесь с переменными типа \Tint).}
\EN{The x64 version is mostly the same and uses 32-bit registers instead of 64-bit ones 
(because we are working with \Tint values here).}
\RU{Но функция \TT{my\_srand()} берет входной аргумент из регистра \ECX, а не из стека:}
\EN{But \TT{my\_srand()} takes its input argument from the \ECX register rather than from stack:}

\lstinputlisting[caption=\Optimizing MSVC 2013 x64]{patterns/145_LCG/rand_MSVC_2013_x64_Ox.asm.\LANG}

\ifdefined\IncludeGCC
\RU{GCC делает почти такой же код}\EN{GCC compiler generates mostly the same code}.
\fi

\ifdefined\IncludeARM
\section{32-bit ARM}

\lstinputlisting[caption=\OptimizingKeilVI (\ARMMode)]{patterns/145_LCG/rand.s_Keil_ARM_O3.s.\LANG}

\RU{В ARM инструкцию невозможно встроить 32-битную константу, так что Keil-у приходится размещать
их отдельно и дополнительно загружать.}
\EN{It's not possible to embed 32-bit constants into ARM instructions, so Keil has to place them externally
and load them additionally.}

\RU{Вот еще что интересно: константу 0x7FFF также нельзя встроить.}
\EN{One interesting thing is that it's not possible to embed the 0x7FFF constant as well.}
\RU{Поэтому Keil сдвигает \TT{rand\_state} влево на 17 бит и затем сдвигает вправо на 17 бит.}
\EN{So what Keil does is shifting \TT{rand\_state} left by 17 bits and then shifting it right by 17 bits.}
\RU{Это аналогично \CCpp{}-выражению $(rand\_state \ll 17) \gg 17$.}
\EN{This is analogous to the $(rand\_state \ll 17) \gg 17$ statement in \CCpp.}
\RU{Выглядит как бессмысленная операция, но тем не менее, что она делает это очищает старшие 17 бит, оставляя
младшие 15 бит нетронутыми, и это наша цель, в конце концов.}
\EN{It seems to be useless operation, but
what it does is clearing the high 17 bits, leaving the low 15 bits intact, and that's our goal after all.}
\ESph{}\PTBRph{}\PLph{}\ITAph{}\\
\\
\Optimizing Keil \RU{для режима Thumb делает почти такой же код}\EN{for Thumb mode generates mostly the same code}.
\fi

\ifdefined\IncludeMIPS
\subsection{MIPS}

\index{MIPS!\Registers!FCCR}
\EN{The co-processor of the MIPS processor has a condition bit which can be set in the FPU and checked in the CPU.}
\RU{В сопроцессоре MIPS есть бит результата, который устанавливается в FPU и проверяется в CPU.}
\EN{Earlier MIPS-es have only one condition bit (called FCC0), later ones have 8 (called FCC7-FCC0).}
\RU{Ранние MIPS имели только один бит (с названием FCC0), а у поздних их 8 (с названием FCC7-FCC0).}
\RU{Этот бит (или биты) находятся в регистре с названием FCCR.}
\EN{This bit (or bits) are located in the register called FCCR.}

\lstinputlisting[caption=\Optimizing GCC 4.4.5 (IDA)]{patterns/12_FPU/3_comparison/MIPS_O3_IDA.lst.\LANG}

\index{MIPS!\Instructions!C.LT.D}
\TT{C.LT.D} \EN{compares two values}\RU{сравнивает два значения}. 
\TT{LT} \EN{is the condition}\RU{это условие} \q{Less Than}\RU{ (меньше чем)}.
\TT{D} \EN{implies values of type}\RU{означает переменные типа} \Tdouble.
\EN{Depending on the result of the comparison, the FCC0 condition bit is either set or cleared.}
\RU{В зависимости от результата сравнения, бит FCC0 устанавливается или очищается.}

\index{MIPS!\Instructions!BC1T}
\index{MIPS!\Instructions!BC1F}
\TT{BC1T} \EN{checks the FCC0 bit and jumps if the bit is set}\RU{проверяет бит FCC0 и делает переход, если бит выставлен}.
\TT{T} \EN{mean that the jump is to be taken if the bit is set}\RU{означает что переход произойдет если бит выставлен} (\q{True}).
\EN{There is also the instruction}\RU{Имеется также инструкция} \q{BC1F} \EN{which jumps if the bit is cleared}\RU{которая сработает, если бит сброшен} (\q{False}).

\RU{В зависимости от перехода один из аргументов функции помещается в регистр \$F0.}
\EN{Depending on the jump, one of function arguments is placed into \$F0.}

\fi

\ifx\LITE\undefined
\section{\RU{Версия этого примера для многопоточной среды}\EN{Thread-safe version of the example}}

\RU{Версия примера для многопоточной среды будет рассмотрена позже}%
\EN{The thread-safe version of the example is to be demonstrated later}: \myref{LCG_TLS}.
\fi
