\chapter{\RU{Оператор GOTO}\EN{GOTO operator}}

\RU{Оператор GOTO считается анти-паттерном}\EN{The GOTO operator is generally considered as anti-pattern.} 
\cite{Dijkstra:1968:LEG:362929.362947}, 
\RU{но тем не менее, его можно использовать в разумных пределах}
\EN{Nevertheless, it can be used reasonably} \cite{Knuth:1974:SPG:356635.356640}, \cite[1.3.2]{CBook}.

\RU{Вот простейший пример}\EN{Here is a very simple example}:

\lstinputlisting{patterns/065_GOTO/goto.c}

\RU{Вот что мы получаем в}\EN{Here is what we have got in} MSVC 2012:

\lstinputlisting[caption=MSVC 2012]{patterns/065_GOTO/MSVC_goto.asm}

\RU{Выражение \IT{goto} заменяется инструкцией \JMP, которая работает точно также:
безусловный переход в другое место.}
\EN{The \IT{goto} statement has been simply replaced by a \JMP instruction, which has the same
effect: unconditional jump to another place.}

\RU{Вызов второго \printf может исполнится только при помощи человеческого вмешательства,
используя отладчик или модифицирование кода.}
\EN{The second \printf could be executed only with human intervention, 
by using a debugger or by patching the code.}
\PTBRph{}\ESph{}\PLph{}\ITAph{}\\
\\
\ifdefined\IncludeHiew
\clearpage
\RU{Это также может быть простым упражнением на модификацию кода.}
\EN{This could also be useful as a simple patching exercise.}
\RU{Откроем исполняемый файл в}\EN{Let's open the resulting executable in} Hiew:

\begin{figure}[H]
\centering
\includegraphics[scale=\FigScale]{patterns/065_GOTO/hiew1.png}
\caption{Hiew}
\label{fig:goto_hiew1}
\end{figure}

\clearpage
\RU{Поместите курсор по адресу}\EN{Place the cursor to address} \JMP (\TT{0x410}), 
\RU{нажмите}\EN{press} F3 (\RU{редактирование}\EN{edit}), \RU{нажмите два нуля, так что
опкод становится}\EN{press zero twice, so the opcode becomes} \TT{EB 00}:

\begin{figure}[H]
\centering
\includegraphics[scale=\FigScale]{patterns/065_GOTO/hiew2.png}
\caption{Hiew}
\label{fig:goto_hiew2}
\end{figure}

\RU{Второй байт опкода \JMP это относительное смещение от перехода. 0 означает место
прямо после текущей инструкции.}
\EN{The second byte of the \JMP opcode denotes the relative offset for the jump, 0 means the point
right after the current instruction.}
\RU{Теперь \JMP не будет пропускать следующий вызов \printf.}
\EN{So now \JMP not skipping the second \printf call.}

\RU{Нажмите F9 (запись) и выйдите.}
\EN{Press F9 (save) and exit.}
\RU{Теперь мы запускаем исполняемый файл и видим это}\EN{Now if we run the executable we should see 
this}:

\begin{figure}[H]
\centering
\includegraphics[scale=\NormalScale]{patterns/065_GOTO/result.png}
\caption{\RU{Результат}\EN{Patched executable output}}
\label{fig:goto_result}
\end{figure}

\RU{Подобного же эффекта можно достичь, если заменить инструкцию \JMP на две инструкции \NOP.}
\EN{The same result could be achieved by replacing the \JMP instruction with 2 \NOP instructions.}
\RU{\NOP имеет опкод \TT{0x90} и длину в 1 байт, так что нужно 2 инструкции для замены.}
\EN{\NOP has an opcode of \TT{0x90} and length of 1 byte, so we need 2 instructions as \JMP replacement (which is 2 bytes in size).}
\fi

\section{\RU{Мертвый код}\EN{Dead code}}

\RU{Вызов второго \printf также называется \q{мертвым кодом} (\q{dead code}) 
в терминах компиляторов.}
\EN{The second \printf call is also called \q{dead code} in compiler terms.}
\RU{Это значит, что он никогда не будет исполнен.}
\EN{This means that the code will never be executed.}
\EN{So when you compile this example with optimizations, the compiler removes \q{dead code}, leaving
no trace of it:}
\RU{Так что если вы компилируете этот пример с оптимизацией, компилятор удаляет \q{мертвый
код} не оставляя следа:}

\lstinputlisting[caption=\Optimizing MSVC 2012]{patterns/065_GOTO/MSVC_goto_Ox.asm}

\RU{Впрочем, строку}\EN{However, the compiler forgot to remove the} \q{skip me!} \RU{компилятор 
убрать забыл}\EN{string}.

%Note: cl "/Ox" option for maximum optimisation does get rid of "skip me" string as well

\ifdefined\IncludeExercises
\section{\Exercise}

% TODO debugger example can fit here
\RU{Попробуйте добиться того же самого в вашем любимом компиляторе и отладчике.}
\EN{Try to achieve the same result using your favorite compiler and debugger.}
\fi
