%FIXME: requires PTBR and ES revision (dbmussi)
\part{\RU{Образцы кода}\EN{Code patterns}\PTBR{Padrões de código}\ES{Patrones de código}}

\RU{\epigraph{Всё познается в сравнении}{Автор неизвестен}}
\EN{\epigraph{Everything is comprehended in comparison}{Author unknown}}
\PTBR{\epigraph{Tudo é relativo}{Autor desconhecido}}
\ES{\epigraph{Todo es relativo}{Autor desconocido}}
% FIXME: english sentence added. (dbmussi) 
% not sure it's correct. (yurichev)
% this is popular Russian proverb and is close to "everything is comprehended in comparison", but the source is lost, however, 
% it's traditionally attributed to all sorts of philosophers..
% I don't know exact analgoue in English language, but OK, let it be so.

\RU{Когда автор этой книги учил Си, а затем \Cpp, он просто писал небольшие фрагменты кода, компилировал и смотрел, что 
получилось на ассемблере. Так было намного проще понять%
\footnote{Честно говоря, он и до сих пор так делаю, когда не понимают, как работает некий код.}.
Он делал это такое количество раз, что связь между кодом на \CCpp и тем, что генерирует компилятор, вбилась в его подсознание достаточно глубоко.
После этого не трудно, глядя на код на ассемблере, сразу в общих чертах понимать, что там было написано на Си. 
Возможно это поможет кому-то ещё.}
\EN{When the author of this book first started learning C and, later, \Cpp, he used to write small pieces of code, compile them, 
and then look at the assembly language output. This made it very easy for him to understand what was going on in the code that he had written.
\footnote{In fact, he still does it when he can't understand what a particular bit of code does.}. 
He did it so many times that the relationship between the \CCpp code and what the compiler produced was imprinted deeply in his mind. 
It's easy to imagine instantly a rough outline of C code's appearance and function. 
Perhaps this technique could be helpful for others.}
\PTBR{Quando o autor deste livro começou a aprender C e, mais tarde, \Cpp, ele costumava escrever pequenos pedaços de código, compilá-los, 
e então olhar a saída em linguagem assembly. Isso tornou muito fácil para ele entender o que estava acontecendo no código que ele tinha escrito.
\footnote{Na verdade, ele ainda faz isso quando não consegue entender o que faz um determinado pedaço de código.}. 
Ele fez isso tantas vezes que o relacionamento entre o código \CCpp code e o que o compilador produzia ficou registrado profundamente em sua mente. 
É fácil imaginar de imediato um esboço da aparência e função do código C. 
Talvez essa técnica poderia ser útil para mais alguém.}
\ES{Cuando el autor de este libro comenzó a aprender C y, más tarde, \Cpp, él solía escribir pequeños trozos de código, compilarlos, 
y luego ver los resultados en lenguaje assembly. Esto lo hizo muy fácil para él entender lo que estaba pasando en el código que había escrito.
\footnote{De hecho, todavia lo hace cuando no puede entender lo que hace una determinada pieza de código.}. 
Él lo hizo tantas veces que la relación entre el código \CCpp y lo que el compilador producido se imprimió profundamente en su mente. 
És fácil imaginar al instante un esbozo de la aparencia y función del código C. 
Quizás esta técnica podría ser útil para otra persona.}

%
%\RU{Здесь много примеров и для x86/x64 и для ARM}\EN{There are a lot of examples for both x86/x64 
%and ARM}.\PTBR{Há uma série de exemplos para ambos x86/x64 e ARM}.\ES{Hay una serie de ejemplos, tanto para x86/x64 y ARM} 
%\RU{Те, кто уже хорошо знаком с одной из архитектур, могут легко пролистывать страницы}
%\EN{Those who already familiar with one of architectures, may freely skim over pages}.
%\PTBR{Aqueles já familiarizados com alguma das arquiteturas, pode ler superficialmente as próximas páginas}.
%\ES{Los que ya están familiarizados con alguna de las arquitecturas, pueden leer superficialmente las páginas siguientes}.

\RU{Иногда здесь используются достаточно древние компиляторы, чтобы получить самый короткий (или простой) фрагмент кода.}%
\EN{Sometimes ancient compilers are used here, in order to get the shortest (or simplest) possible code snippet.}
\PTBR{Em determinadas partes foram usados aqui compiladores muito antigos, para se obter o menor (ou mais simples) snippet possível.}
\ES{En ciertas partes, se han empleado aquí compiladores muy antiguas, con el fin de obtener lo mas corta (o simple) posible snippet.}
\

\ifdefined\IncludeExercises
\section*{\Exercises}

\RU{Когда автор этой книги учил ассемблер, он также часто компилировал короткие функции на Си и затем постепенно 
переписывал их на ассемблер, с целью получить как можно более короткий код.}%
\EN{When the author of this book studied assembly language, he also often compiled small C-functions and then rewrote
them gradually to assembly, trying to make their code as short as possible.}
\PTBR{Quando o autor deste livro estudou a linguagem assembly, ele também frequentemente compilava pequenas funções em C e então as reescrevia gradualmente em assembly, tentando fazer seu código o menor possível.}
\ES{Cuando el autor de este libro estudió la lenguaje assembly, también con frecuencia compilaba pequeñas funciones en C, y reescribia gradualmente en assembly, tratando de hacer el código lo más pequeño posible}
\RU{Наверное, этим не стоит заниматься в наше время на практике (потому что конкурировать с современными
компиляторами в плане эффективности очень трудно), но это очень хороший способ разобраться в ассемблере
лучше.}%
\EN{This probably is not worth doing in real-world scenarios today, 
because it's hard to compete with modern compilers in terms of efficiency. It is, however, a very good way to gain a better understanding of assembly.}
\PTBR{Provavelmente não vale mais à pena fazer isso em cenários reais atualmente, 
porque é difícil competir com os compiladores modernos em termos de eficiência. É, no entanto, uma forma muito boa de obter um melhor entendimento de assembly.}
\ES{Probablemente no vale la pena hacer esto en escenarios reales actualmente, 
porque es dificil competir con los compiladores modernos en términos de eficiencia. Es, sin embargo, una muy buena manera de obtener una mejor compreensión de la assembly}

\RU{Так что вы можете взять любой фрагмент кода на ассемблере в этой книге и постараться сделать его короче.}%
\EN{Feel free, therefore, to take any assembly code from this book and try to make it shorter.}
\PTBR{Sinta-se livre, portanto, para pegar qualquer código assembly deste livro e tentar torná-lo menor.}
\ES{Siéntase libre, por lo tanto, para tomar cualquier código de este libro y tratar de hacerlo más pequeño.}
\RU{Но не забывайте о тестировании своих результатов.}%
\EN{However, don't forget to test what you have written.}
\PTBR{No entanto, não esqueça de testar o que você tiver escrito.}
\ES{Sin embargo, no se olvide de probar lo que has escrito.}
\fi

% rewrote to show that debug\release and optimisations levels are orthogonal concepts.
\section*{\RU{Уровни оптимизации и отладочная информация}\EN{Optimization levels and debug information}\PTBR{Níveis de otimização e informação de depuração}\ES{Níveles de optimización y la información de depuración}}

\RU{Исходный код можно компилировать различными компиляторами с различными уровнями оптимизации.
В типичном компиляторе этих уровней около трёх, где нулевой уровень~--- отключить оптимизацию.
Различают также направления оптимизации кода по размеру и по скорости.}
\EN{Source code can be compiled by different compilers with various optimization levels.
A typical compiler has about three such levels, where level zero means disable optimization.
Optimization can also be targeted towards code size or code speed.}
\PTBR{O código-fonte pode ser compilado por diferentes compiladores com vários níveis de otimização.
Um compilador típico tem cerca de três destes níveis, onde o nível zero significa desativar a otimização.
A otimização também pode ser direcionada para o tamanho do código ou para a velocidade do código.}
\ES{El código fuente puede ser compilado por diferentes compiladores com varios niveles de optimización.
Un compilador típico tiene alredor de tres de esos niveles, donde el nivel cero significa desactivar la optimización.
La optimización también puede dirigirse hacia el tamaño del código o la velocidad de código.}

\RU{Неоптимизирующий компилятор работает быстрее, генерирует более понятный (хотя и более объемный) код.
Оптимизирующий компилятор работает медленнее и старается сгенерировать более быстрый (хотя и не обязательно краткий) код.}
\EN{A non-optimizing compiler is faster and produces more understandable (albeit verbose) code,
whereas an optimizing compiler is slower and tries to produce code that runs faster (but is not necessarily more compact).}
\PTBR{Um compilador sem otimização é mais rápido e produz código mais inteligível (embora maior),
enquanto que um compilador com otimização é mais lento e tenta produzir um código que execute mais rápido (mas não é necessariamente mais compacto).}
\ES{Un compilador sin optimización es más rápido y produce código más inteligible (aunque más grande), 
mientras un compilador con optimización es más lento y trata de producir un código que corre más rápido (pero no necesariamente más compacto).}

\RU{Наряду с уровнями и направлениями оптимизации компилятор может включать в конечный файл отладочную информацию,
производя таким образом код, который легче отлаживать.}
\EN{In addition to optimization levels and direction, a compiler can include in the resulting file some debug information,
thus producing code for easy debugging.}
\PTBR{Além dos níveis e direcionamento da otimização, o compilador pode incluir no arquivo resultante algumas informações de depuração, produzindo assim código para fácil depuração.}
\ES{Además de los niveles y dirección de la otimización, el compilador puede incluir informaciones de depuración en el archivo resultante, produciendo así código para fácil depuración.}

\RU{Одна очень важная черта отладочного кода в том, что он может содержать
связи между каждой строкой в исходном коде и адресом в машинном коде.}
\EN{One of the important features of the ´debug' code is that it might contain links
between each line of the source code and the respective machine code addresses.}
\PTBR{Uma das características importantes do código de ´debug' é que ele pode conter 
ligações entre cada linha do código-fonte e os respectivos endereços de código de máquina.}
\ES{Una de los características importantes del código de ´debug' és que puede contener enlaces entre
cada línea del código fuente y las direcciones de código de máquina respectivos.}
\RU{Оптимизирующие компиляторы обычно генерируют код, где целые строки из исходного кода
могут быть оптимизированы и не присутствовать в итоговом машинном коде.}
\EN{Optimizing compilers, on the other hand, tend to produce output where entire lines of source code
can be optimized away and thus not even be present in the resulting machine code.}
\PTBR{Compiladores com otimização, por outro lado, tendem a produzir uma saída onde linhas inteiras de código-fonte podem ser otimizadas a ponto de serem removidas e portanto não estarem presentes no código de máquina resultante.}
\ES{Compiladores con optimización, por otro lado, tienden a producir una salida donde líneas enteras de código fuente pueden ser optimizados al punto de ser eliminados y por consiguiente no estar presentes en el código de máquina resultante.}

\RU{Практикующий reverse engineer обычно сталкивается с обоими версиями, потому что некоторые разработчики
включают оптимизацию, некоторые другие\EMDASH{}нет. Вот почему мы постараемся поработать с примерами для обоих версий.}
\EN{Reverse engineers can encounter either version, simply because some developers turn on the compiler's optimization flags and others do not. 
Because of this, we'll try to work on examples of both debug and release versions of the code featured in this book, where possible.}
\PTBR{Engenheiros Reversos podem encontrar ambas as versões, simplesmente porque alguns desenvolvedores ativam as flags de otimização do compilador e outros não ativam. 
Por causa disso, nós tentaremos trabalhar em exemplos de ambas as versões de debug e release do código destacado neste livro, onde possível.}
\ES{Ingenieros Inversos pueden encontrar ambas versiones, simplesmente porque alguns desarrolladores activan los flags de optimización del compilador, y otros no activan. 
Debido a esto, vamos a tratar de trabajar con ejemplos de ambas versiones de debug y release del código resaltado en este libro, cuando sea posible.}

\chapter{\RU{Краткое введение в CPU}\EN{A short introduction to the CPU}\PTBR{Uma breve introdução à CPU}\ES{Una breve introducción a la CPU}}

\EN{The}\PTBR{A}\ES{La} \ac{CPU} \RU{это устройство исполняющее все программы}\EN{is the device that executes the machine code a program consists of}\PTBR{é o dispositivo que executa o código de máquina que consiste num programa}\ES{es el dispositivo que ejecuta el código de máquina que constituye un programa}.

\textbf{\RU{Немного терминологии}\EN{A short glossary}\PTBR{Um pequeno glossário}\ES{Un breve glosario}:}

\begin{description}
\item[\RU{Инструкция}\EN{Instruction}\PTBR{Instrução}\ES{Instrucción}]: \RU{примитивная команда}\EN{A primitive}\PTBR{Um primitivo}\ES{Una primitiva}
	\ac{CPU}\RU{.} \EN{command.}\PTBR{comando.}\ES{comando.}
\RU{Простейшие примеры: перемещение между регистрами, работа с памятью, примитивные арифметические операции}%
\EN{The simplest examples include: moving data between registers, working with memory, primitive arithmetic operations}
\PTBR{Os exemplos mais simples incluem: mover dados entre registradores, trabalhar com a memória, operações aritiméticas primitivas}
\ES{Los ejemplos más simples incluyen: mover datos entre registros, trabajar con la memoria, operaciones aritméticas primitivas}.
\RU{Как правило, каждый}\EN{As a rule, each}\PTBR{Como regra geral, cada}\ES{Como regla general, cada} \ac{CPU} \RU{имеет свой набор инструкций}\EN{has its own instruction set architecture}\PTBR{tem seu próprio conjunto de instruções}\ES{tiene su proprio conjunto de instrucciones} 
(\ac{ISA}).

\item[\RU{Машинный код}\EN{Machine code}]: \RU{код понимаемый}\EN{Code that the}\PTBR{Código que a}\ES{Código que la} \ac{CPU}\EN{ directly processes}\PTBR{ processa diretamente}\ES{ procesa directamente}. 
\RU{Каждая инструкция обычно кодируется несколькими байтами}\EN{Each instruction is usually encoded by several bytes}\PTBR{Cada instrução é normalmente codificada em vários bytes}\ES{Cada instrucción generalmente se codifica por vários bytes}.

\item[\RU{Язык ассемблера}\EN{Assembly language}\PTBR{Linguagem assembly}\ES{Lenguaje assembly}]: 
\RU{машинный код плюс некоторые расширения, призванные облегчить труд программиста: макросы, имена, \etc.}
\EN{Mnemonic code and some extensions like macros that are intended to make a programmer's life easier.}
\PTBR{Código mnemônico e algumas extensões como macros que têm a finalidade de facilitar a vida do programamdor.}
\ES{Código mnemónico y algunas extensiones como macros que destinados a hacer la vida del programador más fácil.}

\item[\RU{Регистр CPU}\EN{CPU register}\PTBR{Registradores da CPU}\ES{Registros de la CPU}]: 
\RU{Каждый}\EN{Each}\PTBR{Cada}\ES{Cada} \ac{CPU} \RU{имеет некоторый фиксированный набор регистров общего назначения}\EN{has a fixed set of general purpose registers}\PTBR{tem um conjunto fixo de registradores de propósito geral}\ES{tiene un conjunto fijo de registros de propósito general} (\ac{GPR}).
$\approx 8$ \InENRU x86, $\approx 16$ \InENRU x86-64, $\approx 16$ \InENRU ARM.
\RU{Проще всего понимать регистр как временную переменную без типа}%
\EN{The easiest way to understand a register is to think of it as an untyped temporary variable}
\PTBR{A forma mais fácil de entender um registrador é pensar nele como uma variável temporária não tipada}
\ES{La forma más fácil de entender un registro es pensar en ello como una variable temporal sin tipo}.
\RU{Можно представить, что вы пишете на \ac{PL} высокого уровня и у вас только 8 переменных шириной 32 (или 64) бита}%
\EN{Imagine if you were working with a high-level \ac{PL} and could only use eight 32-bit (or 64-bit) variables}
\PTBR{Imagine que você estivesse trabalhando com uma \ac{PL} de alto nível e pudesse usar apenas oito variáveis de 32-bit (ou de 64-bit)}
\ES{Imagine si estuviera trabajando con una \ac{PL} de alto nivel y sólo podría utilizar ocho variables de 32-bit (o de 64-bit)}.
\RU{Можно сделать очень много используя только их}\EN{Yet a lot can be done using just these}\PTBR{No entanto, muito ainda pode ser feito usando apenas eles}\ES{Sin embargo mucho se puede hacer usando sólo estos}!
\end{description}

\RU{Откуда взялась разница между машинным кодом и \ac{PL} высокого уровня?
Ответ в том, что люди и \ac{CPU}-ы отличаются друг от друга\EMDASH{}}
\EN{One might wonder why there needs to be a difference between machine code and a \ac{PL}.
The answer lies in the fact that humans and \ac{CPU}s are not alike\EMDASH{}}
\PTBR{Alguém poderia perguntar por que é preciso haver diferença entre código de máquina e uma \ac{PL} de alto nível.
A resposta reside no fato de que humanos e \ac{CPU}s não são iguais\EMDASH{}}
\ES{Uno podría perguntarse por qué es necessário que haya diferencia entre el código de la máquina y una lenguaje de programación de alto nivel.
La respuesta está en el hecho de que los seres humanos y CPUs no son iguales\EMDASH{}}
\RU{человеку проще писать на \ac{PL} высокого уровня вроде \CCpp, Java, Python, 
а \ac{CPU} проще работать с абстракциями куда более низкого уровня}%
\EN{it is much easier for humans to use a high-level \ac{PL} like \CCpp, Java, Python, etc., 
but it is easier for a \ac{CPU} to use a much lower level of abstraction.}
\PTBR{É muito mais fácil para os humanos usar uma \ac{PL} de alto nível como \CCpp, Java, Python, etc.,
mas é muito mais fácil para a \ac{CPU} usar um nível de abstração muito menor.}
\ES{És mucho más fácil para los humanos utilizar un \ac{PL} de alto nivel como \CCpp, Java, Python, etc., 
pero és más fácil para una \ac{CPU} utilizar un nivel mucho más bajo de abstración.}
\RU{Возможно, можно было бы придумать \ac{CPU} исполняющий код \ac{PL} высокого уровня, но он был бы значительно сложнее, чем те, что мы имеем сегодня.}
\EN{Perhaps it would be possible to invent a \ac{CPU} that can execute high-level \ac{PL} code, but it would be many times more complex than the \ac{CPU}s we know of today.}
\PTBR{talvez fosse possível inventar uma \ac{CPU} que pudesse executar código feito em \ac{PL} alto nível, mas seria inúmeras vezes mais complexa do que as \ac{CPU}s que conhecemos hoje.}
\ES{Tal vez sería posible inventar una \ac{CPU} que podría ejecutar código de \ac{PL} de alto nivel, pero sería muchas veces más compleja que las \ac{CPU}s que conocemos hoy.}
% A note on the experiments in this area (like the LISP machines http://en.wikipedia.org/wiki/Lisp_machine
% might be useful
\RU{И наоборот, человеку очень неудобно писать на ассемблере из-за его низкоуровневости,
к тому же, крайне трудно обойтись без мелких ошибок.}
\EN{In a similar fashion, it is very inconvenient for humans to write in assembly language,
due to it being so low-level and difficult to write in without making a huge number of annoying mistakes.}
\PTBR{De forma semelhante, é muito inconveninente para os seres humanos escrever em linguagem assembly, 
devido ao fato dela ser tão baixo nível e difícil de escrever sem comenter uma enorme quantidade de erros irritantes.}
\ES{En uma manera similar, es muy incómodo para los seres humanos escribir en lenguaje assembly, 
debido a que es tan bajo nivel y difícil escribir sin hacer una gran cantidade de errores molestos.}
\RU{Программа, переводящая код из \ac{PL} высокого уровня в ассемблер называется \IT{компилятором}%
\footnote{
	\RU{В более старой русскоязычной литературе также часто встречается термин \q{транслятор}.}
	\EN{Old-school Russian literature also use term \q{translator}.}
	\ESph{}
	\PTBRph{}\PLph{}\ITAph{}
}.}
\EN{The program that converts the high-level \ac{PL} code into assembly is called a \IT{compiler}.}
\PTBR{O programa que converte o código de \ac{PL} de alto nível em assembly é chamado \IT{compiler}.}
\ES{El programa que convierte el código de \ac{PL} de alto nivel en assembly se llama \IT{compiler}.}
% TODO1 add about linker: "компоновщик" и "редактор связей" в русскоязычной лит-ре

\ifx\LITE\undefined
\section{\RU{Несколько слов о разнице между \ac{ISA}}\EN{A couple of words about different \ac{ISA}s}\PTBR{Algumas palavras a respeito de diferentes \ac{ISA}s}\ES{Algunas palabras sobre diferentes \ac{ISA}s}}

\RU{x86 всегда был архитектурой с опкодами переменной длины, так что когда пришла 64-битная эра,
расширения x64 не очень сильно повлияли на \ac{ISA}.}
\RU{В x86 до сих пор есть масса инструкций, появившихся в 16-битном 8086 и присутствующих в самых последних
процессорах.}
\EN{The x86 \ac{ISA} has always been one with variable-length opcodes, so when the 64-bit era came, 
the x64 extensions did not impact the \ac{ISA} very significantly. In fact, the x86 \ac{ISA} still contains a lot of instructions that first appeared in 16-bit 8086 CPU, yet are still found in the CPUs of today.}
\PTBR{O \ac{ISA} x86 sempre possuiu opcodes de tamanho variável, então com a chegada da era do 64-bit, 
as extensões x64 não impactaram a \ac{ISA} de forma muito significante. De fato, o \ac{ISA} x86 ainda contém uma série de instrucões que surgiram inicialmente na CPU 8086 16-bit, mas ainda são encontradas nas CPUs de hoje em dia.}
\ES{El \ac{ISA} x86 siempre ha tenido opcodes de tamaño variable, de modo que cuanco llegó la era de 64-bit, 
las extensiones x64 no impactan el \ac{ISA} de manera muy significativa. De hecho, el \ac{ISA} x86 aún contiene una gran cantidade de instrucciones que primero aparecieron en CPU 8086 16-bit, pero aún se encuentran en las CPUs de hoy.}
\PLph{}\ITAph{}\\
\\
\index{ARM!\ARMMode}%
\index{ARM!\ThumbMode}%
\index{ARM!\ThumbTwoMode}%

\RU{ARM это \ac{RISC}-процессор разработанный с учетом опкодов одинаковой длины, что было некоторым преимуществом в прошлом.}
\EN{ARM is a \ac{RISC} \ac{CPU} designed with constant-length opcode in mind, which had some advantages in the past.}
\PTBR{ARM é uma \ac{CPU} \ac{RISC} desenvolvido com a idéia de opcodes com tamanho constante, o que trouxe algumas vantagens no passado.}
\ES{ARM és una \ac{CPU} \ac{RISC} diseñado con la idea de opcodes con tamaño constante, que tenía algunas ventajas en el pasado.}
\RU{Так что в самом начале все инструкции ARM кодировались 4-мя байтами}%
\EN{In the very beginning, all ARM instructions were encoded in 4 bytes}%
\PTBR{Bem no início, todas as instruções ARM foram codificadas em 4 bytes}%
\ES{En el principio, todas las instrucciones ARM fueron codificados en 4 bytes}%
\ifx\LITE\undefined
\footnote{\RU{Кстати,
инструкции фиксированного размера удобны тем, что всегда можно легко узнать адрес 
следующей (или предыдущей) инструкции. Эта особенность будет рассмотрена в секции об операторе 
switch()~(\myref{sec:SwitchARMLot}).}
\EN{By the way, fixed-length instructions are handy because one can calculate the next (or previous) 
instruction address without effort. This feature will be discussed in the switch() operator~(\myref{sec:SwitchARMLot}) section.}
\PTBR{A propósito, instruções de tamanho fixo são úteis porque se pode calcular o endereço da próxima instrução (ou da anterior) sem esforço. Esta característica será discutida na seção do operador switch() ~(\myref{sec:SwitchARMLot}).}
\ES{Dicho sea de paso, las instrucciones de longitud fija son muy útiles porque se puede calcular la dirección de instrucción siguiente (o anterior) sin esfuerzo. Esta característica se discutirá en la sección de el operador switch() ~(\myref{sec:SwitchARMLot}).}
}%
\fi
.
\RU{Это то, что сейчас называется \q{режим ARM}}\EN{This is now referred to as \q{ARM mode}}\PTBR{Este é atualmente referenciado como \q{ARM mode}}\ES{Esto actualmente se conoce como \q{ARM mode}}.

\RU{Потом они подумали, что это не очень экономично}\EN{Then they thought it wasn't as frugal as they first imagined}\PTBR{Então concluiu-se que não era tão econômico quanto se imaginou a princípio.}\ES{Entonces se llegó a la conclusión que no era tan económico como se imaginó al princípio}.
\RU{На самом деле, самые используемые инструкции\footnote{А это MOV/PUSH/CALL/Jcc} процессора на практике могут быть закодированы
c использованием меньшего количества информации.}
\EN{In fact, most used \ac{CPU} instructions\footnote{These are MOV/PUSH/CALL/Jcc} in real world applications can be encoded using less information.}
\PTBR{Na verdade, as instruções de \ac{CPU} mais utilizadas \footnote{São estas MOV/PUSH/CALL/Jcc} em aplicações do mundo real podem ser codificadas usando menos informação.}
\ES{En realidad, la mayoría de las instrucciones de \ac{CPU} utilizados \footnote{Son estos MOV/PUSH/CALL/Jcc} en aplicaciones del mundo real pueden ser codificados utilizando menos información.}
\RU{Так что они добавили другую \ac{ISA} с названием Thumb, где каждая инструкция кодируется всего лишь
2-мя байтами.}
\EN{They therefore added another \ac{ISA}, called Thumb, where each instruction was encoded in just 2 bytes.}
\PTBR{Foi adicionado então outro \ac{ISA}, chamado Thumb, onde cada instrução era codificada em apenas 2 bytes.}
\ES{Por lo tanto añadieron otra \ac{ISA}, llamado Thumb, donde cada instrucción fue codificada en sólo 2 bytes.}
\RU{Теперь это называется \q{режим Thumb}}\EN{This is now referred as \q{Thumb mode}}\PTBR{Este é conhecido como \q{Thumb mode}}\ES{Esto se conoce como \q{Thumb mode}}.
\RU{Но не все инструкции ARM могут быть закодированы в двух байтах, так что набор инструкций Thumb ограниченный.}
\EN{However, not \IT{all} ARM instructions can be encoded in just 2 bytes, so the Thumb instruction set is somewhat limited.}
\PTBR{No entanto, nem \IT{all} instruções ARM podem ser codificadas em apenas 2 bytes, então o conjunto de instruções Thumb é de certa forma limitado.}
\ES{No obstante, no todas las instrucciones ARM pueden ser codificadas en apenas 2 bytes, entonces el conjunto de instrucciones Thumb es algo limitada.}
\RU{Код, скомпилированный для режима ARM и Thumb может сосуществовать в одной программе.}
\EN{It is worth noting that code compiled for ARM mode and Thumb mode may of course coexist within one single program.}
\PTBR{É interessante notar que códigos compilados para os modos ARM e Thumb podem, conforme esperado, coexistir num mesmo programa.}
\ES{Es importante destacar que el código compilado para el modo ARM y para el modo Thumb pueden, por supuesto, coexistir dentro de un solo programa.}

\RU{Затем создатели ARM решили, что Thumb можно расширить: так появился Thumb-2 (в ARMv7).}
\EN{The ARM creators thought Thumb could be extended, giving rise to Thumb-2, which appeared in ARMv7.}
\PTBR{Os criadores do ARM concluíram que o Thumb poderia ser extendido, dando origem ao Thumb-2, que apareceu no ARMv7.}
\ES{Los creadores de ARM concluyeron que se podría extender el Thumb, dando origem al Thumb-2, que apareció en el ARMv7.}
\RU{Thumb-2 это всё ещё двухбайтные инструкции, но некоторые новые инструкции имеют длину 4 байта.}
\EN{Thumb-2 still uses 2-byte instructions, but has some new instructions which have the size of 4 bytes.}
\PTBR{Thumb-2 ainda usa instruções de 2 bytes, mas possui algumas novas instruções com 4 bytes de tamanho.}
\ES{Thumb-2 sigue utilizando instrucciones de 2 bytes, pero tiene algunas nuevas instrucciones que tienen el tamaño de 4 bytes.}
\RU{Распространено заблуждение, что Thumb-2\EMDASH{}это смесь ARM и Thumb. Это не верно. Режим Thumb-2 был дополнен до
более полной поддержки возможностей процессора и теперь может легко конкурировать с режимом ARM.
Основное количество приложений для \idevices скомпилировано для набора инструкций Thumb-2, потому что Xcode
делает так по умолчанию.}
\EN{There is a common misconception that Thumb-2 is a mix of ARM and Thumb. This is incorrect. 
Rather, Thumb-2 was extended to fully support all processor features so it could
compete with ARM mode\EMDASH{}a goal that was clearly achieved, as the majority of applications for \idevices are compiled for the Thumb-2 instruction set (admittedly, largely due to the fact that Xcode does this by default).}
\PTBR{Há um equívoco comum que Thumb-2 é uma mistura de ARM e Thumb. Isso é incorreto.
Em vez disso, Thumb-2 foi extendido para suportar completamente todos os recursos de processador de forma que ele pudesse competir com o modo ARM\EMDASH{}um objetivo que foi claramente alcançado, uma vez que a maioria das aplicações para \idevices são compiladas para o conjunto de instruções do Thumb-2 (admitidamente, principalmente devido ao fato que o Xcode faz isso por padrão).}
\ES{Hay una idea errónea de que Thumb-2 es una mezcla de ARM y Thumb. Esto es incorrecto. 
Más bien, se extendió Thumb-2 para apoyar plenamente todas las características de processador por lo que podría 
competir con el modo ARM\EMDASH{}un objetivo que se logró con claridad, ya que la mayoria de aplicacciones para \idevices son compmilados para el conjunto de instrucciones del Thumb-2 (la verdade es, en gran parte debido al hecho de que Xcode hace esto por defecto).}
\RU{Потом появился 64-битный ARM. Это \ac{ISA} снова с 4-байтными опкодами, без дополнительного режима Thumb.}
\EN{Later the 64-bit ARM came out. This \ac{ISA} has 4-byte opcodes, and lacked the need of any additional Thumb mode.}
\PTBR{Posteriormente o ARM 64-bit foi lançado. Este \ac{ISA} tem opcodes de 4 bytes, e descarta a necessidade de qualquer modo Thumb adicional.}
\ES{Más tarde, el ARM 64-bit salió. Este \ac{ISA} tiene opcodes de 4 bytes, y descarta la necesidade de cualquier modo Thumb adicional.}
\RU{Но 64-битные требования повлияли на \ac{ISA}, так что теперь у нас 3 набора инструкций ARM:
режим ARM, режим Thumb (включая Thumb-2) и ARM64.}
\EN{However, the 64-bit requirements affected the \ac{ISA}, resulting in us now having three ARM instruction sets: ARM mode, Thumb mode (including Thumb-2) and ARM64.}
\PTBR{No entanto, os requisitos de 64-bit afetaram o \ac{ISA}, resultando em termos atualmente três conjuntos de instruções ARM: ARM mode, Thumb mode (incluindo Thumb-2) e ARM64.}
\ES{Pero, los requisitos de 64-bit afectaron la \ac{ISA}, resultando en ahora tenermos tres conjuntos de instrucciones ARM: ARM mode, Thumb mode (incluyendo Thumb-2) y ARM64.}
\RU{Эти наборы инструкций частично пересекаются, но можно сказать, это скорее разные наборы, нежели вариации одного.}%
\EN{These \ac{ISA}s intersect partially, but it can be said that they are different \ac{ISA}s, rather than variations of the same one.}
\PTBR{Estes \ac{ISA}s se intersecionam parcialmente, porém podemos dizer que são \ac{ISA}s diferentes, ao invés de variações do mesmo.}
\ES{Estos \ac{ISA}s se intersectan parcialmente, pero puede ser más bien decir que son \ac{ISA}s diferentes, en lugar de variaciones de lo mismo.}
\RU{Следовательно, в этой книге постараемся добавлять фрагменты кода на всех трех ARM \ac{ISA}.}
\EN{Therefore, we would try to add fragments of code in all three ARM \ac{ISA}s in this book.}
\PTBR{Portanto, gostaríamos de tentar adicionar pedaços de código dos três \ac{ISA}s do ARM neste livro.}
\ES{Por lo tanto, nos gustaría intentar añadir fragmentos de código de los tres \ac{ISA}s del ARM en este libro.}

\index{PowerPC}%
\index{MIPS}%
\index{Alpha AXP}%

\RU{Существует ещё много \ac{RISC} \ac{ISA} с опкодами фиксированной 32-битной длины~--- это как минимум}
\EN{There are, by the way, many other \ac{RISC} \ac{ISA}s with fixed length 32-bit opcodes, such as}
\PTBR{Existem, a propósito, muitos outros \ac{RISC} \ac{ISA}s com opcodes de tamanho fixo de 32-bit, como}
\ES{Hay, por cierto, muchos otros \ac{RISC} \ac{ISA}s con opcodes de tamaño fijo de 32-bit, tales como}
MIPS, PowerPC \AndENRU Alpha AXP.
\fi

% chapters
\chapter{%
\RU{Простейшая функция}%
\EN{The simplest Function}%
\ES{Spanish text here}%
\PTBR{Brazilian portuguese text here}%
}

\RU{Наверное, простейшая из возможных функций это та что возвращает некоторую константу:}%
\EN{The simplest possible function is arguably one that simply returns a constant value:}

\RU{Вот, например}\EN{Here it is}:

\lstinputlisting[caption=\EN{\CCpp Code}\RU{Код на \CCpp}]{patterns/00_ret/1.c}

\RU{Скомпилируем её!}
\EN{Lets compile it!}

\section{x86}

\RU{И вот что делает оптимизирующий GCC}\EN{Here's what both the optimizing GCC and MSVC compilers produce on the x86 platform}:

\lstinputlisting[caption=\Optimizing GCC/MSVC (\assemblyOutput)]{patterns/00_ret/1.s}

\index{x86!\Instructions!RET}
\RU{Здесь только две инструкции. Первая помещает значение 123 в регистр \EAX, который используется
для передачи возвращаемых значений. Вторая это \RET, которая возвращает управление в вызывающую функцию.}
\EN{There are just two instructions: the first places the value 123 into the \EAX register, which is used by convention for storing the return
value and the second one is \RET, which returns execution to the \gls{caller}.}
\RU{Вызывающая функция возьмет результат из регистра \EAX.}
\EN{The caller will take the result from the \EAX register.}

\ifdefined\IncludeARM
\section{ARM}

\RU{А что насчет ARM?}\EN{There are a few differences on the ARM platform:}

\lstinputlisting[caption=\OptimizingKeilVI (\ARMMode) ASM Output]{patterns/00_ret/1_Keil_ARM_O3.s}

\RU{ARM использует регистр \Reg{0} для возврата значений, так что здесь 123 помещается в \Reg{0}.}
\EN{ARM uses the register \Reg{0} for returning the results of functions, so 123 is copied into \Reg{0}.}

\RU{Адрес возврата (\ac{RA}) в ARM не сохраняется в локальном стеке, а в регистре \ac{LR}.
Так что инструкция \TT{BX LR} делает переход по этому адресу, и это то же самое что и вернуть управление
в вызывающую ф-цию.}
%Maybe explain what a link register is, or if it is just a normal register, say so?
\EN{The return address is not saved on the local stack in the ARM \ac{ISA}, but rather in the link register, 
so the \TT{BX LR} instruction causes execution to jump to that address\EMDASH{}effectively returning execution to the \gls{caller}.}
\fi

\index{ARM!\Instructions!MOV}
\index{x86!\Instructions!MOV}
\RU{Нужно отметить, что название инструкции \MOV в x86 и ARM сбивает с толку.}
\EN{It is worth noting that \MOV is a misleading name for the instruction in both x86 and ARM \ac{ISA}s. }
\RU{На самом деле, данные не \IT{перемещаются}, а скорее \IT{копируются}.}
\EN{The data is not in fact \IT{moved}, but \IT{copied}.}

\ifdefined\IncludeMIPS
\section{MIPS}

\label{MIPS_leaf_function_ex1}
\RU{Есть два способа называть регистры в мире MIPS.}
\EN{There are two naming conventions used in the world of MIPS when naming registers:}
\RU{По номеру (от \$0 до \$31) или по псевдоимени (\$V0, \$A0, \etc{}.).}
\EN{by number (from \$0 to \$31) or by pseudoname (\$V0, \$A0, \etc{}).}
\RU{Вывод на ассемблере в GCC показывает регистры по номерам:}
\EN{The GCC assembly output below lists registers by number:}

\lstinputlisting[caption=\Optimizing GCC 4.4.5 (\assemblyOutput)]{patterns/00_ret/MIPS.s}

\dots \RU{а \IDA\EMDASH{}по псевдоименам}\EN{while \IDA does it\EMDASH{}by their pseudonames}:

\lstinputlisting[caption=\Optimizing GCC 4.4.5 (IDA)]{patterns/00_ret/MIPS_IDA.lst}

\RU{Так что регистр \$2 (или \$V0) используется для возврата значений.}
\EN{The \$2 (or \$V0) register is used to store the function's return value.}
\index{MIPS!\Pseudoinstructions!LI}
LI \RU{это}\EN{stands for} ``Load Immediate'' \EN{and is the MIPS equivalent to MOV}.

\index{MIPS!\Instructions!J}
\RU{Другая инструкция это инструкция перехода (J или JR), которая возвращает управление в 
\glslink{caller}{вызывающую ф-цию}, переходя по адресу в регистре \$31 (или \$RA).}
\EN{The other instruction is the jump instruction (J or JR) which returns the execution flow to the \gls{caller},
jumping to the address in the \$31 (or \$RA) register.}
\RU{Это аналог регистра \ac{LR} в ARM.}
\EN{This is the register analogous to \ac{LR} in ARM.}

\RU{Но почему инструкция загрузки (LI) и инструкция перехода (J или JR) поменены местами?}
\index{MIPS!Branch delay slot}
\RU{Это артефакт \ac{RISC} и называется он}
\EN{You might be wondering why positions of the the load instruction (LI) and the jump instruction (J or JR) are swapped. This is due to a \ac{RISC} feature called} ``branch delay slot''.
\RU{На самом деле, нам не нужно вникать в эти детали.}
\RU{Нужно просто запомнить: в MIPS инструкция после инструкции перехода исполняется \IT{перед} 
инструкцией перехода.}
\EN{The reason this happens is a quirk in the architecture of some RISC \ac{ISA}s and isn't important for our purposes - we just need to remember that in MIPS, the instruction following a jump or branch instruction
is executed \IT{before} the jump/brunch instruction itself.}
\RU{Таким образом, инструкция перехода всегда поменена местами с той, которая должна быть исполнена перед ней.}
\EN{As a consequence, branch instructions always swap places with the instruction which must be executed beforehand.}
% A footnote/link to http://en.wikipedia.org/wiki/Delay_slot#Branch_delay_slots or
% something similar might be useful for the people more interested in it.

\subsection{\RU{Еще кое-что об именах инструкций и регистров в MIPS}\EN{A note about MIPS instruction/register names}}

\RU{Имена регистров и инструкций в мире MIPS традиционно пишутся в нижнем регистре.}
\EN{Register and instruction names in the world of MIPS are traditionally written in lowercase.}
\RU{Но мы будем использовать верхний регистр, потому что имена инструкций и регистров других \ac{ISA} в этой книге так же в верхнем регистре.}
\EN{However, for the sake of consistency, we'll stick to using uppercase letters, as it is the convention followed by all other \ac{ISA}s featured this book.}

\fi

\chapter{\HelloWorldSectionName}
\label{sec:helloworld}

\RU{Продолжим, используя знаменитый пример из книги}
\EN{Let's use the famous example from the book}
``The C programming Language''\cite{Kernighan:1988:CPL:576122}:

\lstinputlisting{patterns/01_helloworld/hw.c}

\section{x86}

\subsection{MSVC}

\RU{Компилируем в}\EN{Let's compile it in} MSVC 2010:

\begin{lstlisting}
cl 1.cpp /Fa1.asm
\end{lstlisting}

\RU{(Ключ /Fa означает сгенерировать листинг на ассемблере)}
\EN{(/Fa option instructs the compiler to generate assembly listing file)}

\begin{lstlisting}[caption=MSVC 2010]
CONST	SEGMENT
$SG3830	DB	'hello, world', 0AH, 00H
CONST	ENDS
PUBLIC	_main
EXTRN	_printf:PROC
; Function compile flags: /Odtp
_TEXT	SEGMENT
_main	PROC
	push	ebp
	mov	ebp, esp
	push	OFFSET $SG3830
	call	_printf
	add	esp, 4
	xor	eax, eax
	pop	ebp
	ret	0
_main	ENDP
_TEXT	ENDS
\end{lstlisting}

\ifx\LITE\undefined
\RU{MSVC выдает листинги в синтаксисе Intel.}\EN{MSVC produces assembly listings in Intel-syntax.} 
\RU{Разница между синтаксисом Intel и AT\&T будет рассмотрена немного позже:}
\EN{The difference between Intel-syntax and AT\&T-syntax will be discussed in} \myref{ATT_syntax}.
\fi

\RU{Компилятор сгенерировал файл \TT{1.obj}, который впоследствии будет слинкован линкером в \TT{1.exe}.}
\EN{The compiler generated the file, \TT{1.obj}, which is to be linked into \TT{1.exe}.}
\RU{В нашем случае этот файл состоит из двух сегментов: \TT{CONST} (для данных-констант) и \TT{\_TEXT} (для кода).}
\EN{In our case, the file contains two segments: \TT{CONST} (for data constants) and \TT{\_TEXT} (for code).}

\index{\CLanguageElements!const}
\label{string_is_const_char}
\RU{Строка \TT{hello, world} в \CCpp имеет тип \TT{const char[]} \cite[p176, 7.3.2]{TCPPPL}, 
однако не имеет имени.}
\EN{The string \TT{hello, world} in \CCpp has type \TT{const char[]} \cite[p176, 7.3.2]{TCPPPL},
but it does not have its own name.}
\RU{Но компилятору нужно как-то с ней работать, поэтому он дает ей внутреннее имя \TT{\$SG3830}.}%
\EN{The compiler needs to deal with the string somehow so it defines the internal name \TT{\$SG3830} for it.}%

\RU{Поэтому пример можно было бы переписать вот так}\EN{That is why the example may be rewritten as follows}:

\lstinputlisting{patterns/01_helloworld/hw_2.c}

\RU{Вернемся к листингу на ассемблере. Как видно, строка заканчивается нулевым байтом~--- 
это требования стандарта \CCpp для строк}%
\EN{Let's go back to the assembly listing. 
As we can see, the string is terminated by a zero byte, which is standard for \CCpp strings}.
\RU{Больше о строках в Си}\EN{More about C strings}: \myref{C_strings}.

\RU{В сегменте кода \TT{\_TEXT} находится пока только одна функция}%
\EN{In the code segment, \TT{\_TEXT}, there is only one function so far}: \main.
\RU{Функция \main, как и практически все функции, начинается с пролога и заканчивается эпилогом}%
\EN{The function \main starts with prologue code and ends with epilogue code (like almost any function)}%
\footnote{\RU{Об этом смотрите подробнее в разделе о прологе и эпилоге функции}%
\EN{You can read more about it in the section about function prologues and epilogues}%
~(\myref{sec:prologepilog}).}.

\index{x86!\Instructions!CALL}
\RU{Далее следует вызов функции \printf}
\EN{After the function prologue we see the call to the \printf function}: \TT{CALL \_printf}. 
\index{x86!\Instructions!PUSH}
\RU{Перед этим вызовом адрес строки (или указатель на неё) с нашим приветствием при помощи инструкции \PUSH помещается в стек.}
\EN{Before the call the string address (or a pointer to it) containing our greeting is placed on the stack with the help of the \PUSH instruction.}

\RU{После того, как функция \printf возвращает управление в функцию \main, адрес строки (или указатель на неё) всё ещё лежит в стеке.}
\EN{When the \printf function returns the control to the \main function, the string address (or a pointer to it) is still on the stack.}
\RU{Так как он больше не нужен, то \glslink{stack pointer}{указатель стека} (регистр \ESP) корректируется.} 
\EN{Since we do not need it anymore, the \gls{stack pointer} (the \ESP register) needs to be corrected.}

\index{x86!\Instructions!ADD}
\TT{ADD ESP, 4} \RU{означает прибавить 4 к значению в регистре \ESP.}
\EN{means add 4 to the \ESP register value.}
\RU{Почему 4? Так как это 32-битный код, для передачи адреса нужно 4 байта. В x64-коде это 8 байт.}
\EN{Why 4? Since this is a 32-bit program, we need exactly 4 bytes for address passing through the stack. 
If it was x64 code we would need 8 bytes.}
\TT{ADD ESP, 4} \RU{эквивалентно \TT{POP регистр}, но без использования какого-либо регистра\footnote{Флаги
процессора, впрочем, модифицируются}.}
\EN{is effectively equivalent to \TT{POP register} but without using any register\footnote{CPU flags, however, are modified}.}

\index{Intel C++}
\index{\oracle}
\index{x86!\Instructions!POP}

\RU{Некоторые компиляторы, например, Intel C++ Compiler, в этой же ситуации могут вместо 
\ADD сгенерировать \TT{POP ECX} (подобное можно встретить, например, в коде \oracle{}, им скомпилированном),
что почти то же самое, только портится значение в регистре \ECX.}
\EN{For the same purpose, some compilers (like the Intel C++ Compiler) may emit \TT{POP ECX} 
instead of \ADD (e.g., such a pattern can be observed in the \oracle{} code as it is compiled with the Intel C++ compiler).
This instruction has almost the same effect but the \ECX register contents will be overwritten.}
\RU{Возможно, компилятор применяет \TT{POP ECX}, потому что эта инструкция короче (1 байт у \TT{POP} против 3 у \TT{ADD}).}
\EN{The Intel C++ compiler probably uses \TT{POP ECX} since this instruction's opcode is shorter than 
\TT{ADD ESP, x} (1 byte for \TT{POP} against 3 for \TT{ADD}).}

\RU{Вот пример использования \TT{POP} вместо \TT{ADD} из \oracle}\EN{Here is an example of using \TT{POP} instead of \TT{ADD} from \oracle}:

\begin{lstlisting}[caption=\oracle 10.2 Linux (\RU{файл }app.o\EN{ file})]
.text:0800029A                 push    ebx
.text:0800029B                 call    qksfroChild
.text:080002A0                 pop     ecx
\end{lstlisting}

%\RU{О стеке можно прочитать в соответствующем разделе}
%\EN{Read more about the stack in section}~(\myref{sec:stack}).
\index{\CLanguageElements!return}
\RU{После вызова \printf в оригинальном коде на \CCpp указано \TT{return 0}~--- вернуть 0 
в качестве результата функции \main.} 
\EN{After calling \printf, the original \CCpp code contains the statement \TT{return 0}~---return 0 as the result of the \main function.}
\index{x86!\Instructions!XOR}
\RU{В сгенерированном коде это обеспечивается инструкцией}
\EN{In the generated code this is implemented by the instruction} \TT{XOR EAX, EAX}.
\index{x86!\Instructions!MOV}
\RU{\XOR, как легко догадаться~--- \q{исключающее ИЛИ}}%
\EN{\XOR is in fact just \q{eXclusive OR}}%
\footnote{\href{http://go.yurichev.com/17118}{wikipedia}}
\RU{, но компиляторы часто используют его вместо простого}
\EN{but the compilers often use it instead of}
\TT{MOV EAX, 0}\EMDASH{}\RU{снова потому, что опкод короче (2 байта у \TT{XOR} против 5 у \TT{MOV}).}
\EN{again because it is a slightly shorter opcode (2 bytes for \TT{XOR} against 5 for \TT{MOV}).}

\index{x86!\Instructions!SUB}
\RU{Некоторые компиляторы генерируют}\EN{Some compilers emit}
\INS{SUB EAX, EAX},
\RU{что значит \IT{отнять значение в} \EAX \IT{от значения в }\EAX,
что в любом случае даст 0 в результате.}
\EN{which means \IT{SUBtract the value in the} \EAX \IT{from the value in} \EAX,
which, in any case, results in zero.}

\index{x86!\Instructions!RET}
\RU{Самая последняя инструкция \RET возвращает управление в вызывающую функцию.
Обычно это код \CCpp \ac{CRT}, который, в свою очередь, 
вернёт управление операционной системе.}
\EN{The last instruction \RET returns the control to the \gls{caller}.
Usually, this is \CCpp \ac{CRT} code, which, in turn, returns control to the \ac{OS}.}

\ifdefined\IncludeGCC
\subsection{GCC}

\RU{Теперь скомпилируем то же самое компилятором GCC 4.4.1 в Linux}%
\EN{Now let's try to compile the same \CCpp code in the GCC 4.4.1 compiler in Linux}: \TT{gcc 1.c -o 1}
\RU{Затем при помощи \IDA посмотрим как скомпилировалась функция \main.}
\EN{Next, with the assistance of the \IDA disassembler, let's see how the \main function was created.} 
\IDA, \RU{как и MSVC, показывает код в синтаксисе Intel}\EN{like MSVC, uses Intel-syntax}%
\footnote{\RU{Мы также можем заставить GCC генерировать листинги в этом формате при помощи ключей}
\EN{We could also have GCC produce assembly listings in Intel-syntax by applying the options} 
\TT{-S -masm=intel}.}.

\begin{lstlisting}[caption=\RU{код в}\EN{code in} \IDA]
main            proc near

var_10          = dword ptr -10h

                push    ebp
                mov     ebp, esp
                and     esp, 0FFFFFFF0h
                sub     esp, 10h
                mov     eax, offset aHelloWorld ; "hello, world\n"
                mov     [esp+10h+var_10], eax
                call    _printf
                mov     eax, 0
                leave
                retn
main            endp
\end{lstlisting}

\index{Function prologue}
\index{x86!\Instructions!AND}
\RU{Почти то же самое. 
Адрес строки \TT{hello, world}, лежащей в сегменте данных, вначале сохраняется в \EAX, затем записывается в стек.
А ещё в прологе функции мы видим \TT{AND ESP, 0FFFFFFF0h}~--- 
эта инструкция выравнивает значение в \ESP по 16-байтной границе, делая все значения 
в стеке также выровненными по этой границе (процессор более эффективно работает с переменными, расположенными
в памяти по адресам кратным 4 или 16)\footnote{\URLWPDA}.}
\EN{The result is almost the same.
The address of the \TT{hello, world} string (stored in the data segment) is loaded in the \EAX register first and then it is saved onto the stack.
In addition, the function prologue contains \TT{AND ESP, 0FFFFFFF0h}~---this 
instruction aligns the \ESP register value on a 16-byte boundary.
This results in all values in the stack being aligned the same way (The CPU performs better if the values it is dealing with are located in memory at addresses aligned 
on a 4-byte or 16-byte boundary)\footnote{\URLWPDA}.}

\index{x86!\Instructions!SUB}
\TT{SUB ESP, 10h} \RU{выделяет в стеке 16 байт. Хотя, как будет видно далее, здесь достаточно только 4.}
\EN{allocates 16 bytes on the stack. Although, as we can see hereafter, only 4 are necessary here.} 

\RU{Это происходит потому, что количество выделяемого места в локальном стеке тоже выровнено по 
16-байтной границе.}
\EN{This is because the size of the allocated stack is also aligned on a 16-byte boundary.}

% TODO1: rewrite.
\index{x86!\Instructions!PUSH}
\RU{Адрес строки (или указатель на строку) затем записывается прямо в стек без помощи инструкции \PUSH.
\IT{var\_10} одновременно и локальная переменная и аргумент для \printf{}. Подробнее об этом будет ниже.}
\EN{The string address (or a pointer to the string) is then stored directly onto the stack without using the \PUSH instruction.
\IT{var\_10}~---is a local variable and is also an argument for \printf{}.
Read about it below.}

\RU{Затем вызывается \printf.}\EN{Then the \printf function is called.}

\RU{В отличие от MSVC, GCC в компиляции без включенной оптимизации генерирует \TT{MOV EAX, 0} вместо 
более короткого опкода.}\EN{Unlike MSVC, when GCC is compiling without optimization turned on,
it emits \TT{MOV EAX, 0} instead of a shorter opcode.}

\index{x86!\Instructions!LEAVE}
\RU{Последняя инструкция \LEAVE~--- это аналог команд \TT{MOV ESP, EBP} и \TT{POP EBP}~--- 
то есть возврат \glslink{stack pointer}{указателя стека} и регистра \EBP в первоначальное состояние.} 
\EN{The last instruction, \LEAVE~---is the equivalent of the \TT{MOV ESP, EBP} and \TT{POP EBP} instruction 
pair~---in other words, this instruction sets the \gls{stack pointer} (\ESP) back and restores 
the \EBP register to its initial state.}
\RU{Это необходимо, т.к. в начале функции мы модифицировали регистры \ESP и \EBP (при помощи}
\EN{This is necessary since we modified these register values (\ESP and \EBP) at the 
beginning of the function (by executing}
\TT{\MOV EBP, ESP} / \TT{AND ESP, \ldots}).

\subsection{GCC: \ATTSyntax}
\label{ATT_syntax}

\RU{Попробуем посмотреть, как выглядит то же самое в синтаксисе AT\&T языка ассемблера.}
\EN{Let's see how this can be represented in assembly language AT\&T syntax.}
\RU{Этот синтаксис больше распространен в UNIX-мире.}
\EN{This syntax is much more popular in the UNIX-world.}

\begin{lstlisting}[caption=\RU{компилируем в}\EN{let's compile in} GCC 4.7.3]
gcc -S 1_1.c
\end{lstlisting}

\RU{Получим такой файл:}\EN{We get this:}

\lstinputlisting[caption=GCC 4.7.3]{patterns/01_helloworld/GCC.s}

\RU{Здесь много макросов (начинающихся с точки). Они нам пока не интересны.}
\EN{The listing contains many macros (beginning with dot). These are not interesting for us at the moment.}
\RU{Пока что, ради упрощения, мы можем
их игнорировать (кроме макроса \IT{.string}, при помощи которого кодируется последовательность символов, 
оканчивающихся нулем~--- такие же строки как в Си). И тогда получится следующее}%
\EN{For now, for the sake of simplification, we can ignore them (except the \IT{.string} macro which
encodes a null-terminated character sequence just like a C-string). Then we'll see this}%
%
% TODO: I would suggest moving this particular footnote to the main text. IMHO this will improve the readability.
\footnote{\RU{Кстати, для уменьшения генерации \q{лишних} макросов, можно использовать такой ключ GCC}%
\EN{This GCC option can be used to eliminate \q{unnecessary} macros}: 
\IT{-fno-asynchronous-unwind-tables}}:

\lstinputlisting[caption=GCC 4.7.3]{patterns/01_helloworld/GCC_refined.s}

\index{\ATTSyntax}
\index{\IntelSyntax}
\RU{Основные отличия синтаксиса Intel и AT\&T следующие:}
\EN{Some of the major differences between Intel and AT\&T syntax are:}

\begin{itemize}

\item
\RU{Операнды записываются наоборот.}\EN{Source and destination operands are written in opposite order.}

\RU{В Intel-синтаксисе: <инструкция> <операнд назначения> <операнд-источник>.}
\EN{In Intel-syntax: <instruction> <destination operand> <source operand>.}

\RU{В AT\&T-синтаксисе: <инструкция> <операнд-источник> <операнд назначения>.}
\EN{In AT\&T syntax: <instruction> <source operand> <destination operand>.}

\RU{Чтобы легче понимать разницу, можно запомнить следующее}%
\EN{Here is an easy way to memorise the difference}: \RU{когда вы работаете с синтаксисом Intel~--- можете в уме ставить знак равенства ($=$) между операндами,}
\EN{when you deal with Intel-syntax, you can imagine that there is an equality sign ($=$) between operands}
\RU{а когда с синтаксисом AT\&T~--- мысленно ставьте стрелку направо}
\EN{and when you deal with AT\&T-syntax imagine there is a right arrow} 
($\rightarrow$)
\footnote{
\index{\CStandardLibrary!memcpy()}
\index{\CStandardLibrary!strcpy()}
\RU{Кстати, в некоторых стандартных функциях библиотеки Си (например, memcpy(), strcpy()) также применяется 
расстановка аргументов как в синтаксисе Intel: вначале указатель в памяти на блок назначения, 
затем указатель на блок-источник.}\EN{By the way, in some C standard functions (e.g., memcpy(), strcpy()) the arguments
are listed in the same way as in Intel-syntax: first the pointer to the destination memory block, and then
the pointer to the source memory block.}}.

\item
AT\&T: \RU{Перед именами регистров ставится знак процента (\%), а перед числами знак доллара (\$).}
\EN{Before register names, a percent sign must be written (\%) and before numbers a dollar sign (\$).}
\RU{Вместо квадратных скобок применяются круглые.}\EN{Parentheses are used instead of brackets.}

\item
AT\&T: \RU{К каждой инструкции добавляется специальный символ, определяющий тип данных:}
\EN{A suffix is added to instructions to define the operand size:}

\begin{itemize}
\item q --- quad (64 \RU{бита}\EN{bits})
\item l --- long (32 \RU{бита}\EN{bits})
\item w --- word (16 \RU{бит}\EN{bits})
\item b --- byte (8 \RU{бит}\EN{bits})
\end{itemize}

% TODO1 simple example may be? \RU{Например mov\textbf{l}, movb, movw представляют различые версии инсструкция mov} \EN {For example: movl, movb, movw are variations of the mov instruciton}

\end{itemize}

\RU{Возвращаясь к результату компиляции: он идентичен тому, который мы посмотрели в \IDA.}
\EN{Let's go back to the compiled result: it is identical to what we saw in \IDA.}
\RU{Одна мелочь}\EN{With one subtle difference}: \TT{0FFFFFFF0h} \RU{записывается как}\EN{is presented as} \TT{\$-16}.
\RU{Это то же самое}\EN{It is the same thing}: \TT{16} \RU{в десятичной системе это}\EN{in the decimal system is} \TT{0x10} 
\RU{в шестнадцатеричной}\EN{in hexadecimal}. 
\TT{-0x10} \RU{будет как раз}\EN{is equal to} \TT{0xFFFFFFF0} 
(\RU{в рамках 32-битных чисел}\EN{for a 32-bit data type}).

\index{x86!\Instructions!MOV}
\RU{Возвращаемый результат устанавливается в 0 обычной инструкцией \MOV, а не \XOR}%
\EN{One more thing: the return value is to be set to 0 by using the usual \MOV, not \XOR}.
\MOV \RU{просто загружает значение в регистр}\EN{just loads a value to a register}. 
\RU{Её название не очень удачное (данные не перемещаются, а копируются). 
В других архитектурах подобная инструкция обычно носит название 
\q{LOAD} или \q{STORE} или что-то в этом роде.}
\EN{Its name is a misnomer (data is not moved but rather copied).
In other architectures, this instruction is named \q{LOAD} or \q{STORE} or something similar.}

\fi

\section{x86-64}
\subsection{MSVC\EMDASH{}x86-64}

\index{x86-64}
\RU{Попробуем также 64-битный MSVC}\EN{Let's also try 64-bit MSVC}:

\lstinputlisting[caption=MSVC 2012 x64]{patterns/01_helloworld/MSVC_x64.asm}

\RU{В x86-64 все регистры были расширены до 64-х бит и теперь имеют префикс \TT{R-}}%
\EN{In x86-64, all registers were extended to 64-bit and now their names have an \TT{R-} prefix}.
\index{fastcall}
\RU{Чтобы поменьше задействовать стек (иными словами, поменьше обращаться кэшу и внешней памяти), уже давно имелся
довольно популярный метод передачи аргументов функции через регистры}
\EN{In order to use the stack less often (in other words, to access external memory/cache less often), there exists
a popular way to pass function arguments via registers} 
(fastcall%
\ifx\LITE\undefined%
: \myref{fastcall}%
\fi
).
\RU{Т.е. часть аргументов функции передается через регистры и часть}\EN{I.e., a part
of the function arguments is passed in registers, the rest}\EMDASH{}\RU{через стек}\EN{via the stack}.
\RU{В Win64 первые 4 аргумента функции передаются через регистры}\EN{In Win64, 4 function arguments
are passed in the} \RCX, \RDX, \Reg{8}, \Reg{9}\EN{ registers}.
\RU{Это мы здесь и видим: указатель на строку в \printf теперь передается не через стек, а через регистр \RCX}%
\EN{That is what we see here: a pointer to the string for \printf is now passed not in the stack, but in the \RCX register}.

\RU{Указатели теперь 64-битные, так что они передаются через 64-битные части регистров (имеющие префикс \TT{R-})}%
\EN{The pointers are 64-bit now, so they are passed in the 64-bit registers (which have the \TT{R-} prefix)}.
\RU{Но для обратной совместимости можно обращаться и к нижним 32 битам регистров используя префикс \TT{E-}}%
\EN{However, for backward compatibility, it is still possible to access the 32-bit parts, using the \TT{E-} prefix}.

\RU{Вот как выглядит регистр}\EN{This is how the} \RAX/\EAX/\AX/\AL \RU{в}\EN{register looks like in} x86-64:

\RegTableOne{RAX}{EAX}{AX}{AH}{AL}

\RU{Функция \main возвращает значение типа \Tint, который в \CCpp, вероятно для лучшей совместимости и переносимости,
оставили 32-битным. Вот почему в конце функции \main обнуляется не \RAX, а \EAX, т.е. 32-битная часть регистра.}
\EN{The \main function returns an \Tint{}-typed value, which is, in \CCpp, for better backward compatibility
and portability, still 32-bit, so that is why the \EAX register is cleared at the function end (i.e., the 32-bit
part of the register) instead of \RAX{}.}

\RU{Также видно, что 40 байт выделяются в локальном стеке}\EN{There are also 40 bytes allocated in the local stack}.
\RU{Это}\EN{This is called the} \q{shadow space}, 
\RU{которое мы будем рассматривать позже}%
\EN{about which we are going to talk later}: \myref{shadow_space}.

\ifdefined\IncludeGCC
\subsection{GCC\EMDASH{}x86-64}

\index{x86-64}
\RU{Попробуем GCC в 64-битном Linux}\EN{Let's also try GCC in 64-bit Linux}:

\lstinputlisting[caption=GCC 4.4.6 x64]{patterns/01_helloworld/GCC_x64.s.\LANG}

\RU{В Linux, *BSD и \MacOSX для x86-64 также принят способ передачи аргументов функции через регистры}
\EN{A method to pass function arguments in registers is also used in Linux, *BSD and \MacOSX}\cite{SysVABI}.
\RU{6 первых аргументов передаются через регистры}\EN{The first 6 arguments are passed in the}
\RDI, \RSI, \RDX, \RCX, \Reg{8}, \Reg{9}\RU{, а остальные}\EN{ registers, and the rest}\EMDASH{}\RU{через стек}\EN{via
the stack}.

\RU{Так что указатель на строку передается через \EDI (32-битную часть регистра)}\EN{So the pointer to the
string is passed in \EDI (the 32-bit part of the register)}.
\RU{Но почему не через 64-битную часть}\EN{But why not use the 64-bit part}, \RDI?

\RU{Важно запомнить что в 64-битном режиме все инструкции \MOV, записывающие что-либо в 
младшую 32-битную часть регистра,
обнуляют старшие 32-бита}\EN{It is important to keep in mind that all \MOV instructions in 64-bit mode
that write something into the lower 32-bit register part also clear the higher 32-bits}\cite{Intel}.
\RU{То есть, инструкция}\EN{I.e., the} \TT{MOV EAX, 011223344h} \RU{корректно запишет это значение в \RAX, 
старшие биты сбросятся в ноль}\EN{writes a value into \RAX correctly, since the higher bits will be cleared}.

\RU{Если посмотреть в \IDA скомпилированный объектный файл (.o), увидим также опкоды всех инструкций}%
\EN{If we open the compiled object file (.o), we can also see all the instructions' opcodes}%
\footnote{\RU{Это нужно задать в}\EN{This must be enabled in} 
Options $\rightarrow$ Disassembly $\rightarrow$ Number of opcode bytes}:

\lstinputlisting[caption=GCC 4.4.6 x64]{patterns/01_helloworld/GCC_x64.lst}

\label{hw_EDI_instead_of_RDI}
\RU{Как видно, инструкция, записывающая в \EDI по адресу \TT{0x4004D4}, занимает 5 байт}%
\EN{As we can see, the instruction that writes into \EDI at \TT{0x4004D4} occupies 5 bytes}.
\RU{Та же инструкция, записывающая 64-битное значение в \RDI, занимает 7 байт.}
\EN{The same instruction writing a 64-bit value into \RDI occupies 7 bytes.}
\RU{Возможно, GCC решил немного сэкономить}%
\EN{Apparently, GCC is trying to save some space}. 
\RU{К тому же, вероятно, он уверен, что сегмент данных, где хранится строка,
никогда не будет расположен в адресах выше 4\gls{GiB}.}
\EN{Besides, it can be sure that the data segment containing
the string will not be allocated at the addresses higher than 4\gls{GiB}.}

\label{SysVABI_input_EAX}
\RU{Здесь мы также видим обнуление регистра \EAX перед вызовом \printf.
Это делается потому что по стандарту передачи аргументов в *NIX для x86-64 в \EAX передается количество задействованных векторных регистров
(\cite{SysVABI}).}
\EN{We also see that the \EAX register was cleared before the \printf function call.
This is done because the number of used vector registers is passed in \EAX in *NIX systems on x86-64 
(\cite{SysVABI}).}


\fi

\ifdefined\IncludeGCC
\section{GCC\EMDASH{}\EN{one more thing}\RU{ещё кое-что}}
\label{use_parts_of_C_strings}

\RU{Тот факт, что \IT{анонимная} Си-строка имеет тип}\EN{The fact that an \IT{anonymous} C-string has} 
\IT{const}\EN{ type} (\myref{string_is_const_char}), 
\RU{и тот факт, что выделенные в сегменте констант Си-строки гаратировано неизменяемые (immutable), 
ведет к интересному следствию}\EN{and
that C-strings allocated in constants segment are guaranteed to be immutable, has an interesting consequence}:
\RU{компилятор может использовать определенную часть строки}\EN{the compiler may use a specific part of the string}.

\RU{Вот простой пример}\EN{Let's try this example}:

\begin{lstlisting}
#include <stdio.h>

int f1()
{
	printf ("world\n");
}

int f2()
{
	printf ("hello world\n");
}

int main()
{
	f1();
	f2();
}
\end{lstlisting}

\RU{Среднестатистический компилятор с \CCpp (включая MSVC) выделит место для двух строк, 
но вот что делает GCC 4.8.1}%
\EN{Common \CCpp{}-compilers (including MSVC) allocate two strings, 
but let's see what GCC 4.8.1 does}:

\begin{lstlisting}[caption=GCC 4.8.1 + \RU{листинг в }IDA\EN{ listing}]
f1              proc near

s               = dword ptr -1Ch

                sub     esp, 1Ch
                mov     [esp+1Ch+s], offset s ; "world\n"
                call    _puts
                add     esp, 1Ch
                retn
f1              endp

f2              proc near

s               = dword ptr -1Ch

                sub     esp, 1Ch
                mov     [esp+1Ch+s], offset aHello ; "hello "
                call    _puts
                add     esp, 1Ch
                retn
f2              endp

aHello          db 'hello '
s               db 'world',0xa,0
\end{lstlisting}

\RU{Действительно, когда мы выводим строку}\EN{Indeed: when we print the \q{hello world} string}, 
\RU{эти два слова расположены в памяти впритык друг к другу и \puts, вызываясь из функции f2(), вообще не знает,
что эти строки разделены}\EN{these two words are positioned in memory adjacently and \puts called from f2() 
function is not aware that this string is divided}. \RU{Они и не разделены на самом деле, они разделены
только \q{виртуально}, в нашем листинге}\EN{In fact, it's not divided; it's divided only \q{virtually}, in this
listing}.

\RU{Когда}\EN{When} \puts \RU{вызывается из f1(), он использует строку}\EN{is called from f1(), it uses the} 
\q{world} \RU{плюс нулевой байт}\EN{string plus a zero byte}. \puts \RU{не знает, что там ещё есть какая-то строка
перед этой}\EN{is not aware that there is something before this string}!

\RU{Этот трюк часто используется (по крайней мере в GCC) и может сэкономить немного памяти.}
\EN{This clever trick is often used by at least GCC and can save some memory.}

\fi
\ifdefined\IncludeARM
\section{ARM}
\label{sec:hw_ARM}

\index{\idevices}
\index{Raspberry Pi}
\index{Xcode}
\index{LLVM}
\index{Keil}
\RU{Для экспериментов с процессором ARM несколько компиляторов было использовано:}%
\EN{For my experiments with ARM processors, several compilers were used:} 

\begin{itemize}
\item \RU{Популярный в embedded-среде}\EN{Popular in the embedded area:} Keil Release 6/2013.

\item Apple Xcode 4.6.3 \EN{IDE} (\RU{с компилятором}\EN{with the} LLVM-GCC 4.2\EN{ compiler}%
\footnote{\EN{It is indeed so: Apple Xcode 4.6.3 uses open-source GCC as front-end compiler and LLVM 
code generator}\RU{Это действительно так: Apple Xcode 4.6.3 использует опен-сорсный GCC как компилятор
переднего плана и кодогенератор LLVM}}).

%\item GCC 4.8.1 (Linaro) (\RU{для}\EN{for} ARM64).
%
\item GCC 4.9 (Linaro) (\RU{для}\EN{for} ARM64), 
\RU{доступный в виде исполняемого файла для win32 на}\EN{available as win32-executables at} 
\url{http://go.yurichev.com/17325}.

\end{itemize}

\RU{Везде в этой книге, если не указано иное, идет речь о 32-битном ARM (включая режимы Thumb и Thumb-2).}
\EN{32-bit ARM code is used (including Thumb and Thumb-2 modes) in all cases in this book, if not mentioned otherwise.}
\RU{Когда речь идет о 64-битном ARM, он называется здесь ARM64.}
\EN{When we talk about 64-bit ARM here, we call it ARM64.}

% subsections
\subsection{\NonOptimizingKeilVI (\ARMMode)}

\RU{Для начала скомпилируем наш пример в Keil}\EN{Let's start by compiling our example in Keil}:

\begin{lstlisting}
armcc.exe --arm --c90 -O0 1.c 
\end{lstlisting}

\index{\IntelSyntax}
\RU{Компилятор \IT{armcc} генерирует листинг на ассемблере в формате Intel.}
\EN{The \IT{armcc} compiler produces assembly listings in Intel-syntax} 
\RU{Этот листинг содержит некоторые высокоуровневые макросы, связанные с ARM}%
\EN{but it has high-level ARM-processor related macros}\footnote{
\RU{например, он показывает инструкции \PUSH/\POP, отсутствующие в режиме ARM}
\EN{e.g. ARM mode lacks \PUSH/\POP instructions}}, 
\RU{а нам важнее увидеть инструкции \q{как есть}, так что посмотрим скомпилированный результат в \IDA.}
\EN{but it is more important for us to see the instructions \q{as is} so let's see the compiled result in \IDA.}

\begin{lstlisting}[caption=\NonOptimizingKeilVI (\ARMMode) \IDA]
.text:00000000             main
.text:00000000 10 40 2D E9    STMFD   SP!, {R4,LR}
.text:00000004 1E 0E 8F E2    ADR     R0, aHelloWorld ; "hello, world"
.text:00000008 15 19 00 EB    BL      __2printf
.text:0000000C 00 00 A0 E3    MOV     R0, #0
.text:00000010 10 80 BD E8    LDMFD   SP!, {R4,PC}

.text:000001EC 68 65 6C 6C+aHelloWorld  DCB "hello, world",0    ; DATA XREF: main+4
\end{lstlisting}

\RU{В вышеприведённом примере можно легко увидеть, что каждая инструкция имеет размер 4 байта.}
\EN{In the example, we can easily see each instruction has a size of 4 bytes.}
\RU{Действительно, ведь мы же компилировали наш код для режима ARM, а не Thumb.}
\EN{Indeed, we compiled our code for ARM mode, not for Thumb.}

\index{ARM!\Instructions!STMFD}
\index{ARM!\Instructions!POP}
\RU{Самая первая инструкция}\EN{The very first instruction}, \TT{STMFD SP!, \{R4,LR\}}\footnote{\ac{STMFD}}, 
\RU{работает как инструкция}\EN{works as an x86} \PUSH \RU{в x86}\EN{instruction},
\RU{записывая значения двух регистров}\EN{writing the values of two registers}
(\Reg{4} \AndENRU \ac{LR}) \RU{в стек}\EN{into the stack}.
\RU{Действительно, в выдаваемом листинге на ассемблере компилятор \IT{armcc} для упрощения указывает здесь инструкцию}
\EN{Indeed, in the output listing from the \IT{armcc} compiler, for the sake of simplification, 
actually shows the} \TT{PUSH \{r4,lr\}}\EN{ instruction}.
\RU{Но это не совсем точно, инструкция \PUSH доступна только в режиме Thumb, поэтому,
во избежание путаницы, я предложил работать в \IDA}%
\EN{But that is not quite precise. The \PUSH instruction is only available in Thumb mode.
So, to make things less confusing, we're doing this in \IDA}.

\RU{Итак, эта инструкция уменьшает \ac{SP}, чтобы он указывал на место в стеке, свободное для записи
новых значений, затем записывает значения регистров \Reg{4} и \ac{LR} 
по адресу в памяти, на который указывает измененный регистр \ac{SP}}%
\EN{This instruction first \glspl{decrement} the \ac{SP} so it points to the place in the stack
that is free for new entries, then it saves the values of the \Reg{4} and \ac{LR} registers at the address
stored in the modified \ac{SP}}.

\RU{Эта инструкция, как и инструкция \PUSH в режиме Thumb, может сохранить в стеке одновременно несколько значений регистров, что может быть очень удобно}%
\EN{This instruction (like the \PUSH instruction in Thumb mode) is able to save several register values at once which can be very useful}. 
\RU{Кстати, такого в x86 нет}\EN{By the way, this has no equivalent in x86}.
\RU{Также следует заметить, что \TT{STMFD}~--- генерализация инструкции \PUSH (то есть расширяет её возможности), потому что может работать с любым регистром, а не только с \ac{SP}.}
\EN{It can also be noted that the \TT{STMFD} instruction is a generalization 
of the \PUSH instruction (extending its features), since it can work with any register, not just with \ac{SP}.}
\RU{Другими словами, \TT{STMFD} можно использовать для записи набора регистров в указанном месте памяти.}
\EN{In other words, \TT{STMFD} may be used for storing a set of registers at the specified memory address.}

\index{\PICcode}
\index{ARM!\Instructions!ADR}
\RU{Инструкция}\EN{The} \TT{ADR R0, aHelloWorld}
\RU{прибавляет или отнимает значение регистра \ac{PC} к смещению, где хранится строка}
\EN{instruction adds or subtracts the value in the \ac{PC} register to the offset where the}
\TT{hello, world}\EN{ string is located}.
\RU{Причем здесь \ac{PC}, можно спросить}\EN{How is the \TT{PC} register used here, one might ask}?
\RU{Притом, что это так называемый \q{\PICcode}}\EN{This is called \q{\PICcode}.}
\footnote{
	\RU{Читайте больше об этом в соответствующем разделе}
	\EN{Read more about it in relevant section}~(\myref{sec:PIC})
	}
\RU{он предназначен для исполнения будучи не привязанным к каким-либо адресам в памяти}%
\EN{Such code can be be executed at a non-fixed address in memory}.
\EN{In other words, this is \ac{PC}-relative addressing.}
\RU{Другими словами, это относительная от \ac{PC} адресация.}
\RU{В опкоде инструкции \TT{ADR} указывается разница между адресом этой инструкции и местом, где хранится строка}%
\EN{The \TT{ADR} instruction takes into account the difference between the address of this instruction and the address where the string is located}.
\RU{Эта разница всегда будет постоянной, вне зависимости от того, куда был загружен \ac{OS} наш код}%
\EN{This difference (offset) is always to be the same, no matter at what address our code is loaded by the \ac{OS}}.
\RU{Поэтому всё, что нужно~--- это прибавить адрес текущей инструкции (из \ac{PC}), чтобы получить текущий абсолютный адрес нашей Си-строки}%
\EN{That's why all we need is to add the address of the current instruction (from \ac{PC}) in order to get the absolute memory address of our C-string}.

\index{ARM!\Registers!Link Register}
\index{ARM!\Instructions!BL}
\RU{Инструкция} \TT{BL \_\_2printf}\footnote{Branch with Link}
\RU{вызывает функцию \printf}\EN{instruction calls the \printf function}. 
\RU{Работа этой инструкции состоит из двух фаз}%
\EN{Here's how this instruction works}: 
\begin{itemize}
\item
\RU{записать адрес после инструкции \TT{BL} (\TT{0xC}) в регистр \ac{LR}}%
\EN{store the address following the \TT{BL} instruction (\TT{0xC}) into the \ac{LR}};
\item
\RU{передать управление в \printf, записав адрес этой функции в регистр \ac{PC}}%
\EN{then pass the control to \printf by writing its address into the \ac{PC} register}.
\end{itemize}

\RU{Ведь когда функция \printf закончит работу, нужно знать, куда вернуть управление, поэтому закончив работу, всякая функция передает управление по адресу, записанному в регистре \ac{LR}}%
\EN{When \printf finishes its execution it must have information about where it needs to return the control to.
That's why each function passes control to the address stored in the \ac{LR} register}.

\RU{В этом разница между \q{чистыми} \ac{RISC}-процессорами вроде ARM и \ac{CISC}-процессорами как x86,
где адрес возврата обычно записывается в стек}%
\EN{That is a difference between \q{pure} \ac{RISC}-processors like ARM and \ac{CISC}-processors like x86,
where the return address is usually stored on the stack}\footnote{\RU{Подробнее об этом будет описано в следующей главе}\EN{Read more about this in next section}~(\myref{sec:stack})}.

\RU{Кстати, 32-битный абсолютный адрес (либо смещение) невозможно закодировать в 32-битной инструкции \TT{BL}, в ней есть место только для 24-х бит}%
\EN{By the way, an absolute 32-bit address or offset cannot be encoded in the 32-bit \TT{BL} instruction because
it only has space for 24 bits}.
\RU{Поскольку все инструкции в режиме ARM имеют длину 4 байта (32 бита) и инструкции могут находится только по адресам кратным 4, то последние 2 бита (всегда нулевых) можно не кодировать.}
\EN{As we may remember, all ARM-mode instructions have a size of 4 bytes (32 bits).
Hence, they can only be located on 4-byte boundary addresses.
This implies that the last 2 bits of the instruction address (which are always zero bits) may be omitted.}
\RU{В итоге имеем 26 бит, при помощи которых можно закодировать}
\EN{In summary, we have 26 bits for offset encoding. This is enough to encode} $current\_PC \pm{} \approx{}32M$.

\index{ARM!\Instructions!MOV}
\RU{Следующая инструкция}\EN{Next, the} \TT{MOV R0, \#0}\footnote{MOVe}
\RU{просто записывает 0 в регистр \Reg{0}}\EN{instruction just writes 0 into the \Reg{0} register}.
\RU{Ведь наша Си-функция возвращает 0, а возвращаемое значение всякая функция оставляет в \Reg{0}}%
\EN{That's because our C-function returns 0 and the return value is to be placed in the \Reg{0} register}.

\index{ARM!\Registers!Link Register}
\index{ARM!\Instructions!LDMFD}
\index{ARM!\Instructions!POP}
\RU{Последняя инструкция}\EN{The last instruction} \TT{LDMFD SP!, {R4,PC}}\footnote{\ac{LDMFD}}\RU{~--- это инструкция, обратная}\EN{ is an inverse instruction of} \TT{STMFD}. 
\RU{Она загружает из стека (или любого другого места в памяти) значения для сохранения их в \Reg{4} и \ac{PC}, увеличивая \glslink{stack pointer}{указатель стека} \ac{SP}.}
\EN{It loads values from the stack (or any other memory place) in order to save them into \Reg{4} and \ac{PC}, and \glslink{increment}{increments} the \gls{stack pointer} \ac{SP}.}
\RU{Здесь она работает как аналог \POP}\EN{It works like \POP here}.\\
N.B. \RU{Самая первая инструкция \TT{STMFD} сохранила в стеке \Reg{4} и \ac{LR}, а \IT{восстанавливаются} во время исполнения \TT{LDMFD} регистры \Reg{4} и \ac{PC}}%
\EN{The very first instruction \TT{STMFD} saved the \Reg{4} and \ac{LR} registers pair on the stack, but \Reg{4} and \ac{PC} are \IT{restored} during the \TT{LDMFD} execution}.

\RU{Как мы уже знаем, в регистре \ac{LR} обычно сохраняется адрес места, куда нужно всякой функции вернуть управление}%
\EN{As we already know, the address of the place where each function must return control to is usually saved in the \ac{LR} register}.
\RU{Самая первая инструкция сохраняет это значение в стеке, потому что наша функция \main позже будет сама пользоваться этим регистром в момент вызова \printf}%
\EN{The very first instruction saves its value in the stack because the same register will be used by our
\main function when calling \printf}.
\RU{А затем, в конце функции, это значение можно сразу записать прямо в \ac{PC}, таким образом, передав управление туда, откуда была вызвана наша функция}%
\EN{In the function's end, this value can be written directly to the \ac{PC} register, thus passing control to where our function was called}.

\RU{Так как функция \main обычно самая главная в \CCpp, управление будет возвращено в загрузчик \ac{OS}, либо куда-то в \ac{CRT} 
или что-то в этом роде.}
\EN{Since \main is usually the primary function in \CCpp,
the control will be returned to the \ac{OS} loader or to a point in a \ac{CRT},
or something like that.}

\EN{All that allows omitting the \TT{BX LR} instruction at the end of the function.}
\RU{Всё это позволяет избавиться от инструкции \TT{BX LR} в самом конце функции.}

\index{ARM!DCB}
\TT{DCB}\RU{~--- директива ассемблера, описывающая массивы байт или ASCII-строк, аналог директивы DB в 
x86-ассемблере}%
\EN{~is an assembly language directive defining an array of bytes or ASCII strings, akin to the DB directive 
in the x86-assembly language}.


\subsection{\NonOptimizingKeilVI (\ThumbMode)}

\RU{Скомпилируем тот же пример в Keil для режима Thumb}\EN{Let's compile the same example using Keil in Thumb mode}:

\begin{lstlisting}
armcc.exe --thumb --c90 -O0 1.c 
\end{lstlisting}

\RU{Получим (в \IDA)}\EN{We are getting (in \IDA)}:

\begin{lstlisting}[caption=\NonOptimizingKeilVI (\ThumbMode) + \IDA]
.text:00000000             main
.text:00000000 10 B5          PUSH    {R4,LR}
.text:00000002 C0 A0          ADR     R0, aHelloWorld ; "hello, world"
.text:00000004 06 F0 2E F9    BL      __2printf
.text:00000008 00 20          MOVS    R0, #0
.text:0000000A 10 BD          POP     {R4,PC}

.text:00000304 68 65 6C 6C+aHelloWorld  DCB "hello, world",0    ; DATA XREF: main+2
\end{lstlisting}

\RU{Сразу бросаются в глаза двухбайтные (16-битные) опкоды\EMDASH{}это, как уже было отмечено, Thumb.}%
\EN{We can easily spot the 2-byte (16-bit) opcodes. This is, as was already noted, Thumb.}
\index{ARM!\Instructions!BL}
\RU{Кроме инструкции \TT{BL}.}\EN{The \TT{BL} instruction, however, }
\RU{Но на самом деле она состоит из двух 16-битных инструкций}%
\EN{consists of two 16-bit instructions}.
\RU{Это потому что в одном 16-битном опкоде слишком мало места для задания смещения, по которому находится функция \printf}%
\EN{This is because it is impossible to load an offset for the \printf function while using the small space in one 16-bit opcode}.
\RU{Так что первая 16-битная инструкция загружает старшие 10 бит смещения, а вторая~--- младшие 11 бит смещения}%
\EN{Therefore, the first 16-bit instruction loads the higher 10 bits of the offset and the second instruction loads 
the lower 11 bits of the offset}.
% TODO:
% BL has space for 11 bits, so if we don't encode the lowest bit,
% then we should get 11 bits for the upper half, and 12 bits for the lower half.
% And the highest bit encodes the sign, so the destination has to be within
% \pm 4M of current_PC.
% This may be less if adding the lower half does not carry over,
% but I'm not sure --all my programs have 0 for the upper half,
% and don't carry over for the lower half.
% It would be interesting to check where __2printf is located relative to 0x8
% (I think the program counter is the next instruction on a multiple of 4
% for THUMB).
% The lower 11 bytes of the BL instructions and the even bit are
% 000 0000 0110 | 001 0010 1110 0 = 000 0000 0110 0010 0101 1100 = 0x00625c,
% so __2printf should be at 0x006264.
% But if we only have 10 and 11 bits, then the offset would be:
% 00 0000 0110 | 01 0010 1110 0 = 0 0000 0011 0010 0101 1100 = 0x00325c,
% so __2printf should be at 0x003264.
% In this case, though, the new program counter can only be 1M away,
% because of the highest bit is used for the sign.
\RU{Как уже было упомянуто, все инструкции в Thumb-режиме имеют длину 2 байта (или 16 бит)}%
\EN{As was noted, all instructions in Thumb mode have a size of 2 bytes (or 16 bits)}.
\RU{Поэтому невозможна такая ситуация, когда Thumb-инструкция начинается по нечетному адресу.}
\EN{This implies it is impossible for a Thumb-instruction to be at an odd address whatsoever.}
\RU{Учитывая сказанное, последний бит адреса можно не кодировать}%
\EN{Given the above, the last address bit may be omitted while encoding instructions}.
\RU{Таким образом, в Thumb-инструкции \TT{BL} можно закодировать адрес}
\EN{In summary, the \TT{BL} Thumb-instruction can encode an address in} $current\_PC \pm{}\approx{}2M$.

\index{ARM!\Instructions!PUSH}
\index{ARM!\Instructions!POP}
\RU{Остальные инструкции в функции (\PUSH и \POP) здесь работают почти так же, как и описанные \TT{STMFD}/\TT{LDMFD}, только регистр \ac{SP} здесь не указывается явно}%
\EN{As for the other instructions in the function: \PUSH and \POP work here just like the described \TT{STMFD}/\TT{LDMFD} only the \ac{SP} register is not mentioned explicitly here}.
\TT{ADR} \RU{работает так же, как и в предыдущем примере}\EN{works just like in the previous example}.
\TT{MOVS} \RU{записывает 0 в регистр \Reg{0} для возврата нуля}%
\EN{writes 0 into the \Reg{0} register in order to return zero}.

\subsection{\OptimizingXcodeIV (\ARMMode)}

Xcode 4.6.3 \RU{без включенной оптимизации выдает слишком много лишнего кода, поэтому включим оптимизацию компилятора (ключ \Othree), потому что там меньше инструкций.}
\EN{without optimization turned on produces a lot of redundant code so we'll study optimized output, where the instruction count is as small as possible, setting the compiler switch \Othree.}

\begin{lstlisting}[caption=\OptimizingXcodeIV (\ARMMode)]
__text:000028C4             _hello_world
__text:000028C4 80 40 2D E9   STMFD           SP!, {R7,LR}
__text:000028C8 86 06 01 E3   MOV             R0, #0x1686
__text:000028CC 0D 70 A0 E1   MOV             R7, SP
__text:000028D0 00 00 40 E3   MOVT            R0, #0
__text:000028D4 00 00 8F E0   ADD             R0, PC, R0
__text:000028D8 C3 05 00 EB   BL              _puts
__text:000028DC 00 00 A0 E3   MOV             R0, #0
__text:000028E0 80 80 BD E8   LDMFD           SP!, {R7,PC}

__cstring:00003F62 48 65 6C 6C+aHelloWorld_0  DCB "Hello world!",0
\end{lstlisting}

\RU{Инструкции}\EN{The instructions} \TT{STMFD} \AndENRU \TT{LDMFD} \RU{нам уже знакомы}\EN{are already familiar to us}.

\index{ARM!\Instructions!MOV}
\RU{Инструкция \MOV просто записывает число \TT{0x1686} в регистр \Reg{0}~--- это смещение, указывающее на строку \q{Hello world!}}%
\EN{The \MOV instruction just writes the number \TT{0x1686} into the \Reg{0} register.
This is the offset pointing to the \q{Hello world!} string}.

\RU{Регистр \TT{R7} (по стандарту, принятому в \cite{IOSABI}) это frame pointer, о нем будет рассказано позже.}
\EN{The \TT{R7} register (as it is standardized in \cite{IOSABI}) is a frame pointer. More on that below.}

\index{ARM!\Instructions!MOVT}
\RU{Инструкция}\EN{The} \TT{MOVT R0, \#0} (MOVe Top) \RU{записывает 0 в старшие 16 бит регистра}%
\EN{instruction writes 0 into higher 16 bits of the register}.
\RU{Дело в том, что обычная инструкция \MOV в режиме ARM может записывать какое-либо значение только в младшие 16 бит регистра, ведь в ней нельзя закодировать больше}%
\EN{The issue here is that the generic \MOV instruction in ARM mode may write only the lower 16 bits of the register}.
\RU{Помните, что в режиме ARM опкоды всех инструкций ограничены длиной в 32 бита. Конечно, это ограничение не касается перемещений данных между регистрами.}
\EN{Remember, all instruction opcodes in ARM mode are limited in size to 32 bits. Of course, this limitation is not related to moving data between registers.}
\RU{Поэтому для записи в старшие биты (с 16-го по 31-й включительно) существует дополнительная команда \TT{MOVT}}%
\EN{That's why an additional instruction \TT{MOVT} exists for writing into the higher bits (from 16 to 31 inclusive)}.
\RU{Впрочем, здесь её использование избыточно, потому что инструкция \TT{MOV R0, \#0x1686} выше и так обнулила старшую часть регистра}%
\EN{Its usage here, however, is redundant because the \TT{MOV R0, \#0x1686} instruction above cleared the higher part of the register}.
\RU{Возможно, это недочет компилятора}\EN{This is probably a shortcoming of the compiler}.
% TODO:
% I think, more specifically, the string is not put in the text section,
% ie. the compiler is actually not using position-independent code,
% as mentioned in the next paragraph.
% MOVT is used because the assembly code is generated before the relocation,
% so the location of the string is not yet known,
% and the high bits may still be needed.

\index{ARM!\Instructions!ADD}
\RU{Инструкция}\EN{The} \TT{ADD R0, PC, R0} \RU{прибавляет \ac{PC} к \Reg{0} для вычисления действительного адреса строки \q{Hello world!}. Как нам уже известно, это \q{\PICcode}, поэтому такая корректива необходима}%
\EN{instruction adds the value in the \ac{PC} to the value in the \Reg{0}, to calculate the absolute address of the \q{Hello world!} string. 
As we already know, it is \q{\PICcode} so this correction is essential here}.

\RU{Инструкция \TT{BL} вызывает \puts вместо \printf}%
\EN{The \TT{BL} instruction calls the \puts function instead of \printf}.

\label{puts}
\index{\CStandardLibrary!puts()}
\index{puts() \RU{вместо}\EN{instead of} printf()}
\RU{Компилятор заменил вызов \printf на \puts. 
Действительно, \printf с одним аргументом это почти аналог \puts.}
\EN{GCC replaced the first \printf call with \puts.
Indeed: \printf with a sole argument is almost analogous to \puts.} 
\RU{\IT{Почти}, если принять условие, что в строке не будет управляющих символов \printf, 
начинающихся со знака процента. Тогда эффект от работы этих двух функций будет разным}%
\EN{\IT{Almost}, because the two functions are producing the same result only in case the 
string does not contain printf format identifiers starting with \IT{\%}. 
In case it does, the effect of these two functions would be different}%
\footnote{
\RU{Также нужно заметить, что \puts не требует символа перевода строки `\textbackslash{}n' в конце строки,
поэтому его здесь нет.}
\EN{It has also to be noted the \puts does not require a `\textbackslash{}n' new line symbol 
at the end of a string, so we do not see it here.}}.

\RU{Зачем компилятор заменил один вызов на другой? Наверное потому что \puts работает быстрее}%
\EN{Why did the compiler replace the \printf with \puts? Probably because \puts is faster}%
\footnote{\href{http://go.yurichev.com/17063}{ciselant.de/projects/gcc\_printf/gcc\_printf.html}}. 
\RU{Видимо потому что \puts проталкивает символы в \gls{stdout} не сравнивая каждый со знаком процента.}
\EN{Because it just passes characters to \gls{stdout} without comparing every one of them with the \IT{\%} symbol.}

\RU{Далее уже знакомая инструкция}\EN{Next, we see the familiar} 
\TT{MOV R0, \#0}\RU{, служащая для установки в 0 возвращаемого значения функции}%
\EN{instruction intended to set the \Reg{0} register to 0}.

\subsection{\OptimizingXcodeIV (\ThumbTwoMode)}

\RU{По умолчанию}\EN{By default} Xcode 4.6.3 
\RU{генерирует код для режима Thumb-2 примерно в такой манере}%
\EN{generates code for Thumb-2 in this manner}:

\begin{lstlisting}[caption=\OptimizingXcodeIV (\ThumbTwoMode)]
__text:00002B6C                   _hello_world
__text:00002B6C 80 B5          PUSH            {R7,LR}
__text:00002B6E 41 F2 D8 30    MOVW            R0, #0x13D8
__text:00002B72 6F 46          MOV             R7, SP
__text:00002B74 C0 F2 00 00    MOVT.W          R0, #0
__text:00002B78 78 44          ADD             R0, PC
__text:00002B7A 01 F0 38 EA    BLX             _puts
__text:00002B7E 00 20          MOVS            R0, #0
__text:00002B80 80 BD          POP             {R7,PC}

...

__cstring:00003E70 48 65 6C 6C 6F 20+aHelloWorld  DCB "Hello world!",0xA,0
\end{lstlisting}

% Q: If you subtract 0x13D8 from 0x3E70,
% you actually get a location that is not in this function, or in _puts.
% How is PC-relative addressing done in THUMB2?
% A: it's not Thumb-related. there are just mess with two different segments. TODO: rework this listing.

\index{\ThumbTwoMode}
\index{ARM!\Instructions!BL}
\index{ARM!\Instructions!BLX}
\RU{Инструкции \TT{BL} и \TT{BLX} в Thumb, как мы помним, кодируются как пара 16-битных инструкций, 
а в Thumb-2 эти \IT{суррогатные} опкоды расширены так, что новые инструкции кодируются здесь как 
32-битные инструкции}%
\EN{The \TT{BL} and \TT{BLX} instructions in Thumb mode, as we recall, are encoded as a pair
of 16-bit instructions.
In Thumb-2 these \IT{surrogate} opcodes are extended in such a way so that new instructions
may be encoded here as 32-bit instructions}.
\RU{Это можно заметить по тому что опкоды Thumb-2 инструкций всегда начинаются с \TT{0xFx} либо с \TT{0xEx}}%
\EN{That is obvious considering that the opcodes of the Thumb-2 instructions always begin with \TT{0xFx} or \TT{0xEx}}.
\RU{Но в листинге \IDA байты опкода переставлены местами.
Это из-за того, что в процессоре ARM инструкции кодируются так:
в начале последний байт, потом первый (для Thumb и Thumb-2 режима), либо, 
(для инструкций в режиме ARM) в начале четвертый байт, затем третий, второй и первый 
(т.е. другой \gls{endianness})}%
\EN{But in the \IDA listing
the opcode bytes are swapped because for ARM processor the instructions are encoded as follows: 
last byte comes first and after that comes the first one (for Thumb and Thumb-2 modes) 
or for instructions in ARM mode the fourth byte comes first, then the third,
then the second and finally the first (due to different \gls{endianness})}.

\RU{Вот так байты следуют в листингах IDA:}
\EN{So that is how bytes are located in IDA listings:}
\begin{itemize}
\item \RU{для режимов ARM и ARM64}\EN{for ARM and ARM64 modes}: 4-3-2-1;
\item \RU{для режима Thumb}\EN{for Thumb mode}: 2-1;
\item \RU{для пары 16-битных инструкций в режиме Thumb-2}\EN{for 16-bit instructions pair in Thumb-2 mode}: 2-1-4-3.
\end{itemize}

\index{ARM!\Instructions!MOVW}
\index{ARM!\Instructions!MOVT.W}
\index{ARM!\Instructions!BLX}
\RU{Так что мы видим здесь что инструкции \TT{MOVW}, \TT{MOVT.W} и \TT{BLX} начинаются с}
\EN{So as we can see, the \TT{MOVW}, \TT{MOVT.W} and \TT{BLX} instructions begin with} \TT{0xFx}.

\RU{Одна из Thumb-2 инструкций это}\EN{One of the Thumb-2 instructions is}
\TT{MOVW R0, \#0x13D8}\RU{~--- она записывает 16-битное число в младшую часть регистра \Reg{0}, очищая старшие биты.}
\EN{~---it stores a 16-bit value into the lower part of the \Reg{0} register, clearing the higher bits.}

\RU{Ещё}\EN{Also,} \TT{MOVT.W R0, \#0}\RU{~--- эта инструкция работает так же, как и}
\EN{~works just like} 
\TT{MOVT} \RU{из предыдущего примера, но она работает в}\EN{from the previous example only it works in} Thumb-2.

\index{ARM!\RU{переключение режимов}\EN{mode switching}}
\index{ARM!\Instructions!BLX}
\RU{Помимо прочих отличий, здесь используется инструкция}
\EN{Among the other differences, the} \TT{BLX} 
\RU{вместо}\EN{instruction is used in this case instead of the} \TT{BL}.
\RU{Отличие в том, что помимо сохранения адреса возврата в регистре \ac{LR} и передаче управления 
в функцию \puts, происходит смена режима процессора с Thumb/Thumb-2 на режим ARM (либо назад).}
\EN{The difference is that, besides saving the \ac{RA} in the \ac{LR} register and passing control 
to the \puts function, the processor is also switching from Thumb/Thumb-2 mode to ARM mode (or back).}
\RU{Здесь это нужно потому, что инструкция, куда ведет переход, выглядит так (она закодирована в режиме ARM)}%
\EN{This instruction is placed here since the instruction to which control is passed looks like (it is encoded in ARM mode)}:

\begin{lstlisting}
__symbolstub1:00003FEC _puts           ; CODE XREF: _hello_world+E
__symbolstub1:00003FEC 44 F0 9F E5     LDR  PC, =__imp__puts
\end{lstlisting}

\EN{This is essentially a jump to the place where the address of \puts is written in the imports' section.}
\RU{Это просто переход на место, где записан адрес \puts в секции импортов.}

\RU{Итак, внимательный читатель может задать справедливый вопрос: почему бы не вызывать \puts сразу в 
том же месте кода, где он нужен?}
\EN{So, the observant reader may ask: why not call \puts right at the point in the code where it is needed?}

\RU{Но это не очень выгодно из-за экономии места и вот почему}%
\EN{Because it is not very space-efficient}.

\index{\RU{Динамически подгружаемые библиотеки}\EN{Dynamically loaded libraries}}
\RU{Практически любая программа использует внешние динамические библиотеки (будь то DLL в Windows, .so в *NIX 
либо .dylib в \MacOSX)}\EN{Almost any program uses external dynamic libraries (like DLL in Windows, .so in *NIX or .dylib in \MacOSX)}.
\RU{В динамических библиотеках находятся часто используемые библиотечные функции, в том числе стандартная функция Си \puts}%
\EN{The dynamic libraries contain frequently used library functions, including the standard C-function \puts}.

\index{Relocation}
\RU{В исполняемом бинарном файле}\EN{In an executable binary file} 
(Windows PE .exe, ELF \RU{либо}\EN{or} Mach-O) \RU{имеется секция импортов, список символов (функций либо глобальных переменных) импортируемых из внешних модулей, а также названия самих модулей}%
\EN{an import section is present.
This is a list of symbols (functions or global variables) imported from external modules along with the names of the modules themselves}.

\RU{Загрузчик \ac{OS} загружает необходимые модули и, перебирая импортируемые символы в основном модуле, проставляет правильные адреса каждого символа}%
\EN{The \ac{OS} loader loads all modules it needs and, while enumerating import symbols in the primary module, determines the correct addresses of each symbol}.

\RU{В нашем случае,}\EN{In our case,} \IT{\_\_imp\_\_puts} 
\RU{это 32-битная переменная, куда загрузчик \ac{OS} запишет правильный адрес этой же функции во внешней библиотеке}%
\EN{is a 32-bit variable used by the \ac{OS} loader to store the correct address of the function in an external library}. 
\RU{Так что инструкция \TT{LDR} просто берет 32-битное значение из этой переменной, и, записывая его в регистр \ac{PC}, просто передает туда управление}%
\EN{Then the \TT{LDR} instruction just reads the 32-bit value from this variable and writes it into the \ac{PC} register, passing control to it}.

\RU{Чтобы уменьшить время работы загрузчика \ac{OS}, 
нужно чтобы ему пришлось записать адрес каждого символа только один раз, 
в соответствующее, выделенное для них, место.}
\EN{So, in order to reduce the time the \ac{OS} loader needs for completing this procedure, 
it is good idea to write the address of each symbol only once, to a dedicated place.}

\index{thunk-\RU{функции}\EN{functions}}
\RU{К тому же, как мы уже убедились, нельзя одной инструкцией загрузить в регистр 32-битное число без обращений к памяти}%
\EN{Besides, as we have already figured out, it is impossible to load a 32-bit value into a register 
while using only one instruction without a memory access}.
\RU{Так что наиболее оптимально выделить отдельную функцию, работающую в режиме ARM, 
чья единственная цель~--- передавать управление дальше, в динамическую библиотеку.}
\EN{Therefore, the optimal solution is to allocate a separate function working in ARM mode with the sole 
goal of passing control to the dynamic library}
\RU{И затем ссылаться на эту короткую функцию из одной инструкции (так называемую \glslink{thunk function}{thunk-функцию}) из Thumb-кода}%
\EN{and then to jump to this short one-instruction function (the so-called \gls{thunk function}) from the Thumb-code}.

\index{ARM!\Instructions!BL}
\RU{Кстати, в предыдущем примере (скомпилированном для режима ARM), переход при помощи инструкции \TT{BL} ведет 
на такую же \glslink{thunk function}{thunk-функцию}, однако режим процессора не переключается (отсюда отсутствие \q{X} в мнемонике инструкции)}%
\EN{By the way, in the previous example (compiled for ARM mode) the control is passed by the \TT{BL} to the 
same \gls{thunk function}.
The processor mode, however, is not being switched (hence the absence of an \q{X} in the instruction mnemonic)}.

\subsubsection{\EN{More about thunk-functions}\RU{Еще о thunk-функциях}}
\index{thunk-\RU{функции}\EN{functions}}

\RU{Thunk-функции трудновато понять, вероятно, из-за путаницы в терминах.}
\EN{Thunk-functions are hard to understand, apparently, because of a misnomer.}

\RU{Проще всего представлять их как адаптеры-переходники из одного типа разъемов в другой.}
\EN{The simplest way to understand it as adaptors or convertors of one type of jack to another.}
\RU{Например, адаптер позволяющее вставить в американскую розетку британскую вилку, или наоборот.}
\EN{For example, an adaptor allowing the insertion of a British power plug into an American wall socket, or vice-versa.} 

\EN{Thunk functions are also sometimes called \IT{wrappers}.}
\RU{Thunk-функции также иногда называются \IT{wrapper-ами}. \IT{Wrap} в английском языке это \IT{обертывать}, \IT{завертывать}.}

\RU{Вот еще несколько описаний этих функций:}
\EN{Here are a couple more descriptions of these functions:}

\begin{framed}
\begin{quotation}
“A piece of coding which provides an address:”, according to P. Z. Ingerman, 
who invented thunks in 1961 as a way of binding actual parameters to their formal 
definitions in Algol-60 procedure calls. If a procedure is called with an expression 
in the place of a formal parameter, the compiler generates a thunk which computes 
the expression and leaves the address of the result in some standard location.

\dots

Microsoft and IBM have both defined, in their Intel-based systems, a “16-bit environment” 
(with bletcherous segment registers and 64K address limits) and a “32-bit environment” 
(with flat addressing and semi-real memory management). The two environments can both be 
running on the same computer and OS (thanks to what is called, in the Microsoft world, 
WOW which stands for Windows On Windows). MS and IBM have both decided that the process 
of getting from 16- to 32-bit and vice versa is called a “thunk”; for Windows 95, 
there is even a tool, THUNK.EXE, called a “thunk compiler”.
\end{quotation}
\end{framed}
% TODO FIXME move to bibliography and quote properly above the quote
( \href{http://go.yurichev.com/17362}{The Jargon File} )

\subsection{ARM64}

\subsubsection{GCC}

\RU{Компилируем пример в}\EN{Let's compile the example using} GCC 4.8.1 \InENRU ARM64:

\lstinputlisting[numbers=left,label=hw_ARM64_GCC,caption=\NonOptimizing GCC 4.8.1 + objdump]
{patterns/01_helloworld/ARM/hw.lst}

\RU{В ARM64 нет режима Thumb и Thumb-2, только ARM, так что тут только 32-битные инструкции.}
\EN{There are no Thumb and Thumb-2 modes in ARM64, only ARM, so there are 32-bit instructions only.}
\RU{Регистров тут в 2 раза больше}\EN{The Register count is doubled}: \myref{ARM64_GPRs}.
\RU{64-битные регистры теперь имеют префикс}\EN{64-bit registers have} 
\TT{X-}\EN{ prefixes, while its 32-bit parts}\RU{, а их 32-битные части}\EMDASH{}\TT{W-}.

\index{ARM!\Instructions!STP}
\EN{The }\RU{Инструкция }\TT{STP}\EN{ instruction} (\IT{Store Pair}) 
\RU{сохраняет в стеке сразу два регистра}\EN{saves two registers in the stack simultaneously}: \RegX{29} \InENRU \RegX{30}.
\RU{Конечно, эта инструкция может сохранять эту пару где угодно в памяти, но здесь указан регистр \ac{SP}, так что
пара сохраняется именно в стеке.}
\EN{Of course, this instruction is able to save this pair at an arbitrary place in memory, 
but the \ac{SP} register is specified here, so the pair is saved in the stack.}
\RU{Регистры в ARM64 64-битные, каждый имеет длину в 8 байт, так что для хранения двух регистров нужно именно 16 байт.}
\EN{ARM64 registers are 64-bit ones, each has a size of 8 bytes, so one needs 16 bytes for saving two registers.}

\RU{Восклицательный знак после операнда означает, что сначала от \ac{SP} будет отнято 16 и только затем
значения из пары регистров будут записаны в стек.}
\EN{The exclamation mark after the operand means that 16 is to be subtracted from \ac{SP} first, and only then
are values from register pair to be written into the stack.}
\RU{Это называется}\EN{This is also called} \IT{pre-index}.
\RU{Больше о разнице между}\EN{About the difference between} \IT{post-index} \AndENRU \IT{pre-index} 
\RU{описано здесь}\EN{read here}: \myref{ARM_postindex_vs_preindex}.

\RU{Таким образом, в терминах более знакомого всем процессора x86, первая инструкция~--- это просто аналог 
пары инструкций}
\EN{Hence, in terms of the more familiar x86, the first instruction is just an analogue to a pair of}
\TT{PUSH X29} \AndENRU \TT{PUSH X30}.
\RegX{29} \EN{is used as \ac{FP} in ARM64}\RU{в ARM64 используется как \ac{FP}}, \EN{and}\RU{а} \RegX{30} 
\EN{as}\RU{как} \ac{LR}, \RU{поэтому они сохраняются в прологе функции и
восстанавливаются в эпилоге}\EN{so that's why they are saved in the function prologue and restored in the function epilogue}.

\EN{The second instruction copies}\RU{Вторая инструкция копирует} \ac{SP} \InENRU \RegX{29} (\OrENRU \ac{FP}).
\RU{Это нужно для установки стекового фрейма функции}\EN{This is done to set up the function stack frame}.

\label{pointers_ADRP_and_ADD}
\index{ARM!\Instructions!ADRP/ADD pair}
\RU{Инструкции }\TT{ADRP} \AndENRU \ADD \EN{instructions are used to fill the 
address of the string}\RU{нужны для формирования адреса строки} \q{Hello!} \EN{into the \RegX{0} register}\RU{в регистре \RegX{0}}, 
\RU{ведь первый аргумент функции передается через этот регистр}\EN{because the first function argument is passed
in this register}.
\RU{Но в ARM нет инструкций, при помощи которых можно записать в регистр длинное число}\EN{There are
no instructions, whatsoever, in ARM that can store a large number into a register} 
(\RU{потому что сама длина инструкции ограничена 4-я байтами. Больше об этом здесь}\EN{because the instruction
length is limited to 4 bytes, read more about it here}: \myref{ARM_big_constants_loading}).
\RU{Так что нужно использовать несколько инструкций}\EN{So several instructions must be utilised}.
\RU{Первая инструкция}\EN{The first instruction} (\TT{ADRP}) \EN{writes the address of the 4KiB page, where the string is
located, into \RegX{0}}\RU{записывает в \RegX{0} адрес 4-килобайтной страницы где находится строка}, 
\EN{and the second one}\RU{а вторая} (\ADD) \RU{просто прибавляет к этому адресу остаток}\EN{just adds
the remainder to the address}.
\EN{More about that in}\RU{Читайте больше об этом}: \myref{ARM64_relocs}.

\TT{0x400000 + 0x648 = 0x400648}, \EN{and we see our \q{Hello!} C-string in the \TT{.rodata} data segment at this
address}\RU{и мы видим, что в секции данных \TT{.rodata} по этому адресу как раз находится наша
Си-строка \q{Hello!}}.

\index{ARM!\Instructions!BL}
\RU{Затем при помощи инструкции \TT{BL} вызывается \puts. Это уже рассматривалось ранее: \myref{puts}.}
\EN{\puts is called afterwards using the \TT{BL} instruction. This was already discussed: \myref{puts}.}

\RU{Инструкция }\MOV \EN{writes 0 into}\RU{записывает 0 в} \RegW{0}. 
\RegW{0} \RU{это младшие 32 бита 64-битного регистра}\EN{is the lower 32 bits of the 64-bit} \RegX{0}\EN{ register}:

\input{ARM_X0_register}

\RU{А результат функции возвращается через \RegX{0}, и \main возвращает 0, 
так что вот так готовится возвращаемый результат.}
\EN{The function result is returned via \RegX{0} and \main returns 0, so that's how the return
result is prepared.}
\RU{Почему именно 32-битная часть}\EN{But why use the 32-bit part}?
\RU{Потому в ARM64, как и в x86-64, тип \Tint оставили 32-битным, для лучшей совместимости.}
\EN{Because the \Tint data type in ARM64, just like in x86-64, is still 32-bit, for better compatibility.}
\RU{Следовательно, раз уж функция возвращает 32-битный \Tint, то нужно заполнить только 32 младших бита 
регистра \RegX{0}.}
\EN{So if a function returns a 32-bit \Tint, only the lower 32 bits of \RegX{0} register have to be filled.}

\RU{Для того, чтобы удостовериться в этом, немного отредактируем этот пример и перекомпилируем его.}%
\EN{In order to verify this, let's change this example slightly and recompile it.}
\RU{Теперь \main возвращает 64-битное значение:}%
\EN{Now \main returns a 64-bit value:}

\begin{lstlisting}[caption=\main \RU{возвращающая значение типа}\EN{returning a value of} \TT{uint64\_t}\EN{ type}]
#include <stdio.h>
#include <stdint.h>

uint64_t main()
{
        printf ("Hello!\n");
        return 0;
}
\end{lstlisting}

\RU{Результат точно такой же, только \MOV в той строке теперь выглядит так:}
\EN{The result is the same, but that's how \MOV at that line looks like now:}

\begin{lstlisting}[caption=\NonOptimizing GCC 4.8.1 + objdump]
  4005a4:       d2800000        mov     x0, #0x0                        // #0
\end{lstlisting}

\index{ARM!\Instructions!LDP}
\RU{Далее при помощи инструкции \TT{LDP} (\IT{Load Pair}) восстанавливаются регистры \RegX{29} и \RegX{30}.}
\EN{\TT{LDP} (\IT{Load Pair}) then restores the \RegX{29} and \RegX{30} registers.}
\RU{Восклицательного знака после инструкции нет. Это означает, что сначала значения достаются из стека,
и только потом \ac{SP} увеличивается на 16.}
\EN{There is no exclamation mark after the instruction: this implies that the value is first loaded from the stack,
and only then is \ac{SP} increased by 16.}
\RU{Это называется}\EN{This is called} \IT{post-index}.

\index{ARM!\Instructions!RET}
\RU{В ARM64 есть новая инструкция}\EN{A new instruction appeared in ARM64}: \RET. 
\RU{Она работает так же как и}\EN{It works just as} \TT{BX LR}, \RU{но там добавлен специальный бит,
подсказывающий процессору, что это именно выход из функции, а не просто переход, чтобы процессор
мог более оптимально исполнять эту инструкцию}\EN{only a special \IT{hint} bit is added, informing the \ac{CPU}
that this is a return from a function, not just another jump instruction, so it can execute it more optimally}.

\RU{Из-за простоты этой функции оптимизирующий GCC генерирует точно такой же код.}
\EN{Due to the simplicity of the function, optimizing GCC generates the very same code.}


\fi
\ifdefined\IncludeMIPS
\section{MIPS}

\subsection{\RU{О \q{глобальном указателе} (\q{global pointer})}\EN{A word about the \q{global pointer}}}
\label{MIPS_GP}

\index{MIPS!\GlobalPointer}
\RU{\q{Глобальный указатель} (\q{global pointer})~--- это важная концепция в MIPS.}
\EN{One important MIPS concept is the \q{global pointer}.}
\RU{Как мы уже возможно знаем, каждая инструкция в MIPS имеет размер 32 бита, поэтому невозможно
закодировать 32-битный адрес внутри одной инструкции. Вместо этого нужно использовать пару инструкций
(как это сделал GCC для загрузки адреса текстовой строки в нашем примере).}
\EN{As we may already know, each MIPS instruction has a size of 32 bits, so it's impossible to embed a 32-bit
address into one instruction: a pair has to be used for this 
(like GCC did in our example for the text string address loading).}

\RU{С другой стороны, используя только одну инструкцию, 
возможно загружать данные по адресам в пределах $register-32768...register+32767$, потому что 16 бит
знакового смещения можно закодировать в одной инструкции).}
\EN{It's possible, however, to load data from the address in the range of $register-32768...register+32767$ using one
single instruction (because 16 bits of signed offset could be encoded in a single instruction).}
\RU{Так мы можем выделить какой-то регистр для этих целей и ещё выделить буфер в 64KiB для самых 
частоиспользуемых данных.}
\EN{So we can allocate some register for this purpose and also allocate a 64KiB area of most used data.}
\RU{Выделенный регистр называется \q{глобальный указатель} (\q{global pointer}) и он указывает на середину
области 64KiB.}
\EN{This allocated register is called a \q{global pointer} and it points to the middle of the 64KiB area.}
\RU{Эта область обычно содержит глобальные переменные и адреса импортированных функций вроде \printf,
потому что разработчики GCC решили, что получение адреса функции должно быть как можно более быстрой операцией,
исполняющейся за одну инструкцию вместо двух.}
\EN{This area usually contains global variables and addresses of imported functions like \printf, 
because the GCC developers decided that getting the address of some function must be as fast as a single instruction
execution instead of two.}
\RU{В ELF-файле эта 64KiB-область находится частично в секции .sbss (\q{small \ac{BSS}}) для неинициализированных
данных и в секции .sdata (\q{small data}) для инициализированных данных.}
\EN{In an ELF file this 64KiB area is located partly in sections .sbss (\q{small \ac{BSS}}) for uninitialized data and 
.sdata (\q{small data}) for initialized data.}

\RU{Это значит что программист может выбирать, к чему нужен как можно более быстрый доступ, и затем расположить
это в секциях .sdata/.sbss.}
\EN{This implies that the programmer may choose what data he/she wants to be accessed fast and place it into 
.sdata/.sbss.}

\RU{Некоторые программисты \q{старой школы} могут вспомнить модель памяти в MS-DOS \myref{8086_memory_model} 
или в менеджерах памяти вроде XMS/EMS, где вся память делилась на блоки по 64KiB.}
\EN{Some old-school programmers may recall the MS-DOS memory model \myref{8086_memory_model} 
or the MS-DOS memory managers like XMS/EMS where all memory was divided in 64KiB blocks.}

\index{PowerPC}
\RU{Эта концепция применяется не только в MIPS. По крайней мере PowerPC также использует эту технику.}
\EN{This concept is not unique to MIPS. At least PowerPC uses this technique as well.}

\subsection{\Optimizing GCC}

\EN{Lets consider the following example, which illustrates the \q{global pointer} concept.}
\RU{Рассмотрим следующий пример, иллюстрирующий концепцию \q{глобального указателя}.}

\lstinputlisting[caption=\Optimizing GCC 4.4.5 (\assemblyOutput),numbers=left]{patterns/01_helloworld/MIPS/hw_O3.s.\LANG}

\RU{Как видно, регистр \$GP в прологе функции выставляется в середину этой области.}
\EN{As we see, the \$GP register is set in the function prologue to point to the middle of this area.}
\RU{Регистр \ac{RA} сохраняется в локальном стеке.}
\EN{The \ac{RA} register is also saved in the local stack.}
\RU{Здесь также используется \puts вместо \printf.}
\EN{\puts is also used here instead of \printf.}
\index{MIPS!\Instructions!LW}
\RU{Адрес функции \puts загружается в \$25 инструкцией LW (\q{Load Word}).}
\EN{The address of the \puts function is loaded into \$25 using LW the instruction (\q{Load Word}).}
\index{MIPS!\Instructions!LUI}
\index{MIPS!\Instructions!ADDIU}
\RU{Затем адрес текстовой строки загружается в \$4 парой инструкций LUI (\q{Load Upper Immediate}) и
ADDIU (\q{Add Immediate Unsigned Word}).}
\EN{Then the address of the text string is loaded to \$4 using LUI (\q{Load Upper Immediate}) and 
ADDIU (\q{Add Immediate Unsigned Word}) instruction pair.}
\RU{LUI устанавливает старшие 16 бит регистра (поэтому в имени инструкции присутствует \q{upper}) и ADDIU
прибавляет младшие 16 бит к адресу.}
\EN{LUI sets the high 16 bits of the register (hence \q{upper} word in instruction name) and ADDIU adds
the lower 16 bits of the address.}
\RU{ADDIU следует за JALR (помните о \IT{branch delay slots}?).}
\EN{ADDIU follows JALR (remember \IT{branch delay slots}?).}
\RU{Регистр \$4 также называется \$A0, который используется для передачи первого аргумента функции}%
\EN{The register \$4 is also called \$A0, which is used for passing the first function argument}%
\footnote{\RU{Таблица регистров в MIPS доступна в приложении}\EN{The MIPS registers table %
is available in appendix} \myref{MIPS_registers_ref}}.

\index{MIPS!\Instructions!JALR}
\RU{JALR (\q{Jump and Link Register}) делает переход по адресу в регистре \$25 (там адрес \puts) 
при этом сохраняя адрес следующей инструкции (LW) в \ac{RA}.}
\EN{JALR (\q{Jump and Link Register}) jumps to the address stored in the \$25 register (address of \puts) 
while saving the address of the next instruction (LW) in \ac{RA}.}
\RU{Это так же как и в ARM}\EN{This is very similar to ARM}.
\RU{И ещё одна важная вещь: адрес сохраняемый в \ac{RA} это адрес не следующей инструкции (потому что
это \IT{delay slot} и исполняется перед инструкцией перехода),
а инструкции после неё (после \IT{delay slot}).}
\EN{Oh, and one important thing is that the address saved in \ac{RA} is not the address of the next instruction (because
it's in a \IT{delay slot} and is executed before the jump instruction),
but the address of the instruction after the next one (after the \IT{delay slot}).}
\RU{Таким образом во время исполнения \TT{JALR} в \ac{RA} записывается $PC + 8$. В нашем случае это адрес
инструкции LW следующей после ADDIU.}
\EN{Hence, $PC + 8$ is written to \ac{RA} during the execution of \TT{JALR}, in our case, this is the address of the LW
instruction next to ADDIU.}

\RU{LW (\q{Load Word}) в строке 20 восстанавливает \ac{RA} из локального стека 
(эта инструкция скорее часть эпилога функции).}
\EN{LW (\q{Load Word}) at line 20 restores \ac{RA} from the local stack 
(this instruction is actually part of the function epilogue).}

\index{MIPS!\Pseudoinstructions!MOVE}
\RU{MOVE в строке 22 копирует значение из регистра \$0 (\$ZERO) в \$2 (\$V0).}
\EN{MOVE at line 22 copies the value from the \$0 (\$ZERO) register to \$2 (\$V0).}
\label{MIPS_zero_register}
\RU{В MIPS есть \IT{константный} регистр, всегда содержащий ноль.}
\EN{MIPS has a \IT{constant} register, which always holds zero.}
\RU{Должно быть, разработчики MIPS решили что 0 это самая востребованная константа в программировании,
так что пусть будет использоваться регистр \$0, всякий раз, когда будет нужен 0.}
\EN{Apparently, the MIPS developers came up with the idea that zero is in fact the busiest constant in the computer programming,
so let's just use the \$0 register every time zero is needed.}
\RU{Другой интересный факт: в MIPS нет инструкции, копирующей значения из регистра в регистр.}
\EN{Another interesting fact is that MIPS lacks an instruction that transfers data between registers.}
\RU{На самом деле}\EN{In fact}, \TT{MOVE DST, SRC} \RU{это}\EN{is} \TT{ADD DST, SRC, \$ZERO} ($DST=SRC+0$), 
\RU{которая делает тоже самое}\EN{which does the same}.
\RU{Очевидно, разработчики MIPS хотели сделать как можно более компактную таблицу опкодов.}
\EN{Apparently, the MIPS developers wanted to have a compact opcode table.}
\RU{Это не значит, что сложение происходит во время каждой инструкции MOVE.}
\EN{This does not mean an actual addition happens at each MOVE instruction.}
\RU{Скорее всего, эти псевдоинструкции оптимизируются в \ac{CPU} и \ac{ALU} никогда не используется.}
\EN{Most likely, the \ac{CPU} optimizes these pseudoinstructions and the \ac{ALU} is never used.}

\index{MIPS!\Instructions!J}
\RU{J в строке 24 делает переход по адресу в \ac{RA}, и это работает как выход из функции.}
\EN{J at line 24 jumps to the address in \ac{RA}, which is effectively performing a return from the function.}
\RU{ADDIU после J на самом деле исполняется перед J (помните о \IT{branch delay slots}?) 
и это часть эпилога функции.}
\EN{ADDIU after J is in fact executed before J (remember \IT{branch delay slots}?) 
and is part of the function epilogue.}

\RU{Вот листинг сгенерированный \IDA. Каждый регистр имеет свой псевдоним:}
\EN{Here is also a listing generated by \IDA. Each register here has its own pseudoname:}

\lstinputlisting[caption=\Optimizing GCC 4.4.5 (\IDA),numbers=left]{patterns/01_helloworld/MIPS/hw_O3_IDA.lst.\LANG}

\RU{Инструкция в строке 15 сохраняет GP в локальном стеке. Эта инструкция мистическим образом отсутствует
в листинге от GCC, может быть из-за ошибки в самом GCC\footnote{Очевидно, функция вывода листингов не так критична
для пользователей GCC, поэтому там вполне могут быть неисправленные ошибки.}.}
\EN{The instruction at line 15 saves the GP value into the local stack, and this instruction is missing mysteriously from the GCC output listing, maybe by a GCC error\footnote{Apparently, functions generating listings 
are not so critical to GCC users, so some unfixed errors may still exist.}.}
\RU{Значение GP должно быть сохранено, потому что всякая функция может работать со своим собственным окном данных
размером 64KiB.}
\EN{The GP value has to be saved indeed, because each function can use its own 64KiB data window.}

\RU{Регистр, содержащий адрес функции \puts называется \$T9, потому что регистры с префиксом T- называются
\q{temporaries} и их содержимое можно не сохранять.}
\EN{The register containing the \puts address is called \$T9, because registers prefixed with T- are called
\q{temporaries} and their contents may not be preserved.}

\subsection{\NonOptimizing GCC}

\NonOptimizing GCC \RU{более многословный}\EN{is more verbose}.

\lstinputlisting[caption=\NonOptimizing GCC 4.4.5 (\assemblyOutput),numbers=left]{patterns/01_helloworld/MIPS/hw_O0.s.\LANG}

\RU{Мы видим, что регистр FP используется как указатель на фрейм стека.}
\EN{We see here that register FP is used as a pointer to the stack frame.}
\RU{Мы также видим 3 \ac{NOP}-а.}\EN{We also see 3 \ac{NOP}s.}
\RU{Второй и третий следуют за инструкциями перехода.}
\EN{The second and third of which follow the branch instructions.}

\RU{Вероятно, компилятор GCC всегда добавляет \ac{NOP}-ы (из-за \IT{branch delay slots})
после инструкций переходов и затем, если включена оптимизация, от них может избавляться.}%
\EN{Perhaps, the GCC compiler always adds \ac{NOP}s (because of \IT{branch delay slots}) after branch
instructions and then, if optimization is turned on, maybe eliminates them.}
\RU{Так что они остались здесь}\EN{So in this case they are left here}.

\RU{Вот также листинг от \IDA:}
\EN{Here is also \IDA listing:}

\lstinputlisting[caption=\NonOptimizing GCC 4.4.5 (\IDA),numbers=left]{patterns/01_helloworld/MIPS/hw_O0_IDA.lst.\LANG}

\index{MIPS!\Pseudoinstructions!LA}
\RU{Интересно что \IDA распознала пару инструкций LUI/ADDIU и собрала их в одну псевдоинструкцию 
LA (\q{Load Address}) в строке 15.}
\EN{Interestingly, \IDA recognized the LUI/ADDIU instructions pair and coalesced them into one 
LA (\q{Load Address}) pseudoinstruction at line 15.}
\RU{Мы также видим, что размер этой псевдоинструкции 8 байт!}
\EN{We may also see that this pseudoinstruction has a size of 8 bytes!}
\RU{Это псевдоинструкция (или \IT{макрос}), потому что это не настоящая инструкция MIPS, а скорее
просто удобное имя для пары инструкций.}
\EN{This is a pseudoinstruction (or \IT{macro}) because it's not a real MIPS instruction, but rather
a handy name for an instruction pair.}

\index{MIPS!\Pseudoinstructions!NOP}
\index{MIPS!\Instructions!OR}
\RU{Ещё кое что: \IDA не распознала \ac{NOP}-инструкции в строках 22, 26 и 41.}
\EN{Another thing is that \IDA doesn't recognize \ac{NOP} instructions, so here they are at lines 22, 26 and 41.}
\RU{Это}\EN{It is} \TT{OR \$AT, \$ZERO}.
\RU{По своей сути это инструкция, применяющая операцию ИЛИ к содержимому регистра \$AT с нулем, что,
конечно же, холостая операция.}
\EN{Essentially, this instruction applies the OR operation to the contents of the \$AT register
with zero, which is, of course, an idle instruction.}
\RU{MIPS, как и многие другие \ac{ISA}, не имеет отдельной \ac{NOP}-инструкции.}
\EN{MIPS, like many other \ac{ISA}s, doesn't have a separate \ac{NOP} instruction.}

\subsection{\RU{Роль стекового фрейма в этом примере}\EN{Role of the stack frame in this example}}

\RU{Адрес текстовой строки передается в регистре.}
\EN{The address of the text string is passed in the register.}
\RU{Так зачем устанавливать локальный стек?}\EN{Why setup a local stack anyway?}
\RU{Причина в том, что значения регистров \ac{RA} и GP должны быть сохранены где-то
(потому что вызывается \printf) и для этого используется локальный стек.}
\EN{The reason for this lies in the fact that the values of registers \ac{RA} and GP have to be saved somewhere 
(because \printf is called), and the local stack is used for this purpose.}
\RU{Если бы это была \gls{leaf function}, тогда можно было бы избавиться от пролога и эпилога функции. Например:}
\EN{If this was a \gls{leaf function}, it would have been possible to get rid of the function prologue and epilogue,
for example:} \myref{MIPS_leaf_function_ex1}.

\subsection{\Optimizing GCC: \RU{загрузим в}\EN{load it into} GDB}

\index{GDB}
\lstinputlisting[caption=\RU{пример сессии в GDB}\EN{sample GDB session}]{patterns/01_helloworld/MIPS/O3_GDB.txt}

\fi

\section{\Conclusion{}}

\RU{Основная разница между кодом x86/ARM и x64/ARM64 в том, что указатель на строку теперь 64-битный.}
\EN{The main difference between x86/ARM and x64/ARM64 code is that the pointer to the string is now 64-bits in length.}
\RU{Действительно, ведь для того современные \ac{CPU} и стали 64-битными, потому что подешевела память,
её теперь можно поставить в компьютер намного больше, и чтобы её адресовать, 32-х бит уже
недостаточно.}
\EN{Indeed, modern \ac{CPU}s are now 64-bit due to both the reduced cost of memory and the greater demand for it by modern applications. 
We can add much more memory to our computers than 32-bit pointers are able to address.}
\RU{Поэтому все указатели теперь 64-битные.}\EN{As such, all pointers are now 64-bit.}

% sections
\ifdefined\IncludeExercises
\section{\Exercises}

\begin{itemize}
	\item \url{http://challenges.re/48}
	\item \url{http://challenges.re/49}
\end{itemize}


\fi

\chapter{\RU{Пролог и эпилог функций}\EN{Function prologue and epilogue}}
\label{sec:prologepilog}
\index{Function epilogue}
\index{Function prologue}

\RU{Пролог функции это инструкции в самом начале функции. Как правило это что-то вроде такого
фрагмента кода:}
\EN{A function prologue is a sequence of instructions at the start of a function. It often looks something like the following
code fragment:}

\begin{lstlisting}
    push    ebp
    mov     ebp, esp
    sub     esp, X
\end{lstlisting}

\RU{Эти инструкции делают следующее: сохраняют значение регистра \EBP на будущее, выставляют \EBP равным \ESP, 
затем подготавливают место в стеке для хранения локальных переменных.}
\EN{What these instruction do: save the value in the \EBP register,
set the value of the \EBP register to the value of the \ESP and then allocate space on the stack 
for local variables.}

\RU{\EBP сохраняет свое значение на протяжении всей функции, он будет использоваться здесь для доступа 
к локальным переменным и аргументам. Можно было бы использовать и \ESP, но он постоянно меняется и 
это не очень удобно.}
\EN{The value in the \EBP stays the same over the period of the function execution and is to be used for local variables and 
arguments access. 
For the same purpose one can use \ESP, but since it changes over time this approach is not too convenient.}

\RU{Эпилог функции аннулирует выделенное место в стеке, восстанавливает значение \EBP на старое и возвращает 
управление в вызывающую функцию:}
\EN{The function epilogue frees the allocated space in the stack, returns the value in the \EBP register back to its initial state 
and returns the control flow to the \gls{callee}:}

\begin{lstlisting}
    mov    esp, ebp
    pop    ebp
    ret    0
\end{lstlisting}

% what about calling convention?
\RU{Пролог и эпилог функции обычно находятся в дизассемблерах для отделения функций друг от друга.}
\EN{Function prologues and epilogues are usually detected in disassemblers for function delimitation.}

\section{\Recursion}

\index{\Recursion}
\RU{Наличие эпилога и пролога может несколько ухудшить эффективность рекурсии.}
\EN{Epilogues and prologues can negatively affect the recursion performance.}

\EN{More about recursion in this book}\RU{Больше о рекурсии в этой книге}: 
\myref{Recursion_and_tail_call}.

\chapter{\Stack}
\label{sec:stack}
\index{\Stack}

\RU{Стек в информатике~--- это одна из наиболее фундаментальных структур данных}%
\EN{The stack is one of the most fundamental data structures in computer science}%
\footnote{\href{http://go.yurichev.com/17119}{wikipedia.org/wiki/Call\_stack}}.

\RU{Технически это просто блок памяти в памяти процесса + регистр \ESP в x86 или \RSP в x64, либо \ac{SP} в ARM, который указывает где-то в пределах этого блока.}
\EN{Technically, it is just a block of memory in process memory along with the \ESP or \RSP register in x86 or x64, or the \ac{SP} register in ARM, as a pointer within that block.}

\index{ARM!\Instructions!PUSH}
\index{ARM!\Instructions!POP}
\index{x86!\Instructions!PUSH}
\index{x86!\Instructions!POP}
\RU{Часто используемые инструкции для работы со стеком~--- это \PUSH и \POP (в x86 и Thumb-режиме ARM). 
\PUSH уменьшает \ESP/\RSP/\ac{SP} на 4 в 32-битном режиме (или на 8 в 64-битном),
затем записывает по адресу, на который указывает \ESP/\RSP/\ac{SP}, содержимое своего единственного операнда.}
\EN{The most frequently used stack access instructions are \PUSH and \POP (in both x86 and ARM Thumb-mode). 
\PUSH subtracts from \ESP/\RSP/\ac{SP} 4 in 32-bit mode (or 8 in 64-bit mode) and then writes the contents of its sole operand to the memory address pointed by \ESP/\RSP/\ac{SP}.} 

\RU{\POP это обратная операция~--- сначала достает из \glslink{stack pointer}{указателя стека} значение и помещает его в операнд 
(который очень часто является регистром) и затем увеличивает указатель стека на 4 (или 8).}
\EN{\POP is the reverse operation: retrieve the data from the memory location that \ac{SP} points to, 
load it into the instruction operand (often a register) and then add 4 (or 8) to the \gls{stack pointer}.}

\RU{В самом начале \glslink{stack pointer}{регистр-указатель} указывает на конец стека.}
\EN{After stack allocation, the \gls{stack pointer} points at the bottom of the stack.}
\RU{\PUSH уменьшает \glslink{stack pointer}{регистр-указатель}, а \POP~--- увеличивает.}
\EN{\PUSH decreases the \gls{stack pointer} and \POP increases it.}
\RU{Конец стека находится в начале блока памяти, выделенного под стек. Это странно, но это так.}
\EN{The bottom of the stack is actually at the beginning of the memory allocated for the stack block. 
It seems strange, but that's the way it is.}

\ifdefined\IncludeARM
\RU{В процессоре ARM, тем не менее, есть поддержка стеков, растущих как в сторону уменьшения, так и в
сторону увеличения.}
\EN{ARM supports both descending and ascending stacks.}
\index{ARM!\Instructions!STMFD}
\index{ARM!\Instructions!LDMFD}
\index{ARM!\Instructions!STMED}
\index{ARM!\Instructions!LDMED}
\index{ARM!\Instructions!STMFA}
\index{ARM!\Instructions!LDMFA}
\index{ARM!\Instructions!STMEA}
\index{ARM!\Instructions!LDMEA}

\RU{Например, инструкции}\EN{For example the} 
\ac{STMFD}/\ac{LDMFD}, \ac{STMED}/\ac{LDMED} 
\RU{предназначены для descending-стека 
(растет назад, начиная с высоких адресов в сторону низких).}
\EN{instructions are intended to deal with a descending stack 
(grows downwards, starting with a high address and progressing to a lower one).}
\RU{Инструкции}\EN{The}
\ac{STMFA}/\ac{LDMFA}, \ac{STMEA}/\ac{LDMEA} 
\RU{предназначены для ascending-стека 
(растет вперед, начиная с низких адресов в сторону высоких).}
\EN{instructions are intended to deal with an ascending stack 
(grows upwards, starting from a low address and progressing to a higher one).}
\fi

% It might be worth mentioning that STMED and STMEA write first,
% and then move the pointer,
% and that LDMED and LDMEA move the pointer first, and then read.
% In other words, ARM not only lets the stack grow in a non-standard direction,
% but also in a non-standard order.
% Maybe this can be in the glossary, which would explain why E stands for "empty".

\section{\RU{Почему стек растет в обратную сторону?}\EN{Why does the stack grow backwards?}}

\RU{Интуитивно мы можем подумать, что, как и любая другая структура данных, стек мог бы расти вперед, 
т.е. в сторону увеличения адресов}\EN{Intuitively, we might think that the stack grows upwards, i.e. towards
higher addresses, like any other data structure}.

\RU{Причина, почему стек растет назад, вероятно, историческая}%
\EN{The reason that the stack grows backward is probably historical}.
\RU{Когда компьютеры были большие и занимали целую комнату, было очень легко разделить сегмент на две части:
для \glslink{heap}{кучи} и для стека}\EN{When the computers were big and occupied a whole room, 
it was easy to divide memory into two parts, one for the \gls{heap} and one for the stack}.
\RU{Заранее было неизвестно, насколько большой может быть \glslink{heap}{куча} или стек, 
так что это решение было самым простым}\EN{Of course, 
it was unknown how big the \gls{heap} and the stack would be during program execution, 
so this solution was the simplest possible}.

\begin{center}
	\begin{tikzpicture}
	\tikzstyle{every path}=[thick]

	\node [rectangle,draw,minimum width=6cm, minimum height=2cm] (memory) {};
	\node [] [right=0.2cm of memory.west] (heap) {Heap};
	\node [] [left=0.2cm of memory.east] (stack) {Stack};

	\node [] (center1) [right=2cm of memory.west] {};
	\node [] (center2) [left=2cm of memory.east] {};

	\draw [->] (heap) -- (center1);
	\draw [->] (stack) -- (center2);

	\node [] [above left=1.1cm and 0.2cm of heap] (t1) {\RU{Начало кучи}\EN{Start of heap}};
	\node [] [above right=1.1cm and 0.2cm of stack] (t2) {\RU{Вершина стека}\EN{Start of stack}};

	\draw [->] (t1) -- (memory.west);
	\draw [->] (t2) -- (memory.east);

	\end{tikzpicture}
\end{center}

\RU{В}\EN{In} \cite{Ritchie74} \RU{можно прочитать}\EN{we can read}:

\begin{framed}
\begin{quotation}
The user-core part of an image is divided into three logical segments. The program text segment begins at location 0 in the virtual address space. During execution, this segment is write-protected and a single copy of it is shared among all processes executing the same program. At the first 8K byte boundary above the program text segment in the virtual address space begins a nonshared, writable data segment, the size of which may be extended by a system call. Starting at the highest address in the virtual address space is a stack segment, which automatically grows downward as the hardware's stack pointer fluctuates.
\end{quotation}
\end{framed}

\RU{Это немного напоминает как некоторые студенты
пишут два конспекта в одной тетрадке:
первый конспект начинается обычным образом, второй пишется с конца, перевернув тетрадку.
Конспекты могут встретиться где-то посредине, в случае недостатка свободного места.}
\EN{This reminds us how some students write two lecture notes using only one notebook:
notes for the first lecture are written as usual, 
and notes for the second one are written from the end of notebook, by flipping it.
Notes may meet each other somewhere in between, in case of lack of free space.}
% I think if we want to expand on this analogy,
% one might remember that the line number increases as as you go down a page.
% So when you decrease the address when pushing to the stack, visually,
% the stack does grow upwards.
% Of course, the problem is that in most human languages,
% just as with computers,
% we write downwards, so this direction is what makes buffer overflows so messy.

\section{\RU{Для чего используется стек?}\EN{What is the stack used for?}}

% subsections
\subsection{\RU{Сохранение адреса возврата управления}
\EN{Save the function's return address}}

\subsubsection{x86}

\index{x86!\Instructions!CALL}
\RU{При вызове другой функции через \CALL сначала в стек записывается адрес, указывающий на место после 
инструкции \CALL, затем делается безусловный переход (почти как \TT{JMP}) на адрес, указанный в операнде.} 
\EN{When calling another function with a \CALL instruction, the address of the point exactly after the \CALL instruction is saved 
to the stack and then an unconditional jump to the address in the CALL operand is executed.} 

\index{x86!\Instructions!PUSH}
\index{x86!\Instructions!JMP}
\RU{\CALL~--- это аналог пары инструкций \TT{PUSH address\_after\_call / JMP}.}
\EN{The \CALL instruction is equivalent to a \TT{PUSH address\_after\_call / JMP operand} instruction pair.}

\index{x86!\Instructions!RET}
\index{x86!\Instructions!POP}
\RU{\RET вытаскивает из стека значение и передает управление по этому адресу~--- 
это аналог пары инструкций \TT{POP tmp / JMP tmp}.}
\EN{\RET fetches a value from the stack and jumps to it~---that is equivalent to a \TT{POP tmp / JMP tmp} instruction pair.}

\index{\Stack!\RU{Переполнение стека}\EN{Stack overflow}}
\index{\Recursion}
\RU{Крайне легко устроить переполнение стека, запустив бесконечную рекурсию:}
\EN{Overflowing the stack is straightforward. Just run eternal recursion:}

\begin{lstlisting}
void f()
{
	f();
};
\end{lstlisting}

\RU{MSVC 2008 предупреждает о проблеме:}\EN{MSVC 2008 reports the problem:}

\begin{lstlisting}
c:\tmp6>cl ss.cpp /Fass.asm
Microsoft (R) 32-bit C/C++ Optimizing Compiler Version 15.00.21022.08 for 80x86
Copyright (C) Microsoft Corporation.  All rights reserved.

ss.cpp
c:\tmp6\ss.cpp(4) : warning C4717: 'f' : recursive on all control paths, function will cause runtime stack overflow
\end{lstlisting}

\dots \RU{но, тем не менее, создает нужный код}\EN{but generates the right code anyway}:

\begin{lstlisting}
?f@@YAXXZ PROC						; f
; File c:\tmp6\ss.cpp
; Line 2
	push	ebp
	mov	ebp, esp
; Line 3
	call	?f@@YAXXZ				; f
; Line 4
	pop	ebp
	ret	0
?f@@YAXXZ ENDP						; f
\end{lstlisting}

\dots \RU{причем, если включить оптимизацию (\Ox), то будет даже интереснее, без переполнения стека, 
но работать будет \IT{корректно}\footnote{здесь ирония}:}
\EN{Also if we turn on the compiler optimization (\Ox option) the optimized code will not overflow the stack 
and will work \IT{correctly}\footnote{irony here} instead:}

\begin{lstlisting}
?f@@YAXXZ PROC						; f
; File c:\tmp6\ss.cpp
; Line 2
$LL3@f:
; Line 3
	jmp	SHORT $LL3@f
?f@@YAXXZ ENDP						; f
\end{lstlisting}

\ifdefined\IncludeGCC
\RU{GCC 4.4.1 генерирует точно такой же код в обоих случаях, хотя и не предупреждает о проблеме.}
\EN{GCC 4.4.1 generates similar code in both cases without, however,  issuing any warning about the problem.}
\fi

\ifdefined\IncludeARM
\subsubsection{ARM}

\index{ARM!\Registers!Link Register}
\RU{Программы для ARM также используют стек для сохранения \ac{RA}, куда нужно вернуться, но несколько иначе}\EN{ARM
programs also use the stack for saving return addresses, but differently}.
\RU{Как уже упоминалось в секции}\EN{As mentioned in} \q{\HelloWorldSectionName}~(\myref{sec:hw_ARM}),
\RU{\ac{RA} записывается в регистр}\EN{the \ac{RA} is saved to the} \ac{LR} (\gls{link register}).
\RU{Но если есть необходимость вызывать какую-то другую функцию и использовать регистр \ac{LR} ещё
раз, его значение желательно сохранить}%
\EN{If one needs, however, to call another function and use the \ac{LR} register
one more time, its value has to be saved}.
\index{Function prologue}
\RU{Обычно это происходит в прологе функции, часто мы видим там инструкцию вроде}
\EN{Usually it is saved in the function prologue. Often, we see instructions like}
\index{ARM!\Instructions!PUSH}
\index{ARM!\Instructions!POP}
\TT{PUSH {R4-R7,LR}} \RU{, а в эпилоге}\EN{along with this instruction in epilogue}
\TT{POP {R4-R7,PC}}\RU{~--- так сохраняются регистры, которые будут использоваться в текущей функции, в том числе}
\EN{---thus register values
to be used in the function are saved in the stack, including} \ac{LR}.

\index{ARM!Leaf function}
\RU{Тем не менее, если некая функция не вызывает никаких более функций, в терминологии \ac{RISC} она называется}
\EN{Nevertheless, if a function never calls any other function, in \ac{RISC} terminology it is called a}
\IT{\gls{leaf function}}\footnote{\href{http://go.yurichev.com/17064}{infocenter.arm.com/help/index.jsp?topic=/com.arm.doc.faqs/ka13785.html}}. 
\RU{Как следствие, \q{leaf}-функция не сохраняет регистр \ac{LR} (потому что не изменяет его).}
\EN{As a consequence, leaf functions do not save the \ac{LR} register (because they don't modify it).}
\RU{А если эта функция небольшая, использует мало регистров, она может не использовать стек вообще}%
\EN{If such function is small and uses a small number of registers, it may not use the stack at all}.
\RU{Таким образом, в ARM возможен вызов небольших leaf-функций не используя стек.}
\EN{Thus, it is possible to call leaf functions without using the stack,}
\RU{Это может быть быстрее чем в старых x86, ведь внешняя память для стека не используется}%
\EN{which can be faster than on older x86 machines because external RAM is not used for the stack}%
\footnote{\RU{Когда-то, очень давно, на PDP-11 и VAX на инструкцию CALL (вызов других функций) могло тратиться
вплоть до 50\% времени (возможно из-за работы с памятью),
поэтому считалось, что много небольших функций это \glslink{anti-pattern}{анти-паттерн}}%
\EN{Some time ago, on PDP-11 and VAX, the CALL instruction (calling other functions) was expensive; up to 50\%
of execution time might be spent on it, so it was considered that having a big number of small functions is an \gls{anti-pattern}} \cite[Chapter 4, Part II]{Raymond:2003:AUP:829549}.}.
\RU{Либо это может быть полезным для тех ситуаций, когда память для стека ещё не выделена, либо недоступна}%
\EN{This can be also useful for situations when memory for the stack is not yet allocated or not available}.

\EN{Some examples of leaf functions:}\RU{Некоторые примеры таких функций:}
\myref{ARM_leaf_example1}, \myref{ARM_leaf_example2}, 
\myref{ARM_leaf_example3}, \myref{ARM_leaf_example4}, \myref{ARM_leaf_example5},
\myref{ARM_leaf_example6}, \myref{ARM_leaf_example7}, \myref{ARM_leaf_example10}.
\fi

\subsection{\RU{Передача параметров функции}\EN{Passing function arguments}}

\RU{Самый распространенный способ передачи параметров в x86 называется}
\EN{The most popular way to pass parameters in x86 is called} \q{cdecl}:

\begin{lstlisting}
push arg3
push arg2
push arg1
call f
add esp, 12 ; 4*3=12
\end{lstlisting}

\RU{Вызываемая функция получает свои параметры также через указатель стека.}
\EN{\Gls{callee} functions get their arguments via the stack pointer.}

\RU{Следовательно, так расположены значения в стеке перед исполнением самой первой инструкции
функции \ttf{}:}
\EN{Therefore, this is how the argument values are located in the stack before the execution
of the \ttf{} function's very first instruction:}

\begin{center}
\begin{tabular}{ | l | l | }
\hline
ESP & \RU{адрес возврата}\EN{return address} \\
\hline
ESP+4 & \argument \#1, \MarkedInIDAAs{} \TT{arg\_0} \\
\hline
ESP+8 & \argument \#2, \MarkedInIDAAs{} \TT{arg\_4} \\
\hline
ESP+0xC & \argument \#3, \MarkedInIDAAs{} \TT{arg\_8} \\
\hline
\dots & \dots \\
\hline
\end{tabular}
\end{center}

\ifx\LITE\undefined
\RU{См. также в соответствующем разделе о других способах передачи аргументов через стек}
\EN{For more information on other calling conventions see also section}~(\myref{sec:callingconventions}).
\fi
\RU{Важно отметить, что, в общем, никто не заставляет программистов передавать параметры именно через стек,
это не является требованием к исполняемому коду.}
\EN{It is worth noting that nothing obliges programmers to pass arguments through the stack. It is not a requirement.}
\RU{Вы можете делать это совершенно иначе, не используя стек вообще.}
\EN{One could implement any other method without using the stack at all.}

\RU{К примеру, можно выделять в \glslink{heap}{куче} место для аргументов, 
заполнять их и передавать в функцию указатель на это место через \EAX. И это вполне будет работать}%
\EN{For example, it is possible to allocate a space for arguments in the \gls{heap}, fill it and pass it to a function 
via a pointer to this block in the \EAX register. This will work}%
\footnote{\RU{Например, в книге Дональда Кнута \q{Искусство программирования}, в разделе 1.4.1 
посвященном подпрограммам \cite[раздел 1.4.1]{Knuth:1998:ACP:521463}, 
мы можем прочитать о возможности располагать параметры для вызываемой подпрограммы после инструкции \JMP,
передающей управление подпрограмме. Кнут описывает, что это было особенно удобно для компьютеров IBM System/360.}%
\EN{For example, in the \q{The Art of Computer Programming} book by Donald Knuth, 
in section 1.4.1 dedicated to subroutines \cite[section 1.4.1]{Knuth:1998:ACP:521463},
we could read that one way to supply arguments to a subroutine is simply to list them after the \JMP instruction
passing control to subroutine. Knuth explains that this method was particularly convenient on IBM System/360.}}.
\RU{Однако традиционно сложилось, что в x86 и ARM передача аргументов происходит именно через стек.}
% I am unsure about what this comment means.
% My guess is that the arguments are put in the memory position after
% the jump instruction, so you could say:
% "one way to supply arguments to a subroutine is simply to list them in memory
% after the \JMP instruction that passes control to the subroutine."
% Right now, "after" also sounds like it refers to the time after
% the jump happens, which I think is too late.
\EN{However, it is a convenient custom in x86 and ARM to use the stack for this purpose.} \\
\\
\RU{Кстати, вызываемая функция не имеет информации о количестве переданных ей аргументов.}
\EN{By the way, the \gls{callee} function does not have any information about how many arguments were passed.}
\RU{Функции Си с переменным количеством аргументов (как \printf) определяют их количество по 
спецификаторам строки формата (начинающиеся со знака \%).}
\EN{C functions with a variable number of arguments (like \printf) determine their number using format string  specifiers (which begin with the \% symbol).}
\RU{Если написать что-то вроде}\EN{If we write something like} 

\begin{lstlisting}
printf("%d %d %d", 1234);
\end{lstlisting}

\printf \RU{выведет 1234, затем ещё два случайных числа, которые волею случая оказались в стеке рядом.}
\EN{will print 1234, and then two random numbers, which were lying next to it in the stack.}\\
\\
\RU{Вот почему не так уж и важно, как объявлять функцию \main}
\EN{That's why it is not very important how we declare the \main function}: \RU{как}\EN{as} \main, 
\TT{main(int argc, char *argv[])} 
\RU{либо}\EN{or} \TT{main(int argc, char *argv[], char *envp[])}.

\RU{В реальности, \ac{CRT}-код вызывает \main примерно так:}
\EN{In fact, the \ac{CRT}-code is calling \main roughly as:}

\begin{lstlisting}
push envp
push argv
push argc
call main
...
\end{lstlisting}

\RU{Если вы объявляете \main без аргументов, они, тем не менее, присутствуют в стеке, но не используются.}
\EN{If you declare \main as \main without arguments, they are, nevertheless, still present in the stack, but
are not used.}
\RU{Если вы объявите \main как}\EN{If you declare \main as} \TT{main(int argc, char *argv[])}, 
\RU{вы можете использовать два первых аргумента, а третий останется для вашей функции \q{невидимым}.}
\EN{you will be able to use first two arguments, and the third will remain \q{invisible} for your function.}
\RU{Более того, можно даже объявить}\EN{Even more, it is possible to declare} \TT{main(int argc)}, 
\RU{и это будет работать}\EN{and it will work}.


\subsection{\RU{Хранение локальных переменных}\EN{Local variable storage}}

\RU{Функция может выделить для себя некоторое место в стеке для локальных переменных, просто отодвинув 
\glslink{stack pointer}{указатель стека} глубже к концу стека.}
\EN{A function could allocate space in the stack for its local variables just by decreasing 
the \gls{stack pointer} towards the stack bottom.}
% I think here, "stack bottom" means the lowest address in the stack space,
% but the reader might also think it means towards the top of the stack space,
% like in a pop, so you might change "towards the stack bottom" to
% "towards the lowest address of the stack", or just take it out,
% since "decreasing" also suggests that.
\RU{Это очень быстро вне зависимости от количества локальных переменных.}
\EN{Hence, it's very fast, no matter how many local variables are defined.}

\RU{Хранить локальные переменные в стеке не является необходимым требованием. 
Вы можете хранить локальные переменные где угодно. 
Но по традиции всё сложилось так.}
\EN{It is also not a requirement to store local variables in the stack.
You could store local variables wherever you like, 
but traditionally this is how it's done.}

\subsection{x86: \RU{Функция alloca()}\EN{alloca() function}}
\label{alloca}
\index{\CStandardLibrary!alloca()}
\RU{Интересен случай с функцией \TT{alloca()}}%
\EN{It is worth noting the \TT{alloca()} function}\footnote{
\RU{В MSVC, реализацию функции можно посмотреть в файлах}%
\EN{In MSVC, the function implementation can be found in} 
  \TT{alloca16.asm} 
  \AndENRU 
  \TT{chkstk.asm} 
  \InENRU 
  \TT{C:\textbackslash{}Program Files (x86)\textbackslash{}Microsoft Visual Studio 10.0\textbackslash{}VC\textbackslash{}crt\textbackslash{}src\textbackslash{}intel}}. 

\RU{Эта функция работает как \TT{malloc()}, но выделяет память прямо в стеке.} 
\EN{This function works like \TT{malloc()}, but allocates memory directly on the stack.}

\RU{Память освобождать через \TT{free()} не нужно, так как эпилог функции~(\myref{sec:prologepilog})
вернет \ESP в изначальное состояние и выделенная память просто \IT{выкидывается}.}
\EN{The allocated memory chunk does not need to be freed via a \TT{free()} function call, since the 
function epilogue~(\myref{sec:prologepilog}) returns \ESP back to its initial state and 
the allocated memory is just \IT{dropped}.}

\RU{Интересна реализация функции \TT{alloca()}.}
\EN{It is worth noting how \TT{alloca()} is implemented.}

\RU{Эта функция, если упрощенно, просто сдвигает \ESP вглубь стека 
на столько байт, сколько вам нужно и возвращает \ESP в качестве указателя на выделенный блок.}
\EN{In simple terms, this function just shifts \ESP downwards toward the stack bottom by the number of bytes you need and sets \ESP as a pointer to the \IT{allocated} block.}
\RU{Попробуем:}\EN{Let's try:}

\lstinputlisting{patterns/02_stack/04_alloca/2_1.c}

\RU{Функция \TT{\_snprintf()} работает так же, как и \printf, только вместо выдачи результата в \gls{stdout} (т.е. на терминал или в консоль),
записывает его в буфер \TT{buf}. Функция \puts выдает содержимое буфера \TT{buf} в \gls{stdout}. Конечно, можно было бы
заменить оба этих вызова на один \printf, но здесь нужно проиллюстрировать использование небольшого буфера.}%
\EN{\TT{\_snprintf()} function works just like \printf, but instead of dumping the result into \gls{stdout} (e.g., to terminal or 
console), it writes it to the \TT{buf} buffer. Function \puts copies the contents of \TT{buf} to \gls{stdout}. Of course, these two
function calls might be replaced by one \printf call, but we have to illustrate small buffer usage.}

\subsubsection{MSVC}

\RU{Компилируем}\EN{Let's compile} (MSVC 2010):

\lstinputlisting[caption=MSVC 2010]{patterns/02_stack/04_alloca/2_2_msvc.asm}

\index{Compiler intrinsic}
\RU {Единственный параметр в \TT{alloca()} передается через \EAX, а не как обычно через стек}%
\EN{The sole \TT{alloca()} argument is passed via \EAX (instead of pushing it into the stack)}%
\footnote{
\RU{Это потому, что alloca()~--- это не сколько функция, сколько т.н. \IT{compiler intrinsic}%
\ifx\LITE\undefined
(\myref{sec:compiler_intrinsic})
\fi
}%
\EN{It is because alloca() is rather a compiler intrinsic 
\ifx\LITE\undefined
(\myref{sec:compiler_intrinsic}) 
\fi
than a normal function}.

\RU{Одна из причин, почему здесь нужна именно функция, а не несколько инструкций прямо в коде в том, что в реализации 
функции alloca() от \ac{MSVC}
есть также код, читающий из только что выделенной памяти, чтобы \ac{OS} подключила физическую память к этому региону \ac{VM}.}
\EN{One of the reasons we need a separate function instead of just a couple of instructions in the code,
is because the \ac{MSVC} alloca() implementation also has code which reads from the memory just allocated, in order to let the \ac{OS} map
physical memory to this \ac{VM} region.}
}.
\RU{После вызова \TT{alloca()} \ESP указывает на блок в 600 байт, который 
мы можем использовать под \TT{buf}.}
\EN{After the \TT{alloca()} call, \ESP points to the block of 600 bytes and we can 
use it as memory for the \TT{buf} array.}

\ifdefined\IncludeGCC
\subsubsection{GCC + \IntelSyntax}

\RU{А GCC 4.4.1 обходится без вызова других функций:}
\EN{GCC 4.4.1 does the same without calling external functions:}

\lstinputlisting[caption=GCC 4.7.3]{patterns/02_stack/04_alloca/2_1_gcc_intel_O3.asm.\LANG}

\subsubsection{GCC + \ATTSyntax}

\RU{Посмотрим на тот же код, только в синтаксисе AT\&T}\EN{Let's see the same code, but in AT\&T syntax}:

\lstinputlisting[caption=GCC 4.7.3]{patterns/02_stack/04_alloca/2_1_gcc_ATT_O3.s}

\index{\ATTSyntax}
\RU{Всё то же самое, что и в прошлом листинге.}\EN{The code is the same as in the previous listing.}

\RU{Кстати}\EN{By the way}, \TT{movl \$3, 20(\%esp)}%
\RU{~--- это аналог}\EN{corresponds to} \TT{mov DWORD PTR [esp+20], 3}
\RU{в синтаксисе Intel.}\EN{ in Intel-syntax.}
\RU{Адресация памяти в виде \IT{регистр+смещение} записывается в синтаксисе AT\&T как \TT{смещение(\%{регистр})}.}
\EN{In the AT\&T syntax, the register+offset format of addressing memory looks like \TT{offset(\%{register})}.}
\fi

\subsection{(Windows) SEH}
\index{Windows!Structured Exception Handling}

\RU{В стеке хранятся записи \ac{SEH} для функции (если они присутствуют)}%
\EN{\ac{SEH} records are also stored on the stack (if they are present).}.

\ifx\LITE\undefined
\RU{Читайте больше о нем здесь}\EN{Read more about it}: (\myref{sec:SEH}).
\fi

\input{patterns/02_stack/06_BO_protection}

\subsection{\EN{Automatic deallocation of data in stack}\RU{Автоматическое освобождение данных в стеке}}

\RU{Возможно, причина хранения локальных переменных и SEH-записей в стеке в том, что после выхода из функции, всё эти данные освобождаются автоматически,
используя только одну инструкцию корректирования указателя стека (часто это ADD).}
\EN{Perhaps, the reason for storing local variables and SEH records in the stack is that they are freed automatically upon function exit,
using just one instruction to correct the stack pointer (it is often ADD).}
\RU{Аргументы функций, можно сказать, тоже освобождаются автоматически в конце функции.}
\EN{Function arguments, as we could say, are also deallocated automatically at the end of function.}
\RU{А всё что хранится в куче (\IT{heap}) нужно освобождать явно.}
\EN{In contrast, everything stored in the \IT{heap} must be deallocated explicitly.}

% sections
\section{\RU{Разметка типичного стека}\EN{A typical stack layout}}

\RU{Разметка типичного стека в 32-битной среде
перед исполнением самой первой инструкции функции выглядит так:}
\EN{A typical stack layout in a 32-bit environment at the start of a function, 
before the first instruction execution looks like this:}

\begin{center}
\begin{tabular}{ | l | l | }
\hline
\dots & \dots \\
\hline
ESP-0xC & \RU{локальная переменная}\EN{local variable} \#2, \MarkedInIDAAs{} \TT{var\_8} \\
\hline
ESP-8 & \RU{локальная переменная}\EN{local variable} \#1, \MarkedInIDAAs{} \TT{var\_4} \\
\hline
ESP-4 & \RU{сохраненное значение}\EN{saved value of} \EBP \\
\hline
ESP & \RU{адрес возврата}\EN{return address} \\
\hline
ESP+4 & \argument \#1, \MarkedInIDAAs{} \TT{arg\_0} \\
\hline
ESP+8 & \argument \#2, \MarkedInIDAAs{} \TT{arg\_4} \\
\hline
ESP+0xC & \argument \#3, \MarkedInIDAAs{} \TT{arg\_8} \\
\hline
\dots & \dots \\
\hline
\end{tabular}
\end{center}
% I think this only applies to RISC architectures
% that don't have a POP instruction that only lets you read one value
% (ie. ARM and MIPS).
% In x86, the return address is saved before entering the function,
% and the function does not have the chance to save the frame pointer.
% Also, you should mention that this is how the stack looks like
% right after the function prologue,
% which some readers might think is the first instruction,
% but is needed to save the frame pointer.

\ifx\LITE\undefined
\section{\RU{Мусор в стеке}\EN{Noise in stack}}

\RU{Часто в этой книге говорится о \q{шуме} или \q{мусоре} в стеке или памяти.}
\EN{Often in this book \q{noise} or \q{garbage} values in the stack or memory are mentioned.}
\RU{Откуда он берется}\EN{Where do they come from}?
\RU{Это то, что осталось там после исполнения предыдущих функций.}
\EN{These are what was left in there after other functions' executions.}
\RU{Короткий пример}\EN{Short example}:

\lstinputlisting{patterns/02_stack/08_noise/st.c}

\RU{Компилируем}\EN{Compiling}\dots

\lstinputlisting[caption=\NonOptimizing MSVC 2010]{patterns/02_stack/08_noise/st.asm}

\RU{Компилятор поворчит немного}\EN{The compiler will grumble a little bit}\dots

\begin{lstlisting}
c:\Polygon\c>cl st.c /Fast.asm /MD
Microsoft (R) 32-bit C/C++ Optimizing Compiler Version 16.00.40219.01 for 80x86
Copyright (C) Microsoft Corporation.  All rights reserved.

st.c
c:\polygon\c\st.c(11) : warning C4700: uninitialized local variable 'c' used
c:\polygon\c\st.c(11) : warning C4700: uninitialized local variable 'b' used
c:\polygon\c\st.c(11) : warning C4700: uninitialized local variable 'a' used
Microsoft (R) Incremental Linker Version 10.00.40219.01
Copyright (C) Microsoft Corporation.  All rights reserved.

/out:st.exe
st.obj
\end{lstlisting}

\RU{Но когда мы запускаем}\EN{But when we run the compiled program}\dots

\begin{lstlisting}
c:\Polygon\c>st
1, 2, 3
\end{lstlisting}

\RU{Ох. Вот это странно. Мы ведь не устанавливали значения никаких переменных в}\EN{Oh, 
what a weird thing! We did not set any variables in} \TT{f2()}. 
\RU{Эти значения --- это \q{привидения}, которые всё ещё в стеке.}
\EN{These are \q{ghosts} values, which are still in the stack.}

\clearpage
\RU{Загрузим пример в}\EN{Let's load the example into} \olly:

\begin{figure}[H]
\centering
\includegraphics[scale=\FigScale]{patterns/02_stack/08_noise/olly1.png}
\caption{\olly: \TT{f1()}}
\label{fig:stack_noise_olly1}
\end{figure}

\RU{Когда}\EN{When} \TT{f1()} \RU{заполняет переменные}\EN{assigns the variables} $a$, $b$ \AndENRU $c$ 
\RU{они сохраняются по адресу}\EN{, their values are stored at the address} \TT{0x1FF860} 
\RU{\etc{}.}\EN{and so on.}

\clearpage
\RU{А когда исполняется}\EN{And when} \TT{f2()}\EN{ executes}:

\begin{figure}[H]
\centering
\includegraphics[scale=\FigScale]{patterns/02_stack/08_noise/olly2.png}
\caption{\olly: \TT{f2()}}
\label{fig:stack_noise_olly2}
\end{figure}

... $a$, $b$ \AndENRU $c$ \RU{в функции}\EN{of} \TT{f2()} \RU{находятся по тем же адресам!}
\EN{are located at the same addresses!}
\RU{Пока никто не перезаписал их, так что они здесь в нетронутом виде.}
\EN{No one has overwritten the values yet, so at that point they are still untouched.}

\RU{Для создания такой странной ситуации несколько функций должны исполняться друг за другом
и \ac{SP} должен быть одинаковым при входе в функции, т.е. у функций должно быть равное количество
аргументов). Тогда локальные переменные будут расположены в том же месте стека.}
\EN{So, for this weird situation to occur, several functions have to be called one after another and
\ac{SP} has to be the same at each function entry (i.e., they have the same number
of arguments). Then the local variables will be located at the same positions in the stack.}

\RU{Подводя итоги, все значения в стеке (да и памяти вообще) это значения оставшиеся от 
исполнения предыдущих функций.}
\EN{Summarizing, all values in the stack (and memory cells in general) 
have values left there from previous function executions.}
\RU{Строго говоря, они не случайны, они скорее непредсказуемы.}
\EN{They are not random in the strict sense, but rather have unpredictable values.}

\RU{А как иначе}\EN{Is there another option}?
\RU{Можно было бы очищать части стека перед исполнением каждой функции,
но это слишком много лишней (и ненужной) работы.}
\EN{It probably would be possible to clear portions of the stack before each function execution,
but that's too much extra (and unnecessary) work.}

\subsection{MSVC 2013}

\EN{The example was compiled by}\RU{Этот пример был скомпилирован в} MSVC 2010.
\EN{But the reader of this book made attempt to compile this example in MSVC 2013, ran it, and got all 3 numbers reversed:}%
\RU{Но один читатель этой книги сделал попытку скомпилировать пример в MSVC 2013, запустил и увидел 3 числа в обратном порядке:}

\begin{lstlisting}
c:\Polygon\c>st
3, 2, 1
\end{lstlisting}

\EN{Why?}\RU{Почему?}

\EN{I also compiled this example in MSVC 2013 and saw this:}%
\RU{Я также попробовал скомпилировать этот пример в MSVC 2013 и увидел это:}

\begin{lstlisting}[caption=MSVC 2013]
_a$ = -12						; size = 4
_b$ = -8						; size = 4
_c$ = -4						; size = 4
_f2	PROC

...

_f2	ENDP

_c$ = -12						; size = 4
_b$ = -8						; size = 4
_a$ = -4						; size = 4
_f1	PROC

...

_f1	ENDP
\end{lstlisting}

\EN{Unlike MSVC 2010, MSVC 2013 allocated a/b/c variables in function \TT{f2()} in reverse order.}%
\RU{В отличии от MSVC 2010, MSVC 2013 разместил переменные a/b/c в функции \TT{f2()} в обратном порядке.}
\EN{And this is completely correct, because \CCpp standards has no rule, in which order local variables must be allocated in the local stack, if at all.}%
\RU{И это полностью корректно, потому что в стандартах \CCpp нет правила, в каком порядке локальные переменные должны быть размещены в локальном стеке,
если вообще.}
\EN{The reason of difference is because MSVC 2010 has one way to do it, and MSVC 2013 has probably something changed inside of compiler guts, so it behaves
slightly different.}%
\RU{Разница есть из-за того что MSVC 2010 делает это одним способом, а в MSVC 2013, вероятно, что-то немного изменили во внутренностях компилятора,
так что он ведет себя слегка иначе.}


\fi
\ifdefined\IncludeExercises
\section{\Exercises}

\begin{itemize}
	\item \url{http://challenges.re/51}
	\item \url{http://challenges.re/52}
\end{itemize}


\fi

\clearpage
\section{\RU{Простейшее четырехбайтное XOR-шифрование}\EN{Simplest possible 4-byte XOR encryption}}

\RU{Если при XOR-шифровании применялся шаблон длинее байта, например, 4-байтный, то его также легко
увидеть.}
\EN{If longer pattern was used while XOR-encryption, for example, 4 byte pattern, it's easy
to spot it as well.}
\RU{Например, вот начало файла kernel32.dll (32-битная версия из Windows Server 2008):}
\EN{As example, here is beginning of kernel32.dll file (32-bit version from Windows Server 2008):}

\begin{figure}[H]
\centering
\includegraphics[scale=\FigScale]{ff/XOR/4byte/original1.png}
\caption{\EN{Original file}\RU{Оригинальный файл}}
\end{figure}

\clearpage
\RU{Вот он же, но \q{зашифрованный} 4-байтным ключем:}
\EN{Here is it \q{encrypted} by 4-byte key:}

\begin{figure}[H]
\centering
\includegraphics[scale=\FigScale]{ff/XOR/4byte/encrypted1.png}
\caption{\EN{\q{Encrypted} file}\RU{\q{Зашифрованный} файл}}
\end{figure}

\RU{Очень легко увидеть повторяющиеся 4 символа.}
\EN{It's very easy to spot recurring 4 symbols.}
\RU{Ведь в заголовке PE-файла много длинных нулевых областей, из-за которых ключ становится видным.}
\EN{Indeed, PE-file header has a lot of long zero lacunes, which is the reason why key became visible.}

\clearpage
\RU{Вот начало PE-заголовка в 16-ричном виде:}
\EN{Here is beginning of PE-header in hexadecimal form:}

\begin{figure}[H]
\centering
\includegraphics[scale=\FigScale]{ff/XOR/4byte/original2.png}
\caption{PE-\EN{header}\RU{заголовок}}
\end{figure}

\clearpage
\RU{И вот он же, \q{зашифрованный}:}
\EN{Here is it \q{encrypted}:}

\begin{figure}[H]
\centering
\includegraphics[scale=\FigScale]{ff/XOR/4byte/encrypted2.png}
\caption{\EN{\q{Encrypted} PE-header}\RU{\q{Зашифрованный} PE-заголовок}}
\end{figure}

\RU{Легко увидеть визуально, что ключ это следующие 4 байта}
\EN{It's easy to spot that key is the following 4 bytes}: \TT{8C 61 D2 63}.
\RU{Используя эту информацию, довольно легко расшифровать весь файл.}
\EN{It's easy to decrypt the whole file using this information.}

\RU{Таким образом, важно помнить эти свойства PE-файлов:
1) в PE-заголовке много нулевых областей;
2) все PE-секции дополняются нулями до границы страницы (4096 байт), 
так что после всех секций обычно имеются длинные нулевые области.}
\EN{So this is important to remember these property of PE-files:
1) PE-header has many zero lacunas;
2) all PE-sections padded with zeroes by page border (4096 bytes),
so long zero lacunas usually present after all sections.}

\RU{Некоторые другие форматы файлов могут также иметь длинные нулевые области.}
\EN{Some other file formats may contain long zero lacunas.}
\RU{Это очень типично для файлов, используемых научным и инженерным ПО.}
\EN{It's very typical for files used by scientific and engineering software.}

\RU{Для тех, кто самостоятельно хочет изучить эти файлы, то их можно скачать здесь:}
\EN{For those who wants to inspect these files on one's own, they are downloadable there:}
\url{http://go.yurichev.com/17352}.

\subsection{\Exercise}

\begin{itemize}
	\item \url{http://challenges.re/50}
\end{itemize}


\chapter{scanf()}
\index{\CStandardLibrary!scanf()}
\label{label_scanf}

\RU{Теперь попробуем использовать scanf().}\EN{Now let's use scanf().}

% sections
\clearpage
\section{\RU{Простейшее четырехбайтное XOR-шифрование}\EN{Simplest possible 4-byte XOR encryption}}

\RU{Если при XOR-шифровании применялся шаблон длинее байта, например, 4-байтный, то его также легко
увидеть.}
\EN{If longer pattern was used while XOR-encryption, for example, 4 byte pattern, it's easy
to spot it as well.}
\RU{Например, вот начало файла kernel32.dll (32-битная версия из Windows Server 2008):}
\EN{As example, here is beginning of kernel32.dll file (32-bit version from Windows Server 2008):}

\begin{figure}[H]
\centering
\includegraphics[scale=\FigScale]{ff/XOR/4byte/original1.png}
\caption{\EN{Original file}\RU{Оригинальный файл}}
\end{figure}

\clearpage
\RU{Вот он же, но \q{зашифрованный} 4-байтным ключем:}
\EN{Here is it \q{encrypted} by 4-byte key:}

\begin{figure}[H]
\centering
\includegraphics[scale=\FigScale]{ff/XOR/4byte/encrypted1.png}
\caption{\EN{\q{Encrypted} file}\RU{\q{Зашифрованный} файл}}
\end{figure}

\RU{Очень легко увидеть повторяющиеся 4 символа.}
\EN{It's very easy to spot recurring 4 symbols.}
\RU{Ведь в заголовке PE-файла много длинных нулевых областей, из-за которых ключ становится видным.}
\EN{Indeed, PE-file header has a lot of long zero lacunes, which is the reason why key became visible.}

\clearpage
\RU{Вот начало PE-заголовка в 16-ричном виде:}
\EN{Here is beginning of PE-header in hexadecimal form:}

\begin{figure}[H]
\centering
\includegraphics[scale=\FigScale]{ff/XOR/4byte/original2.png}
\caption{PE-\EN{header}\RU{заголовок}}
\end{figure}

\clearpage
\RU{И вот он же, \q{зашифрованный}:}
\EN{Here is it \q{encrypted}:}

\begin{figure}[H]
\centering
\includegraphics[scale=\FigScale]{ff/XOR/4byte/encrypted2.png}
\caption{\EN{\q{Encrypted} PE-header}\RU{\q{Зашифрованный} PE-заголовок}}
\end{figure}

\RU{Легко увидеть визуально, что ключ это следующие 4 байта}
\EN{It's easy to spot that key is the following 4 bytes}: \TT{8C 61 D2 63}.
\RU{Используя эту информацию, довольно легко расшифровать весь файл.}
\EN{It's easy to decrypt the whole file using this information.}

\RU{Таким образом, важно помнить эти свойства PE-файлов:
1) в PE-заголовке много нулевых областей;
2) все PE-секции дополняются нулями до границы страницы (4096 байт), 
так что после всех секций обычно имеются длинные нулевые области.}
\EN{So this is important to remember these property of PE-files:
1) PE-header has many zero lacunas;
2) all PE-sections padded with zeroes by page border (4096 bytes),
so long zero lacunas usually present after all sections.}

\RU{Некоторые другие форматы файлов могут также иметь длинные нулевые области.}
\EN{Some other file formats may contain long zero lacunas.}
\RU{Это очень типично для файлов, используемых научным и инженерным ПО.}
\EN{It's very typical for files used by scientific and engineering software.}

\RU{Для тех, кто самостоятельно хочет изучить эти файлы, то их можно скачать здесь:}
\EN{For those who wants to inspect these files on one's own, they are downloadable there:}
\url{http://go.yurichev.com/17352}.

\subsection{\Exercise}

\begin{itemize}
	\item \url{http://challenges.re/50}
\end{itemize}


\clearpage
\section{\RU{Простейшее четырехбайтное XOR-шифрование}\EN{Simplest possible 4-byte XOR encryption}}

\RU{Если при XOR-шифровании применялся шаблон длинее байта, например, 4-байтный, то его также легко
увидеть.}
\EN{If longer pattern was used while XOR-encryption, for example, 4 byte pattern, it's easy
to spot it as well.}
\RU{Например, вот начало файла kernel32.dll (32-битная версия из Windows Server 2008):}
\EN{As example, here is beginning of kernel32.dll file (32-bit version from Windows Server 2008):}

\begin{figure}[H]
\centering
\includegraphics[scale=\FigScale]{ff/XOR/4byte/original1.png}
\caption{\EN{Original file}\RU{Оригинальный файл}}
\end{figure}

\clearpage
\RU{Вот он же, но \q{зашифрованный} 4-байтным ключем:}
\EN{Here is it \q{encrypted} by 4-byte key:}

\begin{figure}[H]
\centering
\includegraphics[scale=\FigScale]{ff/XOR/4byte/encrypted1.png}
\caption{\EN{\q{Encrypted} file}\RU{\q{Зашифрованный} файл}}
\end{figure}

\RU{Очень легко увидеть повторяющиеся 4 символа.}
\EN{It's very easy to spot recurring 4 symbols.}
\RU{Ведь в заголовке PE-файла много длинных нулевых областей, из-за которых ключ становится видным.}
\EN{Indeed, PE-file header has a lot of long zero lacunes, which is the reason why key became visible.}

\clearpage
\RU{Вот начало PE-заголовка в 16-ричном виде:}
\EN{Here is beginning of PE-header in hexadecimal form:}

\begin{figure}[H]
\centering
\includegraphics[scale=\FigScale]{ff/XOR/4byte/original2.png}
\caption{PE-\EN{header}\RU{заголовок}}
\end{figure}

\clearpage
\RU{И вот он же, \q{зашифрованный}:}
\EN{Here is it \q{encrypted}:}

\begin{figure}[H]
\centering
\includegraphics[scale=\FigScale]{ff/XOR/4byte/encrypted2.png}
\caption{\EN{\q{Encrypted} PE-header}\RU{\q{Зашифрованный} PE-заголовок}}
\end{figure}

\RU{Легко увидеть визуально, что ключ это следующие 4 байта}
\EN{It's easy to spot that key is the following 4 bytes}: \TT{8C 61 D2 63}.
\RU{Используя эту информацию, довольно легко расшифровать весь файл.}
\EN{It's easy to decrypt the whole file using this information.}

\RU{Таким образом, важно помнить эти свойства PE-файлов:
1) в PE-заголовке много нулевых областей;
2) все PE-секции дополняются нулями до границы страницы (4096 байт), 
так что после всех секций обычно имеются длинные нулевые области.}
\EN{So this is important to remember these property of PE-files:
1) PE-header has many zero lacunas;
2) all PE-sections padded with zeroes by page border (4096 bytes),
so long zero lacunas usually present after all sections.}

\RU{Некоторые другие форматы файлов могут также иметь длинные нулевые области.}
\EN{Some other file formats may contain long zero lacunas.}
\RU{Это очень типично для файлов, используемых научным и инженерным ПО.}
\EN{It's very typical for files used by scientific and engineering software.}

\RU{Для тех, кто самостоятельно хочет изучить эти файлы, то их можно скачать здесь:}
\EN{For those who wants to inspect these files on one's own, they are downloadable there:}
\url{http://go.yurichev.com/17352}.

\subsection{\Exercise}

\begin{itemize}
	\item \url{http://challenges.re/50}
\end{itemize}


\clearpage
\section{\RU{Простейшее четырехбайтное XOR-шифрование}\EN{Simplest possible 4-byte XOR encryption}}

\RU{Если при XOR-шифровании применялся шаблон длинее байта, например, 4-байтный, то его также легко
увидеть.}
\EN{If longer pattern was used while XOR-encryption, for example, 4 byte pattern, it's easy
to spot it as well.}
\RU{Например, вот начало файла kernel32.dll (32-битная версия из Windows Server 2008):}
\EN{As example, here is beginning of kernel32.dll file (32-bit version from Windows Server 2008):}

\begin{figure}[H]
\centering
\includegraphics[scale=\FigScale]{ff/XOR/4byte/original1.png}
\caption{\EN{Original file}\RU{Оригинальный файл}}
\end{figure}

\clearpage
\RU{Вот он же, но \q{зашифрованный} 4-байтным ключем:}
\EN{Here is it \q{encrypted} by 4-byte key:}

\begin{figure}[H]
\centering
\includegraphics[scale=\FigScale]{ff/XOR/4byte/encrypted1.png}
\caption{\EN{\q{Encrypted} file}\RU{\q{Зашифрованный} файл}}
\end{figure}

\RU{Очень легко увидеть повторяющиеся 4 символа.}
\EN{It's very easy to spot recurring 4 symbols.}
\RU{Ведь в заголовке PE-файла много длинных нулевых областей, из-за которых ключ становится видным.}
\EN{Indeed, PE-file header has a lot of long zero lacunes, which is the reason why key became visible.}

\clearpage
\RU{Вот начало PE-заголовка в 16-ричном виде:}
\EN{Here is beginning of PE-header in hexadecimal form:}

\begin{figure}[H]
\centering
\includegraphics[scale=\FigScale]{ff/XOR/4byte/original2.png}
\caption{PE-\EN{header}\RU{заголовок}}
\end{figure}

\clearpage
\RU{И вот он же, \q{зашифрованный}:}
\EN{Here is it \q{encrypted}:}

\begin{figure}[H]
\centering
\includegraphics[scale=\FigScale]{ff/XOR/4byte/encrypted2.png}
\caption{\EN{\q{Encrypted} PE-header}\RU{\q{Зашифрованный} PE-заголовок}}
\end{figure}

\RU{Легко увидеть визуально, что ключ это следующие 4 байта}
\EN{It's easy to spot that key is the following 4 bytes}: \TT{8C 61 D2 63}.
\RU{Используя эту информацию, довольно легко расшифровать весь файл.}
\EN{It's easy to decrypt the whole file using this information.}

\RU{Таким образом, важно помнить эти свойства PE-файлов:
1) в PE-заголовке много нулевых областей;
2) все PE-секции дополняются нулями до границы страницы (4096 байт), 
так что после всех секций обычно имеются длинные нулевые области.}
\EN{So this is important to remember these property of PE-files:
1) PE-header has many zero lacunas;
2) all PE-sections padded with zeroes by page border (4096 bytes),
so long zero lacunas usually present after all sections.}

\RU{Некоторые другие форматы файлов могут также иметь длинные нулевые области.}
\EN{Some other file formats may contain long zero lacunas.}
\RU{Это очень типично для файлов, используемых научным и инженерным ПО.}
\EN{It's very typical for files used by scientific and engineering software.}

\RU{Для тех, кто самостоятельно хочет изучить эти файлы, то их можно скачать здесь:}
\EN{For those who wants to inspect these files on one's own, they are downloadable there:}
\url{http://go.yurichev.com/17352}.

\subsection{\Exercise}

\begin{itemize}
	\item \url{http://challenges.re/50}
\end{itemize}



\section{\Exercise}

\begin{itemize}
	\item \url{http://challenges.re/53}
\end{itemize}


\clearpage
\section{\RU{Простейшее четырехбайтное XOR-шифрование}\EN{Simplest possible 4-byte XOR encryption}}

\RU{Если при XOR-шифровании применялся шаблон длинее байта, например, 4-байтный, то его также легко
увидеть.}
\EN{If longer pattern was used while XOR-encryption, for example, 4 byte pattern, it's easy
to spot it as well.}
\RU{Например, вот начало файла kernel32.dll (32-битная версия из Windows Server 2008):}
\EN{As example, here is beginning of kernel32.dll file (32-bit version from Windows Server 2008):}

\begin{figure}[H]
\centering
\includegraphics[scale=\FigScale]{ff/XOR/4byte/original1.png}
\caption{\EN{Original file}\RU{Оригинальный файл}}
\end{figure}

\clearpage
\RU{Вот он же, но \q{зашифрованный} 4-байтным ключем:}
\EN{Here is it \q{encrypted} by 4-byte key:}

\begin{figure}[H]
\centering
\includegraphics[scale=\FigScale]{ff/XOR/4byte/encrypted1.png}
\caption{\EN{\q{Encrypted} file}\RU{\q{Зашифрованный} файл}}
\end{figure}

\RU{Очень легко увидеть повторяющиеся 4 символа.}
\EN{It's very easy to spot recurring 4 symbols.}
\RU{Ведь в заголовке PE-файла много длинных нулевых областей, из-за которых ключ становится видным.}
\EN{Indeed, PE-file header has a lot of long zero lacunes, which is the reason why key became visible.}

\clearpage
\RU{Вот начало PE-заголовка в 16-ричном виде:}
\EN{Here is beginning of PE-header in hexadecimal form:}

\begin{figure}[H]
\centering
\includegraphics[scale=\FigScale]{ff/XOR/4byte/original2.png}
\caption{PE-\EN{header}\RU{заголовок}}
\end{figure}

\clearpage
\RU{И вот он же, \q{зашифрованный}:}
\EN{Here is it \q{encrypted}:}

\begin{figure}[H]
\centering
\includegraphics[scale=\FigScale]{ff/XOR/4byte/encrypted2.png}
\caption{\EN{\q{Encrypted} PE-header}\RU{\q{Зашифрованный} PE-заголовок}}
\end{figure}

\RU{Легко увидеть визуально, что ключ это следующие 4 байта}
\EN{It's easy to spot that key is the following 4 bytes}: \TT{8C 61 D2 63}.
\RU{Используя эту информацию, довольно легко расшифровать весь файл.}
\EN{It's easy to decrypt the whole file using this information.}

\RU{Таким образом, важно помнить эти свойства PE-файлов:
1) в PE-заголовке много нулевых областей;
2) все PE-секции дополняются нулями до границы страницы (4096 байт), 
так что после всех секций обычно имеются длинные нулевые области.}
\EN{So this is important to remember these property of PE-files:
1) PE-header has many zero lacunas;
2) all PE-sections padded with zeroes by page border (4096 bytes),
so long zero lacunas usually present after all sections.}

\RU{Некоторые другие форматы файлов могут также иметь длинные нулевые области.}
\EN{Some other file formats may contain long zero lacunas.}
\RU{Это очень типично для файлов, используемых научным и инженерным ПО.}
\EN{It's very typical for files used by scientific and engineering software.}

\RU{Для тех, кто самостоятельно хочет изучить эти файлы, то их можно скачать здесь:}
\EN{For those who wants to inspect these files on one's own, they are downloadable there:}
\url{http://go.yurichev.com/17352}.

\subsection{\Exercise}

\begin{itemize}
	\item \url{http://challenges.re/50}
\end{itemize}


\chapter{\RU{Ещё о возвращаемых результатах}\EN{More about results returning}}

\index{x86!\Registers!EAX}
\RU{Результат выполнения функции в x86 обычно возвращается}%
\EN{In x86, the result of function execution is usually returned}%
\footnote{\Seealso: 
MSDN: Return Values (C++): \href{http://go.yurichev.com/17258}{MSDN}}
\RU{через регистр \EAX, 
а если результат имеет тип байт или символ (\Tchar), 
то в самой младшей части \EAX~--- \AL. Если функция возвращает число с плавающей запятой, 
то будет использован регистр FPU \ST{0}.
\ifdefined\IncludeARM
\index{ARM!\Registers!R0}
В ARM обычно результат возвращается в регистре \Reg{0}.
\fi
}
\EN{in the \EAX register. 
If it is byte type or a character (\Tchar), then the lowest part of register \EAX (\AL) is used. 
If a function returns a \Tfloat number, the FPU register \ST{0} is used instead.
\ifdefined\IncludeARM
\index{ARM!\Registers!R0}
In ARM, the result is usually returned in the \Reg{0} register.
\fi
}

\section{\RU{Попытка использовать результат функции возвращающей \Tvoid}
\EN{Attempt to use the result of a function returning \Tvoid}}

\RU{Кстати, что будет, если возвращаемое значение в функции \main объявлять не как \Tint, а как \Tvoid?}
\EN{So, what if the \main function return value was declared of type \Tvoid and not \Tint?}

\RU{Т.н. startup-код вызывает \main примерно так:}
\EN{The so-called startup-code is calling \main roughly as follows:}

\begin{lstlisting}
push envp
push argv
push argc
call main
push eax
call exit
\end{lstlisting}

\RU{Иными словами:}\EN{In other words:}

\begin{lstlisting}
exit(main(argc,argv,envp));
\end{lstlisting}

\RU{Если вы объявите \main как \Tvoid, и ничего не будете возвращать явно (при помощи выражения \IT{return}), 
то в единственный аргумент exit() попадет
то, что лежало в регистре \EAX на момент выхода из \main.}
\EN{If you declare \main as \Tvoid, nothing is to be returned explicitly 
(using the \IT{return} statement),
then something random, that was stored in the \EAX register at the end of \main becomes 
the sole argument of the exit() function.}
\RU{Там, скорее всего, будет какие-то случайное число, оставшееся от работы вашей функции.
Так что код завершения программы будет псевдослучайным.}
\EN{Most likely, there will be a random value, left from your function execution,
so the exit code of program is pseudorandom.}\PTBRph{}\ESph{}\PLph{}\ITAph{} \\

\RU{Мы можем это проиллюстрировать}\EN{We can illustrate this fact}. 
\RU{Заметьте, что у функции}\EN{Please note that here the} \main
\RU{тип возвращаемого значения именно}\EN{function has a} \Tvoid\EN{ return type}:

\begin{lstlisting}
#include <stdio.h>

void main()
{
	printf ("Hello, world!\n");
};
\end{lstlisting}

\RU{Скомпилируем в}\EN{Let's compile it in} Linux.

\index{puts() \RU{вместо}\EN{instead of} printf()}
GCC 4.8.1 \RU{заменила}\EN{replaced} \printf \RU{на}\EN{with} \puts 
\ifx\LITE\undefined
(\RU{мы видели это прежде}\EN{we have seen this before}: \myref{puts})
\fi
, \RU{но это нормально, потому что}\EN{but that's OK, since} \puts \RU{возвращает количество
выведенных символов, так же как и}\EN{returns the number of characters printed out, just like} \printf.
\RU{Обратите внимание на то, что}\EN{Please notice that} \EAX \RU{не обнуляется перед выходом их}\EN{is not 
zeroed before} \main\EN{'s end}.
\RU{Это значит что \EAX перед выходом из \main содержит то, что \puts оставляет там.}
\EN{This implies that the value of \EAX at the end of \main contains what \puts has left there.}

\begin{lstlisting}[caption=GCC 4.8.1]
.LC0:
	.string	"Hello, world!"
main:
	push	ebp
	mov	ebp, esp
	and	esp, -16
	sub	esp, 16
	mov	DWORD PTR [esp], OFFSET FLAT:.LC0
	call	puts
	leave
	ret
\end{lstlisting}

\index{bash}
\RU{Напишем небольшой скрипт на bash, показывающий статус возврата (\q{exit status} или \q{exit code})}
\EN{Let' s write a bash script that shows the exit status}:

\begin{lstlisting}[caption=tst.sh]
#!/bin/sh
./hello_world
echo $?
\end{lstlisting}

\RU{И запустим}\EN{And run it}:

\begin{lstlisting}
$ tst.sh 
Hello, world!
14
\end{lstlisting}

14 \RU{это как раз количество выведенных символов}\EN{is the number of characters printed}.

\section{\RU{Что если не использовать результат функции?}\EN{What if we do not use the function result?}}

\RU{\printf возвращает количество успешно выведенных символов, но результат работы этой функции 
редко используется на практике.}
\EN{\printf returns the count of characters successfully output, but the result of this function 
is rarely used in practice.}
\RU{Можно даже явно вызывать функции, чей смысл именно в возвращаемых значениях, но явно не использовать их:}
\EN{It is also possible to call a function whose essence is in returning a value, and not use it:}

\begin{lstlisting}
int f()
{
    // skip first 3 random values
    rand();
    rand();
    rand();
    // and use 4th
    return rand();
};
\end{lstlisting}

\EN{The result of the rand() function is left in \EAX, in all four cases.}
\RU{Результат работы rand() остается в \EAX во всех четырех случаях.}
\EN{But in the first 3 cases, the value in \EAX is just thrown away.}
\RU{Но в первых трех случаях значение, лежащее в \EAX, просто выбрасывается.}

\ifx\LITE\undefined
\section{\RU{Возврат структуры}\EN{Returning a structure}}

\index{\CLanguageElements!return}
\RU{Вернемся к тому факту, что возвращаемое значение остается в регистре \EAX.}
\EN{Let's go back to the fact that the return value is left in the \EAX register.}
\RU{Вот почему старые компиляторы Си не способны создавать функции, возвращающие нечто большее, нежели 
помещается 
в один регистр (обычно тип \Tint), а когда нужно, приходится возвращать через указатели, указываемые 
в аргументах.}
\EN{That is why old C compilers cannot create functions capable of returning something that does not fit in one 
register (usually \Tint), but if one needs it, one have to return information via pointers passed 
as function's arguments.}
\RU{Так что как правило, если функция должна вернуть несколько значений, она возвращает только одно, 
а остальные~--- через указатели.}
\EN{So, usually, if a function needs to return several values, it returns only one, and 
all the rest---via pointers.}
\RU{Хотя позже и стало возможным, вернуть, скажем, целую структуру, но этот метод до сих пор не 
очень популярен. 
Если функция должна вернуть структуру, вызывающая функция должна сама, скрыто и прозрачно для программиста, 
выделить место и передать указатель на него в качестве первого аргумента. Это почти то же самое 
что и сделать это вручную, но компилятор прячет это.}
\EN{Now it has become possible to return, let's say, an entire structure, but that is still not very popular. 
If a function has to return a large structure, the \gls{caller} must allocate it and pass a pointer to it via the first argument, transparently for the programmer. 
That is almost the same as to pass a pointer in the first argument manually, but the compiler hides it.}

\RU{Небольшой пример:}\EN{Small example:}

\lstinputlisting{patterns/06_return_results/6_1.c}

\dots \RU{получим}\EN{what we got} (MSVC 2010 \Ox):

\lstinputlisting{patterns/06_return_results/6_1.asm}

\RU{\TT{\$T3853} это имя внутреннего макроса для передачи указателя на структуру.}
\EN{The macro name for internal passing of pointer to a structure here is \TT{\$T3853}.}

\index{\CLanguageElements!C99}
\RU{Этот пример можно даже переписать, используя расширения C99}\EN{This example can be rewritten using
the C99 language extensions}:

\lstinputlisting{patterns/06_return_results/6_1_C99.c}

\lstinputlisting[caption=GCC 4.8.1]{patterns/06_return_results/6_1_C99.asm}

\RU{Как видно, функция просто заполняет поля в структуре, выделенной вызывающей функцией. 
Как если бы передавался просто указатель на структуру.
Так что никаких проблем с эффективностью нет.}
\EN{As we see, the function is just filling the structure's fields allocated by
the caller function,
as if a pointer to the structure was passed.
So there are no performance drawbacks.}
\fi

\ifx\LITE\undefined
\clearpage
\section{\RU{Простейшее четырехбайтное XOR-шифрование}\EN{Simplest possible 4-byte XOR encryption}}

\RU{Если при XOR-шифровании применялся шаблон длинее байта, например, 4-байтный, то его также легко
увидеть.}
\EN{If longer pattern was used while XOR-encryption, for example, 4 byte pattern, it's easy
to spot it as well.}
\RU{Например, вот начало файла kernel32.dll (32-битная версия из Windows Server 2008):}
\EN{As example, here is beginning of kernel32.dll file (32-bit version from Windows Server 2008):}

\begin{figure}[H]
\centering
\includegraphics[scale=\FigScale]{ff/XOR/4byte/original1.png}
\caption{\EN{Original file}\RU{Оригинальный файл}}
\end{figure}

\clearpage
\RU{Вот он же, но \q{зашифрованный} 4-байтным ключем:}
\EN{Here is it \q{encrypted} by 4-byte key:}

\begin{figure}[H]
\centering
\includegraphics[scale=\FigScale]{ff/XOR/4byte/encrypted1.png}
\caption{\EN{\q{Encrypted} file}\RU{\q{Зашифрованный} файл}}
\end{figure}

\RU{Очень легко увидеть повторяющиеся 4 символа.}
\EN{It's very easy to spot recurring 4 symbols.}
\RU{Ведь в заголовке PE-файла много длинных нулевых областей, из-за которых ключ становится видным.}
\EN{Indeed, PE-file header has a lot of long zero lacunes, which is the reason why key became visible.}

\clearpage
\RU{Вот начало PE-заголовка в 16-ричном виде:}
\EN{Here is beginning of PE-header in hexadecimal form:}

\begin{figure}[H]
\centering
\includegraphics[scale=\FigScale]{ff/XOR/4byte/original2.png}
\caption{PE-\EN{header}\RU{заголовок}}
\end{figure}

\clearpage
\RU{И вот он же, \q{зашифрованный}:}
\EN{Here is it \q{encrypted}:}

\begin{figure}[H]
\centering
\includegraphics[scale=\FigScale]{ff/XOR/4byte/encrypted2.png}
\caption{\EN{\q{Encrypted} PE-header}\RU{\q{Зашифрованный} PE-заголовок}}
\end{figure}

\RU{Легко увидеть визуально, что ключ это следующие 4 байта}
\EN{It's easy to spot that key is the following 4 bytes}: \TT{8C 61 D2 63}.
\RU{Используя эту информацию, довольно легко расшифровать весь файл.}
\EN{It's easy to decrypt the whole file using this information.}

\RU{Таким образом, важно помнить эти свойства PE-файлов:
1) в PE-заголовке много нулевых областей;
2) все PE-секции дополняются нулями до границы страницы (4096 байт), 
так что после всех секций обычно имеются длинные нулевые области.}
\EN{So this is important to remember these property of PE-files:
1) PE-header has many zero lacunas;
2) all PE-sections padded with zeroes by page border (4096 bytes),
so long zero lacunas usually present after all sections.}

\RU{Некоторые другие форматы файлов могут также иметь длинные нулевые области.}
\EN{Some other file formats may contain long zero lacunas.}
\RU{Это очень типично для файлов, используемых научным и инженерным ПО.}
\EN{It's very typical for files used by scientific and engineering software.}

\RU{Для тех, кто самостоятельно хочет изучить эти файлы, то их можно скачать здесь:}
\EN{For those who wants to inspect these files on one's own, they are downloadable there:}
\url{http://go.yurichev.com/17352}.

\subsection{\Exercise}

\begin{itemize}
	\item \url{http://challenges.re/50}
\end{itemize}


\fi
\chapter{\RU{Оператор GOTO}\EN{GOTO operator}}

\RU{Оператор GOTO считается анти-паттерном}\EN{The GOTO operator is generally considered as anti-pattern.} 
\cite{Dijkstra:1968:LEG:362929.362947}, 
\RU{но тем не менее, его можно использовать в разумных пределах}
\EN{Nevertheless, it can be used reasonably} \cite{Knuth:1974:SPG:356635.356640}, \cite[1.3.2]{CBook}.

\RU{Вот простейший пример}\EN{Here is a very simple example}:

\lstinputlisting{patterns/065_GOTO/goto.c}

\RU{Вот что мы получаем в}\EN{Here is what we have got in} MSVC 2012:

\lstinputlisting[caption=MSVC 2012]{patterns/065_GOTO/MSVC_goto.asm}

\RU{Выражение \IT{goto} заменяется инструкцией \JMP, которая работает точно также:
безусловный переход в другое место.}
\EN{The \IT{goto} statement has been simply replaced by a \JMP instruction, which has the same
effect: unconditional jump to another place.}

\RU{Вызов второго \printf может исполнится только при помощи человеческого вмешательства,
используя отладчик или модифицирование кода.}
\EN{The second \printf could be executed only with human intervention, 
by using a debugger or by patching the code.}
\PTBRph{}\ESph{}\PLph{}\ITAph{}\\
\\
\ifdefined\IncludeHiew
\clearpage
\RU{Это также может быть простым упражнением на модификацию кода.}
\EN{This could also be useful as a simple patching exercise.}
\RU{Откроем исполняемый файл в}\EN{Let's open the resulting executable in} Hiew:

\begin{figure}[H]
\centering
\includegraphics[scale=\FigScale]{patterns/065_GOTO/hiew1.png}
\caption{Hiew}
\label{fig:goto_hiew1}
\end{figure}

\clearpage
\RU{Поместите курсор по адресу}\EN{Place the cursor to address} \JMP (\TT{0x410}), 
\RU{нажмите}\EN{press} F3 (\RU{редактирование}\EN{edit}), \RU{нажмите два нуля, так что
опкод становится}\EN{press zero twice, so the opcode becomes} \TT{EB 00}:

\begin{figure}[H]
\centering
\includegraphics[scale=\FigScale]{patterns/065_GOTO/hiew2.png}
\caption{Hiew}
\label{fig:goto_hiew2}
\end{figure}

\RU{Второй байт опкода \JMP это относительное смещение от перехода. 0 означает место
прямо после текущей инструкции.}
\EN{The second byte of the \JMP opcode denotes the relative offset for the jump, 0 means the point
right after the current instruction.}
\RU{Теперь \JMP не будет пропускать следующий вызов \printf.}
\EN{So now \JMP not skipping the second \printf call.}

\RU{Нажмите F9 (запись) и выйдите.}
\EN{Press F9 (save) and exit.}
\RU{Теперь мы запускаем исполняемый файл и видим это}\EN{Now if we run the executable we should see 
this}:

\begin{figure}[H]
\centering
\includegraphics[scale=\NormalScale]{patterns/065_GOTO/result.png}
\caption{\RU{Результат}\EN{Patched executable output}}
\label{fig:goto_result}
\end{figure}

\RU{Подобного же эффекта можно достичь, если заменить инструкцию \JMP на две инструкции \NOP.}
\EN{The same result could be achieved by replacing the \JMP instruction with 2 \NOP instructions.}
\RU{\NOP имеет опкод \TT{0x90} и длину в 1 байт, так что нужно 2 инструкции для замены.}
\EN{\NOP has an opcode of \TT{0x90} and length of 1 byte, so we need 2 instructions as \JMP replacement (which is 2 bytes in size).}
\fi

\section{\RU{Мертвый код}\EN{Dead code}}

\RU{Вызов второго \printf также называется \q{мертвым кодом} (\q{dead code}) 
в терминах компиляторов.}
\EN{The second \printf call is also called \q{dead code} in compiler terms.}
\RU{Это значит, что он никогда не будет исполнен.}
\EN{This means that the code will never be executed.}
\EN{So when you compile this example with optimizations, the compiler removes \q{dead code}, leaving
no trace of it:}
\RU{Так что если вы компилируете этот пример с оптимизацией, компилятор удаляет \q{мертвый
код} не оставляя следа:}

\lstinputlisting[caption=\Optimizing MSVC 2012]{patterns/065_GOTO/MSVC_goto_Ox.asm}

\RU{Впрочем, строку}\EN{However, the compiler forgot to remove the} \q{skip me!} \RU{компилятор 
убрать забыл}\EN{string}.

%Note: cl "/Ox" option for maximum optimisation does get rid of "skip me" string as well

\ifdefined\IncludeExercises
\section{\Exercise}

% TODO debugger example can fit here
\RU{Попробуйте добиться того же самого в вашем любимом компиляторе и отладчике.}
\EN{Try to achieve the same result using your favorite compiler and debugger.}
\fi

\clearpage
\section{\RU{Простейшее четырехбайтное XOR-шифрование}\EN{Simplest possible 4-byte XOR encryption}}

\RU{Если при XOR-шифровании применялся шаблон длинее байта, например, 4-байтный, то его также легко
увидеть.}
\EN{If longer pattern was used while XOR-encryption, for example, 4 byte pattern, it's easy
to spot it as well.}
\RU{Например, вот начало файла kernel32.dll (32-битная версия из Windows Server 2008):}
\EN{As example, here is beginning of kernel32.dll file (32-bit version from Windows Server 2008):}

\begin{figure}[H]
\centering
\includegraphics[scale=\FigScale]{ff/XOR/4byte/original1.png}
\caption{\EN{Original file}\RU{Оригинальный файл}}
\end{figure}

\clearpage
\RU{Вот он же, но \q{зашифрованный} 4-байтным ключем:}
\EN{Here is it \q{encrypted} by 4-byte key:}

\begin{figure}[H]
\centering
\includegraphics[scale=\FigScale]{ff/XOR/4byte/encrypted1.png}
\caption{\EN{\q{Encrypted} file}\RU{\q{Зашифрованный} файл}}
\end{figure}

\RU{Очень легко увидеть повторяющиеся 4 символа.}
\EN{It's very easy to spot recurring 4 symbols.}
\RU{Ведь в заголовке PE-файла много длинных нулевых областей, из-за которых ключ становится видным.}
\EN{Indeed, PE-file header has a lot of long zero lacunes, which is the reason why key became visible.}

\clearpage
\RU{Вот начало PE-заголовка в 16-ричном виде:}
\EN{Here is beginning of PE-header in hexadecimal form:}

\begin{figure}[H]
\centering
\includegraphics[scale=\FigScale]{ff/XOR/4byte/original2.png}
\caption{PE-\EN{header}\RU{заголовок}}
\end{figure}

\clearpage
\RU{И вот он же, \q{зашифрованный}:}
\EN{Here is it \q{encrypted}:}

\begin{figure}[H]
\centering
\includegraphics[scale=\FigScale]{ff/XOR/4byte/encrypted2.png}
\caption{\EN{\q{Encrypted} PE-header}\RU{\q{Зашифрованный} PE-заголовок}}
\end{figure}

\RU{Легко увидеть визуально, что ключ это следующие 4 байта}
\EN{It's easy to spot that key is the following 4 bytes}: \TT{8C 61 D2 63}.
\RU{Используя эту информацию, довольно легко расшифровать весь файл.}
\EN{It's easy to decrypt the whole file using this information.}

\RU{Таким образом, важно помнить эти свойства PE-файлов:
1) в PE-заголовке много нулевых областей;
2) все PE-секции дополняются нулями до границы страницы (4096 байт), 
так что после всех секций обычно имеются длинные нулевые области.}
\EN{So this is important to remember these property of PE-files:
1) PE-header has many zero lacunas;
2) all PE-sections padded with zeroes by page border (4096 bytes),
so long zero lacunas usually present after all sections.}

\RU{Некоторые другие форматы файлов могут также иметь длинные нулевые области.}
\EN{Some other file formats may contain long zero lacunas.}
\RU{Это очень типично для файлов, используемых научным и инженерным ПО.}
\EN{It's very typical for files used by scientific and engineering software.}

\RU{Для тех, кто самостоятельно хочет изучить эти файлы, то их можно скачать здесь:}
\EN{For those who wants to inspect these files on one's own, they are downloadable there:}
\url{http://go.yurichev.com/17352}.

\subsection{\Exercise}

\begin{itemize}
	\item \url{http://challenges.re/50}
\end{itemize}


\chapter{\SwitchCaseDefaultSectionName}
\index{\CLanguageElements!switch}

% sections
\clearpage
\section{\RU{Простейшее четырехбайтное XOR-шифрование}\EN{Simplest possible 4-byte XOR encryption}}

\RU{Если при XOR-шифровании применялся шаблон длинее байта, например, 4-байтный, то его также легко
увидеть.}
\EN{If longer pattern was used while XOR-encryption, for example, 4 byte pattern, it's easy
to spot it as well.}
\RU{Например, вот начало файла kernel32.dll (32-битная версия из Windows Server 2008):}
\EN{As example, here is beginning of kernel32.dll file (32-bit version from Windows Server 2008):}

\begin{figure}[H]
\centering
\includegraphics[scale=\FigScale]{ff/XOR/4byte/original1.png}
\caption{\EN{Original file}\RU{Оригинальный файл}}
\end{figure}

\clearpage
\RU{Вот он же, но \q{зашифрованный} 4-байтным ключем:}
\EN{Here is it \q{encrypted} by 4-byte key:}

\begin{figure}[H]
\centering
\includegraphics[scale=\FigScale]{ff/XOR/4byte/encrypted1.png}
\caption{\EN{\q{Encrypted} file}\RU{\q{Зашифрованный} файл}}
\end{figure}

\RU{Очень легко увидеть повторяющиеся 4 символа.}
\EN{It's very easy to spot recurring 4 symbols.}
\RU{Ведь в заголовке PE-файла много длинных нулевых областей, из-за которых ключ становится видным.}
\EN{Indeed, PE-file header has a lot of long zero lacunes, which is the reason why key became visible.}

\clearpage
\RU{Вот начало PE-заголовка в 16-ричном виде:}
\EN{Here is beginning of PE-header in hexadecimal form:}

\begin{figure}[H]
\centering
\includegraphics[scale=\FigScale]{ff/XOR/4byte/original2.png}
\caption{PE-\EN{header}\RU{заголовок}}
\end{figure}

\clearpage
\RU{И вот он же, \q{зашифрованный}:}
\EN{Here is it \q{encrypted}:}

\begin{figure}[H]
\centering
\includegraphics[scale=\FigScale]{ff/XOR/4byte/encrypted2.png}
\caption{\EN{\q{Encrypted} PE-header}\RU{\q{Зашифрованный} PE-заголовок}}
\end{figure}

\RU{Легко увидеть визуально, что ключ это следующие 4 байта}
\EN{It's easy to spot that key is the following 4 bytes}: \TT{8C 61 D2 63}.
\RU{Используя эту информацию, довольно легко расшифровать весь файл.}
\EN{It's easy to decrypt the whole file using this information.}

\RU{Таким образом, важно помнить эти свойства PE-файлов:
1) в PE-заголовке много нулевых областей;
2) все PE-секции дополняются нулями до границы страницы (4096 байт), 
так что после всех секций обычно имеются длинные нулевые области.}
\EN{So this is important to remember these property of PE-files:
1) PE-header has many zero lacunas;
2) all PE-sections padded with zeroes by page border (4096 bytes),
so long zero lacunas usually present after all sections.}

\RU{Некоторые другие форматы файлов могут также иметь длинные нулевые области.}
\EN{Some other file formats may contain long zero lacunas.}
\RU{Это очень типично для файлов, используемых научным и инженерным ПО.}
\EN{It's very typical for files used by scientific and engineering software.}

\RU{Для тех, кто самостоятельно хочет изучить эти файлы, то их можно скачать здесь:}
\EN{For those who wants to inspect these files on one's own, they are downloadable there:}
\url{http://go.yurichev.com/17352}.

\subsection{\Exercise}

\begin{itemize}
	\item \url{http://challenges.re/50}
\end{itemize}


\clearpage
\section{\RU{Простейшее четырехбайтное XOR-шифрование}\EN{Simplest possible 4-byte XOR encryption}}

\RU{Если при XOR-шифровании применялся шаблон длинее байта, например, 4-байтный, то его также легко
увидеть.}
\EN{If longer pattern was used while XOR-encryption, for example, 4 byte pattern, it's easy
to spot it as well.}
\RU{Например, вот начало файла kernel32.dll (32-битная версия из Windows Server 2008):}
\EN{As example, here is beginning of kernel32.dll file (32-bit version from Windows Server 2008):}

\begin{figure}[H]
\centering
\includegraphics[scale=\FigScale]{ff/XOR/4byte/original1.png}
\caption{\EN{Original file}\RU{Оригинальный файл}}
\end{figure}

\clearpage
\RU{Вот он же, но \q{зашифрованный} 4-байтным ключем:}
\EN{Here is it \q{encrypted} by 4-byte key:}

\begin{figure}[H]
\centering
\includegraphics[scale=\FigScale]{ff/XOR/4byte/encrypted1.png}
\caption{\EN{\q{Encrypted} file}\RU{\q{Зашифрованный} файл}}
\end{figure}

\RU{Очень легко увидеть повторяющиеся 4 символа.}
\EN{It's very easy to spot recurring 4 symbols.}
\RU{Ведь в заголовке PE-файла много длинных нулевых областей, из-за которых ключ становится видным.}
\EN{Indeed, PE-file header has a lot of long zero lacunes, which is the reason why key became visible.}

\clearpage
\RU{Вот начало PE-заголовка в 16-ричном виде:}
\EN{Here is beginning of PE-header in hexadecimal form:}

\begin{figure}[H]
\centering
\includegraphics[scale=\FigScale]{ff/XOR/4byte/original2.png}
\caption{PE-\EN{header}\RU{заголовок}}
\end{figure}

\clearpage
\RU{И вот он же, \q{зашифрованный}:}
\EN{Here is it \q{encrypted}:}

\begin{figure}[H]
\centering
\includegraphics[scale=\FigScale]{ff/XOR/4byte/encrypted2.png}
\caption{\EN{\q{Encrypted} PE-header}\RU{\q{Зашифрованный} PE-заголовок}}
\end{figure}

\RU{Легко увидеть визуально, что ключ это следующие 4 байта}
\EN{It's easy to spot that key is the following 4 bytes}: \TT{8C 61 D2 63}.
\RU{Используя эту информацию, довольно легко расшифровать весь файл.}
\EN{It's easy to decrypt the whole file using this information.}

\RU{Таким образом, важно помнить эти свойства PE-файлов:
1) в PE-заголовке много нулевых областей;
2) все PE-секции дополняются нулями до границы страницы (4096 байт), 
так что после всех секций обычно имеются длинные нулевые области.}
\EN{So this is important to remember these property of PE-files:
1) PE-header has many zero lacunas;
2) all PE-sections padded with zeroes by page border (4096 bytes),
so long zero lacunas usually present after all sections.}

\RU{Некоторые другие форматы файлов могут также иметь длинные нулевые области.}
\EN{Some other file formats may contain long zero lacunas.}
\RU{Это очень типично для файлов, используемых научным и инженерным ПО.}
\EN{It's very typical for files used by scientific and engineering software.}

\RU{Для тех, кто самостоятельно хочет изучить эти файлы, то их можно скачать здесь:}
\EN{For those who wants to inspect these files on one's own, they are downloadable there:}
\url{http://go.yurichev.com/17352}.

\subsection{\Exercise}

\begin{itemize}
	\item \url{http://challenges.re/50}
\end{itemize}


% TODO What's the difference between 3 and 4? Seems to be the same...
% it is fallthrough from 3 to 4 :) --DY
\clearpage
\section{\RU{Простейшее четырехбайтное XOR-шифрование}\EN{Simplest possible 4-byte XOR encryption}}

\RU{Если при XOR-шифровании применялся шаблон длинее байта, например, 4-байтный, то его также легко
увидеть.}
\EN{If longer pattern was used while XOR-encryption, for example, 4 byte pattern, it's easy
to spot it as well.}
\RU{Например, вот начало файла kernel32.dll (32-битная версия из Windows Server 2008):}
\EN{As example, here is beginning of kernel32.dll file (32-bit version from Windows Server 2008):}

\begin{figure}[H]
\centering
\includegraphics[scale=\FigScale]{ff/XOR/4byte/original1.png}
\caption{\EN{Original file}\RU{Оригинальный файл}}
\end{figure}

\clearpage
\RU{Вот он же, но \q{зашифрованный} 4-байтным ключем:}
\EN{Here is it \q{encrypted} by 4-byte key:}

\begin{figure}[H]
\centering
\includegraphics[scale=\FigScale]{ff/XOR/4byte/encrypted1.png}
\caption{\EN{\q{Encrypted} file}\RU{\q{Зашифрованный} файл}}
\end{figure}

\RU{Очень легко увидеть повторяющиеся 4 символа.}
\EN{It's very easy to spot recurring 4 symbols.}
\RU{Ведь в заголовке PE-файла много длинных нулевых областей, из-за которых ключ становится видным.}
\EN{Indeed, PE-file header has a lot of long zero lacunes, which is the reason why key became visible.}

\clearpage
\RU{Вот начало PE-заголовка в 16-ричном виде:}
\EN{Here is beginning of PE-header in hexadecimal form:}

\begin{figure}[H]
\centering
\includegraphics[scale=\FigScale]{ff/XOR/4byte/original2.png}
\caption{PE-\EN{header}\RU{заголовок}}
\end{figure}

\clearpage
\RU{И вот он же, \q{зашифрованный}:}
\EN{Here is it \q{encrypted}:}

\begin{figure}[H]
\centering
\includegraphics[scale=\FigScale]{ff/XOR/4byte/encrypted2.png}
\caption{\EN{\q{Encrypted} PE-header}\RU{\q{Зашифрованный} PE-заголовок}}
\end{figure}

\RU{Легко увидеть визуально, что ключ это следующие 4 байта}
\EN{It's easy to spot that key is the following 4 bytes}: \TT{8C 61 D2 63}.
\RU{Используя эту информацию, довольно легко расшифровать весь файл.}
\EN{It's easy to decrypt the whole file using this information.}

\RU{Таким образом, важно помнить эти свойства PE-файлов:
1) в PE-заголовке много нулевых областей;
2) все PE-секции дополняются нулями до границы страницы (4096 байт), 
так что после всех секций обычно имеются длинные нулевые области.}
\EN{So this is important to remember these property of PE-files:
1) PE-header has many zero lacunas;
2) all PE-sections padded with zeroes by page border (4096 bytes),
so long zero lacunas usually present after all sections.}

\RU{Некоторые другие форматы файлов могут также иметь длинные нулевые области.}
\EN{Some other file formats may contain long zero lacunas.}
\RU{Это очень типично для файлов, используемых научным и инженерным ПО.}
\EN{It's very typical for files used by scientific and engineering software.}

\RU{Для тех, кто самостоятельно хочет изучить эти файлы, то их можно скачать здесь:}
\EN{For those who wants to inspect these files on one's own, they are downloadable there:}
\url{http://go.yurichev.com/17352}.

\subsection{\Exercise}

\begin{itemize}
	\item \url{http://challenges.re/50}
\end{itemize}


\clearpage
\section{\RU{Простейшее четырехбайтное XOR-шифрование}\EN{Simplest possible 4-byte XOR encryption}}

\RU{Если при XOR-шифровании применялся шаблон длинее байта, например, 4-байтный, то его также легко
увидеть.}
\EN{If longer pattern was used while XOR-encryption, for example, 4 byte pattern, it's easy
to spot it as well.}
\RU{Например, вот начало файла kernel32.dll (32-битная версия из Windows Server 2008):}
\EN{As example, here is beginning of kernel32.dll file (32-bit version from Windows Server 2008):}

\begin{figure}[H]
\centering
\includegraphics[scale=\FigScale]{ff/XOR/4byte/original1.png}
\caption{\EN{Original file}\RU{Оригинальный файл}}
\end{figure}

\clearpage
\RU{Вот он же, но \q{зашифрованный} 4-байтным ключем:}
\EN{Here is it \q{encrypted} by 4-byte key:}

\begin{figure}[H]
\centering
\includegraphics[scale=\FigScale]{ff/XOR/4byte/encrypted1.png}
\caption{\EN{\q{Encrypted} file}\RU{\q{Зашифрованный} файл}}
\end{figure}

\RU{Очень легко увидеть повторяющиеся 4 символа.}
\EN{It's very easy to spot recurring 4 symbols.}
\RU{Ведь в заголовке PE-файла много длинных нулевых областей, из-за которых ключ становится видным.}
\EN{Indeed, PE-file header has a lot of long zero lacunes, which is the reason why key became visible.}

\clearpage
\RU{Вот начало PE-заголовка в 16-ричном виде:}
\EN{Here is beginning of PE-header in hexadecimal form:}

\begin{figure}[H]
\centering
\includegraphics[scale=\FigScale]{ff/XOR/4byte/original2.png}
\caption{PE-\EN{header}\RU{заголовок}}
\end{figure}

\clearpage
\RU{И вот он же, \q{зашифрованный}:}
\EN{Here is it \q{encrypted}:}

\begin{figure}[H]
\centering
\includegraphics[scale=\FigScale]{ff/XOR/4byte/encrypted2.png}
\caption{\EN{\q{Encrypted} PE-header}\RU{\q{Зашифрованный} PE-заголовок}}
\end{figure}

\RU{Легко увидеть визуально, что ключ это следующие 4 байта}
\EN{It's easy to spot that key is the following 4 bytes}: \TT{8C 61 D2 63}.
\RU{Используя эту информацию, довольно легко расшифровать весь файл.}
\EN{It's easy to decrypt the whole file using this information.}

\RU{Таким образом, важно помнить эти свойства PE-файлов:
1) в PE-заголовке много нулевых областей;
2) все PE-секции дополняются нулями до границы страницы (4096 байт), 
так что после всех секций обычно имеются длинные нулевые области.}
\EN{So this is important to remember these property of PE-files:
1) PE-header has many zero lacunas;
2) all PE-sections padded with zeroes by page border (4096 bytes),
so long zero lacunas usually present after all sections.}

\RU{Некоторые другие форматы файлов могут также иметь длинные нулевые области.}
\EN{Some other file formats may contain long zero lacunas.}
\RU{Это очень типично для файлов, используемых научным и инженерным ПО.}
\EN{It's very typical for files used by scientific and engineering software.}

\RU{Для тех, кто самостоятельно хочет изучить эти файлы, то их можно скачать здесь:}
\EN{For those who wants to inspect these files on one's own, they are downloadable there:}
\url{http://go.yurichev.com/17352}.

\subsection{\Exercise}

\begin{itemize}
	\item \url{http://challenges.re/50}
\end{itemize}



\ifdefined\IncludeExercises
\section{\Exercises}

\subsection{\Exercise \#1}
\label{exercise_switch_1}

\RU{Вполне возможно переделать пример на Си в листинге \myref{switch_lot_c} так, чтобы при компиляции
получалось даже ещё меньше кода, но работать всё будет точно так же.}
\EN{It's possible to rework the C example in \myref{switch_lot_c} in such way that the compiler
can produce even smaller code, but will work just the same.}
\RU{Попробуйте этого добиться}\EN{Try to achieve it}.

% \RU{Подсказка}\EN{Hint}: \printf \EN{may be called only from a single place}\RU{вполне может 
% вызываться только из одного места}.
\fi

\clearpage
\section{\RU{Простейшее четырехбайтное XOR-шифрование}\EN{Simplest possible 4-byte XOR encryption}}

\RU{Если при XOR-шифровании применялся шаблон длинее байта, например, 4-байтный, то его также легко
увидеть.}
\EN{If longer pattern was used while XOR-encryption, for example, 4 byte pattern, it's easy
to spot it as well.}
\RU{Например, вот начало файла kernel32.dll (32-битная версия из Windows Server 2008):}
\EN{As example, here is beginning of kernel32.dll file (32-bit version from Windows Server 2008):}

\begin{figure}[H]
\centering
\includegraphics[scale=\FigScale]{ff/XOR/4byte/original1.png}
\caption{\EN{Original file}\RU{Оригинальный файл}}
\end{figure}

\clearpage
\RU{Вот он же, но \q{зашифрованный} 4-байтным ключем:}
\EN{Here is it \q{encrypted} by 4-byte key:}

\begin{figure}[H]
\centering
\includegraphics[scale=\FigScale]{ff/XOR/4byte/encrypted1.png}
\caption{\EN{\q{Encrypted} file}\RU{\q{Зашифрованный} файл}}
\end{figure}

\RU{Очень легко увидеть повторяющиеся 4 символа.}
\EN{It's very easy to spot recurring 4 symbols.}
\RU{Ведь в заголовке PE-файла много длинных нулевых областей, из-за которых ключ становится видным.}
\EN{Indeed, PE-file header has a lot of long zero lacunes, which is the reason why key became visible.}

\clearpage
\RU{Вот начало PE-заголовка в 16-ричном виде:}
\EN{Here is beginning of PE-header in hexadecimal form:}

\begin{figure}[H]
\centering
\includegraphics[scale=\FigScale]{ff/XOR/4byte/original2.png}
\caption{PE-\EN{header}\RU{заголовок}}
\end{figure}

\clearpage
\RU{И вот он же, \q{зашифрованный}:}
\EN{Here is it \q{encrypted}:}

\begin{figure}[H]
\centering
\includegraphics[scale=\FigScale]{ff/XOR/4byte/encrypted2.png}
\caption{\EN{\q{Encrypted} PE-header}\RU{\q{Зашифрованный} PE-заголовок}}
\end{figure}

\RU{Легко увидеть визуально, что ключ это следующие 4 байта}
\EN{It's easy to spot that key is the following 4 bytes}: \TT{8C 61 D2 63}.
\RU{Используя эту информацию, довольно легко расшифровать весь файл.}
\EN{It's easy to decrypt the whole file using this information.}

\RU{Таким образом, важно помнить эти свойства PE-файлов:
1) в PE-заголовке много нулевых областей;
2) все PE-секции дополняются нулями до границы страницы (4096 байт), 
так что после всех секций обычно имеются длинные нулевые области.}
\EN{So this is important to remember these property of PE-files:
1) PE-header has many zero lacunas;
2) all PE-sections padded with zeroes by page border (4096 bytes),
so long zero lacunas usually present after all sections.}

\RU{Некоторые другие форматы файлов могут также иметь длинные нулевые области.}
\EN{Some other file formats may contain long zero lacunas.}
\RU{Это очень типично для файлов, используемых научным и инженерным ПО.}
\EN{It's very typical for files used by scientific and engineering software.}

\RU{Для тех, кто самостоятельно хочет изучить эти файлы, то их можно скачать здесь:}
\EN{For those who wants to inspect these files on one's own, they are downloadable there:}
\url{http://go.yurichev.com/17352}.

\subsection{\Exercise}

\begin{itemize}
	\item \url{http://challenges.re/50}
\end{itemize}


\chapter{\SimpleStringsProcessings}
\index{\CStandardLibrary!strlen()}
\index{\CLanguageElements!while}

% sections
\clearpage
\section{\RU{Простейшее четырехбайтное XOR-шифрование}\EN{Simplest possible 4-byte XOR encryption}}

\RU{Если при XOR-шифровании применялся шаблон длинее байта, например, 4-байтный, то его также легко
увидеть.}
\EN{If longer pattern was used while XOR-encryption, for example, 4 byte pattern, it's easy
to spot it as well.}
\RU{Например, вот начало файла kernel32.dll (32-битная версия из Windows Server 2008):}
\EN{As example, here is beginning of kernel32.dll file (32-bit version from Windows Server 2008):}

\begin{figure}[H]
\centering
\includegraphics[scale=\FigScale]{ff/XOR/4byte/original1.png}
\caption{\EN{Original file}\RU{Оригинальный файл}}
\end{figure}

\clearpage
\RU{Вот он же, но \q{зашифрованный} 4-байтным ключем:}
\EN{Here is it \q{encrypted} by 4-byte key:}

\begin{figure}[H]
\centering
\includegraphics[scale=\FigScale]{ff/XOR/4byte/encrypted1.png}
\caption{\EN{\q{Encrypted} file}\RU{\q{Зашифрованный} файл}}
\end{figure}

\RU{Очень легко увидеть повторяющиеся 4 символа.}
\EN{It's very easy to spot recurring 4 symbols.}
\RU{Ведь в заголовке PE-файла много длинных нулевых областей, из-за которых ключ становится видным.}
\EN{Indeed, PE-file header has a lot of long zero lacunes, which is the reason why key became visible.}

\clearpage
\RU{Вот начало PE-заголовка в 16-ричном виде:}
\EN{Here is beginning of PE-header in hexadecimal form:}

\begin{figure}[H]
\centering
\includegraphics[scale=\FigScale]{ff/XOR/4byte/original2.png}
\caption{PE-\EN{header}\RU{заголовок}}
\end{figure}

\clearpage
\RU{И вот он же, \q{зашифрованный}:}
\EN{Here is it \q{encrypted}:}

\begin{figure}[H]
\centering
\includegraphics[scale=\FigScale]{ff/XOR/4byte/encrypted2.png}
\caption{\EN{\q{Encrypted} PE-header}\RU{\q{Зашифрованный} PE-заголовок}}
\end{figure}

\RU{Легко увидеть визуально, что ключ это следующие 4 байта}
\EN{It's easy to spot that key is the following 4 bytes}: \TT{8C 61 D2 63}.
\RU{Используя эту информацию, довольно легко расшифровать весь файл.}
\EN{It's easy to decrypt the whole file using this information.}

\RU{Таким образом, важно помнить эти свойства PE-файлов:
1) в PE-заголовке много нулевых областей;
2) все PE-секции дополняются нулями до границы страницы (4096 байт), 
так что после всех секций обычно имеются длинные нулевые области.}
\EN{So this is important to remember these property of PE-files:
1) PE-header has many zero lacunas;
2) all PE-sections padded with zeroes by page border (4096 bytes),
so long zero lacunas usually present after all sections.}

\RU{Некоторые другие форматы файлов могут также иметь длинные нулевые области.}
\EN{Some other file formats may contain long zero lacunas.}
\RU{Это очень типично для файлов, используемых научным и инженерным ПО.}
\EN{It's very typical for files used by scientific and engineering software.}

\RU{Для тех, кто самостоятельно хочет изучить эти файлы, то их можно скачать здесь:}
\EN{For those who wants to inspect these files on one's own, they are downloadable there:}
\url{http://go.yurichev.com/17352}.

\subsection{\Exercise}

\begin{itemize}
	\item \url{http://challenges.re/50}
\end{itemize}



\clearpage
\section{\RU{Простейшее четырехбайтное XOR-шифрование}\EN{Simplest possible 4-byte XOR encryption}}

\RU{Если при XOR-шифровании применялся шаблон длинее байта, например, 4-байтный, то его также легко
увидеть.}
\EN{If longer pattern was used while XOR-encryption, for example, 4 byte pattern, it's easy
to spot it as well.}
\RU{Например, вот начало файла kernel32.dll (32-битная версия из Windows Server 2008):}
\EN{As example, here is beginning of kernel32.dll file (32-bit version from Windows Server 2008):}

\begin{figure}[H]
\centering
\includegraphics[scale=\FigScale]{ff/XOR/4byte/original1.png}
\caption{\EN{Original file}\RU{Оригинальный файл}}
\end{figure}

\clearpage
\RU{Вот он же, но \q{зашифрованный} 4-байтным ключем:}
\EN{Here is it \q{encrypted} by 4-byte key:}

\begin{figure}[H]
\centering
\includegraphics[scale=\FigScale]{ff/XOR/4byte/encrypted1.png}
\caption{\EN{\q{Encrypted} file}\RU{\q{Зашифрованный} файл}}
\end{figure}

\RU{Очень легко увидеть повторяющиеся 4 символа.}
\EN{It's very easy to spot recurring 4 symbols.}
\RU{Ведь в заголовке PE-файла много длинных нулевых областей, из-за которых ключ становится видным.}
\EN{Indeed, PE-file header has a lot of long zero lacunes, which is the reason why key became visible.}

\clearpage
\RU{Вот начало PE-заголовка в 16-ричном виде:}
\EN{Here is beginning of PE-header in hexadecimal form:}

\begin{figure}[H]
\centering
\includegraphics[scale=\FigScale]{ff/XOR/4byte/original2.png}
\caption{PE-\EN{header}\RU{заголовок}}
\end{figure}

\clearpage
\RU{И вот он же, \q{зашифрованный}:}
\EN{Here is it \q{encrypted}:}

\begin{figure}[H]
\centering
\includegraphics[scale=\FigScale]{ff/XOR/4byte/encrypted2.png}
\caption{\EN{\q{Encrypted} PE-header}\RU{\q{Зашифрованный} PE-заголовок}}
\end{figure}

\RU{Легко увидеть визуально, что ключ это следующие 4 байта}
\EN{It's easy to spot that key is the following 4 bytes}: \TT{8C 61 D2 63}.
\RU{Используя эту информацию, довольно легко расшифровать весь файл.}
\EN{It's easy to decrypt the whole file using this information.}

\RU{Таким образом, важно помнить эти свойства PE-файлов:
1) в PE-заголовке много нулевых областей;
2) все PE-секции дополняются нулями до границы страницы (4096 байт), 
так что после всех секций обычно имеются длинные нулевые области.}
\EN{So this is important to remember these property of PE-files:
1) PE-header has many zero lacunas;
2) all PE-sections padded with zeroes by page border (4096 bytes),
so long zero lacunas usually present after all sections.}

\RU{Некоторые другие форматы файлов могут также иметь длинные нулевые области.}
\EN{Some other file formats may contain long zero lacunas.}
\RU{Это очень типично для файлов, используемых научным и инженерным ПО.}
\EN{It's very typical for files used by scientific and engineering software.}

\RU{Для тех, кто самостоятельно хочет изучить эти файлы, то их можно скачать здесь:}
\EN{For those who wants to inspect these files on one's own, they are downloadable there:}
\url{http://go.yurichev.com/17352}.

\subsection{\Exercise}

\begin{itemize}
	\item \url{http://challenges.re/50}
\end{itemize}


\ifx\LITE\undefined
\chapter{\FPUChapterName}
\label{sec:FPU}

\newcommand{\FNURLSTACK}{\footnote{\href{http://go.yurichev.com/17123}{wikipedia.org/wiki/Stack\_machine}}}
\newcommand{\FNURLFORTH}{\footnote{\href{http://go.yurichev.com/17124}{wikipedia.org/wiki/Forth\_(programming\_language)}}}
\newcommand{\FNURLIEEE}{\footnote{\href{http://go.yurichev.com/17125}{wikipedia.org/wiki/IEEE\_floating\_point}}}
\newcommand{\FNURLSP}{\footnote{\href{http://go.yurichev.com/17126}{wikipedia.org/wiki/Single-precision\_floating-point\_format}}}
\newcommand{\FNURLDP}{\footnote{\href{http://go.yurichev.com/17127}{wikipedia.org/wiki/Double-precision\_floating-point\_format}}}
\newcommand{\FNURLEP}{\footnote{\href{http://go.yurichev.com/17128}{wikipedia.org/wiki/Extended\_precision}}}

\RU{\ac{FPU}\EMDASH блок в процессоре работающий с числами с плавающей запятой.}
\EN{The \ac{FPU} is a device within the main \ac{CPU}, specially designed to deal with floating point numbers.}
\RU{Раньше он назывался \q{сопроцессором} и он стоит немного в стороне от \ac{CPU}.}
\EN{It was called \q{coprocessor} in the past and it stays somewhat aside of the main \ac{CPU}.}

\section{IEEE 754}

\RU{Число с плавающей точкой в формате IEEE 754 состоит из \IT{знака}, \IT{мантиссы}\footnote{\IT{significand} или \IT{fraction} 
в англоязычной литературе} и \IT{экспоненты}.}
\EN{A number in the IEEE 754 format consists of a \IT{sign}, a \IT{significand} (also called \IT{fraction}) and an \IT{exponent}.}

\section{x86}

\RU{Перед изучением \ac{FPU} в x86 полезно ознакомиться с тем как работают стековые машины\FNURLSTACK 
или ознакомиться с основами языка Forth\FNURLFORTH.}
\EN{It is worth looking into stack machines\FNURLSTACK or learning the basics of the Forth language\FNURLFORTH,
before studying the \ac{FPU} in x86.}

\index{Intel!80486}
\index{Intel!FPU}
\RU{Интересен факт, что в свое время (до 80486) сопроцессор был отдельным чипом на материнской плате, 
и вследствие его высокой цены, он не всегда присутствовал. Его можно было докупить и установить отдельно}%
\EN{It is interesting to know that in the past (before the 80486 CPU) the coprocessor was a separate chip 
and it was not always pre-installed on the motherboard. It was possible to buy it separately and install it}%
\footnote{\RU{Например, Джон Кармак использовал в своей игре Doom числа с фиксированной запятой 
(\href{http://go.yurichev.com/17357}{ru.wikipedia.org/wiki/Число\_с\_фиксированной\_запятой}), хранящиеся
в обычных 32-битных \ac{GPR} (16 бит на целую часть и 16 на дробную),
чтобы Doom работал на 32-битных компьютерах без FPU, т.е. 80386 и 80486 SX.}
\EN{For example, John Carmack used fixed-point arithmetic 
(\href{http://go.yurichev.com/17356}{wikipedia.org/wiki/Fixed-point\_arithmetic}) values in his Doom video game, stored in 
32-bit \ac{GPR} registers (16 bit for integral part and another 16 bit for fractional part), so Doom
could work on 32-bit computers without FPU, i.e., 80386 and 80486 SX.}}.
\RU{Начиная с 80486 DX в состав процессора всегда входит FPU.}
\EN{Starting with the 80486 DX CPU, the \ac{FPU} is integrated in the \ac{CPU}.}

\index{x86!\Instructions!FWAIT}
\RU{Этот факт может напоминать такой рудимент как наличие инструкции \TT{FWAIT}, 
которая заставляет
\ac{CPU} ожидать, пока \ac{FPU} закончит работу}\EN{The \TT{FWAIT} instruction reminds us of that fact---it
switches the \ac{CPU} to a waiting state, so it can wait until the \ac{FPU} is done with its work}.
\RU{Другой рудимент это тот факт, что опкоды \ac{FPU}-инструкций начинаются с т.н. \q{escape}-опкодов 
(\TT{D8..DF}) как опкоды, передающиеся в отдельный сопроцессор.}
\EN{Another rudiment is the fact that the \ac{FPU} instruction 
opcodes start with the so called \q{escape}-opcodes (\TT{D8..DF}), i.e., 
opcodes passed to a separate coprocessor.}

\index{IEEE 754}
\label{FPU_is_stack}
\RU{FPU имеет стек из восьми 80-битных регистров:}
\EN{The FPU has a stack capable to holding 8 80-bit registers, and each register can hold a number 
in the IEEE 754\FNURLIEEE format.}
\RU{\ST{0}..\ST{7}. Для краткости, IDA и \olly отображают \ST{0} как \TT{ST},
что в некоторых учебниках и документациях означает \q{Stack Top} (\q{вершина стека}).}
\RU{Каждый регистр может содержать число в формате IEEE 754\FNURLIEEE.}
\EN{They are \ST{0}..\ST{7}. For brevity, IDA and \olly show \ST{0} as \TT{ST}, 
which is represented in some textbooks and manuals as \q{Stack Top}.}

\section{ARM, MIPS, x86/x64 SIMD}

\RU{В ARM и MIPS FPU это не стек, а просто набор регистров.}
\EN{In ARM and MIPS the FPU is not a stack, but a set of registers.}
\RU{Такая же идеология применяется в расширениях SIMD в процессорах x86/x64.}
\EN{The same ideology is used in the SIMD extensions of x86/x64 CPUs.}

\section{\CCpp}

\index{float}
\index{double}
\RU{В стандартных \CCpp имеются два типа для работы с числами с плавающей запятой: 
\Tfloat (\IT{число одинарной точности}\FNURLSP, 32 бита)
\footnote{Формат представления чисел с плавающей точкой одинарной точности затрагивается в разделе 
\IT{\WorkingWithFloatAsWithStructSubSubSectionName}~(\myref{sec:floatasstruct}).}
и \Tdouble (\IT{число двойной точности}\FNURLDP, 64 бита).}
\EN{The standard \CCpp languages offer at least two floating number types, \Tfloat (\IT{single-precision}\FNURLSP, 32 bits)
\footnote{the single precision floating point number format is also addressed in 
the \IT{\WorkingWithFloatAsWithStructSubSubSectionName}~(\myref{sec:floatasstruct}) section}
and \Tdouble (\IT{double-precision}\FNURLDP, 64 bits).}

\index{long double}
\RU{GCC также поддерживает тип \IT{long double} (\IT{extended precision}\FNURLEP, 80 бит), но MSVC~--- нет.}
\EN{GCC also supports the \IT{long double} type (\IT{extended precision}\FNURLEP, 80 bit), which MSVC doesn't.}

\RU{Несмотря на то, что \Tfloat занимает столько же места, сколько и \Tint на 32-битной архитектуре, 
представление чисел, разумеется, совершенно другое.}
\EN{The \Tfloat type requires the same number of bits as the \Tint type in 32-bit environments, 
but the number representation is completely different.}

\clearpage
\section{\RU{Простейшее четырехбайтное XOR-шифрование}\EN{Simplest possible 4-byte XOR encryption}}

\RU{Если при XOR-шифровании применялся шаблон длинее байта, например, 4-байтный, то его также легко
увидеть.}
\EN{If longer pattern was used while XOR-encryption, for example, 4 byte pattern, it's easy
to spot it as well.}
\RU{Например, вот начало файла kernel32.dll (32-битная версия из Windows Server 2008):}
\EN{As example, here is beginning of kernel32.dll file (32-bit version from Windows Server 2008):}

\begin{figure}[H]
\centering
\includegraphics[scale=\FigScale]{ff/XOR/4byte/original1.png}
\caption{\EN{Original file}\RU{Оригинальный файл}}
\end{figure}

\clearpage
\RU{Вот он же, но \q{зашифрованный} 4-байтным ключем:}
\EN{Here is it \q{encrypted} by 4-byte key:}

\begin{figure}[H]
\centering
\includegraphics[scale=\FigScale]{ff/XOR/4byte/encrypted1.png}
\caption{\EN{\q{Encrypted} file}\RU{\q{Зашифрованный} файл}}
\end{figure}

\RU{Очень легко увидеть повторяющиеся 4 символа.}
\EN{It's very easy to spot recurring 4 symbols.}
\RU{Ведь в заголовке PE-файла много длинных нулевых областей, из-за которых ключ становится видным.}
\EN{Indeed, PE-file header has a lot of long zero lacunes, which is the reason why key became visible.}

\clearpage
\RU{Вот начало PE-заголовка в 16-ричном виде:}
\EN{Here is beginning of PE-header in hexadecimal form:}

\begin{figure}[H]
\centering
\includegraphics[scale=\FigScale]{ff/XOR/4byte/original2.png}
\caption{PE-\EN{header}\RU{заголовок}}
\end{figure}

\clearpage
\RU{И вот он же, \q{зашифрованный}:}
\EN{Here is it \q{encrypted}:}

\begin{figure}[H]
\centering
\includegraphics[scale=\FigScale]{ff/XOR/4byte/encrypted2.png}
\caption{\EN{\q{Encrypted} PE-header}\RU{\q{Зашифрованный} PE-заголовок}}
\end{figure}

\RU{Легко увидеть визуально, что ключ это следующие 4 байта}
\EN{It's easy to spot that key is the following 4 bytes}: \TT{8C 61 D2 63}.
\RU{Используя эту информацию, довольно легко расшифровать весь файл.}
\EN{It's easy to decrypt the whole file using this information.}

\RU{Таким образом, важно помнить эти свойства PE-файлов:
1) в PE-заголовке много нулевых областей;
2) все PE-секции дополняются нулями до границы страницы (4096 байт), 
так что после всех секций обычно имеются длинные нулевые области.}
\EN{So this is important to remember these property of PE-files:
1) PE-header has many zero lacunas;
2) all PE-sections padded with zeroes by page border (4096 bytes),
so long zero lacunas usually present after all sections.}

\RU{Некоторые другие форматы файлов могут также иметь длинные нулевые области.}
\EN{Some other file formats may contain long zero lacunas.}
\RU{Это очень типично для файлов, используемых научным и инженерным ПО.}
\EN{It's very typical for files used by scientific and engineering software.}

\RU{Для тех, кто самостоятельно хочет изучить эти файлы, то их можно скачать здесь:}
\EN{For those who wants to inspect these files on one's own, they are downloadable there:}
\url{http://go.yurichev.com/17352}.

\subsection{\Exercise}

\begin{itemize}
	\item \url{http://challenges.re/50}
\end{itemize}


\clearpage
\section{\RU{Простейшее четырехбайтное XOR-шифрование}\EN{Simplest possible 4-byte XOR encryption}}

\RU{Если при XOR-шифровании применялся шаблон длинее байта, например, 4-байтный, то его также легко
увидеть.}
\EN{If longer pattern was used while XOR-encryption, for example, 4 byte pattern, it's easy
to spot it as well.}
\RU{Например, вот начало файла kernel32.dll (32-битная версия из Windows Server 2008):}
\EN{As example, here is beginning of kernel32.dll file (32-bit version from Windows Server 2008):}

\begin{figure}[H]
\centering
\includegraphics[scale=\FigScale]{ff/XOR/4byte/original1.png}
\caption{\EN{Original file}\RU{Оригинальный файл}}
\end{figure}

\clearpage
\RU{Вот он же, но \q{зашифрованный} 4-байтным ключем:}
\EN{Here is it \q{encrypted} by 4-byte key:}

\begin{figure}[H]
\centering
\includegraphics[scale=\FigScale]{ff/XOR/4byte/encrypted1.png}
\caption{\EN{\q{Encrypted} file}\RU{\q{Зашифрованный} файл}}
\end{figure}

\RU{Очень легко увидеть повторяющиеся 4 символа.}
\EN{It's very easy to spot recurring 4 symbols.}
\RU{Ведь в заголовке PE-файла много длинных нулевых областей, из-за которых ключ становится видным.}
\EN{Indeed, PE-file header has a lot of long zero lacunes, which is the reason why key became visible.}

\clearpage
\RU{Вот начало PE-заголовка в 16-ричном виде:}
\EN{Here is beginning of PE-header in hexadecimal form:}

\begin{figure}[H]
\centering
\includegraphics[scale=\FigScale]{ff/XOR/4byte/original2.png}
\caption{PE-\EN{header}\RU{заголовок}}
\end{figure}

\clearpage
\RU{И вот он же, \q{зашифрованный}:}
\EN{Here is it \q{encrypted}:}

\begin{figure}[H]
\centering
\includegraphics[scale=\FigScale]{ff/XOR/4byte/encrypted2.png}
\caption{\EN{\q{Encrypted} PE-header}\RU{\q{Зашифрованный} PE-заголовок}}
\end{figure}

\RU{Легко увидеть визуально, что ключ это следующие 4 байта}
\EN{It's easy to spot that key is the following 4 bytes}: \TT{8C 61 D2 63}.
\RU{Используя эту информацию, довольно легко расшифровать весь файл.}
\EN{It's easy to decrypt the whole file using this information.}

\RU{Таким образом, важно помнить эти свойства PE-файлов:
1) в PE-заголовке много нулевых областей;
2) все PE-секции дополняются нулями до границы страницы (4096 байт), 
так что после всех секций обычно имеются длинные нулевые области.}
\EN{So this is important to remember these property of PE-files:
1) PE-header has many zero lacunas;
2) all PE-sections padded with zeroes by page border (4096 bytes),
so long zero lacunas usually present after all sections.}

\RU{Некоторые другие форматы файлов могут также иметь длинные нулевые области.}
\EN{Some other file formats may contain long zero lacunas.}
\RU{Это очень типично для файлов, используемых научным и инженерным ПО.}
\EN{It's very typical for files used by scientific and engineering software.}

\RU{Для тех, кто самостоятельно хочет изучить эти файлы, то их можно скачать здесь:}
\EN{For those who wants to inspect these files on one's own, they are downloadable there:}
\url{http://go.yurichev.com/17352}.

\subsection{\Exercise}

\begin{itemize}
	\item \url{http://challenges.re/50}
\end{itemize}


\clearpage
\section{\RU{Простейшее четырехбайтное XOR-шифрование}\EN{Simplest possible 4-byte XOR encryption}}

\RU{Если при XOR-шифровании применялся шаблон длинее байта, например, 4-байтный, то его также легко
увидеть.}
\EN{If longer pattern was used while XOR-encryption, for example, 4 byte pattern, it's easy
to spot it as well.}
\RU{Например, вот начало файла kernel32.dll (32-битная версия из Windows Server 2008):}
\EN{As example, here is beginning of kernel32.dll file (32-bit version from Windows Server 2008):}

\begin{figure}[H]
\centering
\includegraphics[scale=\FigScale]{ff/XOR/4byte/original1.png}
\caption{\EN{Original file}\RU{Оригинальный файл}}
\end{figure}

\clearpage
\RU{Вот он же, но \q{зашифрованный} 4-байтным ключем:}
\EN{Here is it \q{encrypted} by 4-byte key:}

\begin{figure}[H]
\centering
\includegraphics[scale=\FigScale]{ff/XOR/4byte/encrypted1.png}
\caption{\EN{\q{Encrypted} file}\RU{\q{Зашифрованный} файл}}
\end{figure}

\RU{Очень легко увидеть повторяющиеся 4 символа.}
\EN{It's very easy to spot recurring 4 symbols.}
\RU{Ведь в заголовке PE-файла много длинных нулевых областей, из-за которых ключ становится видным.}
\EN{Indeed, PE-file header has a lot of long zero lacunes, which is the reason why key became visible.}

\clearpage
\RU{Вот начало PE-заголовка в 16-ричном виде:}
\EN{Here is beginning of PE-header in hexadecimal form:}

\begin{figure}[H]
\centering
\includegraphics[scale=\FigScale]{ff/XOR/4byte/original2.png}
\caption{PE-\EN{header}\RU{заголовок}}
\end{figure}

\clearpage
\RU{И вот он же, \q{зашифрованный}:}
\EN{Here is it \q{encrypted}:}

\begin{figure}[H]
\centering
\includegraphics[scale=\FigScale]{ff/XOR/4byte/encrypted2.png}
\caption{\EN{\q{Encrypted} PE-header}\RU{\q{Зашифрованный} PE-заголовок}}
\end{figure}

\RU{Легко увидеть визуально, что ключ это следующие 4 байта}
\EN{It's easy to spot that key is the following 4 bytes}: \TT{8C 61 D2 63}.
\RU{Используя эту информацию, довольно легко расшифровать весь файл.}
\EN{It's easy to decrypt the whole file using this information.}

\RU{Таким образом, важно помнить эти свойства PE-файлов:
1) в PE-заголовке много нулевых областей;
2) все PE-секции дополняются нулями до границы страницы (4096 байт), 
так что после всех секций обычно имеются длинные нулевые области.}
\EN{So this is important to remember these property of PE-files:
1) PE-header has many zero lacunas;
2) all PE-sections padded with zeroes by page border (4096 bytes),
so long zero lacunas usually present after all sections.}

\RU{Некоторые другие форматы файлов могут также иметь длинные нулевые области.}
\EN{Some other file formats may contain long zero lacunas.}
\RU{Это очень типично для файлов, используемых научным и инженерным ПО.}
\EN{It's very typical for files used by scientific and engineering software.}

\RU{Для тех, кто самостоятельно хочет изучить эти файлы, то их можно скачать здесь:}
\EN{For those who wants to inspect these files on one's own, they are downloadable there:}
\url{http://go.yurichev.com/17352}.

\subsection{\Exercise}

\begin{itemize}
	\item \url{http://challenges.re/50}
\end{itemize}



\section{\RU{Стек, калькуляторы и обратная польская запись}\EN{Stack, calculators and reverse Polish notation}}

\index{\RU{Обратная польская запись}\EN{Reverse Polish notation}}
\RU{Теперь понятно, почему некоторые старые калькуляторы использовали обратную польскую запись%
\footnote{\href{http://go.yurichev.com/17355}{ru.wikipedia.org/wiki/Обратная\_польская\_запись}}.}
\EN{Now we undestand why some old calculators used reverse Polish notation
\footnote{\href{http://go.yurichev.com/17354}{wikipedia.org/wiki/Reverse\_Polish\_notation}}.}
\RU{Например для сложения 12 и 34 нужно было набрать 12, потом 34, потом нажать знак \q{плюс}.}
\EN{For example, for addition of 12 and 34 one has to enter 12, then 34, then press \q{plus} sign.}
\RU{Это потому что старые калькуляторы просто реализовали стековую машину и это было куда проще, 
чем обрабатывать сложные выражения со скобками.}
\EN{It's because old calculators were just stack machine implementations, and this was much simpler
than to handle complex parenthesized expressions.}
\section{x64}

\RU{О том, как происходит работа с числами с плавающей запятой в x86-64, читайте здесь: \myref{floating_SIMD}.}
\EN{On how floating point numbers are processed in x86-64, read more here: \myref{floating_SIMD}.}

% sections
\ifdefined\IncludeExercises
\section{\Exercises}

\begin{itemize}
	\item \url{http://challenges.re/60}
	\item \url{http://challenges.re/61}
\end{itemize}


\fi

\fi
\clearpage
\section{\RU{Простейшее четырехбайтное XOR-шифрование}\EN{Simplest possible 4-byte XOR encryption}}

\RU{Если при XOR-шифровании применялся шаблон длинее байта, например, 4-байтный, то его также легко
увидеть.}
\EN{If longer pattern was used while XOR-encryption, for example, 4 byte pattern, it's easy
to spot it as well.}
\RU{Например, вот начало файла kernel32.dll (32-битная версия из Windows Server 2008):}
\EN{As example, here is beginning of kernel32.dll file (32-bit version from Windows Server 2008):}

\begin{figure}[H]
\centering
\includegraphics[scale=\FigScale]{ff/XOR/4byte/original1.png}
\caption{\EN{Original file}\RU{Оригинальный файл}}
\end{figure}

\clearpage
\RU{Вот он же, но \q{зашифрованный} 4-байтным ключем:}
\EN{Here is it \q{encrypted} by 4-byte key:}

\begin{figure}[H]
\centering
\includegraphics[scale=\FigScale]{ff/XOR/4byte/encrypted1.png}
\caption{\EN{\q{Encrypted} file}\RU{\q{Зашифрованный} файл}}
\end{figure}

\RU{Очень легко увидеть повторяющиеся 4 символа.}
\EN{It's very easy to spot recurring 4 symbols.}
\RU{Ведь в заголовке PE-файла много длинных нулевых областей, из-за которых ключ становится видным.}
\EN{Indeed, PE-file header has a lot of long zero lacunes, which is the reason why key became visible.}

\clearpage
\RU{Вот начало PE-заголовка в 16-ричном виде:}
\EN{Here is beginning of PE-header in hexadecimal form:}

\begin{figure}[H]
\centering
\includegraphics[scale=\FigScale]{ff/XOR/4byte/original2.png}
\caption{PE-\EN{header}\RU{заголовок}}
\end{figure}

\clearpage
\RU{И вот он же, \q{зашифрованный}:}
\EN{Here is it \q{encrypted}:}

\begin{figure}[H]
\centering
\includegraphics[scale=\FigScale]{ff/XOR/4byte/encrypted2.png}
\caption{\EN{\q{Encrypted} PE-header}\RU{\q{Зашифрованный} PE-заголовок}}
\end{figure}

\RU{Легко увидеть визуально, что ключ это следующие 4 байта}
\EN{It's easy to spot that key is the following 4 bytes}: \TT{8C 61 D2 63}.
\RU{Используя эту информацию, довольно легко расшифровать весь файл.}
\EN{It's easy to decrypt the whole file using this information.}

\RU{Таким образом, важно помнить эти свойства PE-файлов:
1) в PE-заголовке много нулевых областей;
2) все PE-секции дополняются нулями до границы страницы (4096 байт), 
так что после всех секций обычно имеются длинные нулевые области.}
\EN{So this is important to remember these property of PE-files:
1) PE-header has many zero lacunas;
2) all PE-sections padded with zeroes by page border (4096 bytes),
so long zero lacunas usually present after all sections.}

\RU{Некоторые другие форматы файлов могут также иметь длинные нулевые области.}
\EN{Some other file formats may contain long zero lacunas.}
\RU{Это очень типично для файлов, используемых научным и инженерным ПО.}
\EN{It's very typical for files used by scientific and engineering software.}

\RU{Для тех, кто самостоятельно хочет изучить эти файлы, то их можно скачать здесь:}
\EN{For those who wants to inspect these files on one's own, they are downloadable there:}
\url{http://go.yurichev.com/17352}.

\subsection{\Exercise}

\begin{itemize}
	\item \url{http://challenges.re/50}
\end{itemize}


\clearpage
\section{\RU{Простейшее четырехбайтное XOR-шифрование}\EN{Simplest possible 4-byte XOR encryption}}

\RU{Если при XOR-шифровании применялся шаблон длинее байта, например, 4-байтный, то его также легко
увидеть.}
\EN{If longer pattern was used while XOR-encryption, for example, 4 byte pattern, it's easy
to spot it as well.}
\RU{Например, вот начало файла kernel32.dll (32-битная версия из Windows Server 2008):}
\EN{As example, here is beginning of kernel32.dll file (32-bit version from Windows Server 2008):}

\begin{figure}[H]
\centering
\includegraphics[scale=\FigScale]{ff/XOR/4byte/original1.png}
\caption{\EN{Original file}\RU{Оригинальный файл}}
\end{figure}

\clearpage
\RU{Вот он же, но \q{зашифрованный} 4-байтным ключем:}
\EN{Here is it \q{encrypted} by 4-byte key:}

\begin{figure}[H]
\centering
\includegraphics[scale=\FigScale]{ff/XOR/4byte/encrypted1.png}
\caption{\EN{\q{Encrypted} file}\RU{\q{Зашифрованный} файл}}
\end{figure}

\RU{Очень легко увидеть повторяющиеся 4 символа.}
\EN{It's very easy to spot recurring 4 symbols.}
\RU{Ведь в заголовке PE-файла много длинных нулевых областей, из-за которых ключ становится видным.}
\EN{Indeed, PE-file header has a lot of long zero lacunes, which is the reason why key became visible.}

\clearpage
\RU{Вот начало PE-заголовка в 16-ричном виде:}
\EN{Here is beginning of PE-header in hexadecimal form:}

\begin{figure}[H]
\centering
\includegraphics[scale=\FigScale]{ff/XOR/4byte/original2.png}
\caption{PE-\EN{header}\RU{заголовок}}
\end{figure}

\clearpage
\RU{И вот он же, \q{зашифрованный}:}
\EN{Here is it \q{encrypted}:}

\begin{figure}[H]
\centering
\includegraphics[scale=\FigScale]{ff/XOR/4byte/encrypted2.png}
\caption{\EN{\q{Encrypted} PE-header}\RU{\q{Зашифрованный} PE-заголовок}}
\end{figure}

\RU{Легко увидеть визуально, что ключ это следующие 4 байта}
\EN{It's easy to spot that key is the following 4 bytes}: \TT{8C 61 D2 63}.
\RU{Используя эту информацию, довольно легко расшифровать весь файл.}
\EN{It's easy to decrypt the whole file using this information.}

\RU{Таким образом, важно помнить эти свойства PE-файлов:
1) в PE-заголовке много нулевых областей;
2) все PE-секции дополняются нулями до границы страницы (4096 байт), 
так что после всех секций обычно имеются длинные нулевые области.}
\EN{So this is important to remember these property of PE-files:
1) PE-header has many zero lacunas;
2) all PE-sections padded with zeroes by page border (4096 bytes),
so long zero lacunas usually present after all sections.}

\RU{Некоторые другие форматы файлов могут также иметь длинные нулевые области.}
\EN{Some other file formats may contain long zero lacunas.}
\RU{Это очень типично для файлов, используемых научным и инженерным ПО.}
\EN{It's very typical for files used by scientific and engineering software.}

\RU{Для тех, кто самостоятельно хочет изучить эти файлы, то их можно скачать здесь:}
\EN{For those who wants to inspect these files on one's own, they are downloadable there:}
\url{http://go.yurichev.com/17352}.

\subsection{\Exercise}

\begin{itemize}
	\item \url{http://challenges.re/50}
\end{itemize}


\chapter[\RU{Линейный конгруэнтный генератор}\EN{Linear congruential generator}]
{\RU{Линейный конгруэнтный генератор как генератор псевдослучайных чисел}\EN{Linear congruential generator as pseudorandom number generator}}
\index{\CStandardLibrary!rand()}
\label{LCG_simple}

\RU{Линейный конгруэнтный генератор, пожалуй, самый простой способ генерировать псевдослучайные числа.}
\EN{The linear congruential generator is probably the simplest possible way to generate random numbers.}
\RU{Он не в почете в наше время\footnote{Вихрь Мерсенна куда лучше}, но он настолько прост
(только одно умножение, одно сложение и одна операция \q{И}),
что мы можем использовать его в качестве примера.}
\EN{It's not in favour in modern times\footnote{Mersenne twister is better}, but it's so simple 
(just one multiplication, one addition and one AND operation), 
we can use it as an example.}

\lstinputlisting{patterns/145_LCG/rand.c.\LANG}

\RU{Здесь две функции: одна используется для инициализации внутреннего состояния, а вторая
вызывается собственно для генерации псевдослучайных чисел.}
\EN{There are two functions: the first one is used to initialize the internal state, and the second one is called
to generate pseudorandom numbers.}

\RU{Мы видим что в алгоритме применяются две константы}\EN{We see that two constants are used in the algorithm}.
\RU{Они взяты из}\EN{They are taken from} \cite{Numerical}.
\RU{Определим их используя выражение \CCpp \TT{\#define}. Это макрос.}
\EN{Let's define them using a \TT{\#define} \CCpp statement. It's a macro.}
\RU{Разница между макросом в \CCpp и константой в том, что все макросы заменяются на значения препроцессором
\CCpp и они не занимают места в памяти как переменные.}
\EN{The difference between a \CCpp macro and a constant is that all macros are replaced 
with their value by \CCpp preprocessor,
and they don't take any memory, unlike variables.}
\RU{А константы, напротив, это переменные только для чтения.}
\EN{In contrast, a constant is a read-only variable.}
\RU{Можно взять указатель (или адрес) переменной-константы, но это невозможно сделать с макросом.}
\EN{It's possible to take a pointer (or address) of a constant variable, but impossible to do so with a macro.}

\RU{Последняя операция \q{И} нужна, потому что согласно стандарту Си \TT{my\_rand()} должна возвращать значение в пределах
0..32767.}
\EN{The last AND operation is needed because by C-standard \TT{my\_rand()} has to return a value in 
the 0..32767 range.}
\RU{Если вы хотите получать 32-битные псевдослучайные значения, просто уберите последнюю операцию \q{И}.}
\EN{If you want to get 32-bit pseudorandom values, just omit the last AND operation.}

\section{x86}

\lstinputlisting[caption=\Optimizing MSVC 2013]{patterns/145_LCG/rand_MSVC_2013_x86_Ox.asm}

\RU{Вот мы это и видим: обе константы встроены в код.}
\EN{Here we see it: both constants are embedded into the code.}
\RU{Память для них не выделяется.}\EN{There is no memory allocated for them.}
\RU{Функция \TT{my\_srand()} просто копирует входное значение во внутреннюю переменную \TT{rand\_state}.}
\EN{The \TT{my\_srand()} function just copies its input value into the internal \TT{rand\_state} variable.}

\RU{\TT{my\_rand()} берет её, вычисляет следующее состояние \TT{rand\_state}, 
обрезает его и оставляет в регистре EAX.}
\EN{\TT{my\_rand()} takes it, calculates the next \TT{rand\_state}, cuts it and leaves it in the EAX register.}

\RU{Неоптимизированная версия побольше}\EN{The non-optimized version is more verbose}:

\lstinputlisting[caption=\NonOptimizing MSVC 2013]{patterns/145_LCG/rand_MSVC_2013_x86.asm}

\section{x64}

\RU{Версия для x64 почти такая же, и использует 32-битные регистры вместо 64-битных
(потому что мы работаем здесь с переменными типа \Tint).}
\EN{The x64 version is mostly the same and uses 32-bit registers instead of 64-bit ones 
(because we are working with \Tint values here).}
\RU{Но функция \TT{my\_srand()} берет входной аргумент из регистра \ECX, а не из стека:}
\EN{But \TT{my\_srand()} takes its input argument from the \ECX register rather than from stack:}

\lstinputlisting[caption=\Optimizing MSVC 2013 x64]{patterns/145_LCG/rand_MSVC_2013_x64_Ox.asm.\LANG}

\ifdefined\IncludeGCC
\RU{GCC делает почти такой же код}\EN{GCC compiler generates mostly the same code}.
\fi

\ifdefined\IncludeARM
\section{32-bit ARM}

\lstinputlisting[caption=\OptimizingKeilVI (\ARMMode)]{patterns/145_LCG/rand.s_Keil_ARM_O3.s.\LANG}

\RU{В ARM инструкцию невозможно встроить 32-битную константу, так что Keil-у приходится размещать
их отдельно и дополнительно загружать.}
\EN{It's not possible to embed 32-bit constants into ARM instructions, so Keil has to place them externally
and load them additionally.}

\RU{Вот еще что интересно: константу 0x7FFF также нельзя встроить.}
\EN{One interesting thing is that it's not possible to embed the 0x7FFF constant as well.}
\RU{Поэтому Keil сдвигает \TT{rand\_state} влево на 17 бит и затем сдвигает вправо на 17 бит.}
\EN{So what Keil does is shifting \TT{rand\_state} left by 17 bits and then shifting it right by 17 bits.}
\RU{Это аналогично \CCpp{}-выражению $(rand\_state \ll 17) \gg 17$.}
\EN{This is analogous to the $(rand\_state \ll 17) \gg 17$ statement in \CCpp.}
\RU{Выглядит как бессмысленная операция, но тем не менее, что она делает это очищает старшие 17 бит, оставляя
младшие 15 бит нетронутыми, и это наша цель, в конце концов.}
\EN{It seems to be useless operation, but
what it does is clearing the high 17 bits, leaving the low 15 bits intact, and that's our goal after all.}
\ESph{}\PTBRph{}\PLph{}\ITAph{}\\
\\
\Optimizing Keil \RU{для режима Thumb делает почти такой же код}\EN{for Thumb mode generates mostly the same code}.
\fi

\ifdefined\IncludeMIPS
\subsection{MIPS}

\index{MIPS!\Registers!FCCR}
\EN{The co-processor of the MIPS processor has a condition bit which can be set in the FPU and checked in the CPU.}
\RU{В сопроцессоре MIPS есть бит результата, который устанавливается в FPU и проверяется в CPU.}
\EN{Earlier MIPS-es have only one condition bit (called FCC0), later ones have 8 (called FCC7-FCC0).}
\RU{Ранние MIPS имели только один бит (с названием FCC0), а у поздних их 8 (с названием FCC7-FCC0).}
\RU{Этот бит (или биты) находятся в регистре с названием FCCR.}
\EN{This bit (or bits) are located in the register called FCCR.}

\lstinputlisting[caption=\Optimizing GCC 4.4.5 (IDA)]{patterns/12_FPU/3_comparison/MIPS_O3_IDA.lst.\LANG}

\index{MIPS!\Instructions!C.LT.D}
\TT{C.LT.D} \EN{compares two values}\RU{сравнивает два значения}. 
\TT{LT} \EN{is the condition}\RU{это условие} \q{Less Than}\RU{ (меньше чем)}.
\TT{D} \EN{implies values of type}\RU{означает переменные типа} \Tdouble.
\EN{Depending on the result of the comparison, the FCC0 condition bit is either set or cleared.}
\RU{В зависимости от результата сравнения, бит FCC0 устанавливается или очищается.}

\index{MIPS!\Instructions!BC1T}
\index{MIPS!\Instructions!BC1F}
\TT{BC1T} \EN{checks the FCC0 bit and jumps if the bit is set}\RU{проверяет бит FCC0 и делает переход, если бит выставлен}.
\TT{T} \EN{mean that the jump is to be taken if the bit is set}\RU{означает что переход произойдет если бит выставлен} (\q{True}).
\EN{There is also the instruction}\RU{Имеется также инструкция} \q{BC1F} \EN{which jumps if the bit is cleared}\RU{которая сработает, если бит сброшен} (\q{False}).

\RU{В зависимости от перехода один из аргументов функции помещается в регистр \$F0.}
\EN{Depending on the jump, one of function arguments is placed into \$F0.}

\fi

\ifx\LITE\undefined
\section{\RU{Версия этого примера для многопоточной среды}\EN{Thread-safe version of the example}}

\RU{Версия примера для многопоточной среды будет рассмотрена позже}%
\EN{The thread-safe version of the example is to be demonstrated later}: \myref{LCG_TLS}.
\fi

\clearpage
\section{\RU{Простейшее четырехбайтное XOR-шифрование}\EN{Simplest possible 4-byte XOR encryption}}

\RU{Если при XOR-шифровании применялся шаблон длинее байта, например, 4-байтный, то его также легко
увидеть.}
\EN{If longer pattern was used while XOR-encryption, for example, 4 byte pattern, it's easy
to spot it as well.}
\RU{Например, вот начало файла kernel32.dll (32-битная версия из Windows Server 2008):}
\EN{As example, here is beginning of kernel32.dll file (32-bit version from Windows Server 2008):}

\begin{figure}[H]
\centering
\includegraphics[scale=\FigScale]{ff/XOR/4byte/original1.png}
\caption{\EN{Original file}\RU{Оригинальный файл}}
\end{figure}

\clearpage
\RU{Вот он же, но \q{зашифрованный} 4-байтным ключем:}
\EN{Here is it \q{encrypted} by 4-byte key:}

\begin{figure}[H]
\centering
\includegraphics[scale=\FigScale]{ff/XOR/4byte/encrypted1.png}
\caption{\EN{\q{Encrypted} file}\RU{\q{Зашифрованный} файл}}
\end{figure}

\RU{Очень легко увидеть повторяющиеся 4 символа.}
\EN{It's very easy to spot recurring 4 symbols.}
\RU{Ведь в заголовке PE-файла много длинных нулевых областей, из-за которых ключ становится видным.}
\EN{Indeed, PE-file header has a lot of long zero lacunes, which is the reason why key became visible.}

\clearpage
\RU{Вот начало PE-заголовка в 16-ричном виде:}
\EN{Here is beginning of PE-header in hexadecimal form:}

\begin{figure}[H]
\centering
\includegraphics[scale=\FigScale]{ff/XOR/4byte/original2.png}
\caption{PE-\EN{header}\RU{заголовок}}
\end{figure}

\clearpage
\RU{И вот он же, \q{зашифрованный}:}
\EN{Here is it \q{encrypted}:}

\begin{figure}[H]
\centering
\includegraphics[scale=\FigScale]{ff/XOR/4byte/encrypted2.png}
\caption{\EN{\q{Encrypted} PE-header}\RU{\q{Зашифрованный} PE-заголовок}}
\end{figure}

\RU{Легко увидеть визуально, что ключ это следующие 4 байта}
\EN{It's easy to spot that key is the following 4 bytes}: \TT{8C 61 D2 63}.
\RU{Используя эту информацию, довольно легко расшифровать весь файл.}
\EN{It's easy to decrypt the whole file using this information.}

\RU{Таким образом, важно помнить эти свойства PE-файлов:
1) в PE-заголовке много нулевых областей;
2) все PE-секции дополняются нулями до границы страницы (4096 байт), 
так что после всех секций обычно имеются длинные нулевые области.}
\EN{So this is important to remember these property of PE-files:
1) PE-header has many zero lacunas;
2) all PE-sections padded with zeroes by page border (4096 bytes),
so long zero lacunas usually present after all sections.}

\RU{Некоторые другие форматы файлов могут также иметь длинные нулевые области.}
\EN{Some other file formats may contain long zero lacunas.}
\RU{Это очень типично для файлов, используемых научным и инженерным ПО.}
\EN{It's very typical for files used by scientific and engineering software.}

\RU{Для тех, кто самостоятельно хочет изучить эти файлы, то их можно скачать здесь:}
\EN{For those who wants to inspect these files on one's own, they are downloadable there:}
\url{http://go.yurichev.com/17352}.

\subsection{\Exercise}

\begin{itemize}
	\item \url{http://challenges.re/50}
\end{itemize}


\ifx\LITE\undefined
\chapter{\RU{Объединения (union)}\EN{Unions}}

\EN{\CCpp \IT{union} is mostly used for interpreting a variable (or memory block) of one data type as a variable of another data type.}
\RU{\IT{union} в \CCpp используется в основном для интерпертации переменной (или блока памяти) одного типа как переменной другого типа.}

% sections
\clearpage
\section{\RU{Простейшее четырехбайтное XOR-шифрование}\EN{Simplest possible 4-byte XOR encryption}}

\RU{Если при XOR-шифровании применялся шаблон длинее байта, например, 4-байтный, то его также легко
увидеть.}
\EN{If longer pattern was used while XOR-encryption, for example, 4 byte pattern, it's easy
to spot it as well.}
\RU{Например, вот начало файла kernel32.dll (32-битная версия из Windows Server 2008):}
\EN{As example, here is beginning of kernel32.dll file (32-bit version from Windows Server 2008):}

\begin{figure}[H]
\centering
\includegraphics[scale=\FigScale]{ff/XOR/4byte/original1.png}
\caption{\EN{Original file}\RU{Оригинальный файл}}
\end{figure}

\clearpage
\RU{Вот он же, но \q{зашифрованный} 4-байтным ключем:}
\EN{Here is it \q{encrypted} by 4-byte key:}

\begin{figure}[H]
\centering
\includegraphics[scale=\FigScale]{ff/XOR/4byte/encrypted1.png}
\caption{\EN{\q{Encrypted} file}\RU{\q{Зашифрованный} файл}}
\end{figure}

\RU{Очень легко увидеть повторяющиеся 4 символа.}
\EN{It's very easy to spot recurring 4 symbols.}
\RU{Ведь в заголовке PE-файла много длинных нулевых областей, из-за которых ключ становится видным.}
\EN{Indeed, PE-file header has a lot of long zero lacunes, which is the reason why key became visible.}

\clearpage
\RU{Вот начало PE-заголовка в 16-ричном виде:}
\EN{Here is beginning of PE-header in hexadecimal form:}

\begin{figure}[H]
\centering
\includegraphics[scale=\FigScale]{ff/XOR/4byte/original2.png}
\caption{PE-\EN{header}\RU{заголовок}}
\end{figure}

\clearpage
\RU{И вот он же, \q{зашифрованный}:}
\EN{Here is it \q{encrypted}:}

\begin{figure}[H]
\centering
\includegraphics[scale=\FigScale]{ff/XOR/4byte/encrypted2.png}
\caption{\EN{\q{Encrypted} PE-header}\RU{\q{Зашифрованный} PE-заголовок}}
\end{figure}

\RU{Легко увидеть визуально, что ключ это следующие 4 байта}
\EN{It's easy to spot that key is the following 4 bytes}: \TT{8C 61 D2 63}.
\RU{Используя эту информацию, довольно легко расшифровать весь файл.}
\EN{It's easy to decrypt the whole file using this information.}

\RU{Таким образом, важно помнить эти свойства PE-файлов:
1) в PE-заголовке много нулевых областей;
2) все PE-секции дополняются нулями до границы страницы (4096 байт), 
так что после всех секций обычно имеются длинные нулевые области.}
\EN{So this is important to remember these property of PE-files:
1) PE-header has many zero lacunas;
2) all PE-sections padded with zeroes by page border (4096 bytes),
so long zero lacunas usually present after all sections.}

\RU{Некоторые другие форматы файлов могут также иметь длинные нулевые области.}
\EN{Some other file formats may contain long zero lacunas.}
\RU{Это очень типично для файлов, используемых научным и инженерным ПО.}
\EN{It's very typical for files used by scientific and engineering software.}

\RU{Для тех, кто самостоятельно хочет изучить эти файлы, то их можно скачать здесь:}
\EN{For those who wants to inspect these files on one's own, they are downloadable there:}
\url{http://go.yurichev.com/17352}.

\subsection{\Exercise}

\begin{itemize}
	\item \url{http://challenges.re/50}
\end{itemize}


\clearpage
\section{\RU{Простейшее четырехбайтное XOR-шифрование}\EN{Simplest possible 4-byte XOR encryption}}

\RU{Если при XOR-шифровании применялся шаблон длинее байта, например, 4-байтный, то его также легко
увидеть.}
\EN{If longer pattern was used while XOR-encryption, for example, 4 byte pattern, it's easy
to spot it as well.}
\RU{Например, вот начало файла kernel32.dll (32-битная версия из Windows Server 2008):}
\EN{As example, here is beginning of kernel32.dll file (32-bit version from Windows Server 2008):}

\begin{figure}[H]
\centering
\includegraphics[scale=\FigScale]{ff/XOR/4byte/original1.png}
\caption{\EN{Original file}\RU{Оригинальный файл}}
\end{figure}

\clearpage
\RU{Вот он же, но \q{зашифрованный} 4-байтным ключем:}
\EN{Here is it \q{encrypted} by 4-byte key:}

\begin{figure}[H]
\centering
\includegraphics[scale=\FigScale]{ff/XOR/4byte/encrypted1.png}
\caption{\EN{\q{Encrypted} file}\RU{\q{Зашифрованный} файл}}
\end{figure}

\RU{Очень легко увидеть повторяющиеся 4 символа.}
\EN{It's very easy to spot recurring 4 symbols.}
\RU{Ведь в заголовке PE-файла много длинных нулевых областей, из-за которых ключ становится видным.}
\EN{Indeed, PE-file header has a lot of long zero lacunes, which is the reason why key became visible.}

\clearpage
\RU{Вот начало PE-заголовка в 16-ричном виде:}
\EN{Here is beginning of PE-header in hexadecimal form:}

\begin{figure}[H]
\centering
\includegraphics[scale=\FigScale]{ff/XOR/4byte/original2.png}
\caption{PE-\EN{header}\RU{заголовок}}
\end{figure}

\clearpage
\RU{И вот он же, \q{зашифрованный}:}
\EN{Here is it \q{encrypted}:}

\begin{figure}[H]
\centering
\includegraphics[scale=\FigScale]{ff/XOR/4byte/encrypted2.png}
\caption{\EN{\q{Encrypted} PE-header}\RU{\q{Зашифрованный} PE-заголовок}}
\end{figure}

\RU{Легко увидеть визуально, что ключ это следующие 4 байта}
\EN{It's easy to spot that key is the following 4 bytes}: \TT{8C 61 D2 63}.
\RU{Используя эту информацию, довольно легко расшифровать весь файл.}
\EN{It's easy to decrypt the whole file using this information.}

\RU{Таким образом, важно помнить эти свойства PE-файлов:
1) в PE-заголовке много нулевых областей;
2) все PE-секции дополняются нулями до границы страницы (4096 байт), 
так что после всех секций обычно имеются длинные нулевые области.}
\EN{So this is important to remember these property of PE-files:
1) PE-header has many zero lacunas;
2) all PE-sections padded with zeroes by page border (4096 bytes),
so long zero lacunas usually present after all sections.}

\RU{Некоторые другие форматы файлов могут также иметь длинные нулевые области.}
\EN{Some other file formats may contain long zero lacunas.}
\RU{Это очень типично для файлов, используемых научным и инженерным ПО.}
\EN{It's very typical for files used by scientific and engineering software.}

\RU{Для тех, кто самостоятельно хочет изучить эти файлы, то их можно скачать здесь:}
\EN{For those who wants to inspect these files on one's own, they are downloadable there:}
\url{http://go.yurichev.com/17352}.

\subsection{\Exercise}

\begin{itemize}
	\item \url{http://challenges.re/50}
\end{itemize}



\section{\RU{Быстрое вычисление квадратного корня}\EN{Fast square root calculation}}

\RU{Вот где еще можно на практике применить трактовку типа \Tfloat как целочисленного, это быстрое вычисление квадратного корня.}%
\EN{Another well-known algorithm where \Tfloat is interpreted as integer is fast calculation of square root.}

\begin{lstlisting}[caption=\EN{The source code is taken from Wikipedia}\RU{Исходный код взят из Wikipedia}: \url{http://go.yurichev.com/17364}]
/* Assumes that float is in the IEEE 754 single precision floating point format
 * and that int is 32 bits. */
float sqrt_approx(float z)
{
    int val_int = *(int*)&z; /* Same bits, but as an int */
    /*
     * To justify the following code, prove that
     *
     * ((((val_int / 2^m) - b) / 2) + b) * 2^m = ((val_int - 2^m) / 2) + ((b + 1) / 2) * 2^m)
     *
     * where
     *
     * b = exponent bias
     * m = number of mantissa bits
     *
     * .
     */
 
    val_int -= 1 << 23; /* Subtract 2^m. */
    val_int >>= 1; /* Divide by 2. */
    val_int += 1 << 29; /* Add ((b + 1) / 2) * 2^m. */
 
    return *(float*)&val_int; /* Interpret again as float */
}
\end{lstlisting}

\RU{В качестве упражнения, вы можете попробовать скомпилировать эту функцию и разобраться, как она работает.}
\EN{As an exercise, you can try to compile this function and to understand, how it works.}\ESph{}\PTBRph{}\PLph{}\ITAph{}\\
\\
\RU{Имеется также известный алгоритм быстрого вычисления}\EN{There is also well-known algorithm of fast calculation of} $\frac{1}{\sqrt{x}}$.
\index{Quake III Arena}
\RU{Алгоритм стал известным, вероятно потому, что был применен в Quake III Arena.}%
\EN{Algorithm became popular, supposedly, because it was used in Quake III Arena.}

\RU{Описание алгоритма есть в}\EN{Algorithm description is present in} Wikipedia:
\EN{\url{http://go.yurichev.com/17360}}\RU{\url{http://go.yurichev.com/17361}}.


\clearpage
\section{\RU{Простейшее четырехбайтное XOR-шифрование}\EN{Simplest possible 4-byte XOR encryption}}

\RU{Если при XOR-шифровании применялся шаблон длинее байта, например, 4-байтный, то его также легко
увидеть.}
\EN{If longer pattern was used while XOR-encryption, for example, 4 byte pattern, it's easy
to spot it as well.}
\RU{Например, вот начало файла kernel32.dll (32-битная версия из Windows Server 2008):}
\EN{As example, here is beginning of kernel32.dll file (32-bit version from Windows Server 2008):}

\begin{figure}[H]
\centering
\includegraphics[scale=\FigScale]{ff/XOR/4byte/original1.png}
\caption{\EN{Original file}\RU{Оригинальный файл}}
\end{figure}

\clearpage
\RU{Вот он же, но \q{зашифрованный} 4-байтным ключем:}
\EN{Here is it \q{encrypted} by 4-byte key:}

\begin{figure}[H]
\centering
\includegraphics[scale=\FigScale]{ff/XOR/4byte/encrypted1.png}
\caption{\EN{\q{Encrypted} file}\RU{\q{Зашифрованный} файл}}
\end{figure}

\RU{Очень легко увидеть повторяющиеся 4 символа.}
\EN{It's very easy to spot recurring 4 symbols.}
\RU{Ведь в заголовке PE-файла много длинных нулевых областей, из-за которых ключ становится видным.}
\EN{Indeed, PE-file header has a lot of long zero lacunes, which is the reason why key became visible.}

\clearpage
\RU{Вот начало PE-заголовка в 16-ричном виде:}
\EN{Here is beginning of PE-header in hexadecimal form:}

\begin{figure}[H]
\centering
\includegraphics[scale=\FigScale]{ff/XOR/4byte/original2.png}
\caption{PE-\EN{header}\RU{заголовок}}
\end{figure}

\clearpage
\RU{И вот он же, \q{зашифрованный}:}
\EN{Here is it \q{encrypted}:}

\begin{figure}[H]
\centering
\includegraphics[scale=\FigScale]{ff/XOR/4byte/encrypted2.png}
\caption{\EN{\q{Encrypted} PE-header}\RU{\q{Зашифрованный} PE-заголовок}}
\end{figure}

\RU{Легко увидеть визуально, что ключ это следующие 4 байта}
\EN{It's easy to spot that key is the following 4 bytes}: \TT{8C 61 D2 63}.
\RU{Используя эту информацию, довольно легко расшифровать весь файл.}
\EN{It's easy to decrypt the whole file using this information.}

\RU{Таким образом, важно помнить эти свойства PE-файлов:
1) в PE-заголовке много нулевых областей;
2) все PE-секции дополняются нулями до границы страницы (4096 байт), 
так что после всех секций обычно имеются длинные нулевые области.}
\EN{So this is important to remember these property of PE-files:
1) PE-header has many zero lacunas;
2) all PE-sections padded with zeroes by page border (4096 bytes),
so long zero lacunas usually present after all sections.}

\RU{Некоторые другие форматы файлов могут также иметь длинные нулевые области.}
\EN{Some other file formats may contain long zero lacunas.}
\RU{Это очень типично для файлов, используемых научным и инженерным ПО.}
\EN{It's very typical for files used by scientific and engineering software.}

\RU{Для тех, кто самостоятельно хочет изучить эти файлы, то их можно скачать здесь:}
\EN{For those who wants to inspect these files on one's own, they are downloadable there:}
\url{http://go.yurichev.com/17352}.

\subsection{\Exercise}

\begin{itemize}
	\item \url{http://challenges.re/50}
\end{itemize}


\fi
\clearpage
\section{\RU{Простейшее четырехбайтное XOR-шифрование}\EN{Simplest possible 4-byte XOR encryption}}

\RU{Если при XOR-шифровании применялся шаблон длинее байта, например, 4-байтный, то его также легко
увидеть.}
\EN{If longer pattern was used while XOR-encryption, for example, 4 byte pattern, it's easy
to spot it as well.}
\RU{Например, вот начало файла kernel32.dll (32-битная версия из Windows Server 2008):}
\EN{As example, here is beginning of kernel32.dll file (32-bit version from Windows Server 2008):}

\begin{figure}[H]
\centering
\includegraphics[scale=\FigScale]{ff/XOR/4byte/original1.png}
\caption{\EN{Original file}\RU{Оригинальный файл}}
\end{figure}

\clearpage
\RU{Вот он же, но \q{зашифрованный} 4-байтным ключем:}
\EN{Here is it \q{encrypted} by 4-byte key:}

\begin{figure}[H]
\centering
\includegraphics[scale=\FigScale]{ff/XOR/4byte/encrypted1.png}
\caption{\EN{\q{Encrypted} file}\RU{\q{Зашифрованный} файл}}
\end{figure}

\RU{Очень легко увидеть повторяющиеся 4 символа.}
\EN{It's very easy to spot recurring 4 symbols.}
\RU{Ведь в заголовке PE-файла много длинных нулевых областей, из-за которых ключ становится видным.}
\EN{Indeed, PE-file header has a lot of long zero lacunes, which is the reason why key became visible.}

\clearpage
\RU{Вот начало PE-заголовка в 16-ричном виде:}
\EN{Here is beginning of PE-header in hexadecimal form:}

\begin{figure}[H]
\centering
\includegraphics[scale=\FigScale]{ff/XOR/4byte/original2.png}
\caption{PE-\EN{header}\RU{заголовок}}
\end{figure}

\clearpage
\RU{И вот он же, \q{зашифрованный}:}
\EN{Here is it \q{encrypted}:}

\begin{figure}[H]
\centering
\includegraphics[scale=\FigScale]{ff/XOR/4byte/encrypted2.png}
\caption{\EN{\q{Encrypted} PE-header}\RU{\q{Зашифрованный} PE-заголовок}}
\end{figure}

\RU{Легко увидеть визуально, что ключ это следующие 4 байта}
\EN{It's easy to spot that key is the following 4 bytes}: \TT{8C 61 D2 63}.
\RU{Используя эту информацию, довольно легко расшифровать весь файл.}
\EN{It's easy to decrypt the whole file using this information.}

\RU{Таким образом, важно помнить эти свойства PE-файлов:
1) в PE-заголовке много нулевых областей;
2) все PE-секции дополняются нулями до границы страницы (4096 байт), 
так что после всех секций обычно имеются длинные нулевые области.}
\EN{So this is important to remember these property of PE-files:
1) PE-header has many zero lacunas;
2) all PE-sections padded with zeroes by page border (4096 bytes),
so long zero lacunas usually present after all sections.}

\RU{Некоторые другие форматы файлов могут также иметь длинные нулевые области.}
\EN{Some other file formats may contain long zero lacunas.}
\RU{Это очень типично для файлов, используемых научным и инженерным ПО.}
\EN{It's very typical for files used by scientific and engineering software.}

\RU{Для тех, кто самостоятельно хочет изучить эти файлы, то их можно скачать здесь:}
\EN{For those who wants to inspect these files on one's own, they are downloadable there:}
\url{http://go.yurichev.com/17352}.

\subsection{\Exercise}

\begin{itemize}
	\item \url{http://challenges.re/50}
\end{itemize}


\ifx\LITE\undefined
\clearpage
\section{\RU{Простейшее четырехбайтное XOR-шифрование}\EN{Simplest possible 4-byte XOR encryption}}

\RU{Если при XOR-шифровании применялся шаблон длинее байта, например, 4-байтный, то его также легко
увидеть.}
\EN{If longer pattern was used while XOR-encryption, for example, 4 byte pattern, it's easy
to spot it as well.}
\RU{Например, вот начало файла kernel32.dll (32-битная версия из Windows Server 2008):}
\EN{As example, here is beginning of kernel32.dll file (32-bit version from Windows Server 2008):}

\begin{figure}[H]
\centering
\includegraphics[scale=\FigScale]{ff/XOR/4byte/original1.png}
\caption{\EN{Original file}\RU{Оригинальный файл}}
\end{figure}

\clearpage
\RU{Вот он же, но \q{зашифрованный} 4-байтным ключем:}
\EN{Here is it \q{encrypted} by 4-byte key:}

\begin{figure}[H]
\centering
\includegraphics[scale=\FigScale]{ff/XOR/4byte/encrypted1.png}
\caption{\EN{\q{Encrypted} file}\RU{\q{Зашифрованный} файл}}
\end{figure}

\RU{Очень легко увидеть повторяющиеся 4 символа.}
\EN{It's very easy to spot recurring 4 symbols.}
\RU{Ведь в заголовке PE-файла много длинных нулевых областей, из-за которых ключ становится видным.}
\EN{Indeed, PE-file header has a lot of long zero lacunes, which is the reason why key became visible.}

\clearpage
\RU{Вот начало PE-заголовка в 16-ричном виде:}
\EN{Here is beginning of PE-header in hexadecimal form:}

\begin{figure}[H]
\centering
\includegraphics[scale=\FigScale]{ff/XOR/4byte/original2.png}
\caption{PE-\EN{header}\RU{заголовок}}
\end{figure}

\clearpage
\RU{И вот он же, \q{зашифрованный}:}
\EN{Here is it \q{encrypted}:}

\begin{figure}[H]
\centering
\includegraphics[scale=\FigScale]{ff/XOR/4byte/encrypted2.png}
\caption{\EN{\q{Encrypted} PE-header}\RU{\q{Зашифрованный} PE-заголовок}}
\end{figure}

\RU{Легко увидеть визуально, что ключ это следующие 4 байта}
\EN{It's easy to spot that key is the following 4 bytes}: \TT{8C 61 D2 63}.
\RU{Используя эту информацию, довольно легко расшифровать весь файл.}
\EN{It's easy to decrypt the whole file using this information.}

\RU{Таким образом, важно помнить эти свойства PE-файлов:
1) в PE-заголовке много нулевых областей;
2) все PE-секции дополняются нулями до границы страницы (4096 байт), 
так что после всех секций обычно имеются длинные нулевые области.}
\EN{So this is important to remember these property of PE-files:
1) PE-header has many zero lacunas;
2) all PE-sections padded with zeroes by page border (4096 bytes),
so long zero lacunas usually present after all sections.}

\RU{Некоторые другие форматы файлов могут также иметь длинные нулевые области.}
\EN{Some other file formats may contain long zero lacunas.}
\RU{Это очень типично для файлов, используемых научным и инженерным ПО.}
\EN{It's very typical for files used by scientific and engineering software.}

\RU{Для тех, кто самостоятельно хочет изучить эти файлы, то их можно скачать здесь:}
\EN{For those who wants to inspect these files on one's own, they are downloadable there:}
\url{http://go.yurichev.com/17352}.

\subsection{\Exercise}

\begin{itemize}
	\item \url{http://challenges.re/50}
\end{itemize}


\fi
\chapter{\RU{64 бита}\EN{64 bits}}

\section{x86-64}
\index{x86-64}
\label{x86-64}

\RU{Это расширение x86-архитуктуры до 64 бит.}\EN{It is a 64-bit extension to the x86 architecture.}

\RU{С точки зрения начинающего reverse engineer-а, наиболее важные отличия от 32-битного x86 это:}
\EN{From the reverse engineer's perspective, the most important changes are:}

\index{\CLanguageElements!\Pointers}
\begin{itemize}

\item
\RU{Почти все регистры (кроме FPU и SIMD) расширены до 64-бит и получили префикс R-. 
И еще 8 регистров добавлено. 
В итоге имеются эти \ac{GPR}-ы:}
\EN{Almost all registers (except FPU and SIMD) were extended to 64 bits and got a R- prefix.
8 additional registers wer added.
Now \ac{GPR}'s are:} \RAX, \RBX, \RCX, \RDX, 
\RBP, \RSP, \RSI, \RDI, \Reg{8}, \Reg{9}, \Reg{10}, 
\Reg{11}, \Reg{12}, \Reg{13}, \Reg{14}, \Reg{15}. 

\RU{К ним также можно обращаться так же, как и прежде. Например, для доступа к младшим 32 битам \TT{RAX} 
можно использовать \EAX:}
\EN{It is still possible to access the \IT{older} register parts as usual. 
For example, it is possible to access the lower 32-bit part of the \TT{RAX} register using \EAX:}

\RegTableOne{RAX}{EAX}{AX}{AH}{AL}

\RU{У новых регистров \TT{R8-R15} также имеются их \IT{младшие части}: \TT{R8D-R15D} 
(младшие 32-битные части), 
\TT{R8W-R15W} (младшие 16-битные части), \TT{R8L-R15L} (младшие 8-битные части).}
\EN{The new \TT{R8-R15} registers also have their \IT{lower parts}: \TT{R8D-R15D} (lower 32-bit parts),
\TT{R8W-R15W} (lower 16-bit parts), \TT{R8L-R15L} (lower 8-bit parts).}

\RegTableFour{R8}{R8D}{R8W}{R8L}

\RU{Удвоено количество SIMD-регистров: с 8 до 16:}
\EN{The number of SIMD registers was doubled from 8 to 16:} \XMM{0}-\XMM{15}.

\item
\RU{В win64 передача всех параметров немного иная, это немного похоже на fastcall 
\ifx\LITE\undefined
(\myref{fastcall})
\fi
.
Первые 4 аргумента записываются в регистры \RCX, \RDX, \Reg{8}, \Reg{9}, а остальные ~--- в стек. 
Вызывающая функция также должна подготовить место из 32 байт чтобы вызываемая функция могла сохранить 
там первые 4 аргумента и использовать эти регистры по своему усмотрению. 
Короткие функции могут использовать аргументы прямо из регистров, но б\'{о}льшие функции могут сохранять 
их значения на будущее.}
\EN{In Win64, the function calling convention is slightly different, somewhat resembling fastcall
\ifx\LITE\undefined
(\myref{fastcall})
\fi
.
The first 4 arguments are stored in the \RCX, \RDX, \Reg{8}, \Reg{9} registers, the rest~---in the stack.
The \gls{caller} function must also allocate 32 bytes so the \gls{callee} may save there 4 first arguments and use these 
registers for its own needs.
Short functions may use arguments just from registers, but larger ones may save their values on the stack.}

\RU{Соглашение }System V AMD64 ABI (Linux, *BSD, \MacOSX)\cite{SysVABI} \RU{также напоминает}\EN{also somewhat resembles}
fastcall, \RU{использует 6 регистров}\EN{it uses 6 registers} 
\RDI, \RSI, \RDX, \RCX, \Reg{8}, \Reg{9} \RU{для первых шести аргументов}\EN{for the first 6 arguments}.
\RU{Остальные передаются через стек}\EN{All the rest are passed via the stack}.

\ifx\LITE\undefined
\RU{См. также в соответствующем разделе о способах передачи аргументов через стек}
\EN{See also the section on calling conventions}~(\myref{sec:callingconventions}).
\fi

\item
\RU{\Tint в \CCpp остается 32-битным для совместимости.}
\EN{The \CCpp \Tint type is still 32-bit for compatibility.}

\item
\RU{Все указатели теперь 64-битные}\EN{All pointers are 64-bit now}.

\RU{На это иногда сетуют: ведь теперь для хранения всех указателей нужно в 2 раза больше места 
в памяти, в т.ч. и в кэш-памяти, не смотря на то что x64-процессоры могут адресовать только 48 бит
внешней \ac{RAM}.}
\EN{This provokes irritation sometimes: now one needs twice as much memory for storing pointers,
including cache memory, despite the fact that x64 \ac{CPU}s can address only 48 bits of external 
\ac{RAM}.}

\end{itemize}

\index{Register allocation}
\RU{Из-за того, что регистров общего пользования теперь вдвое больше, у компиляторов теперь больше 
свободного места для маневра, называемого \glslink{register allocator}{register allocation}.
Для нас это означает, что в итоговом коде будет меньше локальных переменных.}
\EN{Since now the number of registers is doubled, the compilers have more space for maneuvering called 
\glslink{register allocator}{register allocation}.
For us this implies that the emitted code containing less number of local variables.}

\ifx\LITE\undefined
\index{DES}
\RU{Для примера, функция вычисляющая первый S-бокс алгоритма шифрования DES, 
она обрабатывает сразу 32/64/128/256 значений, в зависимости от типа \TT{DES\_type} (uint32, uint64, SSE2 или AVX), 
методом bitslice DES (больше об этом методе читайте здесь~(\myref{bitslicedes})):}
\EN{For example, the function that calculates the first S-box of the DES encryption algorithm processes
32/64/128/256 values at once (depending on \TT{DES\_type} type (uint32, uint64, SSE2 or AVX)) 
using the bitslice DES method
(read more about this technique here ~(\myref{bitslicedes})):}

\lstinputlisting{patterns/20_x64/19_1.c}

\RU{Здесь много локальных переменных. Конечно, далеко не все они будут в локальном стеке. 
Компилируем обычным MSVC 2008 с опцией \Ox:}
\EN{There are a lot of local variables. 
Of course, not all those going into the local stack.
Let's compile it with MSVC 2008 with \Ox option:}

\lstinputlisting[caption=\Optimizing MSVC 2008]{patterns/20_x64/19_2_msvc_Ox.asm}

\RU{5 переменных компилятору пришлось разместить в локальном стеке.}
\EN{5 variables were allocated in the local stack by the compiler.}

\RU{Теперь попробуем то же самое только в 64-битной версии MSVC 2008:}
\EN{Now let's try the same thing in the 64-bit version of MSVC 2008:}

\lstinputlisting[caption=\Optimizing MSVC 2008]{patterns/20_x64/19_3_msvc_x64.asm}

\RU{Компилятор ничего не выделил в локальном стеке, а \TT{x36} это синоним для \TT{a5}.}
\EN{Nothing was allocated in the local stack by the compiler, \TT{x36} is synonym for \TT{a5}.}
\fi

\iffalse
% FIXME1 невнятно
\RU{Кстати, видно, что функция сохраняет регистры \RCX, \RDX в отведенных для 
этого вызываемой функцией местах, 
а \Reg{8} и \Reg{9} не сохраняет, а начинает использовать их сразу.}
\EN{By the way, we can see here that the function saved the \RCX and \RDX registers in space allocated by the \gls{caller},
but \Reg{8} and \Reg{9} were not saved but used from the beginning.}
\fi

\RU{Кстати, существуют процессоры с еще большим количеством \ac{GPR}, например, 
Itanium ~--- 128 регистров.}
\EN{By the way, there are CPUs with much more \ac{GPR}'s, e.g. Itanium (128 registers).}

\ifdefined\IncludeARM
\section{ARM}

\RU{64-битные инструкции появились в}\EN{64-bit instructions appeared in} ARMv8.
\fi

\ifx\LITE\undefined
\section{\RU{Числа с плавающей запятой}\EN{Float point numbers}}

\RU{О том как происходит работа с числами с плавающей запятой в x86-64, читайте здесь: \myref{floating_SIMD}.}
\EN{How floating point numbers are processed in x86-64 is explained here: \myref{floating_SIMD}.}
\fi

\ifx\LITE\undefined
\clearpage
\section{\RU{Простейшее четырехбайтное XOR-шифрование}\EN{Simplest possible 4-byte XOR encryption}}

\RU{Если при XOR-шифровании применялся шаблон длинее байта, например, 4-байтный, то его также легко
увидеть.}
\EN{If longer pattern was used while XOR-encryption, for example, 4 byte pattern, it's easy
to spot it as well.}
\RU{Например, вот начало файла kernel32.dll (32-битная версия из Windows Server 2008):}
\EN{As example, here is beginning of kernel32.dll file (32-bit version from Windows Server 2008):}

\begin{figure}[H]
\centering
\includegraphics[scale=\FigScale]{ff/XOR/4byte/original1.png}
\caption{\EN{Original file}\RU{Оригинальный файл}}
\end{figure}

\clearpage
\RU{Вот он же, но \q{зашифрованный} 4-байтным ключем:}
\EN{Here is it \q{encrypted} by 4-byte key:}

\begin{figure}[H]
\centering
\includegraphics[scale=\FigScale]{ff/XOR/4byte/encrypted1.png}
\caption{\EN{\q{Encrypted} file}\RU{\q{Зашифрованный} файл}}
\end{figure}

\RU{Очень легко увидеть повторяющиеся 4 символа.}
\EN{It's very easy to spot recurring 4 symbols.}
\RU{Ведь в заголовке PE-файла много длинных нулевых областей, из-за которых ключ становится видным.}
\EN{Indeed, PE-file header has a lot of long zero lacunes, which is the reason why key became visible.}

\clearpage
\RU{Вот начало PE-заголовка в 16-ричном виде:}
\EN{Here is beginning of PE-header in hexadecimal form:}

\begin{figure}[H]
\centering
\includegraphics[scale=\FigScale]{ff/XOR/4byte/original2.png}
\caption{PE-\EN{header}\RU{заголовок}}
\end{figure}

\clearpage
\RU{И вот он же, \q{зашифрованный}:}
\EN{Here is it \q{encrypted}:}

\begin{figure}[H]
\centering
\includegraphics[scale=\FigScale]{ff/XOR/4byte/encrypted2.png}
\caption{\EN{\q{Encrypted} PE-header}\RU{\q{Зашифрованный} PE-заголовок}}
\end{figure}

\RU{Легко увидеть визуально, что ключ это следующие 4 байта}
\EN{It's easy to spot that key is the following 4 bytes}: \TT{8C 61 D2 63}.
\RU{Используя эту информацию, довольно легко расшифровать весь файл.}
\EN{It's easy to decrypt the whole file using this information.}

\RU{Таким образом, важно помнить эти свойства PE-файлов:
1) в PE-заголовке много нулевых областей;
2) все PE-секции дополняются нулями до границы страницы (4096 байт), 
так что после всех секций обычно имеются длинные нулевые области.}
\EN{So this is important to remember these property of PE-files:
1) PE-header has many zero lacunas;
2) all PE-sections padded with zeroes by page border (4096 bytes),
so long zero lacunas usually present after all sections.}

\RU{Некоторые другие форматы файлов могут также иметь длинные нулевые области.}
\EN{Some other file formats may contain long zero lacunas.}
\RU{Это очень типично для файлов, используемых научным и инженерным ПО.}
\EN{It's very typical for files used by scientific and engineering software.}

\RU{Для тех, кто самостоятельно хочет изучить эти файлы, то их можно скачать здесь:}
\EN{For those who wants to inspect these files on one's own, they are downloadable there:}
\url{http://go.yurichev.com/17352}.

\subsection{\Exercise}

\begin{itemize}
	\item \url{http://challenges.re/50}
\end{itemize}


\fi
\ifdefined\IncludeARM
\clearpage
\section{\RU{Простейшее четырехбайтное XOR-шифрование}\EN{Simplest possible 4-byte XOR encryption}}

\RU{Если при XOR-шифровании применялся шаблон длинее байта, например, 4-байтный, то его также легко
увидеть.}
\EN{If longer pattern was used while XOR-encryption, for example, 4 byte pattern, it's easy
to spot it as well.}
\RU{Например, вот начало файла kernel32.dll (32-битная версия из Windows Server 2008):}
\EN{As example, here is beginning of kernel32.dll file (32-bit version from Windows Server 2008):}

\begin{figure}[H]
\centering
\includegraphics[scale=\FigScale]{ff/XOR/4byte/original1.png}
\caption{\EN{Original file}\RU{Оригинальный файл}}
\end{figure}

\clearpage
\RU{Вот он же, но \q{зашифрованный} 4-байтным ключем:}
\EN{Here is it \q{encrypted} by 4-byte key:}

\begin{figure}[H]
\centering
\includegraphics[scale=\FigScale]{ff/XOR/4byte/encrypted1.png}
\caption{\EN{\q{Encrypted} file}\RU{\q{Зашифрованный} файл}}
\end{figure}

\RU{Очень легко увидеть повторяющиеся 4 символа.}
\EN{It's very easy to spot recurring 4 symbols.}
\RU{Ведь в заголовке PE-файла много длинных нулевых областей, из-за которых ключ становится видным.}
\EN{Indeed, PE-file header has a lot of long zero lacunes, which is the reason why key became visible.}

\clearpage
\RU{Вот начало PE-заголовка в 16-ричном виде:}
\EN{Here is beginning of PE-header in hexadecimal form:}

\begin{figure}[H]
\centering
\includegraphics[scale=\FigScale]{ff/XOR/4byte/original2.png}
\caption{PE-\EN{header}\RU{заголовок}}
\end{figure}

\clearpage
\RU{И вот он же, \q{зашифрованный}:}
\EN{Here is it \q{encrypted}:}

\begin{figure}[H]
\centering
\includegraphics[scale=\FigScale]{ff/XOR/4byte/encrypted2.png}
\caption{\EN{\q{Encrypted} PE-header}\RU{\q{Зашифрованный} PE-заголовок}}
\end{figure}

\RU{Легко увидеть визуально, что ключ это следующие 4 байта}
\EN{It's easy to spot that key is the following 4 bytes}: \TT{8C 61 D2 63}.
\RU{Используя эту информацию, довольно легко расшифровать весь файл.}
\EN{It's easy to decrypt the whole file using this information.}

\RU{Таким образом, важно помнить эти свойства PE-файлов:
1) в PE-заголовке много нулевых областей;
2) все PE-секции дополняются нулями до границы страницы (4096 байт), 
так что после всех секций обычно имеются длинные нулевые области.}
\EN{So this is important to remember these property of PE-files:
1) PE-header has many zero lacunas;
2) all PE-sections padded with zeroes by page border (4096 bytes),
so long zero lacunas usually present after all sections.}

\RU{Некоторые другие форматы файлов могут также иметь длинные нулевые области.}
\EN{Some other file formats may contain long zero lacunas.}
\RU{Это очень типично для файлов, используемых научным и инженерным ПО.}
\EN{It's very typical for files used by scientific and engineering software.}

\RU{Для тех, кто самостоятельно хочет изучить эти файлы, то их можно скачать здесь:}
\EN{For those who wants to inspect these files on one's own, they are downloadable there:}
\url{http://go.yurichev.com/17352}.

\subsection{\Exercise}

\begin{itemize}
	\item \url{http://challenges.re/50}
\end{itemize}


\fi
\ifdefined\IncludeMIPS
\clearpage
\section{\RU{Простейшее четырехбайтное XOR-шифрование}\EN{Simplest possible 4-byte XOR encryption}}

\RU{Если при XOR-шифровании применялся шаблон длинее байта, например, 4-байтный, то его также легко
увидеть.}
\EN{If longer pattern was used while XOR-encryption, for example, 4 byte pattern, it's easy
to spot it as well.}
\RU{Например, вот начало файла kernel32.dll (32-битная версия из Windows Server 2008):}
\EN{As example, here is beginning of kernel32.dll file (32-bit version from Windows Server 2008):}

\begin{figure}[H]
\centering
\includegraphics[scale=\FigScale]{ff/XOR/4byte/original1.png}
\caption{\EN{Original file}\RU{Оригинальный файл}}
\end{figure}

\clearpage
\RU{Вот он же, но \q{зашифрованный} 4-байтным ключем:}
\EN{Here is it \q{encrypted} by 4-byte key:}

\begin{figure}[H]
\centering
\includegraphics[scale=\FigScale]{ff/XOR/4byte/encrypted1.png}
\caption{\EN{\q{Encrypted} file}\RU{\q{Зашифрованный} файл}}
\end{figure}

\RU{Очень легко увидеть повторяющиеся 4 символа.}
\EN{It's very easy to spot recurring 4 symbols.}
\RU{Ведь в заголовке PE-файла много длинных нулевых областей, из-за которых ключ становится видным.}
\EN{Indeed, PE-file header has a lot of long zero lacunes, which is the reason why key became visible.}

\clearpage
\RU{Вот начало PE-заголовка в 16-ричном виде:}
\EN{Here is beginning of PE-header in hexadecimal form:}

\begin{figure}[H]
\centering
\includegraphics[scale=\FigScale]{ff/XOR/4byte/original2.png}
\caption{PE-\EN{header}\RU{заголовок}}
\end{figure}

\clearpage
\RU{И вот он же, \q{зашифрованный}:}
\EN{Here is it \q{encrypted}:}

\begin{figure}[H]
\centering
\includegraphics[scale=\FigScale]{ff/XOR/4byte/encrypted2.png}
\caption{\EN{\q{Encrypted} PE-header}\RU{\q{Зашифрованный} PE-заголовок}}
\end{figure}

\RU{Легко увидеть визуально, что ключ это следующие 4 байта}
\EN{It's easy to spot that key is the following 4 bytes}: \TT{8C 61 D2 63}.
\RU{Используя эту информацию, довольно легко расшифровать весь файл.}
\EN{It's easy to decrypt the whole file using this information.}

\RU{Таким образом, важно помнить эти свойства PE-файлов:
1) в PE-заголовке много нулевых областей;
2) все PE-секции дополняются нулями до границы страницы (4096 байт), 
так что после всех секций обычно имеются длинные нулевые области.}
\EN{So this is important to remember these property of PE-files:
1) PE-header has many zero lacunas;
2) all PE-sections padded with zeroes by page border (4096 bytes),
so long zero lacunas usually present after all sections.}

\RU{Некоторые другие форматы файлов могут также иметь длинные нулевые области.}
\EN{Some other file formats may contain long zero lacunas.}
\RU{Это очень типично для файлов, используемых научным и инженерным ПО.}
\EN{It's very typical for files used by scientific and engineering software.}

\RU{Для тех, кто самостоятельно хочет изучить эти файлы, то их можно скачать здесь:}
\EN{For those who wants to inspect these files on one's own, they are downloadable there:}
\url{http://go.yurichev.com/17352}.

\subsection{\Exercise}

\begin{itemize}
	\item \url{http://challenges.re/50}
\end{itemize}


\fi
